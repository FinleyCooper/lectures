\documentclass{article}
\usepackage{../header}
\title{Analysis II - Example Sheet 2}
\author{Solutions by Finley Cooper}
\begin{document}
\maketitle
\newpage
\section*{Question 1}
\subsection*{Part (a)}
Suppose the norms $ ||\cdot || $ and $ ||\cdot ||' $ are Lipschitz equivalent. Then there exists real numbers $ A,B $ positive such that
\[
	A||x ||\le ||x ||' \le B||x|| \quad \text{for all} \ x\in V.\tag{\dagger}
\]
Let $ r=\frac 1A $ and $ R=\frac 1B $. Supose that $ x \in B_1'$. Hence we have that $ ||x||'< 1 $, so by $ (\dagger) $ we get $ A||x||< 1$ Hence it follows that $ ||x||< r $ so $ x\in B_r $. This gives $ B_r\subseteq B_1' $. Now take some $ x\in B_R $. Hence $ ||x||< R=\frac 1B $. So again by $ (\dagger) $ $ ||x||'< 1 $ so $ x\in B_1' $. So $ B_r\subseteq B_1'\subseteq B_R $.\par
Conversely suppose that there exists real numbers $ r,R $ such that
\[
  B_r\subseteq B_1'\subseteq B_R.
\]
Fix a $ x\in V $. We have that
\begin{align*}
	\frac {x}{||x||'}\in B_1' &\implies \frac{x}{||x||'}\in B_r\\
				  &\implies \left|\left|\frac{x}{||x||'}\right|\right|< r\\
				  &\implies \frac 1r ||x||< ||x||'.
\end{align*}
Similarly,
\begin{align*}
	\frac{Rx}{||x||}\in B_R & \implies \frac{Rx}{||x||}\in B_1' \\
				& \implies \left|\left|\frac{Rx}{||x||}\right|\right|'< 1\\
				& \implies ||x||'< \frac 1R ||x||.
\end{align*}
Hence $ ||\cdot || $ and $ ||\cdot ||' $ are Lipschitz equivalent.
\subsection*{Part (b)}
Suppose the norms $ ||\cdot || $ and $ ||\cdot ||' $ are Lipschitz equivalent. \[
	A||x ||\le ||x ||' \le B||x|| \quad \text{for all} \ x\in V.\]

Suppose now $ (x_n) $ is a sequence in $ V $ converging to $ x $ with respect to $ ||\cdot || $. Then we have that
\[
  ||x_n-x||'\le B||x_n-x||\to B\cdot 0 = 0
\]
as $ n\to \infty $. Symmetrically we have the converse statement.\par
Now suppose that $ x_n\to x $ with respect to $ ||\cdot || \iff x_n\to x$ with respect to $ ||\cdot ||' $. Take $ \alpha $ as the identity map from $ (V,||\cdot ||) $ to $ (V, ||\cdot ||') $. By the hypothesis it is continuous, hence there is some $ \delta>0 $ such that
\[
  ||x||<\delta \implies ||x||'<1
\]
and taking the inverse map there is some $ \varepsilon $ such that
\[
  ||x||'<1\implies ||x||<\varepsilon.
\]
Hence by part (a) the norms are Lipschitz equivalent.
\subsection*{Part (c)}
Let's show that $ ||\cdot || + |\varphi(\cdot )| $ defines a norm on $ V $. Firstly,
\[
  ||0||+|\varphi(0)|=0+0=0
\]
since $ \varphi $ is linear. The norm is clearly positive definite since $ ||x||=0 \iff x=0 $. $ \varphi $ is linear, so the norm has linearity in scalar multiplication. We're just left to prove the triangle inequality.
\begin{align*}
	||x+y||+|\varphi(x+y)| &\le ||x||+||y||+|\varphi(x)+\varphi(y)|\\
			       &\le ||x||+|\varphi(x)|+||y||+|\varphi(y)|.
\end{align*}
So this does define a norm on $ V $.\par
Suppose that $ \varphi $ is not continuous. So there exists a point $ x\in V $ and a sequence such that $ x_n\to x $ and $ \varphi(x_n)\to y $ with $ y\ne \varphi(x) $ (with the first limit being taken with respect to the $ ||\cdot || $ norm). We'll show the norms are not Lipschitz equivalent using (c). Call our newly define norm $ ||\cdot ||' $. Then
\begin{align*}
	||x_n-x||' &= ||x_n-x||+|\varphi(x_n-x)|.
		   &= ||x_n-x||+|\varphi(x_n)-\varphi(x)|
\end{align*}
But as $ n\to \infty $ the first term vanishes and the second term doesn't go to zero since $ \varphi(x)\ne \varphi(y) $ as limits are unique. Hence $ x_n $ doesn't converge to $ x $ in this new norm, so by (c) the norms are not Lipschitz equivalent.
\subsection*{Part (d)}
For this part we'll assume the Axiom of Choice so we can construct a basis for any vector space through Zorn's lemma. Let $ V $ be an infinite dimensional vector space. Let $ B $ be a basis for $ V $, take $ (e_n) $ to be some countable subset of the basis vectors and define a $ \varphi: V\to \R $ on this vector space by
\[
	\varphi(e_n)=n, \quad \varphi = 0 \ \text{ on other basis}
\]
so we can then extend $ \varphi $ to make it linear. $ \varphi $ is not bounded on the unit ball so not continuous. Then by (c) we have at least two non-Lipschitz equivalent norms, hence if we have a vector space with exactly one norm up to Lipschitz equivalence, $ V $ \textit{must} be finite-dimensional.
\section*{Question 2}
Suppose that $ f:X\to X' $ is continuous. Let $ V $ be some open set in $ X' $ and let's show that $ X\setminus \inv f(V) $ is closed instead. Take some sequence $ (x_n) $ in $ X\setminus \inv f(V) $ with $ x_n\to x $. So it is sufficient to show that $ x\notin \inv f(V) $.
\par
$ x_n\notin \inv f(V) $, so $ f(x_n)\in X' \setminus V $. Since $ f $ is continuous, $ f(x_n)\to f(x) $ and since $ V $ is open, $ X'\setminus V $ is closed, therefore $ f(x)\in X'\setminus V $. So $ f(x)\notin V $ so $ x\notin \inv f(V) $, hence $ \inv f(V) $ is open.
\section*{Question 3}
Since we're working over the vector space $ \R^n $ it is enough to show that $ X $ is closed and bounded. If we let $ f $ be the function describing the Euclidean metric (which is continuous), then it has a bounded image, so $ X $ is bounded. Now we're left to show $ X $ is closed. Suppose not. Then there is a point $ x $ which is a limit point of $ X $ but not in $ X $, so we can make $ ||y-x||_2 $ as small as we like for $ y\in X $. Hence the function
\begin{align*}
	f:&X\to \R\\
	 &y \to \frac{1}{||y-x||_2}
\end{align*}
is clearly continuous but unbounded. So $ X $ must be closed, so it is compact.\par
For a general metric space, $ (X,d) $.

\section*{Question 4}
\subsection*{Part (a)}
Let $ (x_k) $ be a sequence in $ X $. Given some $ \varepsilon>0 $ let \[
	X=\bigcup_{j=1}^NB_\varepsilon(x_j).
\]
So since we have finitely many balls, some ball must contain infintely many terms. So we have a subsequence $ (x_{n_i}) $ fully contained in some ball, say the ball centred at $ x_0 $. Hence for all $ p,q $ we have that
\[
  d(x_p,x_0)<\varepsilon,\quad d(x_q,x_0)<\varepsilon
\]
which implies that
\[
  d(x_p,x_q)\le d(x_p,x_0) + d(x_q,x_0) < 2\varepsilon.
\]
Hence $ (x_{n_i}) $ is Cauchy.
\subsection*{Part (b)}
Suppose $ (X,d) $ is complete and totally bounded. Let $ (x_n) $ be a sequence in the space. Then by (a) we have a Cauchy subsequence $ (x_{n_i}) $. By completeness this subsequence converges, so the space is compact.
\par
Conversely suppose the space is compact. Then from lectures we have that it is complete. Suppose that $ (X,d) $ is not totally bounded. We want to find some sequence with no convergent subsequence. Given some $ \varepsilon>0 $, define the sequence $ (x_n) $ inductively as follows.
\begin{itemize}
	\item Pick $ x_0\in X $ arbitrarily.
	\item Given we have defined $ x_0,x_1,\dots, x_i $ define
		\[
		A=\bigcup_{j=0}^iB_\varepsilon(x_j).
		\]
		We've supposed that $ X $ is not totally bounded, so $ A $ does not cover $ X $, hence pick $ x_{i+1}\in X\setminus A $, so $ x $ is at least $ \varepsilon $ away from all other elements. Hence all subsequences don't converge since not Cauchy, since we're always $ \varepsilon $ away from all other elements.
\end{itemize}
This is a contradicts compactness, hence $ (X,d) $ must be totally bounded and complete.
\section*{Question 5}
\begin{enumerate}
	\item Not topological. We can relate the open sets $ (0,1) $ and $ (1,\infty) $ by a the transformation $ \frac 1x $, which preserves the topology, but not the boundedness.
	\item Topological. A set is closed $ \iff $ it's complement is open, so the closed-ness of a set is completely determined by not being in the collection of open sets.
	\item Not topological. If we have $ \R $ under the Euclidean metric, we can define a new metric as
		\[
			d(x,y)=\frac{|x-y|}{1+|x+y|}
		\]
		which is bounded. The set $ A=[0,\infty) $ is closed in both metric spaces, but bounded in our new metric, but unbounded in the Euclidean metric, hence the property is dependent on our metric, so not a topological property. (The new metric generates the same open sets since the identity map between our metric spaces preserves open sets in both directions since it has an inverse which is also continuous; this is given by the continuity of the bounded metric.)
	\item Not topological. Use the same example as above. $ \R $ is not totally bounded under the Euclidean metric, but is under our new bounded metric.
	\item Not topological. Use the same example as above. We know that $ \R $ is complete under the Euclidean metric, but $ \R $ under our new metric isn't complete. For example consider the sequence $ x_n=n $, it's clearly Cauchy from looking at the metric, but there is no $ \infty $ for the sequence to tend to.
	\item Topological. Take $ X $ under some metric. Then $ X $ is sequentially compact by Question 4. Sequential compactness is a topological property so it's also sequentially compact under a new metric. So from the other direction of Question 4, $ X $ is totally bounded and complete under the new metric.
\end{enumerate}
\section*{Question 6}


















\end{document}.
