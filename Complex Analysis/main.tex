\documentclass{article}
\usepackage{../header}
\title{Complex Analysis}
\author{Notes by Finley Cooper}
\newcommand{\Arg}{\mathrm{Arg}}
\newcommand{\Log}{\mathrm{Log}}
\begin{document}
  \maketitle
  \newpage
  \tableofcontents
  \newpage
  \section{Complex Differentiation}
  \subsection{Definitions}
  The goal of this course is to develop the surprisingly rich theory of complex valued functions of one complex variable and the theory of integrating such functions along complex paths. The motivations for investigating such topics are complex polynomials of interest in geometry and number theory.\par
  We will also look at functions defined by power series such as the map from $ s\to \sum_{n\in N}\frac 1{n^s} = \zeta(s) $ will define a complex differentiable function for $ \Re(s)>1 $. There is also a connection to harmonic functions which is developed further in Analysis of Functions. We can also use complex methods to solve classical integrals or and differential equations which is developed more in IB Complex Methods.\par
  For this course we will use $ z\in \C $ and $ x=\Re(z), y=\Im(z) $. We will also use $ \theta $ for the argument of $ z $, $ \arg(z) $ which is well-defined up to adding $ 2\pi \Z $. We will use the principal argument $ \Arg(z)\in(-\pi,\pi] $.
  \begin{definition}
	  (Open disc) An \textit{open disc} or \textit{open ball} centred at $ a $ with radius $ r $ in $ \C $ is the set $ \{z\in \C\mid |z-a|<r\}  = B(a,r)= D(a,r)$.
  \end{definition}
  \begin{remark}
	  We will use $ \mathbb D = B(0,1) $ and $ \bar B(a,r) = \{ z\in \C\mid |z-a|\le r\} $.
  \end{remark}
  We use $ \C^* $ to denote $ \C\setminus \{0\} $.\par
  Recall that a set $ U\subseteq \C $ is open if it contains an open disc about each of its points.
  \begin{definition}
	  (Path) A \textit{path} in $ U\subseteq \C $ is a continuous map $ \gamma:[a,b]\to U $.
  \end{definition}
  \begin{definition}
	  (Path-connected) We say that $ U\subseteq \C $ is \textit{path-connected} if for all $ x,y\in U $ there exists a path $ \gamma:[0,1]\to U $ such that $ \gamma(0) = x $ and $ \gamma(1) = y $.
  \end{definition}
\begin{definition}
	(Domain) A \textit{domain} in $ \C $ is a non-empty path-connected subset of $ \C $.
\end{definition}
\begin{definition}
	(Closed path) If $ \gamma $ is a path and $ \gamma(a)=\gamma(b) $ then we say that $ \gamma $ is a \textit{closed path}.
\end{definition}
\begin{definition}
	($ C^1 $ path) We say a path is $ C^1 $ if it is continuously differentiable. We say a path is \textit{piecewise} $ C^1 $ if it has finitely many non-differentiable points but still globally continuous.
\end{definition}
\begin{definition}
	(Simple path) A path is \textit{simple} if it is injective except perhaps at the endpoints.
\end{definition}
\begin{definition}
  Let $ U\subseteq \C $ be open.
  \begin{enumerate}
	  \item We say that $ f:U\to \C $ is \textit{differentiable} at $ w\in U $ if
		  \[
			  f'(w)=\lim_{z\to w}\frac{f(z)-f(w)}{z-w}
		  \]
		  exists.
	  \item We say that $ f $ is \textit{holomorphic} at $ w\in U $ if $ \exists \varepsilon>0 $ such that $ f $ is differentiable on $ B(w,\varepsilon)\subseteq U $.
	  \item If $ f $ is holomorphic everywhere, we say $ f $ is \textit{entire}.
  \end{enumerate}
\end{definition}
\begin{remark}
  Some authors use analytic for holomorphic.
\end{remark}
\begin{remark}
  The usual rules for differenting sums and products and the inverse of a function (when it exists) apply exactly like we say in IA Analysis I with exactly the same proof.
\end{remark}
Any $ f:U\to \C $ can be written as $ f(z)=f(x+iy) = u(x,y)+iv(x,y) $ where $ u,v:U\to \R $ are the real and imaginary parts of $ f $.\par
Recall that $ u:U\to \R $ is differentiable at $ (c,d)\in U $ with derivative $ Du\mid_{(c,d)}= (\lambda,\mu) $ if and only if
\[
	\frac{u(x,y)-u(c,d) - (\lambda(x-c) + \mu(y-d))}{\sqrt{(x-c)^2+(y-d)^2}}\to 0
\]
as $ (x,y)\to (c,d) $.
\begin{proposition}
	(Cauchy-Riemann equations) Let $ f:U\to \C $ be defined on an open set $ U $ and write $ f=u+iv $, then $ f $ is differentiable t $ w= c+id\in U $ with $ f'(w)=p+iq $ if and only if $ u $ and $ w $ are both differentiable at $ (c,d) $ and $ u_x=v_y = p $ and $ -u_y = v_x=q $ at $ (c,d) $. Then $ f'(w) = u_x(c,d)+iv_x(c,d) $. 
\end{proposition}
\pf $ f $ is differentiable at $ w $ with derivative $ p+iq $ if and only if
\begin{align*}
	\lim_{z\to w}\frac{f(z)-f(w) -(z-w)(p+iq)}{z-w} = 0
\end{align*}
One can check that $ \lim_{z\to a}\frac{f(z)}{g)z)}=0 $ if and only if $ \lim_{z\to a}\frac{f(z)}{|g(z)|} = 0 $ and $ (p+iq)(z-w) = p(x-c)-q(y-d)+i(q(x-c)+p(y-d)) $ so using these and taking real and imaginary parts we get that
\begin{align*}
	\lim_{(x,y)\to (c,d)} \frac{u(x,y)-u(c,d)-(p(x-c)-q(y-d))}{\sqrt{(x-c)^2+(y-c)^2}} = 0
\end{align*}
for the real part. And
\begin{align*}
	\lim_{(x,y)\to (c,d)}\frac{v(x,y)-v(c,d) - (q(x-c)+p(y-d))}{\sqrt{(x-c)^2+(y-c)^2}}=0
\end{align*}
for the imaginary part. This is equivalent to saying that $ u $ is differentiable with $ Du\mid_{(c,d)} = (p,-q) $ and $ v $ is differentiable with $ Dv\mid_{(c,d)} = (q,p) $.\qed
\begin{remark}
  Let's make some remarks about the Cauchy-Riemann equations.
  \begin{enumerate}
	  \item If $ f=u +iv $ and $ u_x=v_u $ and $ u_y = -v_x $ at a point $ w $ we \textit{cannot} conclude that $ f $ is differentiable at $ w $ (Example Sheet 1).
	  \item If the partial derivatives $ u_x,u_y,v_x,v_y $ exist and are continuous in an open neighbourhood of $ w $ then the Cauchy-Riemann equations holding does imply complex differentiability.
  \end{enumerate}
\end{remark}
Let's see some examples.
\begin{enumerate}
	\item Polynomials are sums and products of the identity function, hence they are entire.
	\item If $ P $ and $ Q $ are polynomials, and $ U\subseteq \C\setminus \{x\mid Q(x)= 0\} $ then $ \frac PQ $ is differentiable on $ U $. These are called \textit{rational functions}.
	\item If $ f(x)=|x| $, this is not differentiable anywhere in $ \C $. $ f=u+iv $ with $ u=\sqrt{x^2 +y^2} $ and $ v=0 $. If $ (x,y)\ne (0,0) $ then
		\[
			u_x=\frac{x}{\sqrt{x^2+y^2}}, \qquad u_y = \frac{y}{\sqrt{x^2+y^2}}.
		\]
		So the Cauchy-Riemann equations do not hold, and if $ (x,y)= (0,0) $ we know that this isn't even differentiable in the real case, hence it's also not differentiable in the complex case. So $ f $ isn't differentiable anywhere.
\end{enumerate}
\begin{remark}
  Later we'll see that if $ f $ is holomorphic on $ U $, then $ f' $ is also holomorphic on $ U $. Then from the Cauchy-Riemann equations, we can see that $ f $ is harmonic. Conversely we can later see that every harmonic function on an open set in $ \R^2 $ is locally $ \Re(f) $ for some holomorphic function $ f $.
\end{remark}
\subsection{Conformal maps}
\begin{proposition}
  Let $ U\subseteq \C $ be a domain and suppose that $ f:U\to\C $ is holomorphic and $ f'(z)=0 $ on $ U $. Then $ f $ is constant.
\end{proposition}
\pf We will without proof the following elementary topological fact.
\begin{lemma}
	If $ U $ is a domain and $ \gamma:[0,1]\to U $ is a path with $ \gamma(0)=a $ and $ \gamma(1)=b $. Then there is another path $ \bar\gamma:[0,1]\to U $ with $ \bar\gamma(0)=a $ and $ \bar\gamma(1)=b $ where $ \bar\gamma $ is composed of finitely many segments, each parallel to the $ x $ or $ y $ axis, so $ \bar\gamma $ is piecewise-$ C^1 $.
\end{lemma}
Given this we know that $ u_x = v_y $ and $ u_y=-v_x $ hold. Since $ f' $ is zero, all these partials vanish on $ U $. Now the usual mean value theorem shows that $ u,v $ are constant along the segments of $ \bar\gamma $. So $ f(a)=f(b) $.\qed
\begin{definition}
	(Conformal) If $ f $ is holomorphc at a point $ w $ and $ f'(w)\ne 0 $ we say that $ f $ is \textit{conformal} at $ w $.
\end{definition}
This is a geometric property of $ f $.
\par
Suppose that $ U $ is a domain and $ f: U\to \C $ is conformal at $ w $. Let's take $ \gamma_i:(-\varepsilon,\varepsilon)\to U $ such that $ \gamma_i(0)=w, $ and $ \gamma_i'(0)\ne 0 $. So $ \gamma_i $ have non-zero tangent vectors through $ w $. The angle between these vectors is $ \arg(\gamma_1'(0)-\arg(\gamma_2'(0)) $. But then
\[
	\frac{(f\gamma_1)'(0)}{(f\gamma_2)'(0)} = \frac{f'(w)}{f'(w)}=\frac{\gamma_1'(0)}{\gamma_2'(0)}=1
\]
and so the angle between $ f\gamma_1 $ and $ f\gamma_2 $ at $ f(w) $ the same as for $ \gamma_1 $ and $ \gamma_2 $. So conformal means that $ f $ \textit{preserves} angles.
\begin{definition}
	(Conformal equivalence) If $ U,V $ are open in $ \C $ and $ f:U\to V $ is a holomorphic bijection which is everywhere conformal on $ U $ we say that $ f $ is a \textit{conformal equivalence} and $ U $ and $ V $ are conformally equivalent.
\end{definition}
\begin{remark}
  In this setting, if $ f $ is conformal, the inverse function theorem says that $ f $ is locally invertible and that local inverse is complex differentiable. If $ f $ is a bijection is it globally invertible, and hence the chain rule shows that $ \inv f $ is conformal.
\end{remark}
Let's see some examples.
\begin{enumerate}
	\item A linear map $ f(z) = az+b $ with $ a\ne 0 $ is a conformal equivalence $ \C\to \C $.
	\item $ f(z)=z^n  $ takes $ \{ z\mid 0\le \Arg(z) \le \frac \pi n\} \to \{ z\in \C \mid \Im(z) > 0 \} = \mathcal H$. This fails at the origin, but if we consider the open sector and open upper half plane.
	\item The exponential map, $ z\to \exp z =e^x e^{iy} $ sends verticle lines to circles with radius $ e^{\Re z} $. 
	\item $ z\in \mathcal H\iff z $ is closer to $ i $ than to $ -i$ $ \iff \left|\frac{z-i}{z+1} \right| < 1 $. The Mobius map $ z\to \frac{z-i}{z+i} $ takes $ \mathcal H \to \mathbb D$. Moreover $ f'(z) = \frac{2i}{(z+i)^2} $ is non-zero for $ z\in \mathcal H $, so $ f $ is an conformal equivalence.
\end{enumerate}
Recall that the Mobius group, $ \mathcal M $, is the group of mappings $ z\to \frac{az+b}{cz+d} $ with $ ad-bc\ne 0 $. $ A\in \mathcal M $ defines a conformal equivalence from $ \C\setminus \{-\frac dc\}\to \C\setminus \{\frac ac\} $, but it's much better to think of it as a conformal equivalence to the extended complex plane $ \C_\infty $. Recall that Mobius maps are triply transitive, so
\[
	z\to \frac{(z-z_1)(z_2-z_3)}{(z-z_3)(z_2-z_1)}
\]
sends the triple $ (z_1,z_2,z_3) $ to $ (0,1,\infty) $.
\par
Recall further that Mobius maps send circlines to circlines (where a circline is a circle or a line plus the point at infinity).
\par
Suppose we have a sector bounded between lines $ L_1, L_2 $. What are the images of a sector under a Mobius map? This depends on if our Mobius map $ T $ is such that $ \inv T(\infty) \in L_1 $ or $ \inv T(\infty) \in L_2 $. Any region formed by two non-pallel lines, a circle and a line, or two circles could be image under a Mobius map.
\par
The \textit{Joukowsky transform} is the map sending $ z\to \frac 12\left(z+\frac 1z\right) =\frac{z^2+ 1}{2z}$. We can compute that $ f'(z) = \frac 12-\frac 1{2z^2} $ and $ f $ is holomorphic except at $ 0 $ and is conformal except at $ 0 $ and $ \pm 1 $. Take the circle passing through $ -1 $ and $ -i $ not centred at the origin. Under the map $ f $, creates a kink at $ -1 $ since it's not conformal, but the curve is nice everywhere else since it's conformal on the rest of the circle. The transformed curve resembles an aerofoil. The incompressible fluid equations, $ \nabla\times v = 0, v=\nabla \phi $ implies that $ \nabla^2 \phi $ vanishes, so $ \phi $ is harmonic. So $ \phi $ is locally $ \Re(f) $, with $ f $ holomorphic. This is used to transfer questions about fluid flow about complicated shapes to the much simplier circle.
\par
Let $ S^1 =\partial \mathbb D = \{x\in \C: |x|=1\} $
\begin{definition}
	(Simply connceted) Let $ U\in \C $ be a domain. We say that $ U $ is \textit{simply connected} if every continuous map $ \gamma: S^1 \to U $ extends to a map  $ \hat\gamma:\overline{\mathbb{D}}\to U $ such that $ \hat\gamma\mid_{\partial \mathbb{D}} = \gamma $.
\end{definition}
\begin{remark}
  This is the notion that any loop can be continuously shrunk to a point.
\end{remark}
For example for the annulus, if we take $ \gamma $ to be the closed path going all the way around the annulus, we will later prove that no such $ \hat\gamma $ can exist.
\begin{theorem}
	(Riemann mapping theorem) If $ U\subset \C $ is a proper subdomain and $ U $ is simply connected, then $ U $ is conformally equivalent to the disc, $ \mathbb D = \{z:|z|<1\} $.
\end{theorem}
\begin{remark}
  The notation of being simply connected is invariant under conformal equivalences. Being simply connected is a topological property and is invariant under any homeomorphism (since conformal maps are homeomorphism).
\end{remark}
\begin{remark}
  The condition that $ U\ne \C $ is essential. If $ \C $ was conformally equivalent to the open disc then there would exist a function $ f: \C\to \mathbb D $ conformal, hence a homeomorphism, but from IB Topological Spaces, homeomorphisms are closed maps, hence since $ \C $ is closed, $ \mathbb D $ should be closed, but it isn't hence $ f $ cannot exist. We will see a non-topological proof later in the course.
\end{remark}
A useful construction of holomorphic functions (and we'll see later the universal construction) is convergent power series.
\subsection{Power series}
Let's see some reminders from IB Analysis II.
\begin{enumerate}
	\item A sequence $ f_n $ of functions on a set $ T $ converges uniformly to a function $ f $ on $ T $ if that $ \forall \varepsilon>0\ \exists N\ \st \ \forall \ n\ge N\in \N,\ |f_n(x)-f(x)|<\varepsilon $.
	\item The uniform limit of continuous functions is continuous.
	\item The Weiestress M-test says that if we have values $ M_n\in \R_{>0} $ such that $ |f_n(x)|\le M_n $ for all $ x\in T $, then if $ \sum_{n=1}^\infty M_n $ converges, then the series $ \sum_{n=1}^\infty f_n(x) $ converges absolutely and uniformly on $ T $.
	\item Suppose we have a sequence $ \{a_n\} $ then there is a unique $ R\in [0,\infty] $ such that the series $ \sum_{n=0}^\infty a_n (z-a)^n $ converges absolutely if $ |z-a|<R $ and diverges if $ |z-a|>R $. Moreover if $ 0<r<R $ then the series converges uniformly on $ \{z:|z-a|<r\} $. This is called the radius of convergence. The root test says that
		\[
			R = \frac1{\limsup_{n\to\infty} \sqrt[n]{|a_n|}}.
		\]
\end{enumerate}
\begin{theorem}
  Suppose that
  \[
	  f(z) = \sum_{n=0}^\infty c_n(z-a)^n
  \]
  is a complex power series with radius of convergence $ R>0 $. Then
  \begin{enumerate}
	  \item $ f $ is holomorphic on $ B(a,R) = \{z:|z-a|<R\} $;
	  \item $ f'(z) = \sum nc_n(z-a)^{n-1} $ which also has radius of convergence $ R $;
	  \item $ f $ is infinitely complex differentiable on $ B(a,R) $ with
		  \[
			  f^{(n)}(a)= n!c_n.
		  \]
  \end{enumerate}
\end{theorem}
\pf We can see (iii) follows from (ii) so enoughto prove (i) and (ii). Take $ a=0 $ \textit{wlog}. Since $ n|c_n|> |c_n| $ the radius of convergence of $ \sum_{n>= 1} nc_n(x_a)^{n-1} $ is at most $ R $. But if $ |z|< R_1 < R $ then we can see that
\[
	\frac{|nc_n z^{n-1}|}{|c_n R_1^{n-1}|} = n\left|\frac z{R_1}\right|^{n-1}\to 0
\]
which is true for all $ R_1<R $, hence the radius of convergence must be $ R $.\par
Now to show $ f $ is differentiable with that derivative. Pick $ z,w $ with $ |z|,|w|\le R_1< R $. Define
\[
	\varphi(z,w) = \sum_{n=1}^\infty c_n\sum_{j=0}^{n-1} z^j w^{n-1-j}.
\]
We can see that
\[
	\left|c_n\sum_{j=0}^{n-1} z^j w^{n-1-j}\right|\le n|c_n|R_1^n
\]
so this convergences uniformly with $ |z|\le R_1 $ and $ |w|<R_1 $ so the limit is continuous. If $ z\ne w $ we can sum as a geometric series, so
\[
	\varphi(z,w) = \sum_{n=1}^\infty c_n\left(\frac{z^n - w^n}{z-w}\right) = \frac{f(z)-f(w)}{z-w}.
\]
Taking the limit as $ w\to z $ we get that
\begin{align*}
\lim_{w\to z} \frac{f(z)-f(w)}{z-w} &= \varphi(z,z)\\
				    &= \sum_{n=1}^\infty c_n n z^{n-1} = f'(z).
\end{align*}
Hence we're done. \qed
\begin{corollary}
  Suppose we have a power series
  \[
	  f(z) = \sum_{n=-0}^\infty c_n(z-a)^n
  \]
  with $ R>0 $ and suppose that $ f $ vanishes on $ B(a,\varepsilon) $ with $ \varepsilon\in (0,R) $. Then $ f $ vanishes identically.
\end{corollary}
\pf All derivatives are zero at $ x=a $, hence $ f(z)=0 $\qed.
\par
Now we can define some familiar functions
\subsection{Exponentials, logarithms, and branch cuts}
\begin{definition}
	(Exponential function) We define the function
	\[
		e^z = \exp(z) = \sum_{n=0}^\infty \frac{z^n}{n!},
	\]
	which has the properties
	\begin{enumerate}
		\item Radius of convergence is $ \infty $;
		\item $ \frac d{dz} e^z = e^z $;
		\item $ e^0 =1 $. If we fix $ w $ and let $ F(z) = e^{z+w}e^{-z} $ then $ F'(z) $ vanishes, hence $ F $ is constant, and $ F(0) = e^w $, so $ e^{z+w} = e^ze^w $ holds for all $ z,w\in \C $.
		\item $ e^z $ never vanishes on $ \C $.
	\end{enumerate}
\end{definition}
\begin{remark}
  We define the trigonometic functions again as we did in $ \R $. The exponential function, $ \sin $ and $ \cos $ are entire.
\end{remark}
This function is invertible over $ \R $, but in $ \C $ it is not invertible. In fact for every $ z\in \C $ there exists infinitely many $ w\in \C $ distinct such that $ e^w = z $. This can be seen since $ \exp $ is $ 2\pi i $ peroidic.
\begin{definition}
	(Branch of the logarithm) Let $ U\subseteq \C^* = \C \setminus \{0\} $ open. A continuous function $ \lambda: U\to \C $ is \textit{branch of the logarithm} on $ U $ if $ e^{\lambda(z)} = z $ for all $ z\in U $.
\end{definition}
The classical example is $ U = \C \setminus \R_{\le 0} $, the slit plane. Let $ \Log: U \to \C $ be the defined by $ \Log(z) = \log(z) + i\theta $ where $ \theta= \Arg(z) $. This is called the principal branch of the logarithm.
\begin{proposition}
	$ \Log $ is holomorphic on $ \C\setminus \R_{\le 0} $ with
	\begin{enumerate}
		\item  $ \frac{d}{dz}\Log(z) = \frac 1z $;
		\item If $ |z|<1 $ then
			\[
				\Log(1+z) = \sum_{n=1}^\infty (-1)^{n-1}\frac {z^n}n.
			\]
	\end{enumerate}
\end{proposition}
\pf $ \Log $ is continuous on $ U $ and $ \exp $ is continuous on $ \C $. If $ z= e^x $ and $ w=e^y $, then
\begin{align*}
	\frac{\Log(z)-\Log(w)}{z-w} &= \frac{x-y}{e^x-e^y}
\end{align*}
which as $ x\to y $ converges to $ \frac 1{e^y} = \frac 1w $.\par
By the ratio test, $ \Log(1+z) $ has a radius of convergence $ 1 $. We know that $ \Log(1+z) $ has derivative
\[
	1-z+z^2-\cdot = \frac 1{1+z}
\]
so since the claimed power series and $ \Log(1+z) $ have the same derivative, they differ by a constant, and since they are both zero at $ z=0 $, that constant is zero, so $ \Log(1+z) $ has the power series as claimed.\qed
\begin{remark}
  \
  \begin{enumerate}
	  \item Later we'll see that there is no branch of the logarithm defined on $ \C^* $.
	  \item You can define a continuous branch of $ \log $ on any simply connected domain not containing zero.
  \end{enumerate}
\end{remark}
\begin{definition}
	(Multivalued power function) For a value $ \alpha\in \C $, the multivalued function $ z^\alpha $ is by definition $ \exp(\alpha\log(z)). $.
\end{definition}
\begin{remark}
	If for $ U\subseteq \C^* $ we have a branch of $ \log $ specified on $ U $, then $ z^\alpha $ becomes single-valued on $ U $.
\end{remark}
\begin{remark}
	If $ \alpha\in \Z $, then this is single-valued and if $ \alpha\in \Q $ then it is finitely many valued, for example $ z^{1/2} $ has two values. Note we cannot define a square root function globally on $ \C $.
\end{remark}
The function $ f(z) = z(z-1) $ admits a single valued square root on each of the domains $ \C\setminus [0,1] $ and $ \C\setminus (\R_{\ge 0} \cup \R_{\ge 1}) $. Hence
\begin{align*}
	(z(z_1))^{\frac 12} = \exp(\frac 12 (\log(z) - \log(z-1))).
\end{align*}
So taking a path around the removed interval in the complex plane is fine since both $\log$ terms jump, giving a total jump $ 2\pi i$ which doesn't change the value since $ \exp $ is $ 2\pi i $ periodic.
\begin{definition}
	(Branch point) A point $ p\in \C $ is a \textit{branch point} of a multivalued function $ \phi $ if there is no continuous single-value definition of $ \phi $ in $ B(0,\varepsilon) $ for any $ \varepsilon > 0 $.
\end{definition}
For example $ 0 $ is a branch of $ \log $. $ \{0,1\} $ are branch points of $ (z(z-1))^{1/2} $.
\section{Contour Integration}
\subsection{Basic properties}
\begin{definition}
	(Riemann integrablility in $ \C $) A function $ f:[a,b]\to\C $ is \textit{Riemann integrable} if it's real and imaginary parts are both Riemann integrable and
	\[
		\int_a^b f(t)\mathrm dt = \int_a^b \Re(f(t)) \mathrm dt + i\int_a^b \Im(f(t)) \mathrm dt.
	\]
\end{definition}
Note that we have
\[
	\left|\int_a^b f(t)\mathrm dt\right| \le \int_a^b |f(t)|\mathrm dt.
\]
From now on $ f $ will always be continuous and hence Riemann integrable.
\begin{proposition}
	For $ f:[a,b]\to \C $ we have that
	\[
		\left|\int_a^b f(t)\mathrm dt\right| \le \sup_{t\in[a,b]}|f(t)|(b-a)
	\]
	with equality if and only if $ f $ is constant.
\end{proposition}
\pf Let $ \theta = \Arg\int_a^b f(t)\mathrm dt $ and $ M =\sup_t |f(t)| $. Hence
\begin{align*}
	\left|\int_a^b f(t)\mathrm dt\right| &= e^{-i\theta} \int_a^b f(t)\mathrm dt\\
					     &= \int_a^b \Re(e^{-i\theta} f(t))\mathrm dt\\
					     &\le \int_a^b |e^{-i\theta} f(t)|\mathrm dt \\
					     &\le M(b-a)
\end{align*}
and equality is equivalent to $ \Arg(\theta) $ being constant and $ |f| = M $ everywhere, hence equality occurs if and only if $ f $ is constant.\qed
\par
\begin{definition}
	(Contour) A \textit{contour} is a simple, closed path.
\end{definition}
\begin{definition}
	(Contour integral) If $ U $ is a domain, $ f:U\to \C $ is continuous and $ \gamma:[a,b]\to U $ is a $ C^1 $-smooth curve then
	\[
		\int_\gamma f(z) \mathrm dz = \int_a^b f(\gamma(t))\gamma'(t)\mathrm dt.
	\]
\end{definition}
This definition extends in the obvious way to paths. We have some properties about contour integrals.
\begin{enumerate}
	\item They are linear;
	\item They are additive along paths;
	\item They are independent of parameterisation.
\end{enumerate}
\begin{remark}
  If we set the length of $ \gamma $ as
  \[
    |\gamma| = \int_a^b |\gamma'(t)|\mathrm dt
  \]
  this is \textit{not} independent of parameterisation.
\end{remark}
If $ (-\gamma)(t) = \gamma(-t) $ so $ -\gamma: [-b,-a]\to U $, $ \int_{-\gamma} f = -\int_\gamma f $.
\par
Let $ U=C^* $ and $ f(z) = z^n $, $ n\in \Z $. Let $ \phi:[0,2\pi]\to U $ sending $ t\to e^{it} $ then
\[
  \int_\phi f(z)\mathrm dz = \begin{cases}
     2\pi i & n = -1 \\
     0 & \text{otherwise}
  \end{cases}
\]
Let's prove this.
\begin{align*}
	\int_\phi f(z)\mathrm dz &= \int_0^{2\pi} e^{nit}(ie^{it})\mathrm dt \\
				 &= i\int_0^{2\pi} e^{i(n+1)t} \mathrm dt\\
				 &= \begin{cases}
					 2\pi i & n+1 = 0\\
					 0 & \text{otherwise}
				 \end{cases}.
\end{align*}
\begin{theorem}
	(Fundamental Theorem of Calculus) Suppose that $ U $ is a domain $ F:U\to \C $ is holomorphic and $ F'(z) $ is continuous on $ U $. Then for a path $ \gamma:[a,b]\to U $
	\[
	  \int_\gamma F'(z)\mathrm dz = F(\gamma(b)) - F(\gamma(a)).
	\]
	In particular if $ \gamma $ is closed then we get zero.
\end{theorem}
\pf 
\begin{align*}
	\int_\gamma F'(z)\mathrm dz &= \int_a^b F'(\gamma(t))\gamma'(t)\mathrm dt \\
				    &= \int_a^b (F\circ \gamma)'(t)\mathrm dt\\
				    &= (F\circ \gamma)(b) - (F\circ\gamma)(a)\qed
\end{align*}
If $ n\ne -1 $ then $ z^n =\frac {d}{dt}\left(\frac{z^{n+1}}{n+1}\right) $ so the integral of $ z^n $ around a closed path around the origin, is zero.
\begin{definition}
	(Antiderivative) If $ U $ is a domain and $ f:U\to \C $ is continuous and $ F:U\to \C $ is holomorphic with $ F'(z)=f(z) $ then we say that $ F $ is an \textit{antiderivative} for $ f $ on $ U $.
\end{definition}
So $ f(z) = \frac 1z $ has no antiderivative on $ \C^* $.
\begin{remark}
  Later we'll show (without being circular) that the hypothesis saying $ F'(z) $ is continuous actually always holds (which we already know if our holomorphic function came from a power series).
\end{remark}
\subsection{Cauchy's theorem}
\begin{lemma}
  If $ \gamma $ is a $ C^1 $ path and $ f $ is continuous then
  \[
	  \left|\int_\gamma f(z)\mathrm dz\right|\le \mathrm{length}(\gamma)\sup_\gamma|f|.
  \]
\end{lemma}
\pf
\begin{align*}
	\left|\int_\gamma f(z)\mathrm dz\right| &= \left|\int_a^b f(\gamma(t))\gamma'(t)\mathrm dt\right|\\
						&\le \int_a^b |f(\gamma(t))||\gamma'(t)|\mathrm dt\\
						&\le \sup_{a\in [a,b]} |f(\gamma(t))|\mathrm{length}(\gamma).\qed
\end{align*}
We can think of the following theorem as a converse to FTC.
\begin{theorem}
  Let $ U $ be a domain and $ f:U\to \C $ be continuous. Assume that
  \[
    \int_\gamma f(z)\mathrm dz = 0
  \]
  for all piecewise-$ C_1 $ closed curves $ \gamma $ in $ U $. Then $ f $ has an antiderivative $ F(z) $ on $ U $.
\end{theorem}
\pf Pick $ a_0\in U $. $ U $ is a domain so path-connected. So for $ w\in U $ pick a piecwise-$ C^1 $ path $ \gamma_w $ from $ a_0 $ to $ w $. Set
\[
	F(w)=\int_{\gamma_w}f(z)\mathrm dz.
\]
Our hypothesis that the integral over a closed path of $ f $ is zero implies that $ F(w) $ does not depend on the choice of path hence $ F $ is well-defined. $ U $ is open so for some $ \varepsilon>0 $, $ B(w,\varepsilon)\subseteq U $. If $ |h|<\varepsilon $ let $ \delta_h $ be the radial path inside this ball from $ w $ to $ w+h $. Then
\begin{align*}
	F(w+h) &= \int_{\gamma_w + \delta_h} f(z)\mathrm dz\\
	       &= \int { \gamma_w} f(z)\mathrm dz + \int_{\delta_h}f(z)\mathrm dz\\
	       &= F(w) + hf(w) - \int_{\delta_h}f(z)-f(w)\mathrm dz
\end{align*}
So
\begin{align*}
	\left|\frac{F(w+h)-F(w)}h -f(w)\right| &= \left|\frac 1h \int_{\delta_h} f(z)-f(w)\mathrm dz\right|\\
					       &\le \frac 1{|h|}\mathrm{length}(\delta_h)\sup_{z\in \delta_h}|f(z)-f(w)|
\end{align*}
As $ h\to 0 $ the right hand side goes to zero by continuity of $ f $. So $ F $ is complex differentiable at $ w $ and $ F'(w)=f(w) $ for all $ w\in U $. So $ f $ has an antideriviative.\qed
\begin{definition}
	(Convex) A domain $ U $ is \textit{convex} if $ \forall p,q\in U $, the straight line segment connecting $ p $ and $ q $ lies in $ U $. i.e.
	\[
		pt+(1-t)q \in U,\quad\forall t\in[0,1].
	\]
\end{definition}
\begin{definition}
	(Star-shaped) A domain $ U $ is \textit{star-shaped} if there exists a $ p_0 $ such that for all $ q\in U $, the line segment $ p_0, q$ lies in $ U $.
\end{definition}
We have the implications
\[
	\text{disc} \implies \text{convex}\implies \text{star-shaped}\implies \text{simply connected} \implies \text{path-connected}
\]
\begin{remark}
	All the theorem's we will prove for star-shaped sets related to Cauchy's theorem will extend to further sets. Simply connected is the property for Cauchy's theorem to hold for all curves in a domain, but we'll just prove the theorems for star-shaped sets for now.
\end{remark}
\begin{lemma}
  Let $ U $ be star-shaped and $ f:U\to \C $ be continuous and suppose that
  \[
    \int_\gamma f = 0
  \]
  where $ \gamma $ is a \textit{triangle} in $ U $ where a \textit{triangle} is a piecewise-$ C_1 $ path formed by $ 3 $ edges of a Euclidean triangle and the whole triangle including its interior is contained in $ U $. Then $ f $ has an antiderivative on $ U $.
\end{lemma}
\pf Suppose that $ U $ is star-like with respect to some point $ p_0 $ and for $ w\in U $ let $ \gamma_w $ be the path connecting $ p_0 $ to $ w $ in $ U $. Taking $ h $ as before, the paths $ \gamma_w $, $ \gamma_h $, $ \gamma_{w+h} $ form a triangle in $ U $ so we can use our previous proof.
\begin{theorem}
	(Cauchy's theorem for triangles) Let $ U $ be a domain and $ T\in U $ be a triangle in $ U $. If $ f: U \to \C $ is holomorphic then
	\[
		\int_{\partial T} f = 0
	\]
	where $ \partial T $ is the piecewise-$ C_1 $ closed path given by the boundary of $ T $.
\end{theorem}
\pf Let $ L = \mathrm {length} (\partial T) $ and $ I = \left|\int_{\partial T} f\right| $. By bisecting edges we can $ T $ into $ 4 $ smaller triangles creating all 4 $ \Delta $ anticlockwise, note that
\[
  \partial T = \partial \triangle_0 \cup \partial \triangle_1 \cup \partial\triangle_2 \cup \partial \triangle_3
\]
so there exists an $ i $ such that $ \left| \int_{\partial \triangle_i} f\right| \ge \frac I4 $. Label by $ T_1 $ such a triangle $ \triangle_i $. Now iterate constructing a sequence of triangles $ T\supseteq T_1\supseteq T_2\supseteq \dots $ hence
\begin{enumerate}
	\item $ \left| \int_{\partial T_n} f\right| \ge \frac I{4^n} $;
	\item $ \mathrm{length}(\partial T_n) = 2^{-n} L $.
\end{enumerate}
\begin{claim}
  We have that:
  \[
	  \bigcap_{n \ge 1} T_n = \{w\}.
  \]
\end{claim}
\pf We have that the length of the triangle goes to zero hence the diameter goes to zero, so by uniqueness of limits there can be at most one point in the intersection. The intersection being non-empty is a consequence of compactness and follows from applying Bolzano-Weiestrass to each coordinate. So the claim is true.\par
Define
\[
g(z) = \frac{f(z)-f(w)}{z-w}-f'(w)
\]
which vanishes at $ w $ since $ f $ is holomorphic.
\begin{align*}
	\frac I{4^n} \le \left|\int_{\partial T_n} f\right| &= \left| \int_{\partial T_n} f(z)-f(w)-(z-w)f'(w)\ \mathrm dz\right|\\
							    &\le\left|\frac L{2^n}\sup_{z\in\partial T_n}|(z-w)g(z)|\\
							    &\le \frac {L^2}{4^n}\sup_{z\in\partial T_n}|g(z)|\\
	I &\le L^2 \sup_{z\in\partial T_n}|g(z)|\to0
\end{align*}
as $ n\to\infty $.\qed
\begin{corollary}
	(Convex Cauchy) Let $ f $ be a holomorphic on a star-shaped domain $ U $. Then $ \int_\gamma f =0 $ for any piecewise-$ C_1 $ closed curve $ \gamma $.
\end{corollary}
\pf The previous argument implies that $ \int_{\partial T} f =0	$ where $ T $ is a triangle on $ U $. Thus $ f $ has an antiderivative $ F $ on $ U $. So $ F'(z) = f(z) $. So then
\[
  \int_\gamma f(z)\mathrm dz = \int_\gamma F'(z)\mathrm dz=0
\]
by FTC.\qed
\begin{proposition}
  Let $ U $ be a domain and $ f:U \to \C $ be continuous. Suppose that there exists a finite set $ S\subset U $ such that $ f $ is holomorphic on $ U\setminus S $. THen if $ T \subseteq U $ is a triangle, then
  \[
	  \int_{\partial T}f = 0.
  \]
\end{proposition}
\pf Without loss of generality suppose that $ S $ is singleton, subdividing $ T $ if necessary. If $ T $ contains no points of $ S $ we can use Cauchy for triangles. If it contains more than one point of $ S $ divide into smaller triangles each containing one or no points and adding. Now suppose that $ a\in S $ and $ a\in T'\subseteq T $, a smaller triangle. Through a geometric argument we can subdivide $ T $ into smaller triangles one of which is $ T' $ containing $ a $. Orienting all of these triangles anticlockwise we see that
\[
	\int_{\partial T} f = \pm \int_{\partial T'} f
\]
due to Cauchy's theorem, the integral around all of the other triangles is zero. Now
\begin{align*}
	\left|\int_{\partial T'} f\right| &= \mathrm{length}(\partial T') \sup_{z\in \partial T'}|f(z)| \\
					  &= \mathrm{length}(\partial T') \sup_{z\in T} |f(z)|.
\end{align*}
Now since $ \sup_{z\in T}|f(z)| $ is bounded, we can send $ \mathrm{length}(\partial T') \to 0 $ by choosing $ T' $ to be as small as we like. Hence the integral vanishes. \qed
\begin{corollary}
  If $ U $ is star-shaped and $ S\subset U $ is finite and $ f:U\to \C $ is continuous and holomorphic on $ U\setminus S $ then
  \[
    \int_\gamma f =0
  \]
  for all closed curves $ \gamma\in U $.
\end{corollary}
\pf Using the antiderivative theorem, just as we used before to deduce convex Cauchy from Cauchy for triangles.\qed
\subsection{The Cauchy integral formula}
\begin{theorem}
	Let $ U\in \C $ be a domain, $ f:U\to \C $ be holomorphic on $ U $ and $ \overline{B(a,r)}\subseteq U $. For all $ z\in B(a,r) $ and $ 0<\rho \le r $, $ |z-a|<\rho $, we have that
	\[
		f(z) = \frac 1{2\pi i}\int_{|w-a|=\rho} \frac{f(w)}{w-z}\mathrm dw.
	\]
\end{theorem}
\pf Let
\[
	g(w) = \begin{cases}
		\frac{f(w)-f(z)}{w-z} & w\ne z\\
		f'(z) & w = z
	\end{cases}.
\]
This is continuous on $ \overline{B(a,r)} $ and holomorphic except perhaps at $ z $. So by Cauchy,
\[
	\int_{\partial B(a,r) = |w-a|=\rho}g(w)\ \mathrm dw = 0,
\]
hence
\[
	\int_{|w-a|=\rho} \frac{f(w)}{w-z}\mathrm dw = \int_{|w-a|=\rho} \frac{f(z)}{w-z}\ \mathrm dw.
\]
We'll use that
\[
	\frac1{w-z} =\frac1{w-a}\frac 1{1-\left(\frac{z-a}{w-a}\right)} = \frac 1{w-a}\sum_{n=0}^\infty \left(\frac {z-a}{w-a}\right)^n
\]
which converges uniformly on $ |w-a|=\rho $. So
\begin{align*}
	\int_{|w-a|=\rho}\frac{f(w)}{w-z}\ \mathrm dw &= f(z) \int_{|w-a|=\rho} \sum_{n=0}^\infty \frac{(z-a)^n}{(w-a)^{n+1}}\ \mathrm dw\\
						      &= f(z)\sum_{n=0}^\infty \int_{|w-a|=\rho}\frac{(z-a)^n}{(w-a)^{n+1}}\ \mathrm dw
\end{align*}
swapping the summation and the integral using uniform convergence. Recall that we showed that
\[
	\int_{|z|=1} z^n \ \mathrm dz
\]
gave us something nontrivial only when $ n=-1 $. So by a change of variables from $ w-a\to w $ we see only the $ n=0 $ term contributes and gives $ 2\pi i $. Hence we have that
\[
	\int_{|w-a|=\rho} \frac{f(w)}{w-z}\ \mathrm dw = 2\pi i f(z)
\]
which gives the result.\qed
\begin{corollary}
	(Mean value property) If $ f:U\to \C $ is holomorphic on a domain $ U $ and $ \overline{B(a,r)}\subseteq U $, then
	\[
		f(a) = \int_0^1 f(a+re^{2\pi i t}) \ \mathrm dt.
	\]
	So the value of $ f $ at $ a $ is the average of its values over a small linking circle.
\end{corollary}
\pf Parameterise the circle in the Cauchy integral formula, so
\begin{align*}
	f(a) = \frac 1{2\pi i} \int_{|w-a|= r} \frac {f(w)}{w-a}\ \mathrm dw &= \frac 1{2\pi i} \int_\gamma \frac f(\gamma(t)){\gamma(t)-a}\gamma'(t)\ \mathrm dt
\end{align*}
and the result follows simply by substituing $ \gamma $ in.
\begin{corollary}
	(Local maximum principle) If $ f: B(a,r)\to \C $ is holomorphic and $ |f(z)|\le |f(a)| $ for all $ z\in B(a,r) $ then $ f $ is constant.
\end{corollary}
\pf For $ 0<\rho<r $,
\begin{align*}
	|f(a)| &= \left|\int_0^1 f(a+\rho e^{2\pi it})\ \mathrm dt\right|\\
	       &\le \sup_{|z-a|=\rho}|f(z)|\\
	       &\le |f(a)|
\end{align*}
So the inequalities are equalties, hence $ |f(z)| = |f(a)| $ for all $ z $ such that $ |z-a|=\rho $, and $ \rho \in (0,r) $ is arbitrary hence $ |f| $ is constant so by Example Sheet 1 since $ f $ is holomorphic, $ f $ is constant.\qed
\begin{theorem}
	(Liouville's Theorem) Let $ f:\C\to\C $ be entire. Then if $ f $ is bounded it is constant.
\end{theorem}
\pf Let $ z_1,z_2\in C $ and pick $ R>\max\{2|z_1|,2|z_2|\} $. So
\begin{align*}
	f(z_1)-f(z_2) =\frac 1{2\pi i} \int_{\partial B(0,R)}\left(\frac{f(w)}{w-z_2}-\frac{f(w)}{w-z_2}\right)\ \mathrm dw. 
\end{align*}
Taking the absolute value of both sides,
\begin{align*}
	|f(z_1)-f(z_2)| &= \frac 1{2\pi}\left|\int_{\partial B(0,R)}\frac{f(w)(z_1-z_2)}{(w-z_1)(w-z_2)}\ \mathrm dw\right|\\
			&\le \frac 1{2\pi} 2\pi RM\frac{|z_1-z_2|}{(R/2)^2}\to 0
\end{align*}
as $ R\to\infty $. Hence $ f $ is constant.\qed
\begin{corollary}
	(Fundamental theorem of algebra) Every non-constant complex polynomial has a root in $ \C $.
\end{corollary}
\begin{remark}
  This is equivalent to saying all non-constant complex polynomials factors as a product of linear factors.
\end{remark}
\pf Let $ p(z) = a_0 + a_1z+ \dots + a_d z^d $, $ a_d\ne 0 $ and $ d>0 $. Note that
\[
	\left|\frac{p(z)}{z^d}\right| \to |a_d|
\]
as $ z\to\infty $. So $ |p(z)|\to\infty $ as $ |z| = \infty $. Let $ f(x) = \frac 1{p(z)} $. If $ p $ never vanishes, then $ f $ is entire. Note that $ |f(z)|\to 0 $ as $ |z|\to\infty $, so there exists a $ R>0 $ such that $ |f(z)|\le 1 $ for $ |z|>R $. But $ f $ is continuous, so $ f $ is bounded on $ |z|\le R $. Hence $ f $ is bounded on $ \C $, and entire, so we must have that $ f $ is constant by Liouville's theorem, which implies that $ p(z) $ is constant, which is a contradiction, so $ p $ must have a root in $ \C $.\qed
\subsection{Taylor and Morera's theorem}
\begin{theorem}
	(Taylor's Theorem) Let $ f:B(a,r) \to \C $ be holomorphic. Then $ f $ has a convergent power series representation
	\[
		f(z) = \sum_{n=0}^\infty c_n(z-a)^n
	\]
	with
	\[
		c_n = \frac{f^{(n)}(a)}{n!} = \frac 1{2\pi i} \int_{\partial B(a,\rho)} \frac {f(w)}{(w-a)^{n+1}}\ \mathrm dw
	\]
	for any $ 0<\rho <r $.
\end{theorem}
\pf For $ 0<\rho <r $, and $ z $ such that $ |z-a|<\rho $, the Cauchy integral formula gives that
\[
	f(z) = \frac 1{2\pi i} \int_{\partial B(a,\rho)}\frac{f(w)}{w-z}\mathrm dw.
\]
We'll expand
\[
	\frac 1{w-z} = \frac 1{(w-a)\left(1-\frac{(z-a)}{w-a}\right)}
\]
as a power series, so we get that
\begin{align*}
	f(z) &= \frac 1{2\pi i} \int_{\partial B(a,\rho)}f(w)\sum_{n=0}^\infty \frac{(z-a)^n}{(w-a)^{n+1}}\ \mathrm dw\\
	     &= \sum_{n=0}^\infty \left(\frac 1{2\pi i} \int_{\partial B(a,\rho)} \frac{f(w)}{(w-a)^{n+1}}\ \mathrm dw\right)(z-a)^n.
\end{align*}
And Taylor's theorem follows from here.\qed
\begin{remark}
  If $ f:U\to \C $ is holomorphic on a domain $ U $, at $ z\in U $, there exists a $ r>0 $ such that $ B(a,r)\subseteq U $, and Taylor's theorem gives us a power series representation on $ B(a,r) $, but this may not converge everywhere on $ U $.
\end{remark}
\begin{corollary}
  If $ f $ is holomorphic on a domain $ U $, then $ f $ is infinitely differentiable on $ U $. 
\end{corollary}
\pf For each $ a\in U $, apply Taylor's theorem for a ball around $ a $. Then since all power series are infinitely differentiable, we have that $ f $ is infinitely differentiable at $ a $, hence $ f $ is infinitely differentiable on $ U $. \qed
\begin{theorem}
	(Morera's theorem) Let $ U $ be a domain and $ f:U\to\C $ be continuous. If
	\[
	  \int_\gamma f = 0
	\]
	for all piecewise-$ C^1 $ closed paths in $ U $, the $ f $ is holomorphic on $ U $.
\end{theorem}
\pf By the hypothesis, we know that $ f $ has an antiderivative, $ F $ for some $ F $ holomorphic. And now we know that the derivative of a holomorphic function is holomorphic, so $ f $ is holomorphic.\qed 
\begin{corollary}
  Let $ f_n:U\to \C $ be a sequence of holomorphic functions and suppose $ f_n\to f $ uniformly. Then $ f $ is holomorphic on $ U $ and
  \[
	  f'(z) = \lim_{n\to \infty} f_n'(z)
  \]
  for $ z\in U $.
\end{corollary}
\pf We can say $ \textit{wlog} $ $ U=B(z,r) $ is a disc. If $ \gamma $ is a closed curve in this disc, then
\[
  \int_\gamma f_n = 0
\]
by convex Cauchy and
\[
  \int_\gamma f_n \to \int_\gamma f
\]
by uniform convergence in IB Analysis. So
\[
  \int_\gamma f = 0
\]
hence since $ f $ is continuous (it's the uniform limit of continuous functions) by Morera's theorem it is holomorphic. Then for $ 0<\rho <r $, we have that
\begin{align*}
	f'(z) = \frac 1{2\pi i} \int_{\partial B(z,\rho)} \frac{f(w)}{(w-z)^2}  \ \mathrm dw
\end{align*}
by Taylor's theorem. And for each $ f_n'(z) $ we have similar. Hence
\begin{align*}
	|f'(z) - f_n'(z)| &= \frac 1{2\pi} \left|\int_{\partial B(z,\rho)} \frac{f(w) - f_n(w)}{(w-z)^2}\ \mathrm dw\right|\\
			  &\le \frac 1{2\pi} 2\pi \rho \frac 1{\rho^2}\sup_{|w-z|= \rho} |f_1(w)-f_n(w)|
\end{align*}
and $ \rho $ is fixed so as $ n\to \infty $ we get that the difference tends to zero by uniform convergence. \qed
\begin{corollary}
  If $ U $ is a domain, $ f:U\to \C $ is continuous and for some finite set $ S\subseteq U $, $ f $ is holomorphic on $ U\setminus S $, then $ f $ is holomorphic on $ U $.
\end{corollary}
\pf If $ a\in S $ then there exists an $ \varepsilon>0 $ such that $ B(a,\varepsilon)\subseteq U $ and $ B(a,\varepsilon)\cap S = \{a\} $. Cauchy on a disc gives that
\[
  \int_\gamma f = 0
\]
for all closed piecewise-$ C^1 $ paths inside $ B(a,\varepsilon) $ and $ f $ is continuous so by Morera's theorem, $ f $ is holomorphic on $ B(a,\varepsilon) $.
Let's recap some properties of holomorphic functions:
\begin{enumerate}
	\item We have the fundemental theorem of calculus,
		\[
		  \int_\gamma F' = F(\gamma(b)) - F(\gamma(a)).
		\]
	\item Cauchy's theorem for triangles,
		\[
			\int_{\partial T} f = 0
		\]
		if $ T\subseteq U $ is a triangle.
	\item Star-shaped Cauchy, 
		\[
		  \int_\gamma f = 0
		\]
		for all closed $ \gamma $ in a starlike domain.
	\item Taylor's theorem gives that $ f $ is differentiable and has a local power series expansion.
\end{enumerate}
And some criteria for holomorphicity:
\begin{enumerate}
	\item We have the converse fundemental theorem of calculus. If
		\[
		  \int_\gamma f = 0
		\]
		for all closed $ \gamma $, then $ f=F' $ for $ F $ holomorphic.
	\item If $ U $ is star-shaped, then FTC converse holds if
		\[
			\int_{\partial T} f = 0
		\]
		for all $ T\subseteq U $.
	\item Morera's theorem, if
		\[
		  \int_\gamma f = 0
		\]
		for all closed $ \gamma $, then $ f $ is holomorphic.
\end{enumerate}
\subsection{Homotopy}
We should note that Cauchy's theorem holds on a much larger set of domains.
\begin{theorem}
	(Cauchy for simply connected domains) If $ U\subseteq \C $ is a simply connected domain and $ f:U\to \C $ is holomorphic then
	\[
	  \int_\gamma =0
	\]
	for all closed piecewise-$ C^1 $ curves $ \gamma $ in $ U $.
\end{theorem}
First let's see a definition.
\begin{definition}
	(Homotopic) Let $ \phi,\psi $ be closed paths in a domain $ U $, say $ \phi,\psi: [a,b] \to U $. We say that $ \phi,\psi $ are \textit{homotopic} if there exists a $ \Phi:[a,b]\times [0,1] \to U $ continuous such that $ \Phi\mid_{[a,b] \times \{0\}} = \phi $ and $ \Phi\mid_{[a,b]\times \{1\}} = \psi $ and for all $ t $ $ \Phi\mid_{[a,b]\times \{t\}} $ is a closed path.
\end{definition}
By definition $ U $ is simply connected is every loop is homotopic to a constant loop. We will assume the following claim.
\begin{claim}
	$ U $ is simply connected if whenever $ \phi $ is a piecewise-$ C^1 $ closed path, there exists a homotopy $ \Phi $ to the constant path such that for all $ t $, $ \Phi\mid_{[a,b]\times \{t\}} $ is a piecewise-$ C^1 $ closed path.
\end{claim}
We will assume this without proof.
\begin{proposition}
  If $ U $ is a domain, $ \psi,\phi $ are homotopic piecewise-$ C^1 $ closed paths and if $ f:U\to \C $ is holomorphic, then
  \[
    \int_\phi f = \int_\psi f.
  \]
\end{proposition}
\pf Let $ \phi,\psi:[a,b] \to U $ be close (piecewise-$ C^1 $) paths. We say $ \phi $ and $ \psi $ are \textit{elementary deformations} of one another if there exists $ a = x_0<x_1<\cdots< x_n = b $ and convex open sets $ C_i\subseteq U $ for $ 1\le i \le n $ such that $\underbrace{\phi\mid_{[x_{i-1},x_i]}}_{\phi_i} $ and $ \underbrace{\psi\mid_{[x_{i-1},x_i]}}_{\psi_i} $ have images inside $ C_i $. Let $ \gamma_i \in C_i$ be the straight path connecting $ \phi(x_i) $ to $ \psi(x_i) $ which is contained in $ C_i $ due to convexity.
\par
Let $ \Gamma_i = \phi_i + \gamma_i - \psi_i - \gamma_{i-1} $ which is a closed piecewise-$ C^1 $ loop inside $ C_i $. Apply convex Cauchy so
\[
	\int_{\Gamma_i} f = 0,
\]
hence
\begin{align*}
	\sum_{i=1]^n \int_{\Gamma_i} f = 0.
\end{align*}
The $ \gamma_i $ all appear twice with opposite orientation, so $ \int_\phi f + \int_{-\psi} f = 0 $ so
\[
  \int_\phi f = \int_\psi f
\]
if $ \psi $ and $ \psi $ are elementary deformations of each other.\par
Now it is enough to prove that if $ \phi\cong\psi $ then there exists $ \phi  = \phi_0,\phi_1,\dots, \phi_N = \psi $ such that $ \psi_{i+1} $ is an elementary deformation of $ \phi_i $. Let $ \Phi:[a,b]\times [0,1]\to U $ be the homotopy function from $ \phi = \Phi_0 $ to $ \psi = \Phi_1 $. The domain of $ \Phi $ is closed and bounded, hence $ \Phi([a,b]\times [0,1]) $ is closed and bounded. Hence there exists an $ \ep $ such that $ \mathrm{dist}(\Phi([a,b]\times[0,1]),\C\setminus U) = 2\ep>0 $. So for all $(s,t) \in [a,b]\times [0,1] $, $ B(\Phi(s,t),\ep) \subset U $. Now $ \Phi $ is uniformly continuous so $ \exists\delta >0 $ such that 
\begin{align*}
  |(s,t)-(s',t')|<\delta \implies |\Phi(s,t)-\Phi(s',t')|<\ep.
\end{align*}
Pick $ n $ such that $ \frac {1+(b-a)}n < \delta $ and set $ x_j = a+ \frac{(b-a)j}n $ for $ j=1,\dots, n $. Set $ \phi_o = \Phi\mid_{[a,b]\times i/n}  = \Phi_{i/n} $ so $ \phi = \phi_0, \phi_1,\dots, \phi_n =  \psi $ and let $ C_{ij}= B(\Phi(x_j,i/n),\ep) $ which is convex and contained in $ U $. Now if $ s\in [x_{j-1}, x_j] $ and $ t\in \left[\frac {i-1}n , \frac in\right] $ then $ \Phi(s,t)\in C_i $ then this shows that $ \phi_{i+1} $ is an elementary deformation of $ \phi_i $.\qed
\subsection{Zeros}
Suppose that $ f:B(a,r)\to \C $ is holomorphic. By Taylor's theorem we know that
\[
	f(z) = \sum_{n=0}^\infty c_n(z-a)^n
\]
and if $ f\ne 0 $ there is a least $ m $ such that $ c_m \ne 0 $. if $ m>0 $ then $ f(a) = 0 $ we can write that $ f(z) = (z-a)^mg(z) $ where $ g $ is holomorphic on $ B(a,r) $ such that $ g(a) \ne 0 $. We say that $ f $ has a zero of order $ m $ at $ a $.
\begin{proposition}
	(Principle of isolated zeros) If $ f:B(a,r)\to \C $ is holomorphic and $ f\not\equiv =0 $ then there exists $ 0<\rho <r $ such that
	\[
		f(z) \ne 0 \quad\forall z\in B(a,\rho)\setminus \{a\} = B(a,\rho)^*.
	\]
\end{proposition}
\pf If $ f(a) \ne 0 $ the result is immediate by continuity of $ f $. If $ f(a) = 0 $ we said that $ f(z) = (z-a)^m g(z) $ where $ g(a)\ne 0 $. By continuity of $ g $, there exists a $ \rho $ such that $ g $ doesn't vanish on $ B(a,\rho) $. Hence $ f $ doesn't vanish on $ B(a,\rho)^* $.\qed
\par
Recall an Analysis definition
\begin{definition}
	(Accumulation point) We say that a set $ X\subseteq \C $ has an accumulation point at $ a\in \C $ if for all $ \ep>0 $ we have that $ B(a,\ep)^* \cap X \ne \emptyset $.
\end{definition}
\begin{theorem}
	(Identity theorem) Let $ U\subseteq \C $ be a domain and $ f,g:U\to\C $ be holomorphic functions on $ U $. Let $ S = \{z\in U: f(z) = g(z)\} $. If $ S $ has an accumulation point in $ U $ then $ f(z) =g (z) $ for all $ z\in U $.
\end{theorem}
\pf Let $ h(z) = f(z) -g(z) $. Then $ h $ is holomorphic on $ U $ and vanishes on $ S $. Suppose that $ w $ is an accumulation point of $ S $. So it's a non-isolated zero of $ h $. Hence there exists a $ \ep >0 $ such that $ B(w,\ep)\subseteq U $ and $ h\equiv 0 $ on the ball by the principle of isolated zeros. Let $ z\in U $, so there is a path $ \gamma:[0,1] \to U $ such that $ \gamma(0) = w $ and $ \gamma(1) =z $. Let \begin{align*}
	I &= \{t\in [0,1]\mid h^{(n)}(\gamma(t))= 0 \ \forall n\ge 0 \}\\ 
	  &= \{ t\in [0,1] \mid h \text{ vanishes on a small open neighbourhood of } \gamma(t)\}
\end{align*}
by Taylor's theorem. We know that $ I $ is non-empty since $ 0\in I $. But the first line shows that $ I $ is open and the second line shows that $ I $ is closed. Hence $ I $ is clopen and non-empty, so $ I = [0,1] $ and $ h(z) = 0 $ so $ h\equiv 0 $ on $ U $.\qed
\begin{remark}
Let's see some remarks from this theorem.
  \begin{enumerate}
	  \item If $ f\not\equiv 0 $ where $ f $ is holomorphic on $ U $, then the zeros of $ f $ cannot accumulate in $ U $. However they can accumulate at a point not in $ U $. For example consider $ U = \C^* $ and $ f(z) =\sin\left(\frac 1z\right) $. Then $ f(z) = 0 $ for $ z= \frac 1{k\pi} $ for $ k\in \N $ so zeros do accumulate at $ 0 $.
	  \item Identity between entire functions holding on $ \R $ also hold on $ \C $, since we can rearrange them to zero and hence the function vanishes on the real axis so must vanish everywhere.
  \end{enumerate}
\end{remark}











\end{document}
