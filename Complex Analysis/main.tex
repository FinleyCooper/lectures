\documentclass{article}
\usepackage{../header}
\title{Complex Analysis}
\author{Notes by Finley Cooper}
\newcommand{\Arg}{\mathrm{Arg}}
\newcommand{\Log}{\mathrm{Log}}
\begin{document}
  \maketitle
  \newpage
  \tableofcontents
  \newpage
  \section{Complex Differentiation}
  \subsection{Definitions}
  The goal of this course is to develop the surprisingly rich theory of complex valued functions of one complex variable and the theory of integrating such functions along complex paths. The motivations for investigating such topics are complex polynomials of interest in geometry and number theory.\par
  We will also look at functions defined by power series such as the map from $ s\to \sum_{n\in N}\frac 1{n^s} = \zeta(s) $ will define a complex differentiable function for $ \Re(s)>1 $. There is also a connection to harmonic functions which is developed further in Analysis of Functions. We can also use complex methods to solve classical integrals or and differential equations which is developed more in IB Complex Methods.\par
  For this course we will use $ z\in \C $ and $ x=\Re(z), y=\Im(z) $. We will also use $ \theta $ for the argument of $ z $, $ \arg(z) $ which is well-defined up to adding $ 2\pi \Z $. We will use the principal argument $ \Arg(z)\in(-\pi,\pi] $.
  \begin{definition}
	  (Open disc) An \textit{open disc} or \textit{open ball} centred at $ a $ with radius $ r $ in $ \C $ is the set $ \{z\in \C\mid |z-a|<r\}  = B(a,r)= D(a,r)$.
  \end{definition}
  \begin{remark}
	  We will use $ \mathbb D = B(0,1) $ and $ \bar B(a,r) = \{ z\in \C\mid |z-a|\le r\} $.
  \end{remark}
  We use $ \C^* $ to denote $ \C\setminus \{0\} $.\par
  Recall that a set $ U\subseteq \C $ is open if it contains an open disc about each of its points.
  \begin{definition}
	  (Path) A \textit{path} in $ U\subseteq \C $ is a continuous map $ \gamma:[a,b]\to U $.
  \end{definition}
  \begin{definition}
	  (Path-connected) We say that $ U\subseteq \C $ is \textit{path-connected} if for all $ x,y\in U $ there exists a path $ \gamma:[0,1]\to U $ such that $ \gamma(0) = x $ and $ \gamma(1) = y $.
  \end{definition}
\begin{definition}
	(Domain) A \textit{domain} in $ \C $ is a non-empty path-connected subset of $ \C $.
\end{definition}
\begin{definition}
	(Closed path) If $ \gamma $ is a path and $ \gamma(a)=\gamma(b) $ then we say that $ \gamma $ is a \textit{closed path}.
\end{definition}
\begin{definition}
	($ C^1 $ path) We say a path is $ C^1 $ if it is continuously differentiable. We say a path is \textit{piecewise} $ C^1 $ if it has finitely many non-differentiable points but still globally continuous.
\end{definition}
\begin{definition}
	(Simple path) A path is \textit{simple} if it is injective except perhaps at the endpoints.
\end{definition}
\begin{definition}
  Let $ U\subseteq \C $ be open.
  \begin{enumerate}
	  \item We say that $ f:U\to \C $ is \textit{differentiable} at $ w\in U $ if
		  \[
			  f'(w)=\lim_{z\to w}\frac{f(z)-f(w)}{z-w}
		  \]
		  exists.
	  \item We say that $ f $ is \textit{holomorphic} at $ w\in U $ if $ \exists \varepsilon>0 $ such that $ f $ is differentiable on $ B(w,\varepsilon)\subseteq U $.
	  \item If $ f $ is holomorphic everywhere, we say $ f $ is \textit{entire}.
  \end{enumerate}
\end{definition}
\begin{remark}
  Some authors use analytic for holomorphic.
\end{remark}
\begin{remark}
  The usual rules for differenting sums and products and the inverse of a function (when it exists) apply exactly like we say in IA Analysis I with exactly the same proof.
\end{remark}
Any $ f:U\to \C $ can be written as $ f(z)=f(x+iy) = u(x,y)+iv(x,y) $ where $ u,v:U\to \R $ are the real and imaginary parts of $ f $.\par
Recall that $ u:U\to \R $ is differentiable at $ (c,d)\in U $ with derivative $ Du\mid_{(c,d)}= (\lambda,\mu) $ if and only if
\[
	\frac{u(x,y)-u(c,d) - (\lambda(x-c) + \mu(y-d))}{\sqrt{(x-c)^2+(y-d)^2}}\to 0
\]
as $ (x,y)\to (c,d) $.
\begin{proposition}
	(Cauchy-Riemann equations) Let $ f:U\to \C $ be defined on an open set $ U $ and write $ f=u+iv $, then $ f $ is differentiable t $ w= c+id\in U $ with $ f'(w)=p+iq $ if and only if $ u $ and $ w $ are both differentiable at $ (c,d) $ and $ u_x=v_y = p $ and $ -u_y = v_x=q $ at $ (c,d) $. Then $ f'(w) = u_x(c,d)+iv_x(c,d) $. 
\end{proposition}
\pf $ f $ is differentiable at $ w $ with derivative $ p+iq $ if and only if
\begin{align*}
	\lim_{z\to w}\frac{f(z)-f(w) -(z-w)(p+iq)}{z-w} = 0
\end{align*}
One can check that $ \lim_{z\to a}\frac{f(z)}{g)z)}=0 $ if and only if $ \lim_{z\to a}\frac{f(z)}{|g(z)|} = 0 $ and $ (p+iq)(z-w) = p(x-c)-q(y-d)+i(q(x-c)+p(y-d)) $ so using these and taking real and imaginary parts we get that
\begin{align*}
	\lim_{(x,y)\to (c,d)} \frac{u(x,y)-u(c,d)-(p(x-c)-q(y-d))}{\sqrt{(x-c)^2+(y-c)^2}} = 0
\end{align*}
for the real part. And
\begin{align*}
	\lim_{(x,y)\to (c,d)}\frac{v(x,y)-v(c,d) - (q(x-c)+p(y-d))}{\sqrt{(x-c)^2+(y-c)^2}}=0
\end{align*}
for the imaginary part. This is equivalent to saying that $ u $ is differentiable with $ Du\mid_{(c,d)} = (p,-q) $ and $ v $ is differentiable with $ Dv\mid_{(c,d)} = (q,p) $.\qed
\begin{remark}
  Let's make some remarks about the Cauchy-Riemann equations.
  \begin{enumerate}
	  \item If $ f=u +iv $ and $ u_x=v_u $ and $ u_y = -v_x $ at a point $ w $ we \textit{cannot} conclude that $ f $ is differentiable at $ w $ (Example Sheet 1).
	  \item If the partial derivatives $ u_x,u_y,v_x,v_y $ exist and are continuous in an open neighbourhood of $ w $ then the Cauchy-Riemann equations holding does imply complex differentiability.
  \end{enumerate}
\end{remark}
Let's see some examples.
\begin{enumerate}
	\item Polynomials are sums and products of the identity function, hence they are entire.
	\item If $ P $ and $ Q $ are polynomials, and $ U\subseteq \C\setminus \{x\mid Q(x)= 0\} $ then $ \frac PQ $ is differentiable on $ U $. These are called \textit{rational functions}.
	\item If $ f(x)=|x| $, this is not differentiable anywhere in $ \C $. $ f=u+iv $ with $ u=\sqrt{x^2 +y^2} $ and $ v=0 $. If $ (x,y)\ne (0,0) $ then
		\[
			u_x=\frac{x}{\sqrt{x^2+y^2}}, \qquad u_y = \frac{y}{\sqrt{x^2+y^2}}.
		\]
		So the Cauchy-Riemann equations do not hold, and if $ (x,y)= (0,0) $ we know that this isn't even differentiable in the real case, hence it's also not differentiable in the complex case. So $ f $ isn't differentiable anywhere.
\end{enumerate}
\begin{remark}
  Later we'll see that if $ f $ is holomorphic on $ U $, then $ f' $ is also holomorphic on $ U $. Then from the Cauchy-Riemann equations, we can see that $ f $ is harmonic. Conversely we can later see that every harmonic function on an open set in $ \R^2 $ is locally $ \Re(f) $ for some holomorphic function $ f $.
\end{remark}
\subsection{Conformal maps}
\begin{proposition}
  Let $ U\subseteq \C $ be a domain and suppose that $ f:U\to\C $ is holomorphic and $ f'(z)=0 $ on $ U $. Then $ f $ is constant.
\end{proposition}
\pf We will without proof the following elementary topological fact.
\begin{lemma}
	If $ U $ is a domain and $ \gamma:[0,1]\to U $ is a path with $ \gamma(0)=a $ and $ \gamma(1)=b $. Then there is another path $ \bar\gamma:[0,1]\to U $ with $ \bar\gamma(0)=a $ and $ \bar\gamma(1)=b $ where $ \bar\gamma $ is composed of finitely many segments, each parallel to the $ x $ or $ y $ axis, so $ \bar\gamma $ is piecewise-$ C^1 $.
\end{lemma}
Given this we know that $ u_x = v_y $ and $ u_y=-v_x $ hold. Since $ f' $ is zero, all these partials vanish on $ U $. Now the usual mean value theorem shows that $ u,v $ are constant along the segments of $ \bar\gamma $. So $ f(a)=f(b) $.\qed
\begin{definition}
	(Conformal) If $ f $ is holomorphc at a point $ w $ and $ f'(w)\ne 0 $ we say that $ f $ is \textit{conformal} at $ w $.
\end{definition}
This is a geometric property of $ f $.
\par
Suppose that $ U $ is a domain and $ f: U\to \C $ is conformal at $ w $. Let's take $ \gamma_i:(-\varepsilon,\varepsilon)\to U $ such that $ \gamma_i(0)=w, $ and $ \gamma_i'(0)\ne 0 $. So $ \gamma_i $ have non-zero tangent vectors through $ w $. The angle between these vectors is $ \arg(\gamma_1'(0)-\arg(\gamma_2'(0)) $. But then
\[
	\frac{(f\gamma_1)'(0)}{(f\gamma_2)'(0)} = \frac{f'(w)}{f'(w)}=\frac{\gamma_1'(0)}{\gamma_2'(0)}=1
\]
and so the angle between $ f\gamma_1 $ and $ f\gamma_2 $ at $ f(w) $ the same as for $ \gamma_1 $ and $ \gamma_2 $. So conformal means that $ f $ \textit{preserves} angles.
\begin{definition}
	(Conformal equivalence) If $ U,V $ are open in $ \C $ and $ f:U\to V $ is a holomorphic bijection which is everywhere conformal on $ U $ we say that $ f $ is a \textit{conformal equivalence} and $ U $ and $ V $ are conformally equivalent.
\end{definition}
\begin{remark}
  In this setting, if $ f $ is conformal, the inverse function theorem says that $ f $ is locally invertible and that local inverse is complex differentiable. If $ f $ is a bijection is it globally invertible, and hence the chain rule shows that $ \inv f $ is conformal.
\end{remark}
Let's see some examples.
\begin{enumerate}
	\item A linear map $ f(z) = az+b $ with $ a\ne 0 $ is a conformal equivalence $ \C\to \C $.
	\item $ f(z)=z^n  $ takes $ \{ z\mid 0\le \Arg(z) \le \frac \pi n\} \to \{ z\in \C \mid \Im(z) > 0 \} = \mathcal H$. This fails at the origin, but if we consider the open sector and open upper half plane.
	\item The exponential map, $ z\to \exp z =e^x e^{iy} $ sends verticle lines to circles with radius $ e^{\Re z} $. 
	\item $ z\in \mathcal H\iff z $ is closer to $ i $ than to $ -i$ $ \iff \left|\frac{z-i}{z+1} \right| < 1 $. The Mobius map $ z\to \frac{z-i}{z+i} $ takes $ \mathcal H \to \mathbb D$. Moreover $ f'(z) = \frac{2i}{(z+i)^2} $ is non-zero for $ z\in \mathcal H $, so $ f $ is an conformal equivalence.
\end{enumerate}
Recall that the Mobius group, $ \mathcal M $, is the group of mappings $ z\to \frac{az+b}{cz+d} $ with $ ad-bc\ne 0 $. $ A\in \mathcal M $ defines a conformal equivalence from $ \C\setminus \{-\frac dc\}\to \C\setminus \{\frac ac\} $, but it's much better to think of it as a conformal equivalence to the extended complex plane $ \C_\infty $. Recall that Mobius maps are triply transitive, so
\[
	z\to \frac{(z-z_1)(z_2-z_3)}{(z-z_3)(z_2-z_1)}
\]
sends the triple $ (z_1,z_2,z_3) $ to $ (0,1,\infty) $.
\par
Recall further that Mobius maps send circlines to circlines (where a circline is a circle or a line plus the point at infinity).
\par
Suppose we have a sector bounded between lines $ L_1, L_2 $. What are the images of a sector under a Mobius map? This depends on if our Mobius map $ T $ is such that $ \inv T(\infty) \in L_1 $ or $ \inv T(\infty) \in L_2 $. Any region formed by two non-pallel lines, a circle and a line, or two circles could be image under a Mobius map.
\par
The \textit{Joukowsky transform} is the map sending $ z\to \frac 12\left(z+\frac 1z\right) =\frac{z^2+ 1}{2z}$. We can compute that $ f'(z) = \frac 12-\frac 1{2z^2} $ and $ f $ is holomorphic except at $ 0 $ and is conformal except at $ 0 $ and $ \pm 1 $. Take the circle passing through $ -1 $ and $ -i $ not centred at the origin. Under the map $ f $, creates a kink at $ -1 $ since it's not conformal, but the curve is nice everywhere else since it's conformal on the rest of the circle. The transformed curve resembles an aerofoil. The incompressible fluid equations, $ \nabla\times v = 0, v=\nabla \phi $ implies that $ \nabla^2 \phi $ vanishes, so $ \phi $ is harmonic. So $ \phi $ is locally $ \Re(f) $, with $ f $ holomorphic. This is used to transfer questions about fluid flow about complicated shapes to the much simplier circle.
\par
Let $ S^1 =\partial \mathbb D = \{x\in \C: |x|=1\} $
\begin{definition}
	(Simply connceted) Let $ U\in \C $ be a domain. We say that $ U $ is \textit{simply connected} if every continuous map $ \gamma: S^1 \to U $ extends to a map  $ \hat\gamma:\overline{\mathbb{D}}\to U $ such that $ \hat\gamma\mid_{\partial \mathbb{D}} = \gamma $.
\end{definition}
\begin{remark}
  This is the notion that any loop can be continuously shrunk to a point.
\end{remark}
For example for the annulus, if we take $ \gamma $ to be the closed path going all the way around the annulus, we will later prove that no such $ \hat\gamma $ can exist.
\begin{theorem}
	(Riemann mapping theorem) If $ U\subset \C $ is a proper subdomain and $ U $ is simply connected, then $ U $ is conformally equivalent to the disc, $ \mathbb D = \{z:|z|<1\} $.
\end{theorem}
\begin{remark}
  The notation of being simply connected is invariant under conformal equivalences. Being simply connected is a topological property and is invariant under any homeomorphism (since conformal maps are homeomorphism).
\end{remark}
\begin{remark}
  The condition that $ U\ne \C $ is essential. If $ \C $ was conformally equivalent to the open disc then there would exist a function $ f: \C\to \mathbb D $ conformal, hence a homeomorphism, but from IB Topological Spaces, homeomorphisms are closed maps, hence since $ \C $ is closed, $ \mathbb D $ should be closed, but it isn't hence $ f $ cannot exist. We will see a non-topological proof later in the course.
\end{remark}
A useful construction of holomorphic functions (and we'll see later the universal construction) is convergent power series.
\subsection{Power series}
Let's see some reminders from IB Analysis II.
\begin{enumerate}
	\item A sequence $ f_n $ of functions on a set $ T $ converges uniformly to a function $ f $ on $ T $ if that $ \forall \varepsilon>0\ \exists N\ \st \ \forall \ n\ge N\in \N,\ |f_n(x)-f(x)|<\varepsilon $.
	\item The uniform limit of continuous functions is continuous.
	\item The Weiestress M-test says that if we have values $ M_n\in \R_{>0} $ such that $ |f_n(x)|\le M_n $ for all $ x\in T $, then if $ \sum_{n=1}^\infty M_n $ converges, then the series $ \sum_{n=1}^\infty f_n(x) $ converges absolutely and uniformly on $ T $.
	\item Suppose we have a sequence $ \{a_n\} $ then there is a unique $ R\in [0,\infty] $ such that the series $ \sum_{n=0}^\infty a_n (z-a)^n $ converges absolutely if $ |z-a|<R $ and diverges if $ |z-a|>R $. Moreover if $ 0<r<R $ then the series converges uniformly on $ \{z:|z-a|<r\} $. This is called the radius of convergence. The root test says that
		\[
			R = \frac1{\limsup_{n\to\infty} \sqrt[n]{|a_n|}}.
		\]
\end{enumerate}
\begin{theorem}
  Suppose that
  \[
	  f(z) = \sum_{n=0}^\infty c_n(z-a)^n
  \]
  is a complex power series with radius of convergence $ R>0 $. Then
  \begin{enumerate}
	  \item $ f $ is holomorphic on $ B(a,R) = \{z:|z-a|<R\} $;
	  \item $ f'(z) = \sum nc_n(z-a)^{n-1} $ which also has radius of convergence $ R $;
	  \item $ f $ is infinitely complex differentiable on $ B(a,R) $ with
		  \[
			  f^{(n)}(a)= n!c_n.
		  \]
  \end{enumerate}
\end{theorem}
\pf We can see (iii) follows from (ii) so enoughto prove (i) and (ii). Take $ a=0 $ \textit{wlog}. Since $ n|c_n|> |c_n| $ the radius of convergence of $ \sum_{n>= 1} nc_n(x_a)^{n-1} $ is at most $ R $. But if $ |z|< R_1 < R $ then we can see that
\[
	\frac{|nc_n z^{n-1}|}{|c_n R_1^{n-1}|} = n\left|\frac z{R_1}\right|^{n-1}\to 0
\]
which is true for all $ R_1<R $, hence the radius of convergence must be $ R $.\par
Now to show $ f $ is differentiable with that derivative. Pick $ z,w $ with $ |z|,|w|\le R_1< R $. Define
\[
	\varphi(z,w) = \sum_{n=1}^\infty c_n\sum_{j=0}^{n-1} z^j w^{n-1-j}.
\]
We can see that
\[
	\left|c_n\sum_{j=0}^{n-1} z^j w^{n-1-j}\right|\le n|c_n|R_1^n
\]
so this convergences uniformly with $ |z|\le R_1 $ and $ |w|<R_1 $ so the limit is continuous. If $ z\ne w $ we can sum as a geometric series, so
\[
	\varphi(z,w) = \sum_{n=1}^\infty c_n\left(\frac{z^n - w^n}{z-w}\right) = \frac{f(z)-f(w)}{z-w}.
\]
Taking the limit as $ w\to z $ we get that
\begin{align*}
\lim_{w\to z} \frac{f(z)-f(w)}{z-w} &= \varphi(z,z)\\
				    &= \sum_{n=1}^\infty c_n n z^{n-1} = f'(z).
\end{align*}
Hence we're done. \qed
\begin{corollary}
  Suppose we have a power series
  \[
	  f(z) = \sum_{n=-0}^\infty c_n(z-a)^n
  \]
  with $ R>0 $ and suppose that $ f $ vanishes on $ B(a,\varepsilon) $ with $ \varepsilon\in (0,R) $. Then $ f $ vanishes identically.
\end{corollary}
\pf All derivatives are zero at $ x=a $, hence $ f(z)=0 $\qed.
\par
Now we can define some familiar functions
\subsection{Exponentials, logarithms, and branch cuts}
\begin{definition}
	(Exponential function) We define the function
	\[
		e^z = \exp(z) = \sum_{n=0}^\infty \frac{z^n}{n!},
	\]
	which has the properties
	\begin{enumerate}
		\item Radius of convergence is $ \infty $;
		\item $ \frac d{dz} e^z = e^z $;
		\item $ e^0 =1 $. If we fix $ w $ and let $ F(z) = e^{z+w}e^{-z} $ then $ F'(z) $ vanishes, hence $ F $ is constant, and $ F(0) = e^w $, so $ e^{z+w} = e^ze^w $ holds for all $ z,w\in \C $.
		\item $ e^z $ never vanishes on $ \C $.
	\end{enumerate}
\end{definition}
\begin{remark}
  We define the trigonometic functions again as we did in $ \R $. The exponential function, $ \sin $ and $ \cos $ are entire.
\end{remark}
This function is invertible over $ \R $, but in $ \C $ it is not invertible. In fact for every $ z\in \C $ there exists infinitely many $ w\in \C $ distinct such that $ e^w = z $. This can be seen since $ \exp $ is $ 2\pi i $ peroidic.
\begin{definition}
	(Branch of the logarithm) Let $ U\subseteq \C^* = \C \setminus \{0\} $ open. A continuous function $ \lambda: U\to \C $ is \textit{branch of the logarithm} on $ U $ if $ e^{\lambda(z)} = z $ for all $ z\in U $.
\end{definition}
The classical example is $ U = \C \setminus \R_{\le 0} $, the slit plane. Let $ \Log: U \to \C $ be the defined by $ \Log(z) = \log(z) + i\theta $ where $ \theta= \Arg(z) $. This is called the principal branch of the logarithm.
\begin{proposition}
	$ \Log $ is holomorphic on $ \C\setminus \R_{\le 0} $ with
	\begin{enumerate}
		\item  $ \frac{d}{dz}\Log(z) = \frac 1z $;
		\item If $ |z|<1 $ then
			\[
				\Log(1+z) = \sum_{n=1}^\infty (-1)^{n-1}\frac {z^n}n.
			\]
	\end{enumerate}
\end{proposition}
\pf $ \Log $ is continuous on $ U $ and $ \exp $ is continuous on $ \C $. If $ z= e^x $ and $ w=e^y $, then
\begin{align*}
	\frac{\Log(z)-\Log(w)}{z-w} &= \frac{x-y}{e^x-e^y}
\end{align*}
which as $ x\to y $ converges to $ \frac 1{e^y} = \frac 1w $.\par
By the ratio test, $ \Log(1+z) $ has a radius of convergence $ 1 $. We know that $ \Log(1+z) $ has derivative
\[
	1-z+z^2-\cdot = \frac 1{1+z}
\]
so since the claimed power series and $ \Log(1+z) $ have the same derivative, they differ by a constant, and since they are both zero at $ z=0 $, that constant is zero, so $ \Log(1+z) $ has the power series as claimed.\qed
\begin{remark}
  \
  \begin{enumerate}
	  \item Later we'll see that there is no branch of the logarithm defined on $ \C^* $.
	  \item You can define a continuous branch of $ \log $ on any simply connected domain not containing zero.
  \end{enumerate}
\end{remark}
\begin{definition}
	(Multivalued power function) For a value $ \alpha\in \C $, the multivalued function $ z^\alpha $ is by definition $ \exp(\alpha\log(z)). $.
\end{definition}
\begin{remark}
	If for $ U\subseteq \C^* $ we have a branch of $ \log $ specified on $ U $, then $ z^\alpha $ becomes single-valued on $ U $.
\end{remark}
\begin{remark}
	If $ \alpha\in \Z $, then this is single-valued and if $ \alpha\in \Q $ then it is finitely many valued, for example $ z^{1/2} $ has two values. Note we cannot define a square root function globally on $ \C $.
\end{remark}
The function $ f(z) = z(z-1) $ admits a single valued square root on each of the domains $ \C\setminus [0,1] $ and $ \C\setminus (\R_{\ge 0} \cup \R_{\ge 1}) $. Hence
\begin{align*}
	(z(z_1))^{\frac 12} = \exp(\frac 12 (\log(z) - \log(z-1))).
\end{align*}
So taking a path around the removed interval in the complex plane is fine since both $\log$ terms jump, giving a total jump $ 2\pi i$ which doesn't change the value since $ \exp $ is $ 2\pi i $ periodic.
\begin{definition}
	(Branch point) A point $ p\in \C $ is a \textit{branch point} of a multivalued function $ \phi $ if there is no continuous single-value definition of $ \phi $ in $ B(0,\varepsilon) $ for any $ \varepsilon > 0 $.
\end{definition}
For example $ 0 $ is a branch of $ \log $. $ \{0,1\} $ are branch points of $ (z(z-1))^{1/2} $.
\section{Contour Integration}
\subsection{Basic properties}
\begin{definition}
	(Riemann integrablility in $ \C $) A function $ f:[a,b]\to\C $ is \textit{Riemann integrable} if it's real and imaginary parts are both Riemann integrable and
	\[
		\int_a^b f(t)\mathrm dt = \int_a^b \Re(f(t)) \mathrm dt + i\int_a^b \Im(f(t)) \mathrm dt.
	\]
\end{definition}
Note that we have
\[
	\left|\int_a^b f(t)\mathrm dt\right| \le \int_a^b |f(t)|\mathrm dt.
\]
From now on $ f $ will always be continuous and hence Riemann integrable.
\begin{proposition}
	For $ f:[a,b]\to \C $ we have that
	\[
		\left|\int_a^b f(t)\mathrm dt\right| \le \sup_{t\in[a,b]}|f(t)|(b-a)
	\]
	with equality if and only if $ f $ is constant.
\end{proposition}
\pf Let $ \theta = \Arg\int_a^b f(t)\mathrm dt $ and $ M =\sup_t |f(t)| $. Hence
\begin{align*}
	\left|\int_a^b f(t)\mathrm dt\right| &= e^{-i\theta} \int_a^b f(t)\mathrm dt\\
					     &= \int_a^b \Re(e^{-i\theta} f(t))\mathrm dt\\
					     &\le \int_a^b |e^{-i\theta} f(t)|\mathrm dt \\
					     &\le M(b-a)
\end{align*}
and equality is equivalent to $ \Arg(\theta) $ being constant and $ |f| = M $ everywhere, hence equality occurs if and only if $ f $ is constant.\qed
\par
\begin{definition}
	(Contour) A \textit{contour} is a simple, closed path.
\end{definition}
\begin{definition}
	(Contour integral) If $ U $ is a domain, $ f:U\to \C $ is continuous and $ \gamma:[a,b]\to U $ is a $ C^1 $-smooth curve then
	\[
		\int_\gamma f(z) \mathrm dz = \int_a^b f(\gamma(t))\gamma'(t)\mathrm dt.
	\]
\end{definition}
This definition extends in the obvious way to paths. We have some properties about contour integrals.
\begin{enumerate}
	\item They are linear;
	\item They are additive along paths;
	\item They are independent of parameterisation.
\end{enumerate}
\begin{remark}
  If we set the length of $ \gamma $ as
  \[
    |\gamma| = \int_a^b |\gamma'(t)|\mathrm dt
  \]
  this is \textit{not} independent of parameterisation.
\end{remark}
If $ (-\gamma)(t) = \gamma(-t) $ so $ -\gamma: [-b,-a]\to U $, $ \int_{-\gamma} f = -\int_\gamma f $.
\par
Let $ U=C^* $ and $ f(z) = z^n $, $ n\in \Z $. Let $ \phi:[0,2\pi]\to U $ sending $ t\to e^{it} $ then
\[
  \int_\phi f(z)\mathrm dz = \begin{cases}
     2\pi i & n = -1 \\
     0 & \text{otherwise}
  \end{cases}
\]
Let's prove this.
\begin{align*}
	\int_\phi f(z)\mathrm dz &= \int_0^{2\pi} e^{nit}(ie^{it})\mathrm dt \\
				 &= i\int_0^{2\pi} e^{i(n+1)t} \mathrm dt\\
				 &= \begin{cases}
					 2\pi i & n+1 = 0\\
					 0 & \text{otherwise}
				 \end{cases}.
\end{align*}
\begin{theorem}
	(Fundamental Theorem of Calculus) Suppose that $ U $ is a domain $ F:U\to \C $ is holomorphic and $ F'(z) $ is continuous on $ U $. Then for a path $ \gamma:[a,b]\to U $
	\[
	  \int_\gamma F'(z)\mathrm dz = F(\gamma(b)) - F(\gamma(a)).
	\]
	In particular if $ \gamma $ is closed then we get zero.
\end{theorem}
\pf 
\begin{align*}
	\int_\gamma F'(z)\mathrm dz &= \int_a^b F'(\gamma(t))\gamma'(t)\mathrm dt \\
				    &= \int_a^b (F\circ \gamma)'(t)\mathrm dt\\
				    &= (F\circ \gamma)(b) - (F\circ\gamma)(a)\qed
\end{align*}
If $ n\ne -1 $ then $ z^n =\frac {d}{dt}\left(\frac{z^{n+1}}{n+1}\right) $ so the integral of $ z^n $ around a closed path around the origin, is zero.
\begin{definition}
	(Antiderivative) If $ U $ is a domain and $ f:U\to \C $ is continuous and $ F:U\to \C $ is holomorphic with $ F'(z)=f(z) $ then we say that $ F $ is an \textit{antiderivative} for $ f $ on $ U $.
\end{definition}
So $ f(z) = \frac 1z $ has no antiderivative on $ \C^* $.
\begin{remark}
  Later we'll show (without being circular) that the hypothesis saying $ F'(z) $ is continuous actuallyh always holds (which we already know if our holomorphic function came from a power series).
\end{remark}

















\end{document}
