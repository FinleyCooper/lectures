\documentclass{article}
\usepackage{../header}
\title{Complex Analysis}
\author{Notes by Finley Cooper}
\newcommand{\Arg}{\mathrm{Arg}}
\begin{document}
  \maketitle
  \newpage
  \tableofcontents
  \newpage
  \section{Complex Differentiation}
  The goal of this course is to develop the surprisingly rich theory of complex valued functions of one complex variable and the theory of integrating such functions along complex paths. The motivations for investigating such topics are complex polynomials of interest in geometry and number theory.\par
  We will also look at functions defined by power series such as the map from $ s\to \sum_{n\in N}\frac 1{n^s} = \zeta(s) $ will define a complex differentiable function for $ \Re(s)>1 $. There is also a connection to harmonic functions which is developed further in Analysis of Functions. We can also use complex methods to solve classical integrals or and differential equations which is developed more in IB Complex Methods.\par
  For this course we will use $ z\in \C $ and $ x=\Re(z), y=\Im(z) $. We will also use $ \theta $ for the argument of $ z $, $ \arg(z) $ which is well-defined up to adding $ 2\pi \Z $. We will use the principal argument $ \Arg(z)\in(-\pi,\pi] $.
  \begin{definition}
	  (Open disc) An \textit{open disc} or \textit{open ball} centred at $ a $ with radius $ r $ in $ \C $ is the set $ \{z\in \C\mid |z-a|<r\}  = B(a,r)= D(a,r)$.
  \end{definition}
  \begin{remark}
	  We will use $ \mathbb D = B(0,1) $ and $ \bar B(a,r) = \{ z\in \C\mid |z-a|\le r\} $.
  \end{remark}
  We use $ \C^* $ to denote $ \C\setminus \{0\} $.\par
  Recall that a set $ U\subseteq \C $ is open if it contains an open disc about each of its points.
  \begin{definition}
	  (Path) A \textit{path} in $ U\subseteq \C $ is a continuous map $ \gamma:[a,b]\to U $.
  \end{definition}
  \begin{definition}
	  (Path-connected) We say that $ U\subseteq \C $ is \textit{path-connected} if for all $ x,y\in U $ there exists a path $ \gamma:[0,1]\to U $ such that $ \gamma(0) = x $ and $ \gamma(1) = y $.
  \end{definition}
\begin{definition}
	(Domain) A \textit{domain} in $ \C $ is a non-empty path-connected subset of $ \C $.
\end{definition}
\begin{definition}
	(Closed path) If $ \gamma $ is a path and $ \gamma(a)=\gamma(b) $ then we say that $ \gamma $ is a \textit{closed path}.
\end{definition}
\begin{definition}
	($ C^1 $ path) We say a path is $ C^1 $ if it is continuously differentiable. We say a path is \textit{piecewise} $ C^1 $ if it has finitely many non-differentiable points but still globally continuous.
\end{definition}
\begin{definition}
	(Simple path) A path is \textit{simple} if it is injective except perhaps at the endpoints.
\end{definition}
\begin{definition}
  Let $ U\subseteq \C $ be open.
  \begin{enumerate}
	  \item We say that $ f:U\to \C $ is \textit{differentiable} at $ w\in U $ if
		  \[
			  f'(w)=\lim_{z\to w}\frac{f(z)-f(w)}{z-w}
		  \]
		  exists.
	  \item We say that $ f $ is \textit{holomorphic} at $ w\in U $ if $ \exists \varepsilon>0 $ such that $ f $ is differentiable on $ B(w,\varepsilon)\subseteq U $.
	  \item If $ f $ is holomorphic everywhere, we say $ f $ is \textit{entire}.
  \end{enumerate}
\end{definition}
\begin{remark}
  Some authors use analytic for holomorphic.
\end{remark}
\begin{remark}
  The usual rules for differenting sums and products and the inverse of a function (when it exists) apply exactly like we say in IA Analysis I with exactly the same proof.
\end{remark}
Any $ f:U\to \C $ can be written as $ f(z)=f(x+iy) = u(x,y)+iv(x,y) $ where $ u,v:U\to \R $ are the real and imaginary parts of $ f $.\par
Recall that $ u:U\to \R $ is differentiable at $ (c,d)\in U $ with derivative $ Du\mid_{(c,d)}= (\lambda,\mu) $ if and only if
\[
	\frac{u(x,y)-u(c,d) - (\lambda(x-c) + \mu(y-d))}{\sqrt{(x-c)^2+(y-d)^2}}\to 0
\]
as $ (x,y)\to (c,d) $.
\begin{proposition}
	(Cauchy-Riemann equations) Let $ f:U\to \C $ be defined on an open set $ U $ and write $ f=u+iv $, then $ f $ is differentiable t $ w= c+id\in U $ with $ f'(w)=p+iq $ if and only if $ u $ and $ w $ are both differentiable at $ (c,d) $ and $ u_x=v_y = p $ and $ -u_y = v_x=q $ at $ (c,d) $. Then $ f'(w) = u_x(c,d)+iv_x(c,d) $. 
\end{proposition}
\pf $ f $ is differentiable at $ w $ with derivative $ p+iq $ if and only if
\begin{align*}
	\lim_{z\to w}\frac{f(z)-f(w) -(z-w)(p+iq)}{z-w} = 0
\end{align*}
One can check that $ \lim_{z\to a}\frac{f(z)}{g)z)}=0 $ if and only if $ \lim_{z\to a}\frac{f(z)}{|g(z)|} = 0 $ and $ (p+iq)(z-w) = p(x-c)-q(y-d)+i(q(x-c)+p(y-d)) $ so using these and taking real and imaginary parts we get that
\begin{align*}
	\lim_{(x,y)\to (c,d)} \frac{u(x,y)-u(c,d)-(p(x-c)-q(y-d))}{\sqrt{(x-c)^2+(y-c)^2}} = 0
\end{align*}
for the real part. And
\begin{align*}
	\lim_{(x,y)\to (c,d)}\frac{v(x,y)-v(c,d) - (q(x-c)+p(y-d))}{\sqrt{(x-c)^2+(y-c)^2}}=0
\end{align*}
for the imaginary part. This is equivalent to saying that $ u $ is differentiable with $ Du\mid_{(c,d)} = (p,-q) $ and $ v $ is differentiable with $ Dv\mid_{(c,d)} = (q,p) $.\qed
\begin{remark}
  Let's make some remarks about the Cauchy-Riemann equations.
  \begin{enumerate}
	  \item If $ f=u +iv $ and $ u_x=v_u $ and $ u_y = -v_x $ at a point $ w $ we \textit{cannot} conclude that $ f $ is differentiable at $ w $ (Example Sheet 1).
	  \item If the partial derivatives $ u_x,u_y,v_x,v_y $ exist and are continuous in an open neighbourhood of $ w $ then the Cauchy-Riemann equations holding does imply complex differentiability.
  \end{enumerate}
\end{remark}
Let's see some examples.
\begin{enumerate}
	\item Polynomials are sums and products of the identity function, hence they are entire.
	\item If $ P $ and $ Q $ are polynomials, and $ U\subseteq \C\setminus \{x\mid Q(x)= 0\} $ then $ \frac PQ $ is differentiable on $ U $. These are called \textit{rational functions}.
	\item If $ f(x)=|x| $, this is not differentiable anywhere in $ \C $. $ f=u+iv $ with $ u=\sqrt{x^2 +y^2} $ and $ v=0 $. If $ (x,y)\ne (0,0) $ then
		\[
			u_x=\frac{x}{\sqrt{x^2+y^2}}, \qquad u_y = \frac{y}{\sqrt{x^2+y^2}}.
		\]
		So the Cauchy-Riemann equations do not hold, and if $ (x,y)= (0,0) $ we know that this isn't even differentiable in the real case, hence it's also not differentiable in the complex case. So $ f $ isn't differentiable anywhere.
\end{enumerate}
\begin{remark}
  Later we'll see that if $ f $ is holomorphic on $ U $, then $ f' $ is also holomorphic on $ U $. Then from the Cauchy-Riemann equations, we can see that $ f $ is harmonic. Conversely we can later see that every harmonic function on an open set in $ \R^2 $ is locally $ \Re(f) $ for some holomorphic function $ f $.
\end{remark}
\subsection{Conformal maps}
\begin{proposition}
  Let $ U\subseteq \C $ be a domain and suppose that $ f:U\to\C $ is holomorphic and $ f'(z)=0 $ on $ U $. Then $ f $ is constant.
\end{proposition}
\pf We will without proof the following elementary topological fact.
\begin{lemma}
	If $ U $ is a domain and $ \gamma:[0,1]\to U $ is a path with $ \gamma(0)=a $ and $ \gamma(1)=b $. Then there is another path $ \bar\gamma:[0,1]\to U $ with $ \bar\gamma(0)=a $ and $ \bar\gamma(1)=b $ where $ \bar\gamma $ is composed of finitely many segments, each parallel to the $ x $ or $ y $ axis, so $ \bar\gamma $ is piecewise-$ C^1 $.
\end{lemma}
Given this we know that $ u_x = v_y $ and $ u_y=-v_x $ hold. Since $ f' $ is zero, all these partials vanish on $ U $. Now the usual mean value theorem shows that $ u,v $ are constant along the segments of $ \bar\gamma $. So $ f(a)=f(b) $.\qed
\begin{definition}
	(Conformal) If $ f $ is holomorphc at a point $ w $ and $ f'(w)\ne 0 $ we say that $ f $ is \textit{conformal} at $ w $.
\end{definition}
This is a geometric property of $ f $.
\par
Suppose that $ U $ is a domain and $ f: U\to \C $ is conformal at $ w $. Let's take $ \gamma_i:(-\varepsilon,\varepsilon)\to U $ such that $ \gamma_i(0)=w, $ and $ \gamma_i'(0)\ne 0 $. So $ \gamma_i $ have non-zero tangent vectors through $ w $. The angle between these vectors is $ \arg(\gamma_1'(0)-\arg(\gamma_2'(0)) $. But then
\[
	\frac{(f\gamma_1)'(0)}{(f\gamma_2)'(0)} = \frac{f'(w)}{f'(w)}=\frac{\gamma_1'(0)}{\gamma_2'(0)}=1
\]
and so the angle between $ f\gamma_1 $ and $ f\gamma_2 $ at $ f(w) $ the same as for $ \gamma_1 $ and $ \gamma_2 $. So conformal means that $ f $ \textit{preserves} angles.
\begin{definition}
	(Conformal equivalence) If $ U,V $ are open in $ \C $ and $ f:U\to V $ is a holomorphic bijection which is everywhere conformal on $ U $ we say that $ f $ is a \textit{conformal equivalence} and $ U $ and $ V $ are conformally equivalent.
\end{definition}
\begin{remark}
  In this setting, if $ f $ is conformal, the inverse function theorem says that $ f $ is locally invertible and that local inverse is complex differentiable. If $ f $ is a bijection is it globally invertible, and hence the chain rule shows that $ \inv f $ is conformal.
\end{remark}
Let's see some examples.
\begin{enumerate}
	\item A linear map $ f(z) = az+b $ with $ a\ne 0 $ is a conformal equivalence $ \C\to \C $.
	\item $ f(z)=z^n  $ takes $ \{ z\mid 0\le \Arg(z) \le \frac \pi n\} \to \{ z\in \C \mid \Im(z) > 0 \} = \mathcal H$. This fails at the origin, but if we consider the open sector and open upper half plane.
	\item The exponential map, $ z\to \exp z =e^x e^{iy} $ sends verticle lines to circles with radius $ e^{\Re z} $. 
	\item $ z\in \mathcal H\iff z $ is closer to $ i $ than to $ -i$ $ \iff \left|\frac{z-i}{z+1} \right| < 1 $. The Mobius map $ z\to \frac{z-i}{z+i} $ takes $ \mathcal H \to \mathbb D$. Moreover $ f'(z) = \frac{2i}{(z+i)^2} $ is non-zero for $ z\in \mathcal H $, so $ f $ is an conformal equivalence.
\end{enumerate}
Recall that the Mobius group, $ \mathcal M $, is the group of mappings $ z\to \frac{az+b}{cz+d} $ with $ ad-bc\ne 0 $. $ A\in \mathcal M $ defines a conformal equivalence from $ \C\setminus \{-\frac dc\}\to \C\setminus \{\frac ac\} $, but it's much better to think of it as a conformal equivalence to the extended complex plane $ \C_\infty $. Recall that Mobius maps are triply transitive, so
\[
	z\to \frac{(z-z_1)(z_2-z_3)}{(z-z_3)(z_2-z_1)}
\]
sends the triple $ (z_1,z_2,z_3) $ to $ (0,1,\infty) $.
\par
Recall further that Mobius maps send circlines to circlines (where a circline is a circle or a line plus the point at infinity).
\par
Suppose we have a sector bounded between lines $ L_1, L_2 $. What are the images of a sector under a Mobius map? This depends on if our Mobius map $ T $ is such that $ \inv T(\infty) \in L_1 $ or $ \inv T(\infty) \in L_2 $. Any region formed by two non-pallel lines, a circle and a line, or two circles could be image under a Mobius map.
\par
The \textit{Joukowsky transform} is the map sending $ z\to \frac 12\left(z+\frac 1z\right) =\frac{z^2+ 1}{2z}$. We can compute that $ f'(z) = \frac 12-\frac 1{2z^2} $ and $ f $ is holomorphic except at $ 0 $ and is conformal except at $ 0 $ and $ \pm 1 $. Take the circle passing through $ -1 $ and $ -i $ not centred at the origin. Under the map $ f $, creates a kink at $ -1 $ since it's not conformal, but the curve is nice everywhere else since it's conformal on the rest of the circle. The transformed curve resembles an aerofoil. The incompressible fluid equations, $ \nabla\times v = 0, v=\nabla \phi $ implies that $ \nabla^2 \phi $ vanishes, so $ \phi $ is harmonic. So $ \phi $ is locally $ \Re(f) $, with $ f $ holomorphic. This is used to transfer questions about fluid flow about complicated shapes to the much simplier circle.








\end{document}
