\documentclass{article}
\usepackage{../header}
\title{Quantum Mechanics}
\author{Notes made by Finley Cooper}
\begin{document}
  \maketitle
  \newpage
  \tableofcontents
  \newpage
\section{Historical Introduction}
\subsection{Classical mechanics}
Classical mechanics is based on two distinct concepts.
\begin{enumerate}
	\item Particles - Point-like objects which have positions and velocities fully determinated by Newton's second law as a function of time. $ mx''(t)=F(x(t),x'(t)) $. Once $ x(t_0), x'(t_0) $ is known, the solution of the equation gives the position and velocity at all times. Particles can collide, scatter, but never interfer.
	\item Waves - Spread out objects which are functions of time and position. Periodic in $ t/x $. Propagation is determined by the wave equation which is \[
			\frac{\partial^2f(x,t)}{\partial t^2}-c^2\frac{\partial^2f(x,t)}{\partial x^2}=\nu
	\]
	We have solution $ f_{\pm}(x,t)=A_{\pm}\exp[i(kx-\omega t)] $. And the solutions obey the dispersion relation $ \omega=ck $. We have that $ \omega $ is the angular  frequency related to $ \lambda $ (wavelength) by $ \nu=\frac \omega{2\pi} $ and $ \lambda = \frac {2\pi c}{\omega}=\frac c\nu $. Waves interfer, causing constructive interference when in-phase and destructive interference when out of phase.
\end{enumerate}
\smallskip
However we have some programs with classical mehcanics.
	\begin{enumerate}
		\item Light behaving like particles
			\begin{enumerate}
				\item Black body radiation \textit{(NE)}
				\item Photoelectric effect
				\item Compton scattering \textit{(NE)}
			\end{enumerate}

		\item Stability of atom
		\item Particles behaving like waves
			\begin{enumerate}
				\item De Broglie Principle
				\item Electron diffraction
			\end{enumerate}
	\end{enumerate}
\subsection{Photoelectric effect}
This is the experiement when light hits a metal surface and causes electrons to be emitted. They used monochromatic radiation with fixed wavelength and changed the intensity of light and the wavelength. The classical expectation were:
\begin{enumerate}
	\item Incident light carries $ E\propto I=|A|^2 $. They proposed that as the intensity increases they is enough energy to break the bond of the electron with the metal atoms, causing it to be released.
	\item The emission rate of the electrons should be constant over intensity past a certain point.
\end{enumerate}
However what they actually observered was surprising.
\begin{enumerate}
	\item Below given $ \omega $ there was no emission of electrons.
\item The velocity (i.e. KE) depended on $ \omega $ not on $ I $.
\item The emission rate increased with intensity.
\end{enumerate}
In 1905 Einstein used these observations to propose the following
\begin{enumerate}
	\item Light comes in small quanta (now called photons).
	\item Each photon carries a small packet of energy, $ E $.
		\[
			E=\hbar \omega,\quad \mathbf{p}=\hbar\mathbf{k}
		\]
		where
		\[
			\hbar = \frac h {2\pi}
		\]
		where $ h $ is the Planck constant.
	\item The interaction seen in the photoelectric effect was caused by each photon interacting with each photon in a \textit{one-to-one} interaction. So we have that
		\[
			\text{Kinetic energy of } e^- = \text{Kinetic energy of } \gamma - \text{binding energy of the metal}
		\]
		we write this equation as 
		\[
			E_{min}=0=\hbar\omega_{min}-\phi.
		\]
		So the kinetic energy of the electrons is directly proposition to $ \omega $, not $ I $ which we expected in classical mechanics. Instead increasing the intensity increases the emission rate as we have more photons interacting with each electron.
\end{enumerate}
\subsection{Atomic spectra}
In 1897 Thomson plum-pudding model purposed a uniform distribution of positive charge, with negatively charged electrons inside.\par
Later in 1908, Rutherford performed his gold foil experiement, showing Rutherford scattering (large angle scattering), hence proving that the atom was mostly empty space, and the positively charged part was concentrated at the centre of the atom. This couldn't work since the electrons would radiate energy in an orbit and the electrons would collapse on the nucleus. Also it didn't explain the spectra of light emitted by atoms in a set of discrete values of 
\[
	w_{mn}=2\pi c R_0\left(\frac 1 {n^2}-\frac 1{m^2}\right)\quad n,m\in \N.
\]
But in 1913 Bohr showed that the electron orbits are quantised so that the $ L $ (orbital angular model) takes only these values
\[
  L_n=n\hbar
\]
So it was proposed that $ L $ was quantised hence so is $ r, v, $ and $ E $.\\
\pf $ \mathbf L = m_e\mathbf v \times \mathbf r$ so we have that
\begin{align*}
	|\mathbf L| = L = M_e v r \implies v=\frac{L}{m_er}\implies v_n=n\frac{\hbar}{m_er}\\
	\mathbf F^{\text{Coulomb}}=-\frac{e^2}{4\pi\varepsilon_0}\frac1{r^2}\mathbf e_r
\end{align*}
and we know that
\[
	|\mathbf F^{\text{Coulomb}}|=m_ea_r
\]
so
\[
	\frac{e^2}{4\pi\varepsilon_0}\frac{1}{r^2}=m_e\frac{v^2}r\implies r_h=\frac{4\pi\varepsilon_0\hbar^2}{m_ee^2}n^2
\]
so finally
\[
	r_0=\frac{4\pi\varepsilon_0}{m_ee^2}\hbar^2
\]
similiarly the energy is quantised since
\[
	E=\frac 12 m_ev^2-\frac{e^2}{4\pi\varepsilon_0}\frac 1r\implies E_n=-\frac{e^2}{8\pi \varepsilon_0r_0}\frac{1}{n^2}.
\]
The lowest energy state is called the Ground level, $ E_1 $ and the exicted levels $ E_2,E_3,\dots $ get closer to eachother as energy increases.\par
We have that
\[
	\omega_{mn}=\frac{\Delta E_{mn}}{\hbar}=2\pi c\left(\frac{e^2}{4\pi\varepsilon_0\hbar c}\right)^2\left(\frac{1}{n^2}-\frac1{m^2}\right)
\]
where $ \omega_{mn} $ is the angular frequency of light associated with an electron moving from energy state $ m $ to level $ n $. This the Bohr prediction from Rydberg constatant $ R_0 $. 
\subsection{The wave-like behaviour of particles}
In 1923 De Broglie hypothesised that each particle of any mass is associated with a wave having angular frequency, $ \omega $ given by
\[ \omega=\frac E\hbar \].
In 1927 this was observed by showing electron scattering off crystals which observed an interference pattern consistant with De Broglie's hypothesis.
\section{Foundation of Quantum Mechanics}
In quantum mechanics instead of a vector, we have a state represented by the letter $ \psi $. The basis is a continuous basis $ \{\mathbf x\}\to \psi(\mathbf x, t) $. Instead of a vector space $ \C^n $ we have a space of square-integrable functions, $ L^2(\R^3) $. We define an inner product of two states, $ \psi,\phi\in L^2(\R^3) $ as
\[
	(\psi,\phi)=\int_{\R^3}\phi(\mathbf x,t)\psi(\mathbf x,t)\mathrm d^3t.
\]
An operator $ \hat O $ sends a state $ \psi $ to a state $ \phi $ as shown,
\[
  \phi=\hat O\psi.
\]
\subsection{Wave function and probabilistic interpretation}
In classical mechanics $ \mathbf x $ and $ \dot{\mathbf x} $ determine the dynamics of a particle in a deterministic way.\par
However in quantum mechanics the state $ \psi(\mathbf x,t) $ determinates the dynamics of particles in a probablilistic way.
\begin{definition}
	(State of a particle) We say that $ \psi $ is the \textit{state} of a particle, where $ \psi(\x,t) $ is the complex coefficient of $ \psi $ in the continuous basis of $ \x $ at a given time $ t $. i.e. $ \psi(\x,t) $ is $ \psi $ in the $ \x $ representation and is called a wave function.
	\[
	  \psi(\x,t):R^3\to \C
	\]
	that satisfies mathematical properties dictated by its physical interpretation.
\end{definition}
Born's rule or the probabilistic interpretation for a particle described by a state $ \psi $ to sit at $ \x $ at given time $ t $ is
\[
  \rho(\x,t)\propto|\psi(\x,t)|^2=\psi^*(\x,t)\psi(\x,t)
\]
where
\[
	\rho(\x,t)\mathrm dV=\text{ probability that the particle sits in the some small volume } \mathrm dV \text {cented at } \x.
\]
From this we get properties on $ \psi $ as follows:
\begin{enumerate}
	\item $ \int_{\R^3}|\psi(\x,t)|^2\mathrm d^3x=N\le\infty $ with $ N\in\R $ and $ N\ne 0 $.
	\item Because the total probablity has to be equal to $ 1 $ the wavefunction must be normalised to $ 1 $. So
		\[
			\overline\psi(\x,t)=\frac{1}{\sqrt N}\psi(\x,t)
		\]
		which integrates to $ 1 $ over $ \R^3 $, giving that $ \psi(\x,t)=|\overline\psi(\x,t)|^2 $.
\end{enumerate}
\begin{remark}
Often we drop the $ \overline\psi $ notation and just write $ \psi $, and normalise at the end.
\end{remark}
If $ \tilde{\psi}(\x,t) =e^{i\alpha}\psi(\x,t) $ with $ \alpha\in \R $ then we have that
\[
	|\tilde{\psi}(\x,t)|^2=|\psi(\x,t)|^2
\]
so $ \psi $ and $ \tilde\psi $ are equivalent states.
\end{document}
