\documentclass{article}
\usepackage{../header}
\title{Analysis II}
\author{Notes made by Finley Cooper}
\newcommand{\eps}{\varepsilon}
\begin{document}
  \maketitle
  \newpage
  \tableofcontents
  \newpage
  \section{Uniform Convergence}
For a subset $ E\subseteq \R $, have a sequence $ f_n:E\to \R $. What does it mean for the sequence $ (f_n) $ to converge? The most basic notion for any $ x \in E $ require that the sequence of real numbers $ f_n(x) $ to converge in $ \R $. If this holds we can defined a new function $ f: E\to \R $ by setting each value to the limit of the function.
\begin{definition}
	(Pointwise limit) We say that $ (f_n) $ converges \textit{pointwise} if for all $ x $ in its domain we have that
	\[
		f(x)=\lim_{n\to\infty}f_n(x)
	\]
	converges. We write that $ f_n\to f $ pointwise.
\end{definition}
Are properties such as continuity, differentiability integrability, preserved in the limit? We'll use an example to show that continuity is not preserved.\par
We can see this by taking a sequence of functions which converge to a step function by taking tighter and tighter curvers which get steeper and steeper. For example take,
\[
	f_n:[-1,1]\to \R,\quad f_n(x)=x^{\frac 1{2n+1}}.
\]
So in the limit we get that
\[
  f_n(x)\to f(x)=\begin{cases}
	  1 & 0< x \le 1 \\
	  0 & x = 0 \\
	  -1 & -1\le x < 0 
  \end{cases}
\]
which is not continious.\par
For an example where integability is not preserved, let $ q_1,q_2,q_3,\dots $ be an enumeration of $ \Q\cap [0,1] $ and define
\[
  f_n(x)=\begin{cases}
	  1 & x\in\{q_1,\dots, q_n\} \\
	  0 & \text{otherwise}
  \end{cases}
\]
so we get $ f_n(x) $ continious everywhere on $ [0,1] $ apart from a finite number of points, then $ f_n $ is integrable on $ [0,1] $ (IA Analysis I). But,
\[
	\lim_{n\to\infty}f_n(x)=\boldsymbol{1}_\Q(x)
\]
which we know is not integrable.\par
If $ f_n\to f $ pointwise, $ f_n $ integrable, $ f $ integrable, does it follow that $ \int f_n\to\int f $? (Spoiler: No)
For example take $ f_n $ to be a 'spike' with height $ n $ and width $ \frac 2n $, concretely,
\[
  f_n(x)=\begin{cases}
	  n^2x& 0\le x \le \frac 1n \\
	  n^2(\frac 2n - x) & \frac 1n \le x \le \frac 2n \\
	  0 & \text{otherwise}
  \end{cases}
\]
So the integral of $ f_n $ over $ [0,1] $ is $ 1 $, but we can see that $ f_n $ converges pointwise to zero. So $ \int_0 ^1 f_n\to 1 $ but $ \int_0 ^1f\to0 $.\par So we need a better (stronger) notion for the convergence of a sequence of functions.
We can't use something too strong, such as $ f_n \to f$ if $ f_n $ is eventually $ f $ for large enough $ n $. We've got to find something inbetween. This is uniform convergence.
\begin{definition}
	(Uniform convergence) Let $ f_n,f: E\to \R $, for $ n\in\N $. We say that $ (f_n) $ converges \textit{uniformly} on $ E $ if the following holds. For all $ \eps>0 $, $ \exists N=N(\eps) $ such that for every $ n\ge N $ and for every $ x\in E $ we have that $ |f_n(x)-f(x)|<\eps $.
\end{definition}
\begin{remark}
  This statement is equivalent to the following,
  \[
	  \forall\eps >0,\exists N=N(\eps), \text{ s.t. } \forall n\ge N, \sup_{x\in E}|f_n(x)-f(x)|<\eps.
  \]
\end{remark}
Comparing this to pointwise convergence, $ \forall x \in E $ and $ \forall \eps>0 $, $ \exists N=N(\eps,x) $ such that $ n\ge N\implies |f_n(x)-f(x)|<\eps $. So we can change our $ N $ value for each individual $ x $. However we can't in uniform convergence, which makes this is stronger statement.\par
Hence we see Uniform convergence $ \implies $ Pointwise convergence.
This gives a nice way to compute uniform limits. If a function doesn't converge pointwise then we know it doesn't converge uniformly. If we know a sequence of functions converges pointwise to some limit function, then this function must be the limit of the uniform limit, if it exists.
\begin{definition}
	(Uniformly Cauchy) Let $ f_n:E\to \R $ be a sequence of functions. We say that $ (f_n) $ is \textit{uniformly Cauchy} on $ E $ if
	\[
		\forall \eps >0, \exists N=N(\eps) \st n,m\ge N\implies \sup_{x\in E}|f_n(x)-f_m(x)|<\eps.
	\]
\end{definition}
\begin{theorem}
	(Cauchy criterion for uniform convergence) Let $ (f_n) $ be a sequence of functions with $ f_n:E\to \R $. The $ (f_n) $ converges uniformly on $ E $ if and only if $ (f_n) $ is uniformly Cauchy on $ E $.
\end{theorem}
\pf Suppose that $ (f_n) $ is a sequence converging uniformly in $ E $ to some function $ f $. Given some $ \eps>0 $, there is a $ N $ such that $ \sup_{x\in E}|f_n(x)-f(x)|<\eps $ for all $ n\ge N $. By the triangle inequality $ \forall x\in E $, picking $ n,m\ge N $,
\begin{align*}
	|f_n(x)-f_m(x) & |\le |f_n(x)-f(x)|+|f_m(x)-f(x)| \\
 & \le \sup_E|f_n-f|+\sup_E|f_m-f| \\
 & < \eps + \eps\\
 & < 2\eps
\end{align*}
hence $ (f_n) $ is uniformly Cauchy.\\
For the converse, suppose that $ (f_n) $ is a sequence uniformly Cauchy in $ E $. Then the sequence of real numbers $ (f_n(x)) $ is Cauchy so by IA Analysis I, this sequence has a limit, call it $ f(x) $. So $ (f_n) $ converges pointwise to $ f $. Now we check that $ f_n\to f $ uniformly on $ E $. Pick any $ \eps >0 $ and note that by the hypothesis that $ (f_n) $ is uniformly Cauchy, there exists a number $ N $ such that for all $ n,m\ge N $ we have $ |f_n(x)-f_m(x)|<\eps $. Fix $ n\ge N $ and let $ m\to\infty $ in this. So since $ f_m(x) $ converges to $ f(x) $ pointwise, we get that
\[
  |f_n(x)-f(x)|\le \eps
\]
hence $ (f_n) $ converges uniformly in $ E $.\qed\par
For an example consider $ f_n:\R\to\R $ defined by $ f_n(x)=\frac x n $. So $ f_n\to 0 $ pointwise on $ \R $. But $ |f_n-0| $ is unbounded so the suprenum doesn't exist so $ f_n $ does not converge uniformly on $ \R $. However if we restrict the domain of $ f_n $ to $ [-a,a] $ then we get uniform convergence.
\begin{theorem}
	(Continuity is preserved under uniform limits) Let $ f_n,f:[a, b]\to \R $. Suppose that $ (f_n) $ converges to $ f $ uniformly on $ [a,b] $. If $ x\in [a,b] $ is such that $ f_n $ is continuous at $ x $ for all $ n\in \N $, then $ f $ is continuous at $ x $.
\end{theorem}
\pf Let $ \eps>0 $ by uniform convergence of $ f_n\to f $ we have some $ N\in \N $ such that for all $ n\ge N $,
\[
	\sup_{y\in[a,b]}|f_n(y)-f(y)|<\eps
\].
By continuity of $ f_N $ at $ x $ we have $ \delta=\delta(N,x,\eps)>0 \st y\in[a,b], |x-y|<\delta\implies |f_N(y)-f_N(x)|<\eps $.\\
Then $ y\in[a,b], |x-y|<\delta $ we ] have
\begin{align*}
	|f(y)-f(x)| & \le |f(y)-f_N(y)|+|f_N(y)-f_N(x)|+|f_N(x)-f(x)|\\
         & < \eps + \eps +\eps\\
	 & < 3\eps
\end{align*}
Hence $ f $ is continuous at $ x $. \qed\par
It is instructive to see where this proof goes wrong if we only assume that $ (f_n) $ converges to $ f $ pointwise.
\begin{corollary}
	(Uniform limits of continuous functions are continuous) If $ f_n,f:[a,b]\to \R $, and $ f_n\to f $ uniformly on $ [a,b] $ and if $ f_n $ is continuous on $ [a,b] $ for every $ n $ then $ f $ is continuous on $ [a,b] $.
\end{corollary}
\pf Immediate from the previous theorem.\qed\par
From now on we will denote $ C([a,b]) = \{f:[a,b]\to \R:f\text{ is continuous on } [a,b]\}. $
\begin{theorem}
	Let $ (f_n) $ be a uniformly Cauchy sequence of functions in $ C([a,b]) $ the it converges to a function in $ C([a,b]) $.
\end{theorem}
\pf Trivial from our theorems earlier proved. \qed
\begin{theorem}
	(Uniform convergence implies convergence of integrals) For $ f_n,f:[a,b]\to \R $ be such that $ f_n,f $ are bounded and integrable on $ [a,b] $. If $ f_n\to f $ uniformly on $ [a,b] $ then
	\[
	  \int_a^bf_n(x)\mathrm dx\to \int_a^bf(x)\mathrm dx
	\]
\end{theorem}
\begin{remark}
  The assumption that $ f $ is integrable is redundant. We will see later that integrability of $ f_n $ implies that $ f $ is integrable if $ f_n\to f $ uniformly
\end{remark}
\pf
\begin{align*}
	\left|\int_a^bf_n(x)\mathrm dx-\int_a^bf(x)\mathrm dx|&=\left|\int_a^bf_n(x)-f(x)\mathrm dx \\
							      &\le \int_a^b |f_n(x)-f(x)|\mathrm dx \\
							      &\le \sup_{x\in [a,b]}|f_n(x)-f(x)|(b-a) \to 0
\end{align*}
by assumption.
\subsection{Differentation and uniform convergence}
This is more subtle if $ f_n\to f $ uniformly on some interval and if $ f_n $ are differentiable it does not follow that
\begin{enumerate}
	\item That $ f $ is differentiable.
	\item Even if $ f $ is differentiable that $ f_n'(x)\to f(x) $.
\end{enumerate}
We can view this in the example of $ f_n:[-1,1]\to \R $ with $ f_n(x)=|x|^{1+\frac 1n} $. Hence we have that 
\[
	\lim_{x\to 0}\frac{f_n(x)-f_n(0)}{x}=\lim_{x\to 0}\mathrm{sgn}(x^{\frac 1n})=0
\]
So $ f_n $ is differentialbe at $ 0 $ with $ f_n(0)=0 $ and clearly $ f_n $ is differentiable everywhere where $ x=0 $ too. We can check that $ f_n\to |x| $ uniformly. But $ |x| $ is not differentiable at $ x=0 $.\par
Now consider the example $ f_n:\R\to\R $ with
\[
	f_n(x)=\frac{\sin(nx)}{\sqrt n}.
\]
So $ f_n\to 0 $ uniformly on $ \R $. So we have a differentiable limit but $ f_n'(x)=\sqrt n \cos(nx) $ which is not convergent as $ n\to\infty $. So we don't have $ f_n'(x)\to f'(x) $ pointwise on $ \R $.
\begin{theorem}
	Let $ f_n:[a,b]\to \R $ be a sequence of differentiable functions (at the end points this means that the one-sided derivative exists). Suppose that:
	\smallskip\begin{enumerate}
		\item $ f_n'\to g $ uniformly for some function $ g:[a,b]\to \R $.
		\item For some $ c\in[a,b] $ the sequence $ (f_n(c)) $ converges.
	\end{enumerate}\smallskip
	Then $ (f_n) $ converges uniformly to some function $ f:[a,b]\to \R $ where $ f $ is differentiable everywhere on $ [a,b] $ and $ f'(x)=g(x) $ for all $ x\in[a,b] $.
\end{theorem}
This proves that
\[
	\left(\lim_{n\to \infty}f_n\right)'=\lim_{n\to\infty}f'_n
\]
i.e. we can exchange the derivative and limit in this case.
\begin{remark}
If we assume that $ f'_n $ are continuous, then the proof is more straightforward and can be based on the fundamental theorem of calculus. 
\end{remark}
\pf By the mean value theorem applied to the difference $ (f_n-f_m) $ we have that for any $ x\in[a,b] $
\begin{align*}
	f_n(x)-f_m(x)&=f_n(c)-f_m(c)+(x-c)(f_n-f_m)'(x_{n,m}) \\
	\implies |f_n(x)-f_m(x)|&\le |f_n(c)-f_m(c)|+(b-a)|f_n'(x_{n,m})-f_m'(x_{n,m})|\\
	\implies \sup|f_n-f_m|&<|f_n(c)-f_m(c)|+(b-a)\sup|f_n'-f_m'|\to 0
\end{align*}
as $ n\to\infty $. So $ (f_n) $ is uniformly Cauchy and hence there is an $ f : [a,b]\to \R \st f_n\to f$ uniformly.\par
For the next part fix some $ y\in [a,b] $. Define
\[
  h(x)=\begin{cases}
	  \frac{f(x)-f(y)}{x-y} & x\ne y \\
	  g(y) & x=y
  \end{cases}
\]
Now we only have to estabilish that $ h $ is continuous at $ y $ to show that $ f $ is differentiable at $ y $ with $ f'(y)=g(y) $. Let
\[
  h_n(x)=\begin{cases}
	  \frac{f_n(x)-f_n(y)}{x-y} & x\ne y\\
	  f_n'(y) & x = y 
  \end{cases}
\]
then since $ f_n $ is differentiable at $ y $ we see that $ h_n $ is continuous on $ [a,b] $. The pointwise limit of $ (h_n) $ is $ h $ almost by definition since $ f'_n\to g $ at $ x=y $. Since the uniform limit of sequence of continuous functions is continuous, we just need to show that $ (h_n) $ is uniformly Cauchy on $ [a,b] $ since the limit must be $ h $ since it converges pointwise to $ h $.
\begin{align*}
  h_n(x)-h_m(x)=\begin{cases}
	  \frac{(f_n-f_m)(x)-(f_n-f_m)(y)}{x-y} & x\ne y \\
	  (f_n'-f_m')(y) & x=y
  \end{cases}.
\end{align*}
By the mean value theorem,
\begin{align*}
	h_n(x)-h_m(x)&=\begin{cases}
	  (f_n-f_m)'(x_{n,m}) \text{ for some } x_{n,m} \text{ between } x \text{ and } y & x\ne y\\
	  (f_n-f_m)'(y) & x=y
  \end{cases}\\
		\sup_{[a,b]}|h_n-h_m|&\le \sup_{[a,b]}|f_n'-f_m'|\to 0
\end{align*}
as $ n,m\to \infty $. So $ (h_n) $ is uniformly Cauchy so we're done. \qed
\begin{remark}
  $ f_n' $ need not be continuous consider
  \[
    f(x)=\begin{cases}
	    x^2\sin \frac 1x & x\ne 0 \\
	    0 & x = 0
    \end{cases}
  \]
  the $ f $ is differentiable on $ [-1,1] $ with f'(x) not continuous at $ x=0 $ and we can take $ f_n(x)=f(x) $ for all $ n $ (or $ f_n(x)=f(x)+\frac xn $.
\end{remark}
We have a shorter proof of the above theorem, assuming that $ (f_n') $ are continuous in addition to the hypothesis. For any $ x\in [a,b] $ we can write
\begin{align*}
  f_n(x)=f_n(c)+\int_c^xf_n'(t)\mathrm dt
\end{align*}
by FTC. Then
\begin{align*}
	|f_n(x)-f_m(x)|&=\left|f_n(c)-f_m(c)+\int_c^x(f_n'(c)-f_m'(c))\mathrm dt\right|\\
		       &\le |f_n(c)-f_m(c)|+\sup_{t\in[a,b]}|f_n'(t)-f_m'(t)|(b-a)\to 0
\end{align*} as $ n,m\to\infty $. So $ (f_n) $ is uniformly Cauchy, hence converges uniformly.\par
Note that 
\[
	\int_c^xf_n'(t)\mathrm dt\to \int_c^xg(t)\mathrm dt
\]
by uniform convergence of $ f_n'\to g $ which implies $ g $ is continuous and hence also integrable. We can let $ n\to\infty $ the first equation for $ f_n(x) $ which gives that
\[
  f(x)=f(c)+\int_c^xg(x)\mathrm dt
\]
So we can take the derivative of both sides giving that $ f'(x)=g(x)=\lim f_n'(x) $.\qed
\begin{proposition}
  If $ f_n,g_n:E\to \R $ with $ f_n\to f $ uniformly on $ E $ and $ g_n\to g $ uniformly on $ E $ then $ f_n+g_n $ converges uniformly to $ f+g $ on $ E $, and if $ h:E\to \R $ is a bounded function then $ hf_n\to hf $ uniformly on $ E $ also.
\end{proposition}
\pf On the example sheet.
\section{Series of functions}
\begin{definition}
	(Convergence of a series of functions) Let $ g_n: E\to \R $ for $ n\in\N $ then write
  \[
	  f_n=\sum_{j=1}^ng_j
  \]
  defined pointwise. Then we say that that,
  \begin{enumerate}
	  \item The series of functions $ \sum_{n=1}^\infty g_n $ is convergent at a point $ x\in E $ if the sequence of partial sums $ (f_n(x)) $ converges.
	  \item The series of functions $ \sum_{n=1}^\infty g_n $ uniformly on $ E $ if the sequence $ (f_n) $ converges uniformly on $ E $.
	  \item $ \sum_{n=1}^\infty g_n $ converges absolutely at $ x\in E $ if the series $ \sum_{n=1}^\infty |g_n(x)| $ converges.
	  \item $ \sum_{n=1}^\infty g_n $ converges absolutely uniformly on $ E $ if $ \sum_{n=1}^\infty |g_n| $ converges uniformly on $ E $.
  \end{enumerate}
\end{definition}
We know from IA Analysis I that absolutely convergence $ \implies $ convergence for a sequences in $ \R $. From this we have that if $ \sum_{n=1}^\infty  g_n $ converges absolutely at a point $ x\in E $ then $ \sum_{n=1}^\infty g_n $ converges at $ x $. Similiar to this we have the following proposition relating absolute uniform convergence and uniform convergence.
\begin{proposition}
	(Absolute uniform convergence implies uniform convergence) If $ g_n:E\to \R $ and if $ \sum_{n=1}^\infty g_n $ converges absolutely uniformly on $ E $ then $ \sum_{n=1}^\infty g_n$ converges uniformly on $ E $.
\end{proposition}
\pf Let $ f_n=\sum_{i=1}^n g_i $ Then
\begin{align*}
	|f_n(x)-f_m(x)|&=\left|\sum_{i=m+1}^ng(i)\right|\\
		       &= \sum_{i=m+1}^n|g_i(x)|=h_n(x)-h_m(x),\quad\text{where } h_n(x)=\sum_{i=1}^n|g_i(x)|\\
	\sup_{x\in E}|f_n(x)-f_m(x)|&\le \sup_{x\in E}|h_n(x)-h_m(x)|\to 0
\end{align*}
as $ n,m\to \infty $ so $ (f_n) $ converges uniformly on $ E $.\qed
\begin{remark}
  Uniform convergence and absolute pointwise convergence aren't enough to conclude that the series convergence absolutely uniformly.
\end{remark}
\begin{theorem}
	(Weierstrass M-test) Let  $ g_n:E\to\R $ be a sequence of functions and suppose that $ \exists M_n $ such that
	\[
		\sup_{x\in E}|g_n(x)|\le M_n
	\]
	and that
	\[
		\sum_{n=1}^\infty M_n
	\]
	converges. Then
	\[
		\sum_{n=1}^\infty g_n
	\]
	converges absolutely uniformly on $ E $.
\end{theorem}
\pf Let
\[
	h_n(x)=\sum_{j=1}^n |g_n(x)|
\]
for $ n>m $,
\begin{align*}
	h_n(x)-h_m(x)=\sum_{j=m+1}^n|g_j(x)|\le \sum_{j=k+1}^nM_j=\sum_{j=1}^nM_j -\sum_{j=1}^m M_j\\
	\implies \sup_{x\in E}|h_n(x)-h_m(x)|\le \left|\sum_{j=1}^n M_j-\sum_{j=1}^m M_j\right|\quad\forall n,m
\end{align*}
by assumption the right hand side $ \to 0 $ since $ \sum_{j=1}^\infty M_j $ is convergent, hence $ (h_n) $ is uniformly Cauchy hence converges uniformly.
\subsection{Power series}
We'll now specialise to the case where $ g_n(x)=c_n(x-a)^n $ for $ a,c_n\in \R $. This gives a real power series.
\begin{theorem}
	(Radius of convergence) Let $ \sum_{n=0}^\infty c_n(x-a)^n $ be a real power series then there exists a $ R\in[0,\infty] $ called the \textit{radius of convergence} of the power series such that
	\begin{enumerate}
		\item If $ |x-a|<R $ then the power series converges absolutely.
		\item If $ |x-a|>R $ then the power series diverges.
		\item R is given by
			\[
				R=\frac{1}{\limsup_{n\rightarrow\infty}|c_n|^{\frac 1n}}
			\]
			where if the limit is zero, then $ R=\infty $.
		\item For any $ r\in (0,R) $ we have the power series converges uniformly on $ [a-r,a+r] $, in particular the function that the power series converges to is continuous on $ (a-R,a+R) $.
	\end{enumerate}
\end{theorem}
\pf The proof for (i), (ii), and (iii) are in IA Analysis I. We'll just prove (iv). Note first that the power series converges absolutely at $ x=a+r $ i.e. we have that
\[
	\sum_{n=0}^\infty |c_n|r^n
\]
is convergent. Since $ |c_n(x-a)^n|\le |c_n|r^n $ for any $ x\in[a-r,a+r] $ we can apply the Weierstrass $ M $-test with $ M_n=|c_n|r^n $ to conclude that the series
\[
	\sum_{n=0}^\infty c_n(x-a)^n\to f
\]
converges absolutely uniformly on $ [a-r,a+r] $. It follows that $ f $ is continuous. at any point in $ (a-R,a+R) $ by picking $ r $ small enough.
\begin{remark}
   (Boundary behaviour. Let
   \[
	   f(x)=\sum_{n=0}^\infty c_n(x-a)^n
   \]
   with power series boundary $ R $ with $ 0< R<\infty $. If the power series converges at one of the boundary points of the interval of convergence, say at $ x=a+R $ i.e. $ \sum_{n=0}^\infty c_nR^n $ is convergent then
   \[
	   \lim_{x\to a+R}f(x)=\sum_{n=0}^\infty c_nR^n
   \]
   so $ f $ extends to $ (a-R,a+R] $ as a continuous function.
\end{remark}
Moreover, under the same conditions that $ \sum_{n=0}^\infty c_nR^n $ converges we have that the series converges uniformly on $ [a-r,a+r] $ for any $ r\in (0,R) $. Same discussion applies at the endpoint $ a-R $.
\begin{theorem}
	(Differentation of power series) Let $ \sum_{n=0}^\infty c_n(x-a)^n $ be a power series with radius of convergent $ R>0 $. Let
	\[
		f(x)=\sum_{n=0}^\infty c_n(x-a)^n
	\]
	defined on $ (a-R,a+R) $. We have the following
	\begin{enumerate}
		\item The derived series \[
		  \sum_{n=1}^\infty nc_n(x-a)^{n-1}
	  \]
	  has radius of convergent $ R $.
  \item $ f $ is differentiable on $ (a-R,a+R) $ with
	  \[
		  f'(x)=\sum_{n=1}^\infty nc_n(x-a)^{n-1}\quad\forall x\in(a-R,a+R)
	  \]
	\end{enumerate}
\end{theorem}
\pf
Before we prove the theorem let's give a definition we've seen slightly before.
\begin{definition}
	If $ (a_n) $ is a sequence of reals let 
	\begin{align*}
		p_n&= \sup\{a_m: m\ge n\}\\
		q_n&=\inf\{a_m:m\ge n\}.
	\end{align*}
        Then we define
	\begin{align*}
		\limsup_{n\to\infty}a_n&=\lim_{n\to\infty}p_n\\
		\liminf_{n\to\infty}a_n&=\lim_{n\to\infty}q_n.
	\end{align*}
	which exists in $ \R\cup\{\infty\} $ since $ (q_n) $ and $ (p_n) $ are monotone.
\end{definition}
\begin{align*}
	\limsup_{n\to\infty}(n|c_n|)^{\frac 1n}=\limsup_{n\to\infty}|c_n|^{\frac 1n}
\end{align*}
since we have that $ \lim_{n\to\infty}n^{\frac 1n} = 1$. So we have $ (i) $.\par
Define $ f_n(x)=\sum_{j=0}^jc_j(x-a)^j $ is clearly differentiable on $ \R $ with $ f_n'(x)=\sum_{j=1}^njc_j(x-a)^{j-1} $. By (i) we have that $ f_n'(x) $ converges uniformly on $ [a-r,a+r] $ for all $ r<R $ and $ f_n(a)=c_0 \forall n $ so $ (f_n(a)) $ converges. So the limit is differentiable in $ [a-r,a+r] $, with \[ f'(x)=\lim_{n\to\infty}f_n'(x) =\lim_{n\to\infty}\sum_{j=1}^njc_j(x-a)^{j-1}\].\qed\par
If we have a power series $ \sum_{n=1}^\infty c_n(x-a)^n $ we say the power series converges \textit{locally uniformly} on the interval of convergence $ (a-R,a+R) $ i.e. for all $ 0<r<R $ the power series converges uniformly on $ [a-r,a+r] $.
\begin{remark}
	By repeatedly applying the above theorem we get that if $f(x)= \sum_{n=1}^\infty c_n(x-a)^n $ has radius of convergence $ R>0 $ then $ f $ is differentiable to any order $ k\in\N $ in $ (a-R,a+R) $ and the $ k $th derivative is given by
	\[
		f^{(k)}(x)=\sum_{n=k}^\infty n(n-1)\cdots(n-k+1)c_n(x-a)^{(n-k)}.
	\]
	Plugging in $ x=a $ we get that 
	\[
		c_n=\frac{f^{(k)}(a)}{k!}.
	\]
	This says that $ f $ is uniquely determined by its values in an arbitrarily small interval around the point $ x=a $ since that's all we need to capture it's derivatives and form its power series.
\end{remark}
\section{Uniform continuity and Riemann integrability}
\subsection{Uniform continuity}
\begin{definition}
	(Uniform continuity) Let $ E\subseteq \R $ and let $ f:E\to \R $. We say that $ f $ is \textit{Uniformly continuous} on $ E $ if $ \forall \eps>0 $ there exists a $ \delta>0 $ such that $ \forall x,y\in E $ we have that
	\[
	  |x-y|<\delta\implies |f(x)-f(y)|<\eps
	\]
\end{definition}
This differs from our usual definition of continuity. We require some $ \delta $ to work for \textit{any} $ x,y\in E $ given some $ \eps $, rather than picking a $ \delta $ for each $ \eps $ and $ x $ value. Clearly uniform continuity implies continuity but the converse is not true. For an example consider $ f(x)=\frac 1x $ on $ (0,1) $. Clearly continuous at each $ x $, but not uniformly continuous since it gets too steep around $ 0 $.\par
Not even boundedness and continuity is enough for uniform continuity, consider $ \sin(\frac 1x) $, take $ x_n=\frac 1{2n\pi} $ and $ y_n=\frac 1{2n+\frac 12)\pi} $ then $ |f(x)-f(y)|=1 $, so no $ \delta $ works, we can always choose an $ n $ large enough.
\begin{theorem}
	Let $ [a,b] $ be a closed, bounded interval and $ f:[a,b]\to \R $ a continuous function. Then $ f $ is uniformly continuous.
\end{theorem}
\pf Argue by contradiction. Suppose that $ f $ is not uniformly continuous, so there exists an $ \eps>0 $ such that for all $ \delta>0 $ there is a pair of points $ x,y\in [a,b] $ such that $ |y-x|<\delta $ but $ |f(x)-f(y)|\ge \eps $. Now let $ \delta_n=\frac 1n $, so we get a sequence of functions $ x_n $ and $ y_n $ satisfying the above for each $ \delta_n $. By Bolzano-Weiestrass, there exists a subsequence $ (x_{n_k}) $ that converges to a point $ x\in [a,b] $.
\begin{align*}
	|x-y_{n_k}|\le |x-x_{n_k}|+|x_{n_k}-y_{n_k}\le |x-y_{n_k}|+\frac 1{n_k}\to 0\text{ as } n\to\infty 
\end{align*}
By the continuity of $ f $ at $ x $ we get $ f(x_{n_k})\to f(x) $ and $ f(y_{n_k})\to f(x) $. But this is contradiction since $ f(x) $ and $ f(y_{n_k}) $ are always seperated by some distance $ \eps $.\qed\par
We can actually strengthen this theorem.
\begin{theorem}
	Let $ f:[a,b]\to \R $ where $ -\infty<a<b<\infty $ be any function. Suppose that there is a collection $ \mathcal C $ of open intevals $ I\subseteq \R $ such that if
	\[
		F=[a,b]\setminus\bigcup_{I\in\mathcal C}I
	\]
	then $ f $ is continuous at every point in $ F $ (i.e. the set of discontinuities is contained in the union). Then $ \forall \eps>0 \exists\delta>0\st $ $ x\in F,y\in [a,b] $, with $ |x-y|<\delta \implies |f(x)-f(y)|<\eps $.
\end{theorem}
\pf Same as above, using the fact that $ F $ is \textit{closed} so it contains all of its limit points.\par
Let's show some applications of uniform continuity.
\subsection{Riemann Integration}
We'll do a quick recap of Riemann integration. For full proofs, look at IA Analysis I. Let $ f: [a,b]\to \R $ be a bounded function. Say that $ m\le f(x)\le M $ for $ m,M\in \R $. Let $ P =\{a_0=a, a_1,a_2,\dots, a_n=b\} $ be a partition of the interval $ [a,b] $ with $ a_0<a_1<\cdots<a_n $. We will write $ P=\{a_0=a<a_1<\cdots<a_n=b \} $ as shorthand.\par
We write that $ I_j=[a_j,a_{j+1}] $ for $ 0\le j<n $. Define the upper sum of $ f $ with $ P $ as
\[
	U(P,f)=\sum_{j=0}^{n-1}(a_{j+1}-a_j)\sup_{I_j}f
\]
and the lower sum of $ f $ with $ P $ as
\[
	L(P,f)=\sum_{j=0}^{n-1}(a_{j+1}-a_j)\inf_{I_j}f.
\]
We can see immediately that $ m(b-a) \le L(P,f)\le U(P,f)\le M(b-a) $. When we refine the partition by adding finitely many new points the uppers sum decreases or stays the same, and the lower sum increases or stays the same. So now we can define the upper and lower Riemann integral as
\begin{align*}
	I^*(f)&=\inf_PU(P,f)\\
	I_*(f)&=\sup_PL(P,f).
\end{align*}
We say that $ f $ is Riemann integrable if $ I^*(f)=I_*(f) $. We denote
\[
  \int_a^bf(x)\mathrm dx
\]
as this common value.
\begin{theorem}
	(Riemann criterion for integrability) For $ f:[a,b]\to \R $ bounded, $ f $ is integrable if and only if for all $ \eps>0 $ there exists a partition $ P $ of $ [a,b] $ such that
	\[
	  U(P,f)-L(P,f)<\eps
	\]
\end{theorem}
\pf In IA Analysis I.
\begin{theorem}
	Let $ f:[a,b]\to[A,B] $ be integrable and $ g:[A,B]\to\R $ continuous. Then the composite function $ g\circ f:[a,b]\to\R $ is integrable.
\end{theorem}
We may ask does this hold is we switch the order? i.e. given the both conditions is $ f\circ g$ always be integrable?\par
\pf Since $ g $ is continuous in a bounded interval, it is uniformly continuous. Given any $ \eps >0 $ there is a $ \delta $ such that $ x,y\in [A,B] $ with $ |y-x|<\delta\implies|g(x)-g(y)|<\eps $. We also have by integrability that there exists a partition $ P $ such that $ U(P,f)-L(P,f)<\eps' $ for all $ \eps' >0 $.
\begin{align*}
	U(P,g\circ f)-L(P, g\circ f)=\sum (a_{j+1}-a_j)\left(\sup_{I_j}g\circ f-\inf_{I_j}g\circ f\right)
\end{align*}
Take $ J=\left\{j: \sup_{I_j}f -\inf_{I_j}f\le \delta \right\}$. For any $ j\in J $ for all $ x,y \in I_J $ we must have that
\[
	|f(x)-f(y)|\le \sup_{z_1,z_2\in I_j}(f(z_1)-f(z_2))=\sup_{I_j}-\inf_{I_j}\le \delta.
\]
Hence we get that
\[
  |g\circ f(x)-g\circ f(y)|<\eps
\]
so
\begin{align*}
	\sup_{I_j}\left(g\circ f(x)-g\circ f(y)\right)\le \eps\\
	\sup_{I_j}g\circ f - \inf_{I_j}g\circ f\le \eps
\end{align*}
which gives that
\begin{align*}
	U(P,g\circ f)-L(P,g\circ f)&=\sum_{j=0}^n(a_{j+1}-a_j)\left(\sup_{I_j}g\circ f-\inf_{I_j}g\circ f\right)\\
				   &=\sum_{j\in J}(a_{j+1}-a_j)\left(\sup_{I_j}g\circ f-\inf_{I_j}g\circ f\right)+\sum_{j\notin J}(a_{j+1}-a_j)\left(\sup_{I_j}g\circ f-\inf_{I_j}g\circ f\right),\\
				   &\le \eps(b-a)+2\sup_{[A,B]}|g|\sum_{j\notin J} (a_{j+1}-a_j)
\end{align*}
hence it suffices to make the sum over the $ j $s not in $ J $ small enough. We know that
\[
	\sum_{j\notin J}(a_{j+1}-a_j)<\frac{\eps'}\delta
\]
so if we pick $ \eps'=\eps \delta $ we get that 
\[
	U(P,g\circ f)-L(P,g\circ f)<\left((b-a)+2\sup_{[A,B]}|g|\right)\eps.\qed
\]
\begin{corollary}
  If $ f $ is continuous then it is integrable
\end{corollary}
\pf Apply the theorem with $ g=\mathrm{id} $ which is clearly integrable.\qed
\begin{theorem}
	(Uniform limits of integrable functions are integrable) Suppose we have $ f_n:[a,b]\to\R $ be a sequence of Riemann integrable functions and $ f_n\to f $ uniformly. Then $ f $ is bounded, Riemann integrable and
	\[
		\int_a^bf_n\to \int_a^bf
	\]
\end{theorem}
\pf 
\[
	\sup_{[a,b]}|f|\le \sup_{[a,b]}|f-f_n|+\sup_{[a,b]}|f_n|\le 1+\sup_{[a,b]}|f_n|
\]
for $ n $ sufficently large (setting $ \eps=1 $). Hence $ f $ is bounded.\par
Let $ P=\{a_0,\dots, a_m\} $ be a partition of $ [a,b] $. Given some $ \eps>0 $ and consider
\begin{align*}
	U(P,f)-L(P,f)&=\sum_{j=0}^{m-1}(a_{j+1}-a_j)\left(\sup_{I_j}f-\inf_{I_j}f\right)\\
	&=\sum_{j=0}^{m-1}(a_{j+1}-a_j)\left(\sup_{I_j}(f-f_n+f_n)-\inf_{I_j}(f-f_n+f_n)\right)\\
	&\le \sum_{j=0}^{m-1}(a_{j+1}-a_j)\left(\sup_{I_j}(f-f_n)+\sup_{I_j}(f_n)-\inf_{I_j}(f-f_n)-\inf_{I_j}(f_n)\right)\\
	&\le U(P,f_n)-L(P,f_n)+2(a-b)\sup_{[a,b]}|f-f_n|
\end{align*}
So for our $ \eps> 0$ choose some $ N $ such that $ 2(b-a)\sup_{[a,b]}|f-f_N|\le\frac \eps 2 $ by uniform convergence. Now also choose a partition $ P $ such that $ U(P,f_N)-L(P,f_N)<\frac \eps2 $ since $ f_N $ is Riemann integrable. Hence $ U(P,f)-L(P,f)<\frac \eps 2+ \frac\eps 2=\eps $ for any $ \eps>0 $ so $ f $ is integrable by the Riemann criterion. The last part have been proved previously in the course.\qed\par
\textit{Non-examinable} We'll now prove an equivalent condition for a function to be Riemann integrable. First we'll set up some frameworks. For a function $ f:[a,b]\to\R $ bounded, we use $ \mathcal D_f $ to denote its set of discontinuities. We know that there are functions with $ \mathcal D_f $ non-empty which are still Riemann integrable, such as Thomae's function which has $ \mathcal D_f=\Q $. We also know that all monotone functions are integrable. What condition on $ \mathcal D_f $ do we need for integrability?
\begin{definition}
	(Null set) A subset $ \mathcal R\subseteq \R $ is said to be a \textit{null set} (or a set of \textit{Lebesgue measure zero}) if $ \forall\eps>0 $ there exists an at most countable collection of open intevals $ I_j=(a_i,b_i) $ such that
	\[
		\mathcal D\subseteq \bigcup_{i=1}^nI_i
	\]
	and
\[
	\sum_{j=1}^\infty|I_j|\le \eps
\]
where $ |I_j|=b_j-a_j $.
\end{definition}
We have a few examples of null sets.
\begin{enumerate}
	\item The empty set and singleton sets are null.
	\item Any subset of small enough sets are null.
	\item Any countable union of null sets is null (namely $ \Q $ is a null set and any other countable set like the algebraic numbers).
	\item The (standard) Cantor set is a null set even though it's uncountable.
	\item However not every set is a null set, every (open or closed) interval is not a null set.
\end{enumerate}
Now for the big theorem completely characterising Riemann integrable functions.
\begin{theorem}
	(Lebesgue's theorem on the Riemann integral) Let $ f:[a,b]\to\R $ bounded. Then $ f $ is Riemann integrable if and only if $ \mathcal D_f $ is a null set.
\end{theorem}
\pf See Part II Probability and Measure.
\end{document}
