\documentclass{article}
\usepackage{../header}
\title{Electromagnetism}
\author{Notes by Finley Cooper}
\newcommand{\x}{\mathbf x}
\begin{document}
  \maketitle
  \newpage
  \tableofcontents
  \newpage
  \section{Introduction}
  \subsection{Charges and currents}
  \textit{Electric charge} is a physical property of elementary particles. It is:
  \begin{enumerate}
	  \item A signed quantity, it can either be positive, negative, or zero.
	  \item It is quantised to integer multiplies of the elementary charge.
	  \item It is a conserved quantity even if particles are created or destroyed.
  \end{enumerate}
  By convention the electron has charge $ -e $, the proton has charge $ +e $ and the neutron has no charge. On macroscopic scales, the number of particles is so large that charge can be considered to have a continuous electric charge density $ \rho(\mathbf x,t) $. The total charge in a volume $ V $ is then
  \[
    Q=\int_V \rho\mathrm dV.
  \]
  The \textit{electric current density} $ \mathbf J(\mathbf x,t) $ is the flux of electric charge per unit area. The current folowing through a surface $ S $ is
  \[
    I=\int_S\mathbf J\cdot \mathrm d\mathbf S.
  \]
  Consider a time-independent volume $ V $ with boundary $ S $. Since charge is conserved, we have that
\begin{align*}
	\frac{dQ}{dt} &=-I\\
	\frac{d}{dt}\int_V\rho\mathrm dV+\int_S\mathbf J\cdot \mathrm d\mathbf S&=0\\
	\int_V\left(\frac{\partial \rho}{\partial t}+\nabla \cdot \mathbf J\right)\mathrm dV&=0
\end{align*}
Since this is true for any $ V $, we have that
\[
	\frac{\partial \rho}{\partial t}+\nabla\cdot \mathbf J=0.
\]
This \textit{equation of charge conservation} has the typical form of a conservation law.\par
The discrete charge distribution of a single particle of charge $ q_i $; and position vector $ \x_i(t) $, is
\begin{align*}
	\rho&=q_i\delta(\x-\x_i(t)),\\
	\mathbf J &= q_i\dot \x_i\delta(\x-\x_i(t)).
\end{align*}
For $ N $ particles, it is 
\begin{align*}
	\rho &= \sum_{i=1}^N q_i\delta(\x-\x_i(t))\\
	\mathbf J &= \sum_{i=1}^N q_i\dot\x_i\delta(\x-\x_i(t)).
\end{align*}
As an exercise we can see that these satisfy the equation of charge conservation.
\subsection{Fields and forces}
Electromagnetism is a \textit{field theory}.\par
Charged particles don't interact directly, but rather by generating fields around them, which are then experienced by other charged particles. In general we have two time-dependent vector fields, the electric field $ \mathbf E(\x,t) $, and the magnetic field $ \mathbf B(\x,t) $.\par
The \textit{Lorentz force} on a particle of charge $ q $ and velocity $ v $ is
\[
  \mathbf F=q(\mathbf E+\mathbf v\times \mathbf B).
\]
\subsection{Maxwell's equations}
In this course we will explore some consequences of Maxwell's equations.\begin{definition}
	(Maxwell's equations)
	\begin{align*}
		\nabla \cdot \mathbf E&=\frac{\rho}{\varepsilon_0}\\
		\nabla\cdot \mathbf B &= 0 \\
		\nabla\times \mathbf E &= -\frac{\partial \mathbf B}{\partial t} \\
		\nabla \times  \mathbf B &= \mu_0\left(\mathbf J + \varepsilon_0\frac{\partial \mathbf E}{\partial t}\right).
	\end{align*}
\end{definition}
\begin{remark}
  We have some properties about these equations.
  \begin{itemize}
	  \item Coupled linear PDEs in space and time,
	  \item Involve two positive constants:
		  \begin{enumerate}
			  \item $ \varepsilon_0 $ (vacuum permittivity)
			  \item $ \mu_0 $ (vacuum permeability)
		  \end{enumerate}
	  \item Charges $ (\rho) $ and currents $ (\mathbf J) $ are the sources of electromagnetic fields.
	  \item Each equation is an equivalent integral form (see later) related via the divergece or Stokes' theorem.
	  \item These are the \textit{vacuum} equations that apply on microscopic scales or in a vacuum. A related macroscopic version applies in media (Part II Electrodynamics).
	  \item The equations of consistent with each other and with charge conservation. We will show this now.
		  \begin{enumerate}
			  \item Taking the divergence of the third equation, this agrees with the time derivative of the second equation.
			  \item For charge conversation, we have that
				  \begin{align*}
					  \frac{\partial\rho}{\partial t}+\nabla\cdot\mathbf J &= \frac{\partial }{\partial t}\left(\varepsilon_0\nabla\cdot \mathbf E\right) + \nabla\cdot \left(-\varepsilon_0\frac{\partial \mathbf E}{\partial t}+\frac1{\mu_0}\nabla\times \mathbf B\right)\\
					  &= 0.
			  \end{align*}
		  \end{enumerate}
  \end{itemize}
\end{remark}
\subsection{Units}
The SI unit of electric charge is the coulomb $ (C) $. The elementary charge is exactly
\[
	e=1.602\ 176\ 634\times 10^{-19}\ \text{C}.
\]
The SI unit of electric current is the ampere or amp (A) which is equal to $ 1\ \text{C}\ \text{s}^{-1} $.\par
The SI base units needed in electromagnetism and then the second, metre, kilogram, and ampere. From the Lorentz force law we see that the units of $ \mathbf E $ and $ \mathbf B $ must be
\[
	\text{kg}\ \text{m}\ \text{s}^{-3}\text{A}^{-1}\quad \text{and}\quad \text{kg}\ \text{s}^{-2}\text{A}^{-1}.
\]
We sometimes refer to the units of $ \mathbf B $ as the \textit{Telsa} (T).\par
From Maxwell's equations we can work out the units of $ \varepsilon_0 $ and $ \mu_0 $. The values of these constants can be calculated via experimentation as
\begin{align*}
	\varepsilon_0 &= 8.854\dots\times 10^{-12}\ \text{kg}^{-1}\text{m}^{-3}\text{s}^4\ \text{A}^2\\
	\mu_0 &= 1.256\dots\times 10^{-6}\ \text{kg}\ \text{m}\ \text{s}^{-2}\text{A}^{-2}
\end{align*}
The speed of light the exactly
\[
	c=\frac{1}{\sqrt{\mu_0\varepsilon_0}}=299\ 792\ 458\ \text{m}\ \text{s}^{-1}.
\]












\end{document}
