\documentclass{article}
\usepackage{../header}
\title{Electromagnetism}
\author{Notes by Finley Cooper}
\newcommand{\x}{\mathbf x}
\begin{document}
  \maketitle
  \newpage
  \tableofcontents
  \newpage
  \section{Introduction}
  \subsection{Charges and currents}
  \textit{Electric charge} is a physical property of elementary particles. It is:
  \begin{enumerate}
	  \item A signed quantity, it can either be positive, negative, or zero.
	  \item It is quantised to integer multiplies of the elementary charge.
	  \item It is a conserved quantity even if particles are created or destroyed.
  \end{enumerate}
  By convention the electron has charge $ -e $, the proton has charge $ +e $ and the neutron has no charge. On macroscopic scales, the number of particles is so large that charge can be considered to have a continuous electric charge density $ \rho(\mathbf x,t) $. The total charge in a volume $ V $ is then
  \[
    Q=\int_V \rho\mathrm dV.
  \]
  The \textit{electric current density} $ \mathbf J(\mathbf x,t) $ is the flux of electric charge per unit area. The current folowing through a surface $ S $ is
  \[
    I=\int_S\mathbf J\cdot \mathrm d\mathbf S.
  \]
  Consider a time-independent volume $ V $ with boundary $ S $. Since charge is conserved, we have that
\begin{align*}
	\frac{dQ}{dt} &=-I\\
	\frac{d}{dt}\int_V\rho\mathrm dV+\int_S\mathbf J\cdot \mathrm d\mathbf S&=0\\
	\int_V\left(\frac{\partial \rho}{\partial t}+\nabla \cdot \mathbf J\right)\mathrm dV&=0
\end{align*}
Since this is true for any $ V $, we have that
\[
	\frac{\partial \rho}{\partial t}+\nabla\cdot \mathbf J=0.
\]
This \textit{equation of charge conservation} has the typical form of a conservation law.\par
The discrete charge distribution of a single particle of charge $ q_i $; and position vector $ \x_i(t) $, is
\begin{align*}
	\rho&=q_i\delta(\x-\x_i(t)),\\
	\mathbf J &= q_i\dot \x_i\delta(\x-\x_i(t)).
\end{align*}
For $ N $ particles, it is 
\begin{align*}
	\rho &= \sum_{i=1}^N q_i\delta(\x-\x_i(t))\\
	\mathbf J &= \sum_{i=1}^N q_i\dot\x_i\delta(\x-\x_i(t)).
\end{align*}
As an exercise we can see that these satisfy the equation of charge conservation.
\subsection{Fields and forces}
Electromagnetism is a \textit{field theory}.\par
Charged particles don't interact directly, but rather by generating fields around them, which are then experienced by other charged particles. In general we have two time-dependent vector fields, the electric field $ \mathbf E(\x,t) $, and the magnetic field $ \mathbf B(\x,t) $.\par
The \textit{Lorentz force} on a particle of charge $ q $ and velocity $ v $ is
\[
  \mathbf F=q(\mathbf E+\mathbf v\times \mathbf B).
\]
\subsection{Maxwell's equations}
In this course we will explore some consequences of Maxwell's equations.\begin{definition}
	(Maxwell's equations)
	\begin{align*}
		\nabla \cdot \mathbf E&=\frac{\rho}{\varepsilon_0}\\
		\nabla\cdot \mathbf B &= 0 \\
		\nabla\times \mathbf E &= -\frac{\partial \mathbf B}{\partial t} \\
		\nabla \times  \mathbf B &= \mu_0\left(\mathbf J + \varepsilon_0\frac{\partial \mathbf E}{\partial t}\right).
	\end{align*}
\end{definition}
\begin{remark}
  We have some properties about these equations.
  \begin{itemize}
	  \item Coupled linear PDEs in space and time,
	  \item Involve two positive constants:
		  \begin{enumerate}
			  \item $ \varepsilon_0 $ (vacuum permittivity)
			  \item $ \mu_0 $ (vacuum permeability)
		  \end{enumerate}
	  \item Charges $ (\rho) $ and currents $ (\mathbf J) $ are the sources of electromagnetic fields.
	  \item Each equation is an equivalent integral form (see later) related via the divergece or Stokes' theorem.
	  \item These are the \textit{vacuum} equations that apply on microscopic scales or in a vacuum. A related macroscopic version applies in media (Part II Electrodynamics).
	  \item The equations of consistent with each other and with charge conservation. We will show this now.
		  \begin{enumerate}
			  \item Taking the divergence of the third equation, this agrees with the time derivative of the second equation.
			  \item For charge conversation, we have that
				  \begin{align*}
					  \frac{\partial\rho}{\partial t}+\nabla\cdot\mathbf J &= \frac{\partial }{\partial t}\left(\varepsilon_0\nabla\cdot \mathbf E\right) + \nabla\cdot \left(-\varepsilon_0\frac{\partial \mathbf E}{\partial t}+\frac1{\mu_0}\nabla\times \mathbf B\right)\\
					  &= 0.
			  \end{align*}
		  \end{enumerate}
  \end{itemize}
\end{remark}
\subsection{Units}
The SI unit of electric charge is the coulomb $ (C) $. The elementary charge is exactly
\[
	e=1.602\ 176\ 634\times 10^{-19}\ \text{C}.
\]
The SI unit of electric current is the ampere or amp (A) which is equal to $ 1\ \text{C}\ \text{s}^{-1} $.\par
The SI base units needed in electromagnetism and then the second, metre, kilogram, and ampere. From the Lorentz force law we see that the units of $ \mathbf E $ and $ \mathbf B $ must be
\[
	\text{kg}\ \text{m}\ \text{s}^{-3}\text{A}^{-1}\quad \text{and}\quad \text{kg}\ \text{s}^{-2}\text{A}^{-1}.
\]
We sometimes refer to the units of $ \mathbf B $ as the \textit{Telsa} (T).\par
From Maxwell's equations we can work out the units of $ \varepsilon_0 $ and $ \mu_0 $. The values of these constants can be calculated via experimentation as
\begin{align*}
	\varepsilon_0 &= 8.854\dots\times 10^{-12}\ \text{kg}^{-1}\text{m}^{-3}\text{s}^4\ \text{A}^2\\
	\mu_0 &= 1.256\dots\times 10^{-6}\ \text{kg}\ \text{m}\ \text{s}^{-2}\text{A}^{-2}
\end{align*}
The speed of light is exactly
\[
	c=\frac{1}{\sqrt{\mu_0\varepsilon_0}}=299\ 792\ 458\ \text{m}\ \text{s}^{-1}.
\]
\section{Electrostatics}
In a time-independent situation, Maxwell's equations reduce to
\begin{align*}
	\nabla\cdot\mathbf E &= \frac \rho{\varepsilon_0}\\
	\nabla \cdot \mathbf B &= 0 \\
	\nabla \times \mathbf E &= 0 \\
	\nabla \times \mathbf B &= \mu_0 \mathbf J
\end{align*}
Now $ \mathbf E $ and $ \mathbf B $ are decoupled so we can study them seperately. Electrostatics is the study of the electric field generated by a stationary charge distribution. We'll be looking at
\[
	\nabla\cdot \mathbf E = \frac\rho{\varepsilon_0},\qquad \nabla\times \mathbf E = 0.
\]
\subsection{Gauss' Law}
Consider a closed surface $ S $ enclosing a volume $ V $. Integrate over $ V $ and use the divergence theorem to obtain Gauss' law which is
\[
		\int_S \mathbf E\cdot \mathrm d\mathbf S = \frac{Q}{\varepsilon_0},
\]
Where
\[
  Q = \int_V \rho\mathrm dV
\]
is the total charge in $ V $. Gauss' law is the integral version of the first of Maxwell's equations and is valid generally. We get that electric flux is proportional to the total charge enclosed.\par
In special situations we use Gauss' law together with symmetry to deduce $ \E $ from $ \rho $, by choosing the \textit{Gaussian surface} S appropriately.
\subsubsection{Spherical symmetry}
Consider a spherically symmetric charge distribution, $ \rho(r) $ in spherical polar coordinates with total charge $ Q $ contained within an outer radius $ R $. To have spherical symmetry, the electric field should have the form
\[
  \mathbf E = E(r)\mathbf e_r.
\]
This will satisfy $ \nabla\times \mathbf E = 0 $ as required.\par
To find $ E(r) $ apply Gauss' law to a sphere of radius $ r $. If $ r>R $ then we get that
\begin{align*}
	\int_S\mathbf E\cdot \mathrm d\mathbf S &= E(r)\int_S \mathbf e_r\cdot \mathrm d\mathbf S\\
						&= E(r)\int_S\mathrm dS\\
						&= E(r)4\pi r^2=\frac Q{\varepsilon_0}.
\end{align*}
Thus
\[
	\mathbf E = \frac{Q}{4\pi \varepsilon_0 r^2} \mathbf e_r.
\]
So the external electric field of a spherically symmetric body depends only on the total charge, and is equivalent to a point charge at the origin with all of the charge. The Lorentz force on a particle of charge $ q $ in $ r>R $ is 
\[
	\mathbf F = q \mathbf E = \frac{Qq}{4\pi \varepsilon_0r^2}\mathbf e_r.
\]
This is the \textit{Coulomb force} between charge particles. The force is repulsive if the charges have the same sign and attractive if the charges have different sign.\par
In the limit as $ R\to 0 $ we obtain the electric field at a \textit{point charge} Q, corresponding to
\[
  \rho = Q \delta(\mathbf x).
\]
There is a close analogy between the Coulomb force and the gravitational force between massive particles, recall from IA Dynamics and Relativity that
\[
	\mathbf F = -\frac{GMm}{r^2} \mathbf e_r.
\]
Both involve an inverse-square law and the product of the charges, however there are some differences.
\begin{enumerate}
	\item While gravity is always attractive, electric forces can be repulsive or attractive;
	\item Gravity is very much weaker, due to the much smaller constant of proportionality.
\end{enumerate}
For example if we consider two protons, the ratio of the electric to gravitational force is $ 10^{36} $. On the atom scale, gravity is irrelevant. But the $ + $ and $ - $ charges balance so accurately, that they cancle on the planetary scale, and gravity is much more dominant.
\subsubsection{Cylindrical symmetry}
Consider a cylindrically symmetric charge distribution, with $ \rho(r) $ in cylindrical polar coordinates with totaly charge $ \lambda $ per unit length contained within an outer radius $ R $. To have cylindrical symmetry again we have that
\[
  \mathbf E = E(r)\mathbf e_r.
\]
Again this will satisfy $ \nabla\times \mathbf E = 0 $. To find $ E(r) $, apply Gauss' law to a cylinder of radius $ r $ arbitrary length $ L $.\par
If $ r>R $ then
\begin{align*}
	\int_S\mathbf E\cdot\mathrm d\mathbf S &= E(r)\int_S \mathbf e_r\cdot\mathrm d\mathbf S\\
					       &= E(r)\int_S\mathrm dS\\
					       &= E(r)2\pi rL = \frac{\lambda L}{\varepsilon_0}.
\end{align*}
Thus we have that
\[
	\mathbf E = \frac{\lambda}{2\pi \varepsilon_0 r} \mathbf e_r.
\]
In the limit as $ R\to 0 $ we obtain the electric field of a line charge $ \lambda $ per unit length, corresponding to $ \rho = \lambda\delta(x)\delta(y) $.
\subsubsection{Planar symmetry}
For a planar charge distribution, we have a charge density of $ \rho(z) $ in Cartesian coordinates with total charge $ \sigma $ per unit area contained within a region $ -d<z< d $ of thinkness $ 2d $.\par
We will assume reflective symmetry, so $ \rho(z) $ is even.\par
To have planar symmetry, we have $ \mathbf E = E(z)\mathbf e_z $. Again we have that $ \nabla\times \mathbf E = 0 $. The reflectional symmetry implies that $ E(-z) = -E(z) $.
\par
To find $ E(Z) $ for $ z>0 $ apply Gauss' law to a "Gaussian pillbox" of height $ 2z $ and arbitrary area $ A $. If $ z>d $ then
\begin{align*}
	\int_S \mathbf E \cdot\mathrm \mathbf S &= E(z)A-E(-z)A\\
						&= 2E(z)A\\
						&= \frac{\sigma A}{\varepsilon_0}
\end{align*}
Thus we have that
\[
  \mathbf E = \begin{cases}
	  \frac{\sigma}{2\varepsilon_0}\mathbf e_z & z< d\\
	  -\frac\sigma{2\varepsilon_0}\mathbf e_z & z< -d
  \end{cases}.
\]
In the limit as $ d\to 0 $ we obtain the electric field of a \textit{surface charge} $ \sigma $ per unit area, corresponding to $ \rho = \sigma\delta(z) $.
\subsubsection{Surface charge and discontinuity}
Let $ \mathbf n $ be a unit vector normal to the charged surface, pointing from region 1 to region 2. In our example we have that $ \mathbf n = \mathbf e_z $. This discontinuity in $ \mathbf E $ is given by
\[
	[\mathbf n\cdot \mathbf E] = \frac\sigma{\varepsilon_0}
\]
where $ \sigma $ is the surface charge density and 
\[
	[X] = X_2-X_1
\]
denotes a discontinuity between regions 1 and 2.










\end{document}
