\documentclass{article}
\usepackage{../header}
\title{Electromagnetism}
\author{Notes by Finley Cooper}
\newcommand{\x}{\mathbf x}
\begin{document}
  \maketitle
  \newpage
  \tableofcontents
  \newpage
  \section{Introduction}
  \subsection{Charges and currents}
  \textit{Electric charge} is a physical property of elementary particles. It is:
  \begin{enumerate}
	  \item A signed quantity, it can either be positive, negative, or zero.
	  \item It is quantised to integer multiplies of the elementary charge.
	  \item It is a conserved quantity even if particles are created or destroyed.
  \end{enumerate}
  By convention the electron has charge $ -e $, the proton has charge $ +e $ and the neutron has no charge. On macroscopic scales, the number of particles is so large that charge can be considered to have a continuous electric charge density $ \rho(\mathbf x,t) $. The total charge in a volume $ V $ is then
  \[
    Q=\int_V \rho\mathrm dV.
  \]
  The \textit{electric current density} $ \mathbf J(\mathbf x,t) $ is the flux of electric charge per unit area. The current folowing through a surface $ S $ is
  \[
    I=\int_S\mathbf J\cdot \mathrm d\mathbf S.
  \]
  Consider a time-independent volume $ V $ with boundary $ S $. Since charge is conserved, we have that
\begin{align*}
	\frac{dQ}{dt} &=-I\\
	\frac{d}{dt}\int_V\rho\mathrm dV+\int_S\mathbf J\cdot \mathrm d\mathbf S&=0\\
	\int_V\left(\frac{\partial \rho}{\partial t}+\nabla \cdot \mathbf J\right)\mathrm dV&=0
\end{align*}
Since this is true for any $ V $, we have that
\[
	\frac{\partial \rho}{\partial t}+\nabla\cdot \mathbf J=0.
\]
This \textit{equation of charge conservation} has the typical form of a conservation law.\par
The discrete charge distribution of a single particle of charge $ q_i $; and position vector $ \x_i(t) $, is
\begin{align*}
	\rho&=q_i\delta(\x-\x_i(t)),\\
	\mathbf J &= q_i\dot \x_i\delta(\x-\x_i(t)).
\end{align*}
For $ N $ particles, it is 
\begin{align*}
	\rho &= \sum_{i=1}^N q_i\delta(\x-\x_i(t))\\
	\mathbf J &= \sum_{i=1}^N q_i\dot\x_i\delta(\x-\x_i(t)).
\end{align*}
As an exercise we can see that these satisfy the equation of charge conservation.
\subsection{Fields and forces}
Electromagnetism is a \textit{field theory}.\par
Charged particles don't interact directly, but rather by generating fields around them, which are then experienced by other charged particles. In general we have two time-dependent vector fields, the electric field $ \mathbf E(\x,t) $, and the magnetic field $ \mathbf B(\x,t) $.\par
The \textit{Lorentz force} on a particle of charge $ q $ and velocity $ v $ is
\[
  \mathbf F=q(\mathbf E+\mathbf v\times \mathbf B).
\]
\subsection{Maxwell's equations}
In this course we will explore some consequences of Maxwell's equations.\begin{definition}
	(Maxwell's equations)
	\begin{align*}
		\nabla \cdot \mathbf E&=\frac{\rho}{\varepsilon_0}\\
		\nabla\cdot \mathbf B &= 0 \\
		\nabla\times \mathbf E &= -\frac{\partial \mathbf B}{\partial t} \\
		\nabla \times  \mathbf B &= \mu_0\left(\mathbf J + \varepsilon_0\frac{\partial \mathbf E}{\partial t}\right).
	\end{align*}
\end{definition}
\begin{remark}
  We have some properties about these equations.
  \begin{itemize}
	  \item Coupled linear PDEs in space and time,
	  \item Involve two positive constants:
		  \begin{enumerate}
			  \item $ \varepsilon_0 $ (vacuum permittivity)
			  \item $ \mu_0 $ (vacuum permeability)
		  \end{enumerate}
	  \item Charges $ (\rho) $ and currents $ (\mathbf J) $ are the sources of electromagnetic fields.
	  \item Each equation is an equivalent integral form (see later) related via the divergece or Stokes' theorem.
	  \item These are the \textit{vacuum} equations that apply on microscopic scales or in a vacuum. A related macroscopic version applies in media (Part II Electrodynamics).
	  \item The equations of consistent with each other and with charge conservation. We will show this now.
		  \begin{enumerate}
			  \item Taking the divergence of the third equation, this agrees with the time derivative of the second equation.
			  \item For charge conversation, we have that
				  \begin{align*}
					  \frac{\partial\rho}{\partial t}+\nabla\cdot\mathbf J &= \frac{\partial }{\partial t}\left(\varepsilon_0\nabla\cdot \mathbf E\right) + \nabla\cdot \left(-\varepsilon_0\frac{\partial \mathbf E}{\partial t}+\frac1{\mu_0}\nabla\times \mathbf B\right)\\
					  &= 0.
			  \end{align*}
		  \end{enumerate}
  \end{itemize}
\end{remark}
\subsection{Units}
The SI unit of electric charge is the coulomb $ (C) $. The elementary charge is exactly
\[
	e=1.602\ 176\ 634\times 10^{-19}\ \text{C}.
\]
The SI unit of electric current is the ampere or amp (A) which is equal to $ 1\ \text{C}\ \text{s}^{-1} $.\par
The SI base units needed in electromagnetism and then the second, metre, kilogram, and ampere. From the Lorentz force law we see that the units of $ \mathbf E $ and $ \mathbf B $ must be
\[
	\text{kg}\ \text{m}\ \text{s}^{-3}\text{A}^{-1}\quad \text{and}\quad \text{kg}\ \text{s}^{-2}\text{A}^{-1}.
\]
We sometimes refer to the units of $ \mathbf B $ as the \textit{Telsa} (T).\par
From Maxwell's equations we can work out the units of $ \varepsilon_0 $ and $ \mu_0 $. The values of these constants can be calculated via experimentation as
\begin{align*}
	\varepsilon_0 &= 8.854\dots\times 10^{-12}\ \text{kg}^{-1}\text{m}^{-3}\text{s}^4\ \text{A}^2\\
	\mu_0 &= 1.256\dots\times 10^{-6}\ \text{kg}\ \text{m}\ \text{s}^{-2}\text{A}^{-2}
\end{align*}
The speed of light is exactly
\[
	c=\frac{1}{\sqrt{\mu_0\varepsilon_0}}=299\ 792\ 458\ \text{m}\ \text{s}^{-1}.
\]
\section{Electrostatics}
In a time-independent situation, Maxwell's equations reduce to
\begin{align*}
	\nabla\cdot\mathbf E &= \frac \rho{\varepsilon_0}\\
	\nabla \cdot \mathbf B &= 0 \\
	\nabla \times \mathbf E &= 0 \\
	\nabla \times \mathbf B &= \mu_0 \mathbf J
\end{align*}
Now $ \mathbf E $ and $ \mathbf B $ are decoupled so we can study them seperately. Electrostatics is the study of the electric field generated by a stationary charge distribution. We'll be looking at
\[
	\nabla\cdot \mathbf E = \frac\rho{\varepsilon_0},\qquad \nabla\times \mathbf E = 0.
\]
\subsection{Gauss' Law}
Consider a closed surface $ S $ enclosing a volume $ V $. Integrate over $ V $ and use the divergence theorem to obtain Gauss' law which is
\[
		\int_S \mathbf E\cdot \mathrm d\mathbf S = \frac{Q}{\varepsilon_0},
\]
Where
\[
  Q = \int_V \rho\mathrm dV
\]
is the total charge in $ V $. Gauss' law is the integral version of the first of Maxwell's equations and is valid generally. We get that electric flux is proportional to the total charge enclosed.\par
In special situations we use Gauss' law together with symmetry to deduce $ \E $ from $ \rho $, by choosing the \textit{Gaussian surface} S appropriately.
\subsubsection{Spherical symmetry}
Consider a spherically symmetric charge distribution, $ \rho(r) $ in spherical polar coordinates with total charge $ Q $ contained within an outer radius $ R $. To have spherical symmetry, the electric field should have the form
\[
  \mathbf E = E(r)\mathbf e_r.
\]
This will satisfy $ \nabla\times \mathbf E = 0 $ as required.\par
To find $ E(r) $ apply Gauss' law to a sphere of radius $ r $. If $ r>R $ then we get that
\begin{align*}
	\int_S\mathbf E\cdot \mathrm d\mathbf S &= E(r)\int_S \mathbf e_r\cdot \mathrm d\mathbf S\\
						&= E(r)\int_S\mathrm dS\\
						&= E(r)4\pi r^2=\frac Q{\varepsilon_0}.
\end{align*}
Thus
\[
	\mathbf E = \frac{Q}{4\pi \varepsilon_0 r^2} \mathbf e_r.
\]
So the external electric field of a spherically symmetric body depends only on the total charge, and is equivalent to a point charge at the origin with all of the charge. The Lorentz force on a particle of charge $ q $ in $ r>R $ is 
\[
	\mathbf F = q \mathbf E = \frac{Qq}{4\pi \varepsilon_0r^2}\mathbf e_r.
\]
This is the \textit{Coulomb force} between charge particles. The force is repulsive if the charges have the same sign and attractive if the charges have different sign.\par
In the limit as $ R\to 0 $ we obtain the electric field at a \textit{point charge} Q, corresponding to
\[
  \rho = Q \delta(\mathbf x).
\]
There is a close analogy between the Coulomb force and the gravitational force between massive particles, recall from IA Dynamics and Relativity that
\[
	\mathbf F = -\frac{GMm}{r^2} \mathbf e_r.
\]
Both involve an inverse-square law and the product of the charges, however there are some differences.
\begin{enumerate}
	\item While gravity is always attractive, electric forces can be repulsive or attractive;
	\item Gravity is very much weaker, due to the much smaller constant of proportionality.
\end{enumerate}
For example if we consider two protons, the ratio of the electric to gravitational force is $ 10^{36} $. On the atom scale, gravity is irrelevant. But the $ + $ and $ - $ charges balance so accurately, that they cancle on the planetary scale, and gravity is much more dominant.
\subsubsection{Cylindrical symmetry}
Consider a cylindrically symmetric charge distribution, with $ \rho(r) $ in cylindrical polar coordinates with totaly charge $ \lambda $ per unit length contained within an outer radius $ R $. To have cylindrical symmetry again we have that
\[
  \mathbf E = E(r)\mathbf e_r.
\]
Again this will satisfy $ \nabla\times \mathbf E = 0 $. To find $ E(r) $, apply Gauss' law to a cylinder of radius $ r $ arbitrary length $ L $.\par
If $ r>R $ then
\begin{align*}
	\int_S\mathbf E\cdot\mathrm d\mathbf S &= E(r)\int_S \mathbf e_r\cdot\mathrm d\mathbf S\\
					       &= E(r)\int_S\mathrm dS\\
					       &= E(r)2\pi rL = \frac{\lambda L}{\varepsilon_0}.
\end{align*}
Thus we have that
\[
	\mathbf E = \frac{\lambda}{2\pi \varepsilon_0 r} \mathbf e_r.
\]
In the limit as $ R\to 0 $ we obtain the electric field of a line charge $ \lambda $ per unit length, corresponding to $ \rho = \lambda\delta(x)\delta(y) $.
\subsubsection{Planar symmetry}
For a planar charge distribution, we have a charge density of $ \rho(z) $ in Cartesian coordinates with total charge $ \sigma $ per unit area contained within a region $ -d<z< d $ of thinkness $ 2d $.\par
We will assume reflective symmetry, so $ \rho(z) $ is even.\par
To have planar symmetry, we have $ \mathbf E = E(z)\mathbf e_z $. Again we have that $ \nabla\times \mathbf E = 0 $. The reflectional symmetry implies that $ E(-z) = -E(z) $.
\par
To find $ E(z) $ for $ z>0 $ apply Gauss' law to a "Gaussian pillbox" of height $ 2z $ and arbitrary area $ A $. If $ z>d $ then
\begin{align*}
	\int_S \mathbf E \cdot\mathrm d \mathbf S &= E(z)A-E(-z)A\\
						&= 2E(z)A\\
						&= \frac{\sigma A}{\varepsilon_0}
\end{align*}
Thus we have that
\[
  \mathbf E = \begin{cases}
	  \frac{\sigma}{2\varepsilon_0}\mathbf e_z & z< d\\
	  -\frac\sigma{2\varepsilon_0}\mathbf e_z & z< -d
  \end{cases}.
\]
In the limit as $ d\to 0 $ we obtain the electric field of a \textit{surface charge} $ \sigma $ per unit area, corresponding to $ \rho = \sigma\delta(z) $.
\subsubsection{Surface charge and discontinuity}
Let $ \mathbf n $ be a unit vector normal to the charged surface, pointing from region 1 to region 2. In our example we have that $ \mathbf n = \mathbf e_z $. This discontinuity in $ \mathbf E $ is given by
\[
	[\mathbf n\cdot \mathbf E] = \frac\sigma{\varepsilon_0}
\]
where $ \sigma $ is the surface charge density and 
\[
	[X] = X_2-X_1
\]
denotes a discontinuity between regions 1 and 2.\par
The tangential components are continuous:
\[
	[\mathbf n\times \mathbf E]= 0.
\]
And these two equations apply to any surface surface even if it's curved and non-uniform.
\subsection{The electrostatic potential}
For a general $ \rho(\x) $ we cannot determine $ \mathbf E(\x) $ using Gauss' law alone.  We'll need to use the Maxwell equation $ \nabla\times \mathbf E = 0 $. This implies that $ E $ is irrotational so it has an \textit{electrostatic} potential $ \Phi(\x) $, such that
\[
  \mathbf E = - \nabla \Phi.
\]
\begin{definition}
	(Potential difference) The \textit{potential difference} or \text{voltage} between two points $ \x_1 $ and $ \x_2 $ is
	\[
		\Phi(\x_2)-\Phi(\x_1) &= \int\mathrm d\Phi \\
				      &= - \int_{\x_1}^{\x_2} \mathbf E\cdot\mathrm d\x
	\]
	and is path independent since $ \nabla\times \mathbf E = 0 $ is zero and the region is simply connected, so the field is conservative .
\end{definition}
\begin{definition}
	(Electric force) The \textit{electric force} on a particle of charge $ q $ is
	\[
	  \mathbf F = q\mathbf E = -q\nabla\Phi.
	\]
\end{definition}
\begin{remark}
  This is a conservative force associated with the potential energy
  \[
    U(\x) = q\Phi(\x).
  \]
\end{remark}
Recall that the first Maxwell equation implies that $ \Phi $ satisfies Poisson's equation, so
\[
	-\nabla^2 \Phi =\frac\rho{\varepsilon_0}.
\]
So we have the solution (from IB Methods) as (over all space with boundary conditions that $ \Phi \to 0$ as $ |\x|\to\infty $).
\[
	\Phi(\x)=\frac1{4\pi\varepsilon}\int\frac{\rho(\x')}{|\x-\x'|}\mathrm d^3\x'.
\]
This is the convolution of $ \rho(\x) $ with the potential of a unit point charge (which relates to our Green's function from IB Methods) $ \frac{1}{4\pi\varepsilon |\x|}$. Namely it is the solution to
\[
	-\nabla^2\Phi = \frac{\delta(\x)}{\varepsilon_0}
\]
satisfying $ \Phi\to 0 $ as $ |\x|\to \infty $. Note that $ \E $ is unaffected if we add an arbitrary constant to $ \Phi $ (this makes sense since $ \Phi $ measures a potential difference between two points so increasing the charge uniformly doesn't change $ \E $). We usually choose this such that $ \Phi \to 0 $ as $ |\x|\to \infty $. If $ \rho(\x) $ does not decay sufficiently rapidly this may not be possible. For example if we have a line charge $ E_r\propto \frac 1r $, so we have that $ \Phi\propto \log r $ which doesn't go to zero as $ r\to \infty $.
\subsubsection{Point charge}
The potential due to a point charge $ q $ at the origin is
\[
	\Phi(\x) = \frac q{4\pi\varepsilon_0|\x|} = \frac q{4\pi\varepsilon_0 r}.
\]
\subsubsection{Electric dipole}
Two equal and opposite charges at different positions. Without loss of generality consider charges $ -q $ t $ \x= 0 $ and $ +q $ at $ \x = \mathbf d $. The potential due to the dipole is
\[
	\Phi(\x) = \frac q{4\pi \varepsilon_0}\left(-\frac 1{|\x|} + \frac 1{|\x-\mathbf d|}\right)
\]
Apply Taylor's theorem for a scalar field,
\[
  f(\x+\mathbf h) = f(\x) + (\mathbf h \cdot\nabla)f(\x) + \frac 12 (\mathbf h \cdot \nabla)^2 f(\x) + O(||\mathbf h ||^2).
\]
So we get that
\[
	\Phi(\x) = \frac q{4\pi \varepsilon_0} \left(-\frac 1r + \frac 1r -(\mathbf d \cdot \nabla)\frac 1r + O(|\mathbf d |^2)\right) = \frac{q\mathbf d\cdot\mathbf x}{4\pi\varepsilon_0 |\x|^3}+O(|\mathbf d|^2).
\]
In the limit as $ |\mathbf d| \to 0 $ with $ q\mathbf d $ finite, we obtain a \textit{point dipole} with \textit{electric dipole moment}
\[
  \mathbf p =q\mathbf d.
\]
which has potential
\[
	\Phi(\x) = \frac{\p \cdot \x}{4\pi \varepsilon_0 |\x|^3}
\]
and electric field
\begin{align*}
	\mathbf E = -\nabla\Phi &= \frac {3(\mathbf p \cdot \x)\x-|\x|^2 \mathbf p}{4\pi\varepsilon_0 |\x|^5}.
\end{align*}
In spherical polar coordinates aligned with $ \mathbf p  = p\mathbf e_z $. So
\[
	\Phi = \frac {p\cos\theta}{4\pi\varepsilon_0 r^2}.
\]
Then we get that
\[
	E_r = -\frac{\partial \Phi}{\partial r} = \frac  {2p\cos(\theta)}{4\pi\varepsilon_0 r^3}
\]
and
\[
	E_\theta = -\frac 1r \frac{\partial \Phi}{\partial \theta} = \frac{p\sin\theta}{4\pi\varepsilon_0 r^3}.
\]
From our alignment we have that $ E_\phi = 0 $.
\begin{remark}
  Note that
  \begin{enumerate}
	  \item $ \Phi $ and $ \mathbf E $ are not spherically symmetric.
	  \item They decrease more rapidly with $ r $ than a point charge since the dipole are nearly cancelling eachother out.
  \end{enumerate}
\end{remark}
A point dipole $ \mathbf p  $ at the origin corresponds to
\[
  \rho(\x) = -\mathbf p \cdot \nabla \delta(\x),
\]
So we can find the associated potential $ \Phi $ as
\[
	\Phi(\x) = \mathbf p \cdot\nabla\left(\frac1{4\pi\varepsilon_0 |\x|}\right).
\]
\subsubsection{Field lines and equipotentials}
\textit{Electric field lines} are the integral curves of $ \mathbf E $ being tangent to $ \mathbf E $ everywhere. Since we have that $ \nabla\cdot \mathbf E = \frac \rho{\varepsilon_0} $, field lines begin on positive charges and end on negative charges. In electrostatics, $ \mathbf E = -\nabla\Phi $, so field lines are perpendicular to the equipotential surfaces of which $ \Phi $ are constant.
\subsubsection{Dipole in an external field}
Consider a dipole $ \mathbf p $ in an external field $ \mathbf E_{\text{external}} = - \nabla\Phi $ generated by distinct charges. With $ -q $ at $ \x $ and $ +q $ and $ \x+\mathbf d $, the potential energy at the dipole due to the external field is
\begin{align*}
	U &= -q\Phi(\x) +Q \Phi(\x+\mathbf d)\\
	  &= q(\mathbf d\cdot\nabla)\Phi(\x) + O(|\mathbf d|^2) \\
\end{align*}
In the limit at the point dipole, 
\[
	U=\mathbf p \cdot\nabla\Phi = -\mathbf p \cdot \mathbf E_{\text{external}}
\]
and is minimised when $ \mathbf p  $ is aligned with $ \mathbf E_{\text{external}} $.
\subsubsection{Multipole expansion}
For a general charge distribution $ \rho(\x) $ confined to a ball $ \{V:|\x|<R\} $,
\[
	\Phi(\x) = \frac 1{4\pi\varepsilon_0}\int_V \frac{\rho(\x')}{|\x-\x'|}\mathrm d^3\x'.
\]
We'll look at the external potential at $ \x $ with $ \x\notin V $. Expand 
\[
	\frac 1{|\x-\x'|}= \frac 1r - (\x'\cdot \nabla)\frac 1r + \frac 12(\x'\cdot \nabla)^2\frac 1r + O(|\x'|^3).
\]
Which is
\[
	=\frac 1r \left[1+\frac {\x'\cdot \x}{r^2} + \frac{3(\x'\cdot\x)^2 - |\x'|^2|\x|^2}{2r^4} + O\left(\frac {R^3}{r^3}\right)\right]
\]
This leads to the \textit{multipole expansion} of the potential,
\[
	\Phi(\x) = \frac 1{4\pi \varepsilon_0} \left(\frac Qr + \frac{\mathbf p\cdot\x}{r^2} + \frac 12 \frac{Q_{ij}x_ix_j}{r^5}+\cdots\right).
\]
The first three multipole moments:
\begin{enumerate}
	\item The total charge, $ Q=\int_V \rho(\mathbf x)\  \mathrm d^3 \x $.
	\item The electric dipole moment $ \mathbf p =\int_V \x \rho(\x)\ \mathrm d^3 \x $.
	\item The electric quadrupole moment. This is a second order tensor which is traceless and symmetric,
		\[
			Q_{ij} = \int_V (3x_ix_j - |\x|^2 \delta_{ij})\rho (\x)\ \mathrm d^3 \x.
		\]
\end{enumerate}
For $ \gg R $, $ \Phi $ and $ \mathbf E $ look increasingly like those of a point charge $ Q $, unless $ Q=0 $, in which case they look like those of a point dipole, unless $ \mathbf p = 0 $, etc.
\subsection{Electrostatic energy}
The work done against the electric force, $ \mathbf F = q\mathbf E $, in bringing in a particle of charge q from infinity (where we assume that $ \Phi=0 $ at infinity) to $ \x $ is 
\[
	-\int_{\infty}^{\x} \mathbf F \cdot\mathrm d\x = +q\int_\infty^{\x} \nabla\Phi\cdot\mathrm d\x = q\Phi(\x).
\]
Consider assembling a confriguration of $ N $ point charges one by one. Particle $ i $ of charge $ q_i $ is brought from $ \infty $ to $ \x_i $ while the previous particles remain fixed. For the first particle no work is involved, $ W_1 = 0 $. For the second particle
\[
	W_2 = q_2\left(\frac{q_1}{4\pi\varepsilon_0 |\x_2-\x_1|}\right)
\]
and for the third particle
\[
	W_3 = q_3\left(\frac{q_1}{4\pi\varepsilon_0 |\x_3-\x_1|} + \frac{q_2}{4\pi\varepsilon_0 |\x_3-\x_2|}\right)
\]
So the total work done is
\[
	U= \sum_{i=1}^N W_i = \sum_{i=2}^N\sum_{j=1}^{i-1} \frac{q_i q_j}{4\pi\varepsilon_0 |\x_i-\x_j|}.
\]
This can be rewritten as
\[
U=	\frac 12 \sum_{i=1}^N\sum_{j\ne i} \frac{q_iq_j}{4\pi \varepsilon_0|\x_i-\x_j|}.
\]
or
\[
	U = \frac 12 \sum_{i=1}^N q_i\Phi(\x_i).
\]
We can generalise to a continuous charge distribution $ \rho(\x) $ occupying a finite volume $ V $.
\[
  U= \frac 12 \int_V\rho(\x)\Phi(\x)\mathrm d^3 \x.
\]
Using the first Maxwell equation we ge that
\begin{align*}
	U &= \frac 12 \int_V(\varepsilon_0 \grad\cdot \mathbf E)\Phi\mathrm dV \\
	  &= \frac{\varepsilon_0}2 \int_V\left(\nabla\cdot(\Phi\mathbf E) - \mathbf E \cdot \nabla \Phi\right)\mathrm dV\\
	  &= \frac{\varepsilon_0}2 \int_S \Phi \mathbf E \cdot\mathrm d\mathbf S + \int_V \frac{\varepsilon_0 |\mathbf E|^2}2 \mathrm dV.
\end{align*}
Let $ S=\partial V $ be a sphere of radius $ R\to\infty $. Then $ \Phi= O(\inv R) $ and $ \mathbf E = O(R^{-2}) $ on $ S $ while the area of $ S $ is $ O(R^2) $, so $ \int_S $ is $ O(\inv R) $ and $ \to 0 $ as $ R\to\infty $. Then
\[
	U = \int\frac{\varepsilon_0 |\mathbf E|^2}2 \mathrm dV
\]
where the integral is taken over all space, not just the volume where the charges are contained.
\begin{remark}
  This implies that energy is stored is the electric field, even in a vacuum.
\end{remark}
Any of expression for $ U $ suggests that the self-energy of a point charge is infinite, hence for $ U $ to be useful, we discard all self-energies since it is unchanging and causes no force.
\subsection{Conductors} 
In a \textit{conductor} such as a metal, some charges can move freely. In electrostatics we require
\[
	\mathbf E = 0, \qquad \Phi=\text{constant}
\]
inside a conductor, hence $ \rho = 0 $. Otherwise free charges would move in a response to the electric force and a current would flow.\par
However a surface charge density $ \sigma $ can exist on the surface of a conductor, which is an equipotential.\par
Taking $ \mathbf n $ to point out of the conductor, the condition,
\[
	\mathbf n \cdot\mathbf E = \frac\sigma{\varepsilon_0}
\]
becomes
\[
	\mathbf n \cdot\mathbf E = \frac{\sigma}{\varepsilon_0}\quad\text{ immediately outside the conductor}.
\]
The constant potential of a conductor can be set by connecting it to a battery or another conductor.
\begin{definition}
	(Earthed/Grounded conductor) An \textit{earthed} or \textit{grounded} conductor is connected to the ground, usually taken as $ \Phi = 0 $.
\end{definition}
To find $ \Phi(\x) $ and $ \mathbf E(\x) $ due to the charge distribution $ \rho(\x) $ in the presence of conductors with surface $ S_i $ and potentials $ \Phi_i $ we solve Poisson's equation
\[
	-\nabla^2 \Phi = \frac\rho{\varepsilon_0}
\]
with Dirichlet boundary conditions
\[
	\Phi = \Phi_i\quad\text{on } S_i.
\]
The solution depends linearly on $ \rho $ and $ \{\Phi_i\} $.\par
Let's see an example. Take a point charge $ q $ at position $ (0,0,h) $ in a half space $ (z>0) $ bounded by an earthed conducting wall. Hence we have the boundary condition $ \Phi = 0 $ on $ z=0 $. By the method of images, the solution in $ z>0 $ is identical to that of a dipole, with image charge $ -q $ placed at $ (0,0,-h) $. The wall coincides with an equipotential of the dipole, namely the line with $ \Phi = 0 $ which is the same as our boundary condition. The induced surface charge density on the wall can be worked out from
\[
	\frac\sigma{\varepsilon_0} = \mathbf n\cdot\mathbf E = E_z= -\frac{2qh}{4\pi\varepsilon_0(r^2+h^2)^{3/2}}.
\]
The total induced surface charge is
\begin{align*}
	\int_0^\infty \sigma\ 2\pi r \ \mathrm dr &= -qh \int_0^\infty \frac{r\ \mathrm dr}{(r^2+h^2)^{3/2}}\\
	&= -q
\end{align*}
which is equal to the image charge.
\begin{definition}
	(Capacitor) A simple \textit{capacitor} constants of two seperated conductors carrying charges $ \pm Q $. If the potential difference between them is $ V $, then the capacitance is defined by
	\[
	  C = \frac QV
	\]
	and depends only on the geometry, because $ \Phi $ depends linearly on $ Q $.
\end{definition}
For example, consider two infinite parallel plates seperated by some distance $ d $. Let the plate surfaces at $ z=0, z=d $ have surface charge densities $ \pm \sigma $. Then $ \mathbf E = E \mathbf e_z $ with $ E= \sigma/\varepsilon_0  = \text{const.}$ for $ 0<z<d $, with $ \mathbf E = 0 $ elsewhere. So
\[
	\Phi = -Ez + \text{const},\quad\text{and}\quad V = Ed.
\]
The same solution holds approximately for parallel plates of $ A\gg d^2 $ if end effects are neglected. So 
\[
	C= \frac QV\approx \frac{\sigma A} {Ed}\approx \frac{\varepsilon_0 A}d.
\]
The electrostatic energy stored in a capacitor is
\begin{align*}
	U &= \int \frac{\varepsilon_0 |\mathbf E|^2}2 \mathrm dV\\
	  &\approx \frac{\varepsilon_0 E^2}2 Ad\\
	  &\approx \frac 12 C V^2.
\end{align*}
In general,
\[
	U = \frac 12 CV^2 = \frac {Q^2}{2C}.
\]
The work done in moving an element of charge $ \delta Q $ from one plate to another is $ \delta W = V\ \delta Q $, so the total work done is
\begin{align*}
	\int_0^Q \frac{Q'}C \mathrm Q' = \frac{Q^2}{2C}
\end{align*}
for any geometry of the capactor. Or we can use
\begin{align*}
	U &= \frac 12 \int \rho \Phi \mathrm dV\\
	  &= \frac 12 Q \Phi_+ - \frac 12 Q \Phi_-\\
	  &= \frac 12 QV.
\end{align*}
\section{Magnetostatics}
\textit{Magnetostatics} is the study of t h emagnetic field generated by a stationary current distribution.
\begin{align*}
	\nabla\times \mathbf B &= \mu_0 \mathbf J\\
	\nabla\cdot\mathbf B = 0,
\end{align*}
and the first equation here implies that $ \nabla\cdot\mathbf J = 0 $, the time-independent equation of charge conservation.
\subsection{Ampere's law}
Consider a closed curve $ C $ that is the boundary of an open surface $ S $. Integrate the fourth Maxwell equation over $ S $ and apply Stokes' theorem to obtain Ampere's law.
\[
  \int_C \mathbff B\cdot\mathrm d\x = \mu_0 I
\]
where 
\[
  I = \int_S\mathbf J \cdot\mathrm d\mathbf S.
\]
$ I $ is the total current through the surface $ S $. Since $ \nabla\cdot \mathbf J = 0 $, the same current $ I $ flows through \textit{any} open surface $ S $ such that $ \partial S = C $. Ampere's law is the integral version of the forth Maxwell equation and is valid provided that $ \frac{\partial \mathbf E}{\partial t} = 0 $. Ampere's law is saying that
\[
	\text{circulation of magnetic field around loop}\ \propto\ \text{total current through loop}.
\]
In special situations we can use Ampere's law together with symmetry to deudce $ \mathbf B $ from $ \mathbf J $.\par
A cylindrically symmetric situation could involve
\begin{itemize}
	\item An axial current distribution
		\[
		  J_z(r) \mathbf e_z.
		\]
	\item An azimuthal current distribution
		\[
		  J_\phi(r)\mathbf e_\phi
		\]
\end{itemize}
or a combination. (In fact $ \nabla\cdot\mathbf J = 0 $ excludes a radial current.)\par
The same applies to $ \mathbf B $. The curl in the forth Maxwell equation implies that $ B_\phi $ is lineraly related to $ J_z $ and $ B_z $ is linearly related to $ J_\phi $.
Let's consider a cylindrical wire of radius $ R $ which carries a total current $ I $ parallel to its axis. To find $ B_\phi(r) $ generated by $ J_z(r) $, apply Ampere's law to a circle $ C $ of radius $ R $.\par
If $ r>R $ then
\begin{align*}
	\int_C\mathbf B\cdot\mathrm d\x &= B_\phi(r) \int_C \mathbf e_\phi\cdot\mathrm  d\x\\
  &= B_\phi(r)\int_C\mathrm d\ell\\
  &= B_\phi(r) 2\pi r = \mu_0 I.
\end{align*}
Thus, outside the wire,
\[
	\mathbf B = \frac{\mu_0 I}{2\pi r} \mathbf e_\phi.
\]
Now for a solenoid. A thin wire is coiled around a cylindrical tube of radius $ R $. An \textit{ideal solenoid} is infintely long and tightly wound, having cylindrical symmetry and purely azimuthal current. The wire carries current $ I $ and has $ N $ turns per unit length of the tube.
\par
To find $ B_z(r) $ generated by $ J_\phi(r) $, apply Ampere's law to a recntagular loop $ C $. Taking $ a<b<R $ or $ R<a<b $ gives that
\[
  L(B_z(a)-B_z(b)) = 0.
\]
and taking $ a<R<b $ gives
\[
  L(B_z(a)-B_z(b)) = \mu_0 NLI.
\]
Asumming that $ B_z(r)\to 0 $ as $ r\to\infty $, we deduce that
\[
 B_z(r) = \begin{cases}
	 \mu_0 NI & r< R \\
	 0 & r> R
 \end{cases}   
\]
The ideal solenoid is an example of a surfac current, here of the form
\[
  J_\phi(r) = K_\phi \delta(r-R)
\]
with $ K_\phi = NI $.\par
Generally, a surface current density, $ \mathbf K $, produces a discontinuity in the tangential magnetic field:
\[
	[\mathbf n \times \mathbf B] = \mu_0 \mathbf K.
\]
It follows from Ampere's law that the norm component is continuous, i.e.
\[
	[\mathbf n \cdot\mathbf B] = 0.
\]
\subsection{The magnetic vector potential}
The second Maxwell equation implies that $ \mathbf B $ can be written in terms of a \textit{magnetic vector potential}.
\begin{definition}
	(Magnetic vector potential) For a magnetic field $ \mathbf B $, the \textit{magnetic vector potential} is the vector $ \mathbf A(\x) $ such that
	\[
	  \mathbf B = \nabla\times \mathbf A.
	\]
\end{definition}
\begin{remark}
	$ \mathbf A $ is \textit{not} unique. If we make a \textit{gauge transformation}, replacing $ \mathbf A $ with 
	\[
		\tilde{\mathbf A} = \mathbf A +\nabla\chi,
	\]
	where $ \chi(\x) $ is an arbitrary scalar field, then $ \mathbf B $ is unchanged since $ \mathbf B = \nabla\times \mathbf A = \nabla\time\times \tilde{\mathbf A} $. \par
	A convenient guage for many calculations is the \textit{Coulomb gauge} in which $ \nabla\cdot \mathbf A = 0 $. We can assume this condition \textit{wlog}. If $ \nabla\cdot \nabla\ne 0 $, then we can make a gauge transformation such that $ \nabla\cdot\tilde{\mathbf A} = 0 $ by choosing $ \chi $ to be the solution of Poisson's equation
	\[
	  -\nabla^2 \chi = \nabla\cdot\mathbf A.
	\]
\end{remark}
In terms of $ \mathbf A $ the fourth Maxwell equation becomes
\[
  \nabla\times(\nabla \times\mathbf A ) = \mu_0 \mathbf J.
\]
Using the identity,
\[
  \nabla\times(\nabla\times \mathbf A) = \nabla(\nabla\cdot \mathbf A) - \nabla^2 \mathbf A
\]
and assuming Coulomb gauge with $ \nabla\cdot\mathbf A =0 $, we obtain Poisson' equation in vector form,
\[
  -\nabla^2 \mathbf A = \mu_0 \mathbf J
\]
\subsection{The Biot-Sarat law}
The solution of Poisson's equation is
\[
	\mathbf A(\x) = \frac{\mu_0}{4\pi} \int\frac{\mathbf J(\x')}{|\x - \x'|} \mathrm d^3 \x'
\]




\end{document}
