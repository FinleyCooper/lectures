\documentclass{article}
\usepackage{../header}
\title{Electromagnetism}
\author{Notes by Finley Cooper}
\newcommand{\x}{\mathbf x}
\begin{document}
  \maketitle
  \newpage
  \tableofcontents
  \newpage
  \section{Introduction}
  \subsection{Charges and currents}
  \textit{Electric charge} is a physical property of elementary particles. It is:
  \begin{enumerate}
	  \item A signed quantity, it can either be positive, negative, or zero.
	  \item It is quantised to integer multiplies of the elementary charge.
	  \item It is a conserved quantity even if particles are created or destroyed.
  \end{enumerate}
  By convention the electron has charge $ -e $, the proton has charge $ +e $ and the neutron has no charge. On macroscopic scales, the number of particles is so large that charge can be considered to have a continuous electric charge density $ \rho(\mathbf x,t) $. The total charge in a volume $ V $ is then
  \[
    Q=\int_V \rho\mathrm dV.
  \]
  The \textit{electric current density} $ \mathbf J(\mathbf x,t) $ is the flux of electric charge per unit area. The current folowing through a surface $ S $ is
  \[
    I=\int_S\mathbf J\cdot \mathrm d\mathbf S.
  \]
  Consider a time-independent volume $ V $ with boundary $ S $. Since charge is conserved, we have that
\begin{align*}
	\frac{dQ}{dt} &=-I\\
	\frac{d}{dt}\int_V\rho\mathrm dV+\int_S\mathbf J\cdot \mathrm d\mathbf S&=0\\
	\int_V\left(\frac{\partial \rho}{\partial t}+\nabla \cdot \mathbf J\right)\mathrm dV&=0
\end{align*}
Since this is true for any $ V $, we have that
\[
	\frac{\partial \rho}{\partial t}+\nabla\cdot \mathbf J=0.
\]
This \textit{equation of charge conservation} has the typical form of a conservation law.\par
The discrete charge distribution of a single particle of charge $ q_i $; and position vector $ \x_i(t) $, is
\begin{align*}
	\rho&=q_i\delta(\x-\x_i(t)),\\
	\mathbf J &= q_i\dot \x_i\delta(\x-\x_i(t)).
\end{align*}
For $ N $ particles, it is 
\begin{align*}
	\rho &= \sum_{i=1}^N q_i\delta(\x-\x_i(t))\\
	\mathbf J &= \sum_{i=1}^N q_i\dot\x_i\delta(\x-\x_i(t)).
\end{align*}
As an exercise we can see that these satisfy the equation of charge conservation.
\subsection{Fields and forces}
Electromagnetism is a \textit{field theory}.\par
Charged particles don't interact directly, but rather by generating fields around them, which are then experienced by other charged particles. In general we have two time-dependent vector fields, the electric field $ \mathbf E(\x,t) $, and the magnetic field $ \mathbf B(\x,t) $.\par
The \textit{Lorentz force} on a particle of charge $ q $ and velocity $ v $ is
\[
  \mathbf F=q(\mathbf E+\mathbf v\times \mathbf B).
\]
\subsection{Maxwell's equations}
In this course we will explore some consequences of Maxwell's equations.\begin{definition}
	(Maxwell's equations)
	\begin{align*}
		\nabla \cdot \mathbf E&=\frac{\rho}{\varepsilon_0}\\
		\nabla\cdot \mathbf B &= 0 \\
		\nabla\times \mathbf E &= -\frac{\partial \mathbf B}{\partial t} \\
		\nabla \times  \mathbf B &= \mu_0\left(\mathbf J + \varepsilon_0\frac{\partial \mathbf E}{\partial t}\right).
	\end{align*}
\end{definition}
\begin{remark}
  We have some properties about these equations.
  \begin{itemize}
	  \item Coupled linear PDEs in space and time,
	  \item Involve two positive constants:
		  \begin{enumerate}
			  \item $ \varepsilon_0 $ (vacuum permittivity)
			  \item $ \mu_0 $ (vacuum permeability)
		  \end{enumerate}
	  \item Charges $ (\rho) $ and currents $ (\mathbf J) $ are the sources of electromagnetic fields.
	  \item Each equation is an equivalent integral form (see later) related via the divergece or Stokes' theorem.
	  \item These are the \textit{vacuum} equations that apply on microscopic scales or in a vacuum. A related macroscopic version applies in media (Part II Electrodynamics).
	  \item The equations of consistent with each other and with charge conservation. We will show this now.
		  \begin{enumerate}
			  \item Taking the divergence of the third equation, this agrees with the time derivative of the second equation.
			  \item For charge conversation, we have that
				  \begin{align*}
					  \frac{\partial\rho}{\partial t}+\nabla\cdot\mathbf J &= \frac{\partial }{\partial t}\left(\varepsilon_0\nabla\cdot \mathbf E\right) + \nabla\cdot \left(-\varepsilon_0\frac{\partial \mathbf E}{\partial t}+\frac1{\mu_0}\nabla\times \mathbf B\right)\\
					  &= 0.
			  \end{align*}
		  \end{enumerate}
  \end{itemize}
\end{remark}
\subsection{Units}
The SI unit of electric charge is the coulomb $ (C) $. The elementary charge is exactly
\[
	e=1.602\ 176\ 634\times 10^{-19}\ \text{C}.
\]
The SI unit of electric current is the ampere or amp (A) which is equal to $ 1\ \text{C}\ \text{s}^{-1} $.\par
The SI base units needed in electromagnetism and then the second, metre, kilogram, and ampere. From the Lorentz force law we see that the units of $ \mathbf E $ and $ \mathbf B $ must be
\[
	\text{kg}\ \text{m}\ \text{s}^{-3}\text{A}^{-1}\quad \text{and}\quad \text{kg}\ \text{s}^{-2}\text{A}^{-1}.
\]
We sometimes refer to the units of $ \mathbf B $ as the \textit{Telsa} (T).\par
From Maxwell's equations we can work out the units of $ \varepsilon_0 $ and $ \mu_0 $. The values of these constants can be calculated via experimentation as
\begin{align*}
	\varepsilon_0 &= 8.854\dots\times 10^{-12}\ \text{kg}^{-1}\text{m}^{-3}\text{s}^4\ \text{A}^2\\
	\mu_0 &= 1.256\dots\times 10^{-6}\ \text{kg}\ \text{m}\ \text{s}^{-2}\text{A}^{-2}
\end{align*}
The speed of light is exactly
\[
	c=\frac{1}{\sqrt{\mu_0\varepsilon_0}}=299\ 792\ 458\ \text{m}\ \text{s}^{-1}.
\]
\section{Electrostatics}
In a time-independent situation, Maxwell's equations reduce to
\begin{align*}
	\nabla\cdot\mathbf E &= \frac \rho{\varepsilon_0}\\
	\nabla \cdot \mathbf B &= 0 \\
	\nabla \times \mathbf E &= 0 \\
	\nabla \times \mathbf B &= \mu_0 \mathbf J
\end{align*}
Now $ \mathbf E $ and $ \mathbf B $ are decoupled so we can study them seperately. Electrostatics is the study of the electric field generated by a stationary charge distribution. We'll be looking at
\[
	\nabla\cdot \mathbf E = \frac\rho{\varepsilon_0},\qquad \nabla\times \mathbf E = 0.
\]
\subsection{Gauss' Law}
Consider a closed surface $ S $ enclosing a volume $ V $. Integrate over $ V $ and use the divergence theorem to obtain Gauss' law which is
\[
		\int_S \mathbf E\cdot \mathrm d\mathbf S = \frac{Q}{\varepsilon_0},
\]
Where
\[
  Q = \int_V \rho\mathrm dV
\]
is the total charge in $ V $. Gauss' law is the integral version of the first of Maxwell's equations and is valid generally. We get that electric flux is proportional to the total charge enclosed.\par
In special situations we use Gauss' law together with symmetry to deduce $ \E $ from $ \rho $, by choosing the \textit{Gaussian surface} S appropriately.
\subsubsection{Spherical symmetry}
Consider a spherically symmetric charge distribution, $ \rho(r) $ in spherical polar coordinates with total charge $ Q $ contained within an outer radius $ R $. To have spherical symmetry, the electric field should have the form
\[
  \mathbf E = E(r)\mathbf e_r.
\]
This will satisfy $ \nabla\times \mathbf E = 0 $ as required.\par
To find $ E(r) $ apply Gauss' law to a sphere of radius $ r $. If $ r>R $ then we get that
\begin{align*}
	\int_S\mathbf E\cdot \mathrm d\mathbf S &= E(r)\int_S \mathbf e_r\cdot \mathrm d\mathbf S\\
						&= E(r)\int_S\mathrm dS\\
						&= E(r)4\pi r^2=\frac Q{\varepsilon_0}.
\end{align*}
Thus
\[
	\mathbf E = \frac{Q}{4\pi \varepsilon_0 r^2} \mathbf e_r.
\]
So the external electric field of a spherically symmetric body depends only on the total charge, and is equivalent to a point charge at the origin with all of the charge. The Lorentz force on a particle of charge $ q $ in $ r>R $ is 
\[
	\mathbf F = q \mathbf E = \frac{Qq}{4\pi \varepsilon_0r^2}\mathbf e_r.
\]
This is the \textit{Coulomb force} between charge particles. The force is repulsive if the charges have the same sign and attractive if the charges have different sign.\par
In the limit as $ R\to 0 $ we obtain the electric field at a \textit{point charge} Q, corresponding to
\[
  \rho = Q \delta(\mathbf x).
\]
There is a close analogy between the Coulomb force and the gravitational force between massive particles, recall from IA Dynamics and Relativity that
\[
	\mathbf F = -\frac{GMm}{r^2} \mathbf e_r.
\]
Both involve an inverse-square law and the product of the charges, however there are some differences.
\begin{enumerate}
	\item While gravity is always attractive, electric forces can be repulsive or attractive;
	\item Gravity is very much weaker, due to the much smaller constant of proportionality.
\end{enumerate}
For example if we consider two protons, the ratio of the electric to gravitational force is $ 10^{36} $. On the atom scale, gravity is irrelevant. But the $ + $ and $ - $ charges balance so accurately, that they cancle on the planetary scale, and gravity is much more dominant.
\subsubsection{Cylindrical symmetry}
Consider a cylindrically symmetric charge distribution, with $ \rho(r) $ in cylindrical polar coordinates with totaly charge $ \lambda $ per unit length contained within an outer radius $ R $. To have cylindrical symmetry again we have that
\[
  \mathbf E = E(r)\mathbf e_r.
\]
Again this will satisfy $ \nabla\times \mathbf E = 0 $. To find $ E(r) $, apply Gauss' law to a cylinder of radius $ r $ arbitrary length $ L $.\par
If $ r>R $ then
\begin{align*}
	\int_S\mathbf E\cdot\mathrm d\mathbf S &= E(r)\int_S \mathbf e_r\cdot\mathrm d\mathbf S\\
					       &= E(r)\int_S\mathrm dS\\
					       &= E(r)2\pi rL = \frac{\lambda L}{\varepsilon_0}.
\end{align*}
Thus we have that
\[
	\mathbf E = \frac{\lambda}{2\pi \varepsilon_0 r} \mathbf e_r.
\]
In the limit as $ R\to 0 $ we obtain the electric field of a line charge $ \lambda $ per unit length, corresponding to $ \rho = \lambda\delta(x)\delta(y) $.
\subsubsection{Planar symmetry}
For a planar charge distribution, we have a charge density of $ \rho(z) $ in Cartesian coordinates with total charge $ \sigma $ per unit area contained within a region $ -d<z< d $ of thinkness $ 2d $.\par
We will assume reflective symmetry, so $ \rho(z) $ is even.\par
To have planar symmetry, we have $ \mathbf E = E(z)\mathbf e_z $. Again we have that $ \nabla\times \mathbf E = 0 $. The reflectional symmetry implies that $ E(-z) = -E(z) $.
\par
To find $ E(z) $ for $ z>0 $ apply Gauss' law to a "Gaussian pillbox" of height $ 2z $ and arbitrary area $ A $. If $ z>d $ then
\begin{align*}
	\int_S \mathbf E \cdot\mathrm d \mathbf S &= E(z)A-E(-z)A\\
						&= 2E(z)A\\
						&= \frac{\sigma A}{\varepsilon_0}
\end{align*}
Thus we have that
\[
  \mathbf E = \begin{cases}
	  \frac{\sigma}{2\varepsilon_0}\mathbf e_z & z< d\\
	  -\frac\sigma{2\varepsilon_0}\mathbf e_z & z< -d
  \end{cases}.
\]
In the limit as $ d\to 0 $ we obtain the electric field of a \textit{surface charge} $ \sigma $ per unit area, corresponding to $ \rho = \sigma\delta(z) $.
\subsubsection{Surface charge and discontinuity}
Let $ \mathbf n $ be a unit vector normal to the charged surface, pointing from region 1 to region 2. In our example we have that $ \mathbf n = \mathbf e_z $. This discontinuity in $ \mathbf E $ is given by
\[
	[\mathbf n\cdot \mathbf E] = \frac\sigma{\varepsilon_0}
\]
where $ \sigma $ is the surface charge density and 
\[
	[X] = X_2-X_1
\]
denotes a discontinuity between regions 1 and 2.\par
The tangential components are continuous:
\[
	[\mathbf n\times \mathbf E]= 0.
\]
And these two equations apply to any surface surface even if it's curved and non-uniform.
\subsection{The electrostatic potential}
For a general $ \rho(\x) $ we cannot determine $ \mathbf E(\x) $ using Gauss' law alone.  We'll need to use the Maxwell equation $ \nabla\times \mathbf E = 0 $. This implies that $ E $ is irrotational so it has an \textit{electrostatic} potential $ \Phi(\x) $, such that
\[
  \mathbf E = - \nabla \Phi.
\]
\begin{definition}
	(Potential difference) The \textit{potential difference} or \text{voltage} between two points $ \x_1 $ and $ \x_2 $ is
	\[
		\Phi(\x_2)-\Phi(\x_1) &= \int\mathrm d\Phi \\
				      &= - \int_{\x_1}^{\x_2} \mathbf E\cdot\mathrm d\x
	\]
	and is path independent since $ \nabla\times \mathbf E = 0 $ is zero and the region is simply connected, so the field is conservative .
\end{definition}
\begin{definition}
	(Electric force) The \textit{electric force} on a particle of charge $ q $ is
	\[
	  \mathbf F = q\mathbf E = -q\nabla\Phi.
	\]
\end{definition}
\begin{remark}
  This is a conservative force associated with the potential energy
  \[
    U(\x) = q\Phi(\x).
  \]
\end{remark}
Recall that the first Maxwell equation implies that $ \Phi $ satisfies Poisson's equation, so
\[
	-\nabla^2 \Phi =\frac\rho{\varepsilon_0}.
\]
So we have the solution (from IB Methods) as (over all space with boundary conditions that $ \Phi \to 0$ as $ |\x|\to\infty $).
\[
	\Phi(\x)=\frac1{4\pi\varepsilon}\int\frac{\rho(\x')}{|\x-\x'|}\mathrm d^3\x'.
\]
This is the convolution of $ \rho(\x) $ with the potential of a unit point charge (which relates to our Green's function from IB Methods) $ \frac{1}{4\pi\varepsilon |\x|}$. Namely it is the solution to
\[
	-\nabla^2\Phi = \frac{\delta(\x)}{\varepsilon_0}
\]
satisfying $ \Phi\to 0 $ as $ |\x|\to \infty $. Note that $ \E $ is unaffected if we add an arbitrary constant to $ \Phi $ (this makes sense since $ \Phi $ measures a potential difference between two points so increasing the charge uniformly doesn't change $ \E $). We usually choose this such that $ \Phi \to 0 $ as $ |\x|\to \infty $. If $ \rho(\x) $ does not decay sufficiently rapidly this may not be possible. For example if we have a line charge $ E_r\propto \frac 1r $, so we have that $ \Phi\propto \log r $ which doesn't go to zero as $ r\to \infty $.
\subsubsection{Point charge}
The potential due to a point charge $ q $ at the origin is
\[
	\Phi(\x) = \frac q{4\pi\varepsilon_0|\x|} = \frac q{4\pi\varepsilon_0 r}.
\]
\subsubsection{Electric dipole}
Two equal and opposite charges at different positions. Without loss of generality consider charges $ -q $ t $ \x= 0 $ and $ +q $ at $ \x = \mathbf d $. The potential due to the dipole is
\[
	\Phi(\x) = \frac q{4\pi \varepsilon_0}\left(-\frac 1{|\x|} + \frac 1{|\x-\mathbf d|}\right)
\]
Apply Taylor's theorem for a scalar field,
\[
  f(\x+\mathbf h) = f(\x) + (\mathbf h \cdot\nabla)f(\x) + \frac 12 (\mathbf h \cdot \nabla)^2 f(\x) + O(||\mathbf h ||^2).
\]
So we get that
\[
	\Phi(\x) = \frac q{4\pi \varepsilon_0} \left(-\frac 1r + \frac 1r -(\mathbf d \cdot \nabla)\frac 1r + O(|\mathbf d |^2)\right) = \frac{q\mathbf d\cdot\mathbf x}{4\pi\varepsilon_0 |\x|^3}+O(|\mathbf d|^2).
\]
In the limit as $ |\mathbf d| \to 0 $ with $ q\mathbf d $ finite, we obtain a \textit{point dipole} with \textit{electric dipole moment}
\[
  \mathbf p =q\mathbf d.
\]
which has potential
\[
	\Phi(\x) = \frac{\p \cdot \x}{4\pi \varepsilon_0 |\x|^3}
\]
and electric field
\begin{align*}
	\mathbf E = -\nabla\Phi &= \frac {3(\mathbf p \cdot \x)\x-|\x|^2 \mathbf p}{4\pi\varepsilon_0 |\x|^5}.
\end{align*}
In spherical polar coordinates aligned with $ \mathbf p  = p\mathbf e_z $. So
\[
	\Phi = \frac {p\cos\theta}{4\pi\varepsilon_0 r^2}.
\]
Then we get that
\[
	E_r = -\frac{\partial \Phi}{\partial r} = \frac  {2p\cos(\theta)}{4\pi\varepsilon_0 r^3}
\]
and
\[
	E_\theta = -\frac 1r \frac{\partial \Phi}{\partial \theta} = \frac{p\sin\theta}{4\pi\varepsilon_0 r^3}.
\]
From our alignment we have that $ E_\phi = 0 $.
\begin{remark}
  Note that
  \begin{enumerate}
	  \item $ \Phi $ and $ \mathbf E $ are not spherically symmetric.
	  \item They decrease more rapidly with $ r $ than a point charge since the dipole are nearly cancelling eachother out.
  \end{enumerate}
\end{remark}
A point dipole $ \mathbf p  $ at the origin corresponds to
\[
  \rho(\x) = -\mathbf p \cdot \nabla \delta(\x),
\]
So we can find the associated potential $ \Phi $ as
\[
	\Phi(\x) = \mathbf p \cdot\nabla\left(\frac1{4\pi\varepsilon_0 |\x|}\right).
\]
\subsubsection{Field lines and equipotentials}
\textit{Electric field lines} are the integral curves of $ \mathbf E $ being tangent to $ \mathbf E $ everywhere. Since we have that $ \nabla\cdot \mathbf E = \frac \rho{\varepsilon_0} $, field lines begin on positive charges and end on negative charges. In electrostatics, $ \mathbf E = -\nabla\Phi $, so field lines are perpendicular to the equipotential surfaces of which $ \Phi $ are constant.
\subsubsection{Dipole in an external field}
Consider a dipole $ \mathbf p $ in an external field $ \mathbf E_{\text{external}} = - \nabla\Phi $ generated by distinct charges. With $ -q $ at $ \x $ and $ +q $ and $ \x+\mathbf d $, the potential energy at the dipole due to the external field is
\begin{align*}
	U &= -q\Phi(\x) +Q \Phi(\x+\mathbf d)\\
	  &= q(\mathbf d\cdot\nabla)\Phi(\x) + O(|\mathbf d|^2) \\
\end{align*}
In the limit at the point dipole, 
\[
	U=\mathbf p \cdot\nabla\Phi = -\mathbf p \cdot \mathbf E_{\text{external}}
\]
and is minimised when $ \mathbf p  $ is aligned with $ \mathbf E_{\text{external}} $.
\subsubsection{Multipole expansion}
For a general charge distribution $ \rho(\x) $ confined to a ball $ \{V:|\x|<R\} $,
\[
	\Phi(\x) = \frac 1{4\pi\varepsilon_0}\int_V \frac{\rho(\x')}{|\x-\x'|}\mathrm d^3\x'.
\]
We'll look at the external potential at $ \x $ with $ \x\notin V $. Expand 
\[
	\frac 1{|\x-\x'|}= \frac 1r - (\x'\cdot \nabla)\frac 1r + \frac 12(\x'\cdot \nabla)^2\frac 1r + O(|\x'|^3).
\]
Which is
\[
	=\frac 1r \left[1+\frac {\x'\cdot \x}{r^2} + \frac{3(\x'\cdot\x)^2 - |\x'|^2|\x|^2}{2r^4} + O\left(\frac {R^3}{r^3}\right)\right]
\]
This leads to the \textit{multipole expansion} of the potential,
\[
	\Phi(\x) = \frac 1{4\pi \varepsilon_0} \left(\frac Qr + \frac{\mathbf p\cdot\x}{r^2} + \frac 12 \frac{Q_{ij}x_ix_j}{r^5}+\cdots\right).
\]
The first three multipole moments:
\begin{enumerate}
	\item The total charge, $ Q=\int_V \rho(\mathbf x)\  \mathrm d^3 \x $.
	\item The electric dipole moment $ \mathbf p =\int_V \x \rho(\x)\ \mathrm d^3 \x $.
	\item The electric quadrupole moment. This is a second order tensor which is traceless and symmetric,
		\[
			Q_{ij} = \int_V (3x_ix_j - |\x|^2 \delta_{ij})\rho (\x)\ \mathrm d^3 \x.
		\]
\end{enumerate}
For $ \gg R $, $ \Phi $ and $ \mathbf E $ look increasingly like those of a point charge $ Q $, unless $ Q=0 $, in which case they look like those of a point dipole, unless $ \mathbf p = 0 $, etc.
\subsection{Electrostatic energy}
The work done against the electric force, $ \mathbf F = q\mathbf E $, in bringing in a particle of charge q from infinity (where we assume that $ \Phi=0 $ at infinity) to $ \x $ is 
\[
	-\int_{\infty}^{\x} \mathbf F \cdot\mathrm d\x = +q\int_\infty^{\x} \nabla\Phi\cdot\mathrm d\x = q\Phi(\x).
\]
Consider assembling a confriguration of $ N $ point charges one by one. Particle $ i $ of charge $ q_i $ is brought from $ \infty $ to $ \x_i $ while the previous particles remain fixed. For the first particle no work is involved, $ W_1 = 0 $. For the second particle
\[
	W_2 = q_2\left(\frac{q_1}{4\pi\varepsilon_0 |\x_2-\x_1|}\right)
\]
and for the third particle
\[
	W_3 = q_3\left(\frac{q_1}{4\pi\varepsilon_0 |\x_3-\x_1|} + \frac{q_2}{4\pi\varepsilon_0 |\x_3-\x_2|}\right)
\]
So the total work done is
\[
	U= \sum_{i=1}^N W_i = \sum_{i=2}^N\sum_{j=1}^{i-1} \frac{q_i q_j}{4\pi\varepsilon_0 |\x_i-\x_j|}.
\]
This can be rewritten as
\[
U=	\frac 12 \sum_{i=1}^N\sum_{j\ne i} \frac{q_iq_j}{4\pi \varepsilon_0|\x_i-\x_j|}.
\]
or
\[
	U = \frac 12 \sum_{i=1}^N q_i\Phi(\x_i).
\]
We can generalise to a continuous charge distribution $ \rho(\x) $ occupying a finite volume $ V $.
\[
  U= \frac 12 \int_V\rho(\x)\Phi(\x)\mathrm d^3 \x.
\]
Using the first Maxwell equation we ge that
\begin{align*}
	U &= \frac 12 \int_V(\varepsilon_0 \grad\cdot \mathbf E)\Phi\mathrm dV \\
	  &= \frac{\varepsilon_0}2 \int_V\left(\nabla\cdot(\Phi\mathbf E) - \mathbf E \cdot \nabla \Phi\right)\mathrm dV\\
	  &= \frac{\varepsilon_0}2 \int_S \Phi \mathbf E \cdot\mathrm d\mathbf S + \int_V \frac{\varepsilon_0 |\mathbf E|^2}2 \mathrm dV.
\end{align*}
Let $ S=\partial V $ be a sphere of radius $ R\to\infty $. Then $ \Phi= O(\inv R) $ and $ \mathbf E = O(R^{-2}) $ on $ S $ while the area of $ S $ is $ O(R^2) $, so $ \int_S $ is $ O(\inv R) $ and $ \to 0 $ as $ R\to\infty $. Then
\[
	U = \int\frac{\varepsilon_0 |\mathbf E|^2}2 \mathrm dV
\]
where the integral is taken over all space, not just the volume where the charges are contained.
\begin{remark}
  This implies that energy is stored is the electric field, even in a vacuum.
\end{remark}
Any of expression for $ U $ suggests that the self-energy of a point charge is infinite, hence for $ U $ to be useful, we discard all self-energies since it is unchanging and causes no force.
\subsection{Conductors} 
In a \textit{conductor} such as a metal, some charges can move freely. In electrostatics we require
\[
	\mathbf E = 0, \qquad \Phi=\text{constant}
\]
inside a conductor, hence $ \rho = 0 $. Otherwise free charges would move in a response to the electric force and a current would flow.\par
However a surface charge density $ \sigma $ can exist on the surface of a conductor, which is an equipotential.\par
Taking $ \mathbf n $ to point out of the conductor, the condition,
\[
	\mathbf n \cdot\mathbf E = \frac\sigma{\varepsilon_0}
\]
becomes
\[
	\mathbf n \cdot\mathbf E = \frac{\sigma}{\varepsilon_0}\quad\text{ immediately outside the conductor}.
\]
The constant potential of a conductor can be set by connecting it to a battery or another conductor.
\begin{definition}
	(Earthed/Grounded conductor) An \textit{earthed} or \textit{grounded} conductor is connected to the ground, usually taken as $ \Phi = 0 $.
\end{definition}
To find $ \Phi(\x) $ and $ \mathbf E(\x) $ due to the charge distribution $ \rho(\x) $ in the presence of conductors with surface $ S_i $ and potentials $ \Phi_i $ we solve Poisson's equation
\[
	-\nabla^2 \Phi = \frac\rho{\varepsilon_0}
\]
with Dirichlet boundary conditions
\[
	\Phi = \Phi_i\quad\text{on } S_i.
\]
The solution depends linearly on $ \rho $ and $ \{\Phi_i\} $.\par
Let's see an example. Take a point charge $ q $ at position $ (0,0,h) $ in a half space $ (z>0) $ bounded by an earthed conducting wall. Hence we have the boundary condition $ \Phi = 0 $ on $ z=0 $. By the method of images, the solution in $ z>0 $ is identical to that of a dipole, with image charge $ -q $ placed at $ (0,0,-h) $. The wall coincides with an equipotential of the dipole, namely the line with $ \Phi = 0 $ which is the same as our boundary condition. The induced surface charge density on the wall can be worked out from
\[
	\frac\sigma{\varepsilon_0} = \mathbf n\cdot\mathbf E = E_z= -\frac{2qh}{4\pi\varepsilon_0(r^2+h^2)^{3/2}}.
\]
The total induced surface charge is
\begin{align*}
	\int_0^\infty \sigma\ 2\pi r \ \mathrm dr &= -qh \int_0^\infty \frac{r\ \mathrm dr}{(r^2+h^2)^{3/2}}\\
	&= -q
\end{align*}
which is equal to the image charge.
\begin{definition}
	(Capacitor) A simple \textit{capacitor} constants of two seperated conductors carrying charges $ \pm Q $. If the potential difference between them is $ V $, then the capacitance is defined by
	\[
	  C = \frac QV
	\]
	and depends only on the geometry, because $ \Phi $ depends linearly on $ Q $.
\end{definition}
For example, consider two infinite parallel plates seperated by some distance $ d $. Let the plate surfaces at $ z=0, z=d $ have surface charge densities $ \pm \sigma $. Then $ \mathbf E = E \mathbf e_z $ with $ E= \sigma/\varepsilon_0  = \text{const.}$ for $ 0<z<d $, with $ \mathbf E = 0 $ elsewhere. So
\[
	\Phi = -Ez + \text{const},\quad\text{and}\quad V = Ed.
\]
The same solution holds approximately for parallel plates of $ A\gg d^2 $ if end effects are neglected. So 
\[
	C= \frac QV\approx \frac{\sigma A} {Ed}\approx \frac{\varepsilon_0 A}d.
\]
The electrostatic energy stored in a capacitor is
\begin{align*}
	U &= \int \frac{\varepsilon_0 |\mathbf E|^2}2 \mathrm dV\\
	  &\approx \frac{\varepsilon_0 E^2}2 Ad\\
	  &\approx \frac 12 C V^2.
\end{align*}
In general,
\[
	U = \frac 12 CV^2 = \frac {Q^2}{2C}.
\]
The work done in moving an element of charge $ \delta Q $ from one plate to another is $ \delta W = V\ \delta Q $, so the total work done is
\begin{align*}
	\int_0^Q \frac{Q'}C \mathrm Q' = \frac{Q^2}{2C}
\end{align*}
for any geometry of the capactor. Or we can use
\begin{align*}
	U &= \frac 12 \int \rho \Phi \mathrm dV\\
	  &= \frac 12 Q \Phi_+ - \frac 12 Q \Phi_-\\
	  &= \frac 12 QV.
\end{align*}
\section{Magnetostatics}
\textit{Magnetostatics} is the study of the magnetic field generated by a stationary current distribution.
\begin{align*}
	\nabla\times \mathbf B &= \mu_0 \mathbf J\\
	\nabla\cdot\mathbf B = 0,
\end{align*}
and the first equation here implies that $ \nabla\cdot\mathbf J = 0 $, the time-independent equation of charge conservation.
\subsection{Ampere's law}
Consider a closed curve $ C $ that is the boundary of an open surface $ S $. Integrate the fourth Maxwell equation over $ S $ and apply Stokes' theorem to obtain Ampere's law.
\[
  \int_C \mathbff B\cdot\mathrm d\x = \mu_0 I
\]
where 
\[
  I = \int_S\mathbf J \cdot\mathrm d\mathbf S.
\]
$ I $ is the total current through the surface $ S $. Since $ \nabla\cdot \mathbf J = 0 $, the same current $ I $ flows through \textit{any} open surface $ S $ such that $ \partial S = C $. Ampere's law is the integral version of the fourth Maxwell equation and is valid provided that $ \frac{\partial \mathbf E}{\partial t} = 0 $. Ampere's law is saying that
\[
	\text{circulation of magnetic field around loop}\ \propto\ \text{total current through loop}.
\]
In special situations we can use Ampere's law together with symmetry to deudce $ \mathbf B $ from $ \mathbf J $.\par
A cylindrically symmetric situation could involve
\begin{itemize}
	\item An axial current distribution
		\[
		  J_z(r) \mathbf e_z.
		\]
	\item An azimuthal current distribution
		\[
		  J_\phi(r)\mathbf e_\phi
		\]
\end{itemize}
or a combination. (In fact $ \nabla\cdot\mathbf J = 0 $ excludes a radial current.)\par
The same applies to $ \mathbf B $. The curl in the fourth Maxwell equation implies that $ B_\phi $ is lineraly related to $ J_z $ and $ B_z $ is linearly related to $ J_\phi $.
Let's consider a cylindrical wire of radius $ R $ which carries a total current $ I $ parallel to its axis. To find $ B_\phi(r) $ generated by $ J_z(r) $, apply Ampere's law to a circle $ C $ of radius $ R $.\par
If $ r>R $ then
\begin{align*}
	\int_C\mathbf B\cdot\mathrm d\x &= B_\phi(r) \int_C \mathbf e_\phi\cdot\mathrm  d\x\\
  &= B_\phi(r)\int_C\mathrm d\ell\\
  &= B_\phi(r) 2\pi r = \mu_0 I.
\end{align*}
Thus, outside the wire,
\[
	\mathbf B = \frac{\mu_0 I}{2\pi r} \mathbf e_\phi.
\]
Now for a solenoid. A thin wire is coiled around a cylindrical tube of radius $ R $. An \textit{ideal solenoid} is infintely long and tightly wound, having cylindrical symmetry and purely azimuthal current. The wire carries current $ I $ and has $ N $ turns per unit length of the tube.
\par
To find $ B_z(r) $ generated by $ J_\phi(r) $, apply Ampere's law to a recntagular loop $ C $. Taking $ a<b<R $ or $ R<a<b $ gives that
\[
  L(B_z(a)-B_z(b)) = 0.
\]
and taking $ a<R<b $ gives
\[
  L(B_z(a)-B_z(b)) = \mu_0 NLI.
\]
Asumming that $ B_z(r)\to 0 $ as $ r\to\infty $, we deduce that
\[
 B_z(r) = \begin{cases}
	 \mu_0 NI & r< R \\
	 0 & r> R
 \end{cases}   
\]
The ideal solenoid is an example of a surfac current, here of the form
\[
  J_\phi(r) = K_\phi \delta(r-R)
\]
with $ K_\phi = NI $.\par
Generally, a surface current density, $ \mathbf K $, produces a discontinuity in the tangential magnetic field:
\[
	[\mathbf n \times \mathbf B] = \mu_0 \mathbf K.
\]
It follows from Ampere's law that the norm component is continuous, i.e.
\[
	[\mathbf n \cdot\mathbf B] = 0.
\]
\subsection{The magnetic vector potential}
The second Maxwell equation implies that $ \mathbf B $ can be written in terms of a \textit{magnetic vector potential}.
\begin{definition}
	(Magnetic vector potential) For a magnetic field $ \mathbf B $, the \textit{magnetic vector potential} is the vector $ \mathbf A(\x) $ such that
	\[
	  \mathbf B = \nabla\times \mathbf A.
	\]
\end{definition}
\begin{remark}
	$ \mathbf A $ is \textit{not} unique. If we make a \textit{gauge transformation}, replacing $ \mathbf A $ with 
	\[
		\tilde{\mathbf A} = \mathbf A +\nabla\chi,
	\]
	where $ \chi(\x) $ is an arbitrary scalar field, then $ \mathbf B $ is unchanged since $ \mathbf B = \nabla\times \mathbf A = \nabla\time\times \tilde{\mathbf A} $. \par
	A convenient guage for many calculations is the \textit{Coulomb gauge} in which $ \nabla\cdot \mathbf A = 0 $. We can assume this condition \textit{wlog}. If $ \nabla\cdot \nabla\ne 0 $, then we can make a gauge transformation such that $ \nabla\cdot\tilde{\mathbf A} = 0 $ by choosing $ \chi $ to be the solution of Poisson's equation
	\[
	  -\nabla^2 \chi = \nabla\cdot\mathbf A.
	\]
\end{remark}
In terms of $ \mathbf A $ the fourth Maxwell equation becomes
\[
  \nabla\times(\nabla \times\mathbf A ) = \mu_0 \mathbf J.
\]
Using the identity,
\[
  \nabla\times(\nabla\times \mathbf A) = \nabla(\nabla\cdot \mathbf A) - \nabla^2 \mathbf A
\]
and assuming Coulomb gauge with $ \nabla\cdot\mathbf A =0 $, we obtain Poisson' equation in vector form,
\[
  -\nabla^2 \mathbf A = \mu_0 \mathbf J
\]
\subsection{The Biot-Savart law}
The solution of Poisson's equation is
\[
	\mathbf A(\x) = \frac{\mu_0}{4\pi} \int\frac{\mathbf J(\x')}{|\x - \x'|} \mathrm d^3 \x'.
\]
We should check that the solution satisfies the assumed Coluomb gauge condition
\begin{align*}
	\nabla\cdot\mathbf A(\x)&=\frac{\mu_0}{4\pi}\int_V\nabla\cdot\left\frac{\mathbf J(\x')}{|\x-\x'|}\mathrm d^3\x'\\
				&=\frac{\mu_0}{4\pi}\int_V\mathbf J(\x')\cdot\nabla\left(\frac 1{|\x-\x'|}\right)\mathrm d^3\x'\\
				&=-\frac{\mu_0}{4\pi}\int_V\mathbf J(\x')\cdot\nabla'\left(\frac 1{|\x-\x'|}\right)\mathrm d^3\x'\\
				&=-\frac{\mu_0}{4\pi}\int_V\nabla'\cdot\left(\frac {\mathbf J(\x')}{|\x-\x'|}\right)\mathrm d^3\x'\\
				&=-\frac{\mu_0}{4\pi}\int_{\partial V}\frac{\mathrm J(\x')\cdot\mathrm d\mathbf S'}{|\x-\x'|}.
\end{align*}
Thus $ \nabla\cdot\mathbf A=0 $, as assumed. if the current is contained in some finite volume and we take $ V $ to be at least as alarge, or if $ \mathbf J $ decays sufficiently as $ |\x|\to \infty $.\par
To find the magnetic field, derive $ \mathbf B = \nabla\times \mathbf A $. This gives the following law.
\begin{theorem}
	(Biot-Savart law)
\begin{align*}
	\mathbf B(\x)=\frac{\mu_0}{4\pi}\int\frac{\mathbf J(\x')\times (\x-\x')}{|\x-\x'|^3}\mathrm d^3\x'
\end{align*}
\end{theorem}
A special case is when the current is restricted to a thin wire in the form of a curve $ C $. Then the current element $ \mathbf J \mathrm d^3 \x $ can be replaced by $ I\mathrm d\x $. By charge conservation, we get that $ I $ is constant along the wire, hence we can take it outside of the integral, so the Biot-Savart law for a current carrying wire is
\[
	\mathbf B(\x) = \frac{\mu_0 I}{4\pi} \int_C\frac{\mathrm d\x'\times (\x-\x')}{|\x-\x'|^3}.
\]
Alternatively, the thin wire current density can be represented as
\[
  \mathbf J(\x) = I\int_C\delta(\x-\x')\mathrm d^3\x',
\]
which gives the same equation if substituted in. Charge conservation takes the form
\begin{align*}
	\nabla\cdot\mathbf J(\x) &= I\int_C \nabla\delta(\x-\x')\cdot\mathrm d\x'\\
				 &= -I\int-C\nabla \delta(\x-\x')\cdot\mathrm d\x'\\
				 &= -I[\delta(\x-\x_2)-\delta(\x-\x_1)]
\end{align*}
where $ C $ runs from $ \x_1 $ to $ \x_2 $. If $ C $ is closed then $ \x_2=\x_1 $ and $ \nabla\cdot\mathrm J=0 $ as expected. If $ C $ is infinite then $ \nabla\cdot\mathrm J =0  $ for any finite $ \x $. Let's check the thin-wire version of Biot-Savart's law gives the same result as Ampere's law for a long straight thin wire along the $ z $ azis of cylindrical polars. We have that $ \x = r\mathbf e_r $ (taking $ z=0 $ \textit{wlog}) and $ \x'=z'\mathbf e_z $. So $ \x-\x'=r\mathbf e_r-z'\mathbf e_z $. We also have that $ \mathrm d\x'=\mathrm z'\mathbf e_z $. This gives that 
\[
	\mathbf B(\x) = \frac{\mu_0 I}{4\pi}\mathbf e_\phi\int_{-\infty}^\infty\frac{r\mathrm dz'}{(r^2+z^2')^{3/2}}=\frac{\mu_0 I}{2\pi r}\mathbf e_\phi
\]
as expected.
\subsection{Magnetic dipoles}
For a general current distribution $ \mathbf J(\x) $ confined to a ball $ \{V:|\x|<R\} $,
\[
	\mathbf A(\x) = \frac{\mu_0}{4\pi} \int_V\frac{\mathbf J(\x')}{|\x-\x'|}\mathrm d^3\x'.
\]
The external field for $ |\x|=r>R $ can be evaluated by expanding
\[
	\frac 1{|\x-\x'|}=\frac 1r\left(1+\frac{\x'\cdot\x}{r^2} + O\left(\frac{R^2}{r^2}\right)\right),
\]
leading to a multipole expansion as before. We need to calculate the moments of the current distribution.
\par
Since $ \mathbf J = 0 $, on $ \partial V $ and $ \nabla\cdot\mathbf J = 0 $ the divergence theorem implies that
\begin{align*}
	0 = \int_{\partial V} x_iJ_j dS_j &= \int_V \partial_j (x_iJ_j)\mathrm d^3\x\\
						  &= \int_V(\delta_{ij} J_j +x_i \partial_j J_j)\mathrm d^2\x\\
						  &=\int_V J_i\mathrm d^3\x
\end{align*}
so the zeroth moment vanishes. Similarly,
\begin{align*}
	0 = \int_{\partial V} x_ix_jJ_k\mathrm dS_k &= \int_V\partial_k(x_ix_jJ_k)\mathrm d^3 \x\\
						    &=\int_V(\delta_{ik}x_jJ_k+x_i\delta_{jk}J_k+x_ix_j\partial_kJ_k)\mathrm d^3\x\\
						    &= \int_V x_jJ_i\mathrm d^3\x+\int_V x_iJ_i\mathrm d^3\x.
\end{align*}
So the first moment is an antisymmetric tensor. Like the electric dipole moment we define:
\begin{definition}
	(Magnetic dipole moment)
	\[
	  \mathbf m = \frac 12\int_V \x\times \mathbf J \mathrm d^3 \x
	\]
	or in component form
	\[
		m_i = \frac 12\varepsilon_{ijk}\int_V x_jJ_k\mathrm d^3 \x.
	\]
\end{definition}
This is a vector related  to the antisymmetric matrix by
\[
	\int_V x_iJ_j\mathrm d^3 \x = \varepsilon_{ijk} m_k.
\]
Returning to the multipole expansion for $ \mathbf A $, we have
\[
	A_i(\x)=\frac{\mu_0}{4\pi |\x|}\left(\int_V J_i(\x')\mathrm d^3\x'+\dots \right)
\]
we know the first term is zero so
\[
\mathbf A(\x) \approx \mathbf A_{\text{dipole}} (\x) = \frac{\mu_0}{4\pi}\frac{\mathbf m \times \x}{|\x^3|}
\]
which is the vector potential due to a point dipole $ \mathbf m $ at the origin. The corresponding magnetic field is
\[\mathbf B_{\text{dipole}} = \nabla\times\mathbf A_{\text{dipole}} = \frac{\mu_0}{4\pi}\left(\frac{3(\mathbf m\cdot\x)\x-|\x|^2\mathbf m}{|\x|^5}\right). \]
A point dipole $ \mathbf m $ at the origin corresponds to the current density and vector potential
\[
	\mathbf J = \nabla\times(\mathbf m\delta(\x)),\qquad \mathbf A = \nabla\times\left(\frac{\mu_0 \mathbf m}{4\pi |\x|}\right).
\]
The magnetic dipole moment of a thin wire carrying current $ I $ around a closed curve $ C $ is
\[
	\mathbf m = \frac{I}2 \int_C \x\times \mathrm d\x.
\]
To evaulated this, let $ \mathbf a $ be any constant vector. Then by Stokes' theorem,
\begin{align*}
	\mathbf a \cdot\mathrm m &= \frac I2\int_C\mathbf a\cdot(\x\times \mathrm d\x)\\
  &= \frac I2\int_C(\mathrm a \times \x)\cdot\mathrm d\x\\
  &= \frac I2 \int_S(\nabla\times (\mathrm a \time \x))\cdot\mathrm d\mathbf S\\
  &= I\int_S \mathbf a\cdot\mathrm d\mathbf S
\end{align*}
where $ S $ is an open surface with boundary $ C $ and we use
\[
  \nabla\times(\mathbf a \times \x) = \x\cdot \nabla\mathbf a - \mathbf a \cdot\nabla \x + (\nabla\cdot\x)\mathbf a - (\nabla\cdot\mathbf a)\x = 2\mathbf a.
\]
Since $ \mathbf a $ is arbitrary we conclude that
\[
  \mathbf m = I\mathbf S
\]
where
\[
  \mathbf S = \int_S\mathrm d\mathbf S
\]
is the vector area of the surface $ S $ which depends only on $ C $.
\par
The simplest example is a circular loop, for example $ x^2+y^2 = a^2 $ for which $ \mathbf m = I \pi a^2 \mathbf e_z $. On the $ z $-axis the dipole approximation gives that
\[
	B_z = \frac{\mu_0}{4\pi}\left(\frac{3m_zz^2 - z^2m_z}{|z|^5}\right) = \frac{\mu_0 Ia^2}{2|z|^3}
\]
while the exact solution is
\[
	B_z = \frac{\mu_0 Ia^2}{2(z^2+a^2)^{3/2}}.
\]
Magnetic field lines are the integral curves of $ \mathbf B $. Since $ \nabla\cdot\mathbf B = 0 $, they are continuous.
\begin{definition}
	(Permanent magnets) A bar magnet has north and south poles and a dipole moment. This comes from the superposition of aligned dipoles on the atomic scale. Atoms contain electrons whihc are spinning charges particles with a magnetic dipole moment.
\end{definition}
A classical model of a particle is a spinning charged sphere which is a current loop with a magnetic dipole moment proportional to its charge and spin. As far as we know, there is no magnetic charges (monopoles).\par
The liquid iron outer core of the Earth is a conducting fluid in convective motion and supports electric currents that generate a magnetic field. At the Earth's surface this resembles a dipole field.
\subsection{Magnetic forces}
Recall that the Lorentz force on a particle of charge $ q_i $ at position $ \x_i(t) $ is
\[
  \mathbf F = q_i(\mathbf E+\dot\x_i\times \mathbf B)
\]
where both $ \mathbf E $ and $ \mathbf B $ are evaluated at $ \x_i(t) $.\par
In the limit of continuous charge and current densities, the Lorentz force per unit volume is then
\[
  \rho\mathbf E  + \mathbf J\times \mathbf B.
\]
We can recover the discrete version of the Lorentz force by substituting
\begin{align*}
	\rho &= \sum_i q_i\delta(\x-\x_i(t))\\
	\mathbf J &= \sum_i q_i\dot\x_i(t)\delta(\x-\x_i(t)).
\end{align*}
Let's look at the force between thin wires. Consider two or more thin wires with current $ I_i $ along a curve $ C_i $. The total magnetic field is $ \mathbf B = \sum_i \mathbf B_i $, where
\[
	\mathbf B_i(\x)= \frac{\mu_0 I_i}{4\pi} \int_{C_i} \frac{\mathrm d\x_i \times (\x-\x_i)}{|\x-\x_i|^3}
\]
is the magnetic field due to wire $ i $. The current density if $ \mathbf J = \sum_i \mathbf J_i $, where
\[
	\mathbf J_i(\x) = I_i\int_{C_i} \delta(\x-\x_i)\mathrm d\x_i.
\]
The total magnetic field acting on a volume $ V $ is
\[
  \mathbf F = \int_V \mathbf J\times \mathbf B \mathrm dV.
\]
The force acting on wire $ i $ is
\begin{align*}
	\mathbf F_i &= \int \mathbf J_i(\x)\times\mathbf B(\x)\mathrm d^3\x\\
		    &= I_i\int_{C_i}\mathrm d\x_i\times \mathbf B(\x_i).
\end{align*}
Since $ \mathbf B = \sum_i \mathbf B_i $, we have that
\[
	\mathbf F_i =\sum_j \mathbf F_{ij}
\]
where
\[
	\mathbf F_{ij} = I_i\int_{C_i} \mathrm d\x_i \times \mathbf B_j(\x_i)
\]
is the force on wire $ i $ due to wire $ j $. Using the Biot-Savart law,
\[
	\mathbf F_{ij} = \frac{\mu_0 I_iI_j}{4\pi} \int_{C_i}\int_{C_j}\mathrm d\x_i \times\left(\frac{\mathrm d\x_j\times(\x_i-\x_j)}{|\x_i-\x_j|^3}\right).
\]
This can be rewritten (Example Sheet 3) in a manifestly antisymmetric way that shows that
\[
	\mathbf F_{ji} = -\mathbf F_{ij}
\]
as excepected from Newton's third law. The self force $ \mathbf F_{ii} $ vanishes, although the thin-wire integral is singular when $ i=j $ and it is better treat the case of thick wires.
\par
Let's consider two infinitely long parallel thin wires seperated by a distance $ r $. Use cylindrical polars centred on the second wire. We have that $ \mathbf B_2 = \frac{\mu_0 I_2}{2\pi r}\mathbf e_\phi $.
\[
	\mathbf F_{12} = I_1\int_{-\infty}^\infty \mathrm dz \mathbf e_z \times\mathbf B_2.
\]
The total force is infinite. The force per unit length is
\[
	I_1\mathbf e_z\times\mathbf B_2 = -\frac{\mu_0 I_1I_2}{2\pi r}\mathbf e_r.
\]
This is directed towards wire 2 is $ I_1I_2>0 $ so the force is attractive if the currents are aligned and repulsive otherwise.
\par
Let's consider the force and torque on a magnetic dipole. Consider a localised current distribution confined to a ball $ \{V:|\x|<R\} $. Place this in an external magnetic field $ \mathbf B(\x) $ that varies slowly over the length scale $ R $. the magnetic torque about the origin on the current loop is
\begin{align*}
	\boldsymbol\tau &= \int_V\x \times(\mathbf J(\x) \times \mathbf B(\x)\mathrm d^3\x\\
			&= \int_V ((\x\cdot \mathbf B(\x))\mathbf J(\x)-(\x\cdot \J(\x))\mathbf B(\x))\mathrm d^3\x.
\end{align*}
within $ V $, $ \mathbf B(\x) $ can be expanded as a Taylor series,
\begin{align*}
  \mathbf B_i(\x) = \mathbf B_i(0) +x_j \partial_jB_i(0) + \cdots 
\end{align*}
retaining only the zero order term (uniform field), we have that
\[
  \tau_i \approx B_j(0)\int_V x_j J_i\mathrm d^3\x - B_i(0)\int_V x_jJ_j\mathrm d^3\x.
\]
Recall the first moments of the current distribution:
\begin{align*}
	\int_V x_i J_j\mathrm d^3 \x = \varepsilon_{ijk} m_k.
\end{align*}
Thus
\[
	\tau_i \approx B_j(0)\varepsilon_{jik}m_k.
\]
In general,
\[
  \boldsymbol \tau \approx \mathbf m \times \mathbf B,
\]
where $ \mathbf B $ is evaulated at the dipole's location and $ \boldsymbol \tau $ is measured about this point.
\par
For the force, we need to go to the first order of the Taylor expansion of $ \mathbf B $.
\[
  \mathbf F = \int_V \mathbf J(\x) \times\mathbf B(\x)\mathrm d^3 \x
\]
and in components
\begin{align*}
	F_i &\approx \int_V \varepsilon_{ijk} J_j(\x)(B_k(0) + x_\ell \partial_\ell B_k(0))\mathrm d^3 \x\\
	    &= 0 + \varepsilon_{ijk} \partial_\ell B_k(0)\varepsilon_{\ell jn}m_n\\
	    &= \partial_i B_k(0) m_k - \partial_k B_k(0)m_i\\
	    &= \partial_i(m_kB_k)(0)\qquad\qquad \text{since } \nabla\cdot\mathbf B = 0\\
\end{align*}
In general, $ \mathbf F \approx \nabla(\mathbf m \cdot\mathbf B) $. Which can be written as $ \mathbf F = - \nabla U $, where
\[
  U = -\mathbf m \cdot\mathbf B
\]
is the potential energy of a magnetic dipole in an external field. As in the electric case this is minimised when $ \mathbf m $ is aligned with $ \mathbf B $.
\section{Electrodynamics}
\subsection{Faraday's law of induction}
We're interested in the third Maxwell equation,
\[
	\nabla\times\mathbf E = - \frac{\partial \mathbf B}{\partial t}
\]
which implies that a time-dependent magnetic field, must be accompanied by an electric field. This can induce a current to flow in a conductor. This is called \textit{electromagnetic induction}.
\par
First let's look at a static circuit. Consider a closed curve $ C $ that is the boundary of a time-independent open surface $ S $. Integrate the third Maxwell equation over $ S $ and use Stokes' theorem.
\begin{align*}
	\int_C \mathbf E\cdot\mathrm d\x &= -\int_S \frac{\partial \mathbf B}{\partial t}\cdot\mathrm d \mathbf S \\
					 &= \frac d{dt} \int_S \mathbf B\cdot\mathrm d\mathbf S
\end{align*}

This is Faraday's law of induction for a static circuit.
\[
	\mathcal E = - \frac{d\mathcal F}{dt}.
\]
where
\[
  \mathcal E = \int_C \mathbf E\cdot\mathrm d\x
\]
is the \textit{electromotive force} (emf) around $ C $ and
\[
  \mathcal F = \int_S \mathbf B\cdot\mathrm d\mathbf S
\]
is the \textit{magnetic flux} through $ S $.\par
Since $ \naabla\cdot\mathbf B = 0 $, the flux $ \mathcal F $ is the same through any surface $ S $ such that $ \partial S=C $, so it can be regarded as the magnetic flux through $ C $.
\par
Using $ \mathbf B = \nabla\times \mathbf A $ and Stokes' theorem we can write $ \mathcal F = \int_C\mathbf A\cdot\mathrm d \x $, which is invariant under a gauge transformation. The emf is not actually a force. it is the line integral of the Lorentz force on a particle of unit charge confined to $ C $.
\[
	\mathcal E = \frac 1q \int_C\mathbf F \cdot\mathrm \x = \int_C(\mathbf E + \dot{\x} \times \mathbf B) \cdot\mathrm d\x = \int_C \mathbf E \cdot\mathrm d\x
\]
since $ \dot{\x} $ is tangent to $ C $ for a particle confied to a time-independent curve $ C $. We will see later that if $ C $ coincides with a thin wire of resistance $ R $, then the current induced in the wire is $ I = \mathcal E /R $.
\par
There are several ways in which the magnetic through $ C $ could change with time.
\begin{itemize}
	\item A magnet is moved near $ C $;
	\item A current-carrying circuit is moved near $ C $;
	\item The current in a nearby circuit is changed.
\end{itemize}
All these will induce an emf around $ C $ and cause a current to flow.
\par
Now let's look at Faraday's law for a moving circuit. let $ C(t) $ be time-dependent closed curve that is the boundary of an open surface $ S(t) $. How does the magnetic flux through $ S $,
\[
  \mathcal F = \int_S \mathbf B \cdot\mathrm d\mathbf S
\]
change in time? We have 
\begin{align*}
	\mathcal F(t+\delta t) - \mathcal F(t) &= \int_{S(t+\delta t)} \mathbf B(\x,t+\delta t) \cdot\mathrm d\mathbf S - \int_{S(t)} \mathbf B(\x,t)\cdot\mathrm d\mathbf S\\
					       &= \int_{S(t+\delta t)}\left(\mathbf B(\x,t)+\frac{\partial \mathbf B}{\partial t}\delta t +O(\delta t^2)\right)\cdot\mathrm d\mathbf S - \int_{S(t)}\mathbf B(\x,t)\cdot\mathrm d\mathbf S\\
					       &= \int_{S(t+\delta t)-S(t)}\mathbf B(\x,t)\cdot\mathrm d\mathbf S + \int_{S(t)} \frac{\partial \mathbf B}{\partial t}\cdot\mathrm d\mathbf S \delta t + O(\delta t^2)
\end{align*}
Let $ \delta V $ be the volume swept out by $ S(t) $ in the time interval $ \delta t $. Its boundary is the closed surface $ S(t+\delta t) - S(t) + \Sigma $, where $ \Sigma $ is the surface swept out by $ C(t) $ in the time $ \delta t $.
\par
Looking at a line element $ \mathrm d \x $ on $ C(t) $ which sweeps out a surface of height $ \mathbf v \delta t $ with normal $ \mathrm d\mathbf S $. By the second Maxwell equation and the divergence theorem,
\begin{align*}
   0 &= \int_{\delta V} (\nabla\cdot\mathbf B)\mathrm dV \\
     &= \int_{S(t+\delta t)-S(t)} \mathbf B\cdot\mathrm d\mathbf S + \int_\Sigma \mathbf B \cdot\mathrm d \mathbf S.
\end{align*}
To evaluate the last term, parammetrise $ C $ as $ \x = \x(\lambda,t) $, where $ \lambda $ is a parameter around $ C $. An element of $ C $ is
\begin{align*}
	\mathrm d\x = \frac{\partial \x} {\partial \lambda}\mathrm d\lambda
\end{align*}
and has velocity
\[
	\mathbf v = \frac{\partial \x}{\partial t}.
\]
In time $ \delta t $ it sweeps out the vector area element
\[
  \mathrm d \mathbf S = \mathrm d \x \times (\mathbf v \delta t)
\]
which points out of $ \delta V $ as we require. Thus 
\[
  \int_\sigma \mathbf B\cdot\mathrm d\mathbf S = \int_C \mathbf B \cdot (\mathrm d\x \times \mathbf v)\ \delta t +O(\delta t^2). 
\]
We then have
\[
	\mathcal F(t+\delta t) - \mathcal F(t) = -\int_C (\mathbf v \times \mathbf B)\cdot\mathrm d\x \ \delta t +\int_S \frac{\partial \mathbf B}{\partial t} \cdot\mathrm d \mathbf S \ \delta t + O(\delta t^2)
\]
from which
\begin{align*}
	\frac{d\mathcal F}{dt} &= -\int_C (\mathbf v \times \mathbf B)\cdot\mathrm d \x + \int_S \frac{\partial \mathbf B}{\partial t}\cdot\mathrm d \mathbf S\\
			       &= -\int_C(\mathbf v\times \mathbf B)\cdot\mathrm d\x - \int_S (\nabla\times \mathbf E) \cdot\mathrm d \mathbf S\\
			       &= -\int (\mathbf E +\mathbf v \times \mathbf B)\cdot\mathrm d\x.
\end{align*}
We recover Faraday's law,
\[
	\mathcal E = -\frac{d\mathcal F}{dt}
\]
with the redefined emf
\[
  \mathcal E = \int_C (\mathbf E +\mathbf v \times \mathbf B)\cdot \mathrm d \x.
\]
This is again the line integral around $ C $ of the Lorentz force on a particle of unit charge confined to $ C $. (For which the perpendicular components of $ \dot{\x} $ must agree with those of the curve velocity $ \mathbf v $.)
\begin{law}
	(Lenz's law) The direction of the induced current is always such as to produce a magnetic field that opposes the change in flux that causes the emf.
\end{law}
\begin{example}
  A circular wire in the $ x-y $-plane. If $ B_z $ inside the loop increases in time, then
  \[
	  \mathcal E = - \frac{d\mathcal F}{dt} < 0.
  \]
  This induces a clockwise current $ (I<0) $ that generates a magnetic field $ B_z< 0 $ inside the loop. Hence the minus sign in Faraday's lwa. This avoids an unstable situtation in which the flux grows indefinitely.
\end{example}
\begin{definition}
	(Inductance) If a current $ I $ around a circuit $ C $ generates a magnetic field with flux $ \mathcal F $ then the \textit{inductance} of the circuit is defined by
	\[
		L = \frac {\mathcal F}I
	\]
	and depends only on the geometry.
\end{definition}
\begin{example}
	An ideal solenoid with cross-sectional area $ A $ and $ N $ turns per unit length. The uniform fluid $ B= \mu_0 NI $ inside the solenoid produces a flux $ BA $ per turn, so the inductance per unit length of the solenoid is $ \mu_0 N^2 A $.
\end{example}
\begin{exercise}
	Show that the magnetic flux through a thin wire $ C_i $ due to a current $ I_j $ around another thin wire $ C_j $ is $ \mathcal F_{ij} = L_{ij}I_j $ where the \textit{mutual inductance} is
	\[
		L_{ij} = \frac{\mu_0}{4\pi} \int_{C_i}\int_{C_j}\frac{\mathrm d\x_i \cdot\mathrm \x_j}{|\x_i-\x_j|} = L_{ji}.
	\]
\end{exercise}
When the current $ I $ around a circuit $ C $ is varies, an emf
\[
	\mathcal E = - \frac{d\mathcal F}{dt} = -L\frac{dI}{dt}
\]
is induced. In a small time interval $ \delta t $, a charge $ \delta Q  = I\ \delta t $ flows around $ C $ and the work done on it by the Lorentz force is 
\[
	\delta W = \mathcal E \ \delta Q = -LI \frac{dI}{dt} \ \delta t.
\]
So the rate at which work is done by the current on the EM field is
\[
	-\frac{dW}{dt} = LI \frac{dI}{dt} = \frac d{dt} \left(\frac 12 LI^2\right).
\]
Consider reaching a magnetostatic state by building up the current from $ 0 $ to $ I $ The energy stored is
\begin{align*}
	U &= \frac 12 LI^2 \\
	  &= \frac 12 I\mathcal F \\
	  &= \frac 12 I\int_C \mathbf A \cdot\mathrm d \x \\
	  &= \frac 12 \int_V \mathbf J\cdot\mathbf A \ \mathrm dV.
\end{align*}
This is analogous to $ U = \frac 12 \int \rho \Phi \ \mathrm dV $ in electrostatics. Now using the third Maxwell equation we have
\[
	U = \frac1{2\mu_0} \int (\nabla\times\mathbf B) \cdot\mathbf A \ \mathrm dV
\]
and $ (\nabla\times\mathbf B)\cdot\mathbf A = \nabla\cdot(\mathbf B\times\mathbf A) +\mathbf B \cdot(\nabla\times\mathbf A) $. If we take the integral over all space then the first term gives zero by the divergence theorem, since $ |\mathbf B| = O\left(\frac 1{|\x|^3}\right) $ and $ |\mathbf A| = O\left(\frac 1{|\x|^2}\right) $ as $ |\x|\to\infty $ for a finite current distribution, leaving
\[
	U = \int\frac{|\mathbf B|^2}{2\mu_0}\ \mathrm dV
\]
as the energy stored in the magnetic field.
\subsection{Ohm's law}
In a stationary conductor
\[
  \mathbf J = \sigma\mathbf E
\]
where $ \sigma $ is the electrical conductivity. This is not a fundemental physical law but a constitutive relation, a macroscopic property of a material. 
The inverse relation
\[
  \mathbf E = \inv \sigma \mathbf J
\]
where $ \inv \sigma $ is the \textit{resitivity} (usually denoted by $ \rho $ which are both in conflict with the notation for charge densities). In some materials $ \sigma $ is not a scalar, but instead is a tensor, hence we denote the resitivity by $ \inv \sigma $ to show that sometimes we need to invert a tensor if the material is not isotropic.
\begin{definition}
	(Perfect conductor) A \textit{perfect conductor} corresponds to $ \sigma\to\infty $ so $ \mathbf E = 0 $.
\end{definition}
\begin{definition}
	(Perfect insulator) A \textit{perfect insulator} corresponds to $ \sigma\to 0 $ so $ \mathbf J = 0 $.
\end{definition}

Consider a straight wire of length $ L $ in the direction of the unit vector $ \mathbf n $ and with uniform cross-sectional area $ A $ and conductivity $ \sigma $. If the electric field is $ \mathbf E = E\mathbf n  $ where $ E $ is constant then $ \mathbf J = \sigma E \mathbf n $ and the total current is $ I = \sigma EA $. The potential difference (voltage) along the wire is
\[
	V = \int \mathbf E \cdot\mathrm d \x = EL = \frac{IL}{\sigma A}.
\]
So $ V = IR $  where $ R = \frac L{\sigma A} $ is the \textit{resistance} of the wire.\par
Accompanying the resistance of a wire is \textit{Joule heating} which is the conversion of EM energy into heat at rate $ I^2R $. If the voltage $ V $ is maintained by a battery then $ VI = I^2 R $ is the rate at which the emf of the battery $ (\mathcal E = V) $ does work to maintain the current $ I $. 
\subsection{Time-dependent electric fields}
In electrodynamics we can no longer write
\[
  \mathbf E = -\nabla \Phi
\]
since $ \mathbf E $ is no longer irrotational. But we still have the second Maxwell equation, so $ \mathbf B $ is still divergence free so we can still write
\[
  \mathbf B = \nabla\times\mathbf A
\]
and the third Maxwell equation gives that
\begin{align*}
	\nabla\times\left(\mathbf E +\frac{\partial A}{\partial t}\right) = 0
\end{align*}
which allows us to write
\[
	\mathbf E = - \nabla\Phi - \frac{\partial \mathbf A}{\partial t}
\]
which generalises the electrostatic expression. Let's check this is well-defined by ensuring $ \mathbf E $ is unchanged by a time-dependent gauge transformation. Suppose that
\[
	\tilde {\mathbf A } = \mathbf A + \nabla \chi,\qquad \tilde\Phi = \Phi - \frac{\partial \chi}{\partial t}
\]
where $ \chi(\x,t) $ is any scalar field, then both $ \mathbf E $ and $ \mathbf B $ are unchanged.\par
In magnetostatics we used Ampere's law 
\[
  \int_C \mathbf B \cdot\mathrm d\x = \mu_0 \int_S \mathbf J\cdot\mathrm d \mathbf S = \mu_0 I
\]
or its differential form
\[
  \nabla \times\mathbf B = \mu_0 \mathbf J.
\]
For a time-dependent situation we have to use the whole of the fourth Maxwell equation,
\[
	\nabla\times\mathbf B = \mu_0 \left(\mathbf J +\ep_0 \frac{\partial \mathbf E}{\partial t}\right)
\]
which contains an extra term, the \textit{displacement current}.\par
\textit{Why is this extra term needed?} Without it we would have $ \nabla\cdot\mathbf J = 0 $, which describes charge conservation in a situation where $ \rho $ is constrained to remain constant. But suppose we place free particles of positive charge in some localised region. Repulsive Coulumb forces cause the particles to seperate implying that $ \nabla\cdot\mathbf J > 0 $.\par
We have seen that charge conservation in the correct form
\[
	\frac{\partial \rho}{\partial t} + \nabla\cdot \mathbf J =0 
\]
follows from Maxwell's equations, including the displacement current.
\subsection{Electromagnetic waves}
We will consider freely evolving electric and magnetic fields in a vacuum, in the absence of charges and currents. We have that
\begin{align*}
	\nabla\cdot\mathbf E & = 0 \\
	\nabla\cdot\mathbf B & = 0 \\
	\nabla\times\mathbf E &= - \frac{\partial \mathbf B}{\partial t} \\
	\nabla\times\mathbf B &= \mu_0 \ep_0 \frac{\partial \mathbf B}{\partial t}.
\end{align*}
Let's eliminate $ \mathbf B $ by taking the time derivative of the fourth Maxwell equation and substituting in the third equation. We get
\begin{align*}
	\mu_0 \ep_0 \frac{\partial^2 \mathbf E}{\partial t^2} &= \frac\partial{\partial t} (\nabla\times\mathbf B) \\
							      &=\nabla\times\frac{\partial \mathbf B}{\partial t} \\
							      &= - \nabla\times(\nabla\times \mathbf E) \\
							      &= \nabla^2 \mathbf E
\end{align*}
where we use the identity (and the first Maxwell equation)
\[
  \nabla\times(\nabla\times\mathbf E) = \nabla(\nabla\cdot\mathbf E ) - \nabla^2 \mathbf E.
\]
Alternatively, eliminate $ \mathbf E $ by taking the time derivative of the third Maxwell equation and substituting in the fourth.
\begin{align*}
	\frac{\partial^2 \mathbf B}{\partial t^2} &= -\frac\partial{\partial t} (\nabla\times\mathbf E) \\
						  &= -\nabla\times\frac{\partial\mathbf E}{\partial t} \\
						  &= - \frac 1{\mu_0\ep_0}\nabla\times(\nabla\times\mathbf E)\\
						  &=\frac 1{\mu_0\ep_0} \nabla^2 \mathbf B.
\end{align*}
So each (Cartesian) component of $ \mathbf E $ and $ \mathbf B $ satisfies the wave equation
\[
	\frac{\partial^2 u }{\partial t^2} = c^2\nabla^2 u
\]
with wave speed
\[
	c = \frac 1{\sqrt \mu_0 \ep_0}
\]
which is the speed of light (in a vacuum). We expect this since light is an electromagnetic wave involving oscillations of $ \mathbf E $ and $ \mathbf B $.\par
Depending on the wavelength, EM waves can be radio waves, microwaves, infrared, visible light, ultraviolet, x-rays, and gamma rays.
\begin{example}
  Consider a plane wave in which $ \mathbf E $ and $ \mathbf B $ depend only on $ (x,t) $ and not on $ (y,z) $. A simple example is
  \[
    \mathbf E = E(x,t)\mathbf e_y
  \]
  where $ E(x,t) $ satisfies the 1D wave equation
  \[
	  \frac{\partial^2 E}{\partial t^2} = c^2 \frac{\partial^2 E}{\partial x^2}.
  \]
  The general solution (solving using method of characteristics) is
  \[
    E(x,t) = f(x-ct) + g(x+ct)
  \]
  which can be seens as the sum of a wave travellign without change of form in the positive $ x $ direction and another travelling in the negative $ x $ direction. We can use Maxwell's equation to find the corresponding magnetic field $ \mathbf B $. We have
  \begin{align*}
	  \frac{\partial \mathbf B}{\partial t} &= - \nabla\times\mathbf E \\
						&= - \frac{\partial E}{\partial x} \mathbf e_z \\
						&= (-f'(x-ct) - g'(x+ct))\mathbf e_z
  \end{align*}
  and so
  \[
    \mathbf B = B(x,t)\mathbf e_z
  \]
  with $ B(x,t) = \frac 1c(f(x-ct)-g(x+ct)) $. This also satisfies $ \nabla\times\mathbf B = \mu_0 \ep_0 \frac{\partial \mathbf E}{\partial t}. $
\end{example}
Of particular importance is a \textit{monochromatic wve} of a singular angular frequency $ \omega $ so
\begin{align*}
	E &= E_0 \cos(kx-\omega t) \\
	B &= \frac {E_0}c \cos(kx-\omega t)
\end{align*}
where $ E_0 $ is a constant amplitude and $ k = \frac \omega c $ is the \textit{wavenumber} related to the wavelength $ \lambda $ by $ k = \frac{2\pi}\lambda $.
\begin{remark}
  Let's make some notes about these monochromatic waves
  \begin{itemize}
	  \item The (angular) frequency and wavenumber are related by the \textit{dispersion relation} 
		  \[
		    \omega^2 = c^2 k^2
		  \]
		  so $ \omega = \pm ck $.
	  \item The oscillations of $ \mathbf E $ and $ \mathbf B $ are in phase but in orthogonal directions.
	  \item The waves are \textit{transverse}. The oscilattory fields are orthogonal to the direction in which the wave propagates.
  \end{itemize}
\end{remark}
Because Maxwell's equations are linear, Electromagnetic waves of different amplitudes, frequencies and directions can be superposed.
\par
A more general approach to plane EM waves is to seek solutions of Maxwell's equations of the form 
\begin{align*}
	\mathbf E &= \Re(\mathbf E_0 e^{i(\mathbf k \cdot \x - \omega t)})\\
	\mathbf B &= \Re(\mathbf B_0 e^{i(\mathbf k\cdot \x - \omega t)})
\end{align*}
where $ \mathbf E_0 $ and $ \mathbf B_0 $ are complex constant vector amplitudes, $ \mathbf k $ is the real, constant \textit{wavevector} and $ \omega $ is the real constant angular frequency. The wavenumber is $ k = |\mathbf k| $. The wave equation is satisfied by $ \mathbf E $ and $ \mathbf B $ if $ \omega $ and $ k $ satisfy the dispersion relation
\[
  \omega^2 = c^2k^2.
\]
The individual Maxwell equations reduce to algebraic conditions.
\begin{align*}
	\nabla\cdot\mathbf E \quad &\implies \quad \mathbf k\cdot\mathbf E_0 = 0\\
	\nabla\cdot\mathbf B \quad&\implies \quad \mathbf k \cdot\mathbf B_0 = 0 \\
	\nabla \times\mathbf E = -\frac{\partial \mathbf B}{\partial t} \quad & \implies \quad \mathbf k \times\mathbf E_0 = \omega \mathbf B_0 \\
	\nabla\times\mathbf B = \mu_0\ep_0\frac{\partial\mathbf E}{\partial t} \quad&\implies \quad \mathbf k \times\mathbf B_0 = -\frac \omega{c^2}\mathbf E_0
\end{align*}
The fourth equation is redundant because the third and third together with the dispersion relation imply
\begin{align*}
	\mathbf k \times\mathbf B_0 = \frac 1 \omega \mathbf k \times(\mathbf k \times\mathbf E_0) = -\frac{k^2}\omega \mathbf E_0 = -\frac\omega{c^2} \mathbf E_0.
\end{align*}
Suppose that $ \mathbf E_0 $ is real. Then $ \mathbf B_0 $ is also real and the vectors $ \mathbf k,\mathbf E_0,\mathbf B_0 $ form an arothogonal triad. So $ \mathbf E $ and $ \mathbf B $ oscillate in fixed directions, with are perpendicular to each other and to the direction of propagation. This is similar to the 1D wave considered previously and corresponds to a linearly polarised wave. Polarisation describes the orientation of the electric field.\par
Now suppose that $ \mathbf E_0 $ is complex and of the form
\[ \mathbf E_0 = \mathbf a - i \mathbf b \]
where $ \mathbf a, \mathbf b $ are lineraly independent real vectors in the plane perpendicular to $ \mathbf k $. Then the direction of 
\[
  \mathbf E = \mathbf a \cos(\mathbf k \cdot\x - \omega t) +\mathbf b \sin(\mathbf k \cdot\x-  \omega t)
\]
is not fixed but rotates in the plane perpendicular to $ k $. In general it traces out an ellipse and the wave is said to be \textit{elliptically polarised}.\par
The special case of \textit{circular polarisation} occurs when $ |\mathbf a| = |\mathbf b| $ and $ \mathbf a \cdot\mathbf b= 0 $. Then $ \mathbf E $ traces out a circle.\par
The polarisation is \textit{right-handed} if $ \mathbf b=\hat{\mathbf k} \times\mathbf a $ and \textit{left-handed} if $ \mathbf b = \mathbf a \times{\hat{\mathbf k}} $.\par
Suppose a perfect conductor occupies the half-space $ x>0 $. An \textit{incident wave} approaching the surface $ x=0 $ in the $ +x $ direction has the form
\[
	\mathbf E_{\text{inc}} = \Re\left(E_0 e^{i(kx-\omega t)} \mathbf e_y\right)
\]
where we take $ k>0 $, $ \omega=ck>0 $ and $ E_0\in \R $.\par
Inside the conductor we have $ \mathbf E = 0 $. Since the tangential components of $ \mathbf E $ are continuous at the interface $ ([\mathbf n \times\mathbf E]=0) $ we require $ E_y = 0 $ at $ x=0 $. This can be satisfied by adding a \textit{reflected wave}
\[
	\mathbf E_{\text{reff}} = \Re\left(-E_0 e^{i(-kx-\omega t)}\mathbf e_y\right)
\]
propagating in the $ -x $ direction. The combined solution is
\begin{align*}
	\mathbf E &= \mathbf E_{\text{inc}} +\mathbf E_{\text{ref}} \\
		  &= \Re\left(2iE_0 \sin(kx)e^{-i\omega t} \mathbf e_y\right)\\
		  &= 2E_0 \sin(kt)\sin(\omega t) \mathbf e_y
\end{align*}
which is a standing wave.
\begin{exercise}
  Show that $ \mathbf B $ is discontinuous at $ x=0 $ and work out the surface current density $ \mathbf K(t) $.
\end{exercise}
\subsection{Electromagnetic energy}
In electrostatics we found that the energy per unit volume in $ \mathbf E $ is $ \frac 12 \ep_0 |\mathbf E|^2 $. In magnetostatics we found the energy per unit volume was $ \frac 1{2\mu_0}|\mathbf B|^2 $. \par
Now derive a general conservation law from Maxwell's equations. Consider the rate of change of the electric and magnetic energy per unit volume. \begin{align*}
	\frac\partial{\partial t}\left(\frac 12 \ep |\mathbf E|^2 + \frac 1{2\mu_0} |\mathbf B^2|\right) &= \mathbf E\cdot\left(\ep_0\frac{\partial \mathbf E}{\partial t}\right) + \mathbf B \cdot\left(\frac 1{\mu_0}\frac{\partial \mathbf B}{\partial t}\right)\\
													 &= \mathbf E \cdot\left(\frac 1{\mu_0}\nabla\times\mathbf B - \mathbf J\right) + \mathbf B\cdot\left(-\frac 1{\mu_0} \nabla\times\mathbf E\right).
\end{align*}
An identity from vector calculus:
\[
  \nabla\cdot(\mathbf E\times\mathbf B) = \mathbf B\cdot(\nabla\times\mathbf E)-\mathbf E\cdot(\nabla\times\mathbf B).
\]
Divide by $ \mu_0 $ and add to the previous equation
\[
	\frac{\partial}{\partial t} \left(\frac 12 \ep_0 |\mathbf E|^2+\frac 1{2\mu_0}|\mathbf B|^2\right) + \nabla\cdot\left(\frac{\mathbf E\times\mathbf B}{\mu_0}\right) = -\mathbf E \cdot\mathbf J.
\]
This is the energy equation for the EM field. Write it as
\[
	\frac{\partial w}{\partial t} + \nabla\cdot\mathbf S = -\mathbf E \cdot\mathbf J
\]
where
\[
	w = \frac 12 \ep_0 |\mathbf E|^2 + \frac 1{2\mu_0}|\mathbf B|^2
\]
is the \textit{energy density} and 
\[
	\mathbf S = \frac{\mathbf E\times\mathbf B}{\mu_0}
\]
is the \textit{energy flux density} and it known as the \textit{Poynting vector}.\par
In the absense of currents this becomes
\[
	\frac{\partial w}{\partial t} + \nabla\cdot\mathbf S = 0
\]
and expresses the conservation of electromagnetic energy. Integrate over some time independent volume $ V $ with $ \partial V = S $ and use the divergence theorem
\begin{align*}
	\frac{d}{dt} \int_V w\mathrm dV + \int_S \mathbf S \cdot\mathrm d\mathbf S = 0.
\end{align*}
So the rate of change of energy in $ V $ is equal to minus the flux of energy through the surface $ S $.\par
If we do however have $ \mathbf J \ne 0 $, we can see that $ \mathbf E\cdot\mathbf J $ is the rate at which the EM field loses energy (per unit volume) by doing work on charged particles via the Lorentz force.

\end{document}
