\documentclass{article}
\usepackage{../header}
\title{Topological Spaces}
\author{Notes by Finley Cooper}
\begin{document}
  \maketitle
  \newpage
  \tableofcontents
  \newpage
  \section{Topologies}
  \subsection{Definitions}
  We denote $ \mathcal P(X) $ as the power set of $ X $.
  \begin{definition}
	  (Topology) Let $ X $ be a set. A \textit{topology} on $ X $ is a collection of sets $ T\subseteq \mathcal P(X) $ such that
	  \begin{enumerate}
		  \item $\emptyset, X\in T $,
		  \item $ T $ is closed under (possibly uncountable) unions.
		  \item $ T $ is closed under finite intersections.
	  \end{enumerate}
  \end{definition}


  A set $ X $ with a topology $ T $ is called a \textit{topological space} of $ X $. An element of $ X $ is called a \textit{point} and elements of $ T $ are called \textit{open sets}. If $ x\in U\in T $ we say $ U $ is an open neighbourhood of $ x $. Strictly we should always denote $ (X,T) $ for a topological space, but when $ T $ is clear, we just write $ X $ for the topological space.
\begin{definition}
	(Continuity) If $ (X,T_X) $ and $ (Y,T_Y) $ are topological spaces then a function $ f:X\to Y $ is called \textit{continuous} if for $ U\in T_Y $, $ \inv f(U)\in T_X $.
\end{definition}
\begin{definition}
	(Homeomorphism) A function $ f:(X,T_X)\to (Y,T_Y) $ is a \textit{homeomorphism} if it is continuous and has a continuous inverse.
\end{definition}
\begin{definition}
	If $ T\subseteq T' $ are topologies on $ X $ then we say that $ T $ is \textit{coarser} and $ T' $ is \textit{finer}. The identity function $ d: (X,T)\to (X,T') $ is continuous.
\end{definition}
\subsection{Topologies from metrics}
If $ (X,d) $ is a metric space, recall that a subset $ U\subseteq X $ is called \textit{open} if for every point $ x\in U $ there exists a $ \varepsilon>0 $ such that $ B_\varepsilon(x)\subseteq U $.
\begin{proposition}
  If $ T_d $ is the subset of $ X $ which are open under the metric $ d $, then $ (X,T_d) $ is a topological space. We will call this the topology on $ X $ induced by the metric $ d $.
\end{proposition}
\pf Tautologically we have that $ \emptyset \in T_d $. Clearly we have that $ X\in T_d $ too. Let $ \{U_\alpha\}_{\alpha\in I} $ be a collection of open sets in $ T_d $ with a (possibly uncountable) index set $ I $. Let
\[
	x\in\bigcup_{\alpha\in I}U_\alpha.
\]
Then $ x\in U_\beta $ for some $ \beta\in I $, so $ U_\beta $ is open hence there exists a $ \varepsilon>0 $ such that $ B_\varepsilon\subseteq U_\beta\subseteq \bigcup_{\alpha\in I}U_\alpha $, hence $ \bigcup_{\alpha\in I}U_\alpha $ is open.\par
Now suppoes that $ I $ is finite, and $ x\in \bigcap_{\alpha\in I}U_\alpha $. For each $ \alpha $ there exists a $ \varepsilon_\alpha>0 $ such that $ B_{\varepsilon_\alpha}(x)\subseteq U_\alpha $. Take $ \varepsilon=\inf_{\alpha\in I}\varepsilon_\alpha $, so $ B_\varepsilon(x)\subseteq B_{\varepsilon_\alpha}(x)\subseteq U_\alpha $ for all $ \alpha $, hence we have that $ B_\varepsilon(x) \subseteq \bigcap_{\alpha\in I}U_\alpha$ so it's open. Hence $ T $ is a topology.\qed
\par
Now we have lots of examples we can use for topological spaces. For example we have that topology induced by the Euclidean metric on $ \R^d $ which we will call the Euclidean topology. For any $ X\subseteq \R^d $ we can have a topology induced by the Euclidean metric too, like $ \Q $, $ [0,1], (0,1) $.
\begin{proposition}
  If we have two metric spaces $ (X,d_X), (Y,d_Y) $ and we have $ f:X\to Y $, the $ f $ is continuous in the metric space sense if and only if it is continuous in the topological space sense (with the topologies induced by the metric $ d_X $ and $ d_Y $ respectively).
\end{proposition}
\pf Let $ f:X\to Y $ be continuous in the metric space sense. Let $ U $ be an open set in $ T_{d_Y} $ so we need to show that $ \inv f(U) $ is open. Let $ x\in \inv f(U) $, so $ f(x)\in U $. Hence there exists an $ \varepsilon>0 $ such that $ B_\varepsilon(f(x))\subseteq U $. So since $ f $ is continuous there exists a $ \delta>0 $ such that if $ d_X(x,x')<\delta $, then $ d_Y(f(x),f(x'))<\varepsilon $. Hence $ f(B_\delta(x))\subseteq B_\varepsilon(f(x)) $. So $ B_\delta(x)\in \inv f(U) $, hence $ \inv f(U) $ is open.\par
Now let's do the converse and suppose that $ f:X\to Y $ is continuous in the topological sense. Fix some $ x\in X $ and $ \varepsilon>0 $. Consider $ B_\varepsilon(f(x)) $ which is open in $ Y $. Then $ \inv f(B_\varepsilon(f(x))) $ is in $ T_{d_X} $. It contains $ x $ so there exists a $ \delta>0 $ such that $ x\in B_\delta(x) $, so
\[
  d_X(x,x')<\delta \implies d_Y(f(x),f(x'))<\varepsilon.
\]
So $ f $ is continuous in the metric sense.\qed
\begin{definition}
  Let $ (X,T) $ be a topological space and $ x_1,x_2,\dots \in X $ say. We say that $ x_n $ convergences to $ x $ if for every open neighbourhood $ U $ of $ x $ there exists a $ N $ such that $ x_n\in U $ for all $ n\ge N $.
\end{definition}
\begin{proposition}
  If $ (X,d) $ is a metric space with topology $ T_d $ then a sequence $ (x_n) $ converges in the metric sense if and only if it converges in the topological sense.
\end{proposition}
\pf Suppose it convergens in the metric sense to $ x $. Then for all $ \varepsilon>0 $ there exists a $ N $ such that for all $ n\ge N $ we have that $ x_n\in B_\varepsilon(x) $. If $ U $ is a neighourhood of $ x $ then there there is some $ \varepsilon $ such that the ball of radius $ \varepsilon $ centred at $ x $ is contained in $ U $. Conversely if $ (x_n) $ converges in the topological sense to $ x $, let $ \varepsilon>0 $ and consider the open ball centred at $ x $ with radius $ \varepsilon $. Now $ B_\varepsilon(x) $ is an open neighbourhood of $ x $ so there exists an integer $ N $ such that $ x_n\in B_\varepsilon(x) $ for all $ n> N $. Hence $ (x_n) $ converges to $ x $ in the metric sense.\qed
\par
Consider $ \R $ and $ (0,1) $ with the Euclidean metric and topology. Then the two spaces are related, by the function $ (0,1)\to \R $ by $ \tan^{-1}x $ which is invertible. Hence we say the two spaces are homeomorphic, and $ \R\cong (0,1) $. However the two spaces are not isometric since $ \R $ is not complete under the Euclidean metric and $ (0,1) $ is not. Hence the property of completeness is not a topological property: it is a property induced by the metric.
\par
\begin{definition}
	(Discrete topology) Let $ X $ be a set. The \textit{discrete} topology is the topology $ T_{\text{discrete}} = \mathcal P(X) $ (so every set is open).
\end{definition}
\begin{remark}
	Any function from $ (X,T_{\text{discrete}}) $ to any space si continuous. This toplogy can be induced by the discrete metric, where $ d(x,y) = \begin{cases}
		1 & x\ne y \\
		0 & x = y
	\end{cases} $. So $ B_{\frac 12}(x) = \{x\} $ so $ \{x\} $ is open, hence all the sets are open. 
\end{remark}
\begin{definition}
	(Indiscrete topology) Let $ X $ be a set. The \textit{indiscrete} topology $ T_{\text{indiscrete}} = \{\emptyset, X\} $ (as little as possible sets are open).
\end{definition}
\begin{remark}
	A function from any space to $ (X,T_{\text{indiscrete}}) $ is continuous. This topology does not come from a metric unless $ X $ is a singleton set. This is because if $ x\ne y $ then $ d(x,y)=\varepsilon > 0 $, so $ y\notin B_\varepsilon(x) $ and since $ y $ is arbitrary, then $ B_\varepsilon(x) = \{x\}=X $.
\end{remark}
Let $ X = \{o,c\} $. Then let $ T = \{ \emptyset, \{o,c\}, \{o\}\} $ be a topology of $ X $. This is called the Sierpinski space. It as the property that every sequence converges to $ c $. A continuous function $ f:T\to (X,T_{\text{Sierpinski}}) $ is exactly an open subset of $ Y $.\par
Let $ X = \R $ we'll define the right order topology on $ X $ as
\[
	T_{\text{ord}} = \{(a,\infty) \mid -\infty \le a \le \infty\}.
\]
Let $ \{(a,\infty)\}_{a\in I} $ be a collect of elements of $ T_{\text{ord}} $. Then
\begin{align*}
	\bigcup_{a\in I} (a,\infty) = (\inf_{a\in I}a,\infty)\in T_{\text{ord}}.
\end{align*}
Similarly for finite $ I $,
\begin{align*}
	\bigcap_{a\in I}(a,\infty) = (\max_{a\in I}a,\infty) \in T_{\text{ord}}
\end{align*}
\subsection{Bases and subbases}
\begin{definition}
(Basis) Let $ T $ be a topology of $ X $. A \textit{basis}, $ B\subseteq T $ for $ T $ is a subcollection such that every element of $ T $ is a union of elements in $ B $.
\end{definition}
\begin{definition}
	(Subbasis) Let $ T $ be a topology of $ X $. A \textit{subbasis}, $ S\subseteq T $ for $ T $ is a subcollection such that every element of $ T $ is a union of sets which are finite intersections of elements of $ S $.
\end{definition}
\begin{lemma}
  Let $ f:(X,T_X)\to (Y,T_Y) $ and $ S\subseteq T_Y $ is a subbasis. If $ \inv f(U) $ is open for all $ U\in S $ then $ f $ is continuous.
\end{lemma}
\pf If $ V\subseteq T_Y $, then $ V=\bigcup_{a\in I}V_a $ where $ V_a\in \bigcap_{b\in J_a}U_{a,b} $ with $U_{a,b}\in S $ and $ J_a $ finite. Then
\begin{align*}
	\inv f(V) = \bigcup_{a\in I}V_a &= \bigcup_{a\in I}\left(\bigcap_{b\in J_a}\inv f(U_{a,b})\right)\in T_X,
\end{align*}
by the axioms of the topology.\qed
\par
Consider the Euclidean topology on $ \R^n $. The collection $ B= \{B_r(x)\mid x\in \R^n, r>0\} $ is a basis. Likewise the collection of $ n $-cubes everywhere are also a basis. Interestingly the set $ QB \subseteq B $ with balls at rational points with rational radii is also a basis. This is interesting since $ QB $ is countable while $ B $ is uncountable and $ \mathcal P(\R^n) $ is $ \aleph_2 $.
\begin{definition}
	(Closed set) Let $ (X,T) $ be a topological space. A subset $ C\subseteq X $ is \textit{closed} if $ X\setminus C\in T $.
\end{definition}
\begin{proposition}
	Let $ (X,T) $ be a topological space and $ \mathcal F = \{ C\subseteq X \mid C \ \text{closed} \} $. Then
	\begin{enumerate}
		\item $ \emptyset, X\in \mathcal F $;
		\item $ \mathcal F $ is closed under (possibly uncountable) intersections;
		\item $ \mathcal F $ is closed under finite unions.
	\end{enumerate}
\end{proposition}
\begin{proposition}
  A function $ f:X\to Y $ between topological spaces is continuous if and only if the preimage of every closed set is closed.
\end{proposition}
\begin{definition}
  Let $ (X,T) $ be a topological space. Let $ A\subseteq X $ be a subset of $ X $. Then
  \begin{enumerate}
	  \item The closure $ \bar A $ is the smallest (by inclusion) closed set containing $ A $ so
		  \[
			  \bar A = \bigcap_{S\ \text{closed}, A\subseteq S} S.
		  \]
	  \item We say that $ A $ is dense in $ X $ if $ A=\bar A $.
	  \item The interior $ \mathring{A} $ is the largest open set contained in $ A $ so
		  \[
			  \mathring{A} = \bigcup_{S\ \text{open}, S\subseteq A} S.
		  \]
  \end{enumerate}
\end{definition}
\begin{definition}
	(Limit point) Let $ X $ be a topological space and $ A\subseteq X $. A \textit{limit point} of $ A $ is a point in $ X $ which is a limit of a sequence in $ A $.
\end{definition}
\begin{proposition}
  If $ C $ is a closed subset of $ (X,T) $, then the limit points of $ C $ lie in $ C $.
\end{proposition}
\pf Let $ \{x_n\} $ be a sequence in $ C $ with limit $ x_\infty $. If $ x_\infty\notin C $, then $ x_\infty \in X\setminus C $ which is open. Then if $ x_n\to x_\infty $ then we should have that $ x_n\in X\setminus C $ for $ n\ge N $ but $ x_n\in C $ so $ x_n\notin X\setminus C $ which is a contradiction.\qed
\begin{corollary}
  A limit point of a $ A $ lies in $ \bar A $.
\end{corollary}








\end{document}
