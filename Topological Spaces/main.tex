\documentclass{article}
\usepackage{../header}
\title{Topological Spaces}
\author{Notes by Finley Cooper}
\begin{document}
  \maketitle
  \newpage
  \tableofcontents
  \newpage
  \section{Topologies}
  \subsection{Definitions}
  We denote $ \mathcal P(X) $ as the power set of $ X $.
  \begin{definition}
	  (Topology) Let $ X $ be a set. A \textit{topology} on $ X $ is a collection of sets $ T\subseteq \mathcal P(X) $ such that
	  \begin{enumerate}
		  \item $\emptyset, X\in T $,
		  \item $ T $ is closed under (possibly uncountable) unions.
		  \item $ T $ is closed under finite intersections.
	  \end{enumerate}
  \end{definition}


  A set $ X $ with a topology $ T $ is called a \textit{topological space} of $ X $. An element of $ X $ is called a \textit{point} and elements of $ T $ are called \textit{open sets}. If $ x\in U\in T $ we say $ U $ is an open neighbourhood of $ x $. Strictly we should always denote $ (X,T) $ for a topological space, but when $ T $ is clear, we just write $ X $ for the topological space.
\begin{definition}
	(Continuity) If $ (X,T_X) $ and $ (Y,T_Y) $ are topological spaces then a function $ f:X\to Y $ is called \textit{continuous} if for $ U\in T_Y $, $ \inv f(U)\in T_X $.
\end{definition}
\begin{definition}
	(Homeomorphism) A function $ f:(X,T_X)\to (Y,T_Y) $ is a \textit{homeomorphism} if it is continuous and has a continuous inverse.
\end{definition}
\begin{definition}
	If $ T\subseteq T' $ are topologies on $ X $ then we say that $ T $ is \textit{coarser} and $ T' $ is \textit{finer}. The identity function $ d: (X,T)\to (X,T') $ is continuous.
\end{definition}
\subsection{Topologies from metrics}
If $ (X,d) $ is a metric space, recall that a subset $ U\subseteq X $ is called \textit{open} if for every point $ x\in U $ there exists a $ \varepsilon>0 $ such that $ B_\varepsilon(x)\subseteq U $.
\begin{proposition}
  If $ T_d $ is the subset of $ X $ which are open under the metric $ d $, then $ (X,T_d) $ is a topological space. We will call this the topology on $ X $ induced by the metric $ d $.
\end{proposition}
\pf Tautologically we have that $ \emptyset \in T_d $. Clearly we have that $ X\in T_d $ too. Let $ \{U_\alpha\}_{\alpha\in I} $ be a collection of open sets in $ T_d $ with a (possibly uncountable) index set $ I $. Let
\[
	x\in\bigcup_{\alpha\in I}U_\alpha.
\]
Then $ x\in U_\beta $ for some $ \beta\in I $, so $ U_\beta $ is open hence there exists a $ \varepsilon>0 $ such that $ B_\varepsilon\subseteq U_\beta\subseteq \bigcup_{\alpha\in I}U_\alpha $, hence $ \bigcup_{\alpha\in I}U_\alpha $ is open.\par
Now suppoes that $ I $ is finite, and $ x\in \bigcap_{\alpha\in I}U_\alpha $. For each $ \alpha $ there exists a $ \varepsilon_\alpha>0 $ such that $ B_{\varepsilon_\alpha}(x)\subseteq U_\alpha $. Take $ \varepsilon=\inf_{\alpha\in I}\varepsilon_\alpha $, so $ B_\varepsilon(x)\subseteq B_{\varepsilon_\alpha}(x)\subseteq U_\alpha $ for all $ \alpha $, hence we have that $ B_\varepsilon(x) \subseteq \bigcap_{\alpha\in I}U_\alpha$ so it's open. Hence $ T $ is a topology.\qed
\par
Now we have lots of examples we can use for topological spaces. For example we have that topology induced by the Euclidean metric on $ \R^d $ which we will call the Euclidean topology. For any $ X\subseteq \R^d $ we can have a topology induced by the Euclidean metric too, like $ \Q $, $ [0,1], (0,1) $.
\begin{proposition}
  If we have two metric spaces $ (X,d_X), (Y,d_Y) $ and we have $ f:X\to Y $, the $ f $ is continuous in the metric space sense if and only if it is continuous in the topological space sense (with the topologies induced by the metric $ d_X $ and $ d_Y $ respectively).
\end{proposition}
\pf Let $ f:X\to Y $ be continuous in the metric space sense. Let $ U $ be an open set in $ T_{d_Y} $ so we need to show that $ \inv f(U) $ is open. Let $ x\in \inv f(U) $, so $ f(x)\in U $. Hence there exists an $ \varepsilon>0 $ such that $ B_\varepsilon(f(x))\subseteq U $. So since $ f $ is continuous there exists a $ \delta>0 $ such that if $ d_X(x,x')<\delta $, then $ d_Y(f(x),f(x'))<\varepsilon $. Hence $ f(B_\delta(x))\subseteq B_\varepsilon(f(x)) $. So $ B_\delta(x)\in \inv f(U) $, hence $ \inv f(U) $ is open.\par
Now let's do the converse and suppose that $ f:X\to Y $ is continuous in the topological sense. Fix some $ x\in X $ and $ \varepsilon>0 $. Consider $ B_\varepsilon(f(x)) $ which is open in $ Y $. Then $ \inv f(B_\varepsilon(f(x))) $ is in $ T_{d_X} $. It contains $ x $ so there exists a $ \delta>0 $ such that $ x\in B_\delta(x) $, so
\[
  d_X(x,x')<\delta \implies d_Y(f(x),f(x'))<\varepsilon.
\]
So $ f $ is continuous in the metric sense.\qed
\begin{definition}
  Let $ (X,T) $ be a topological space and $ x_1,x_2,\dots \in X $ say. We say that $ x_n $ convergences to $ x $ if for every open neighbourhood $ U $ of $ x $ there exists a $ N $ such that $ x_n\in U $ for all $ n\ge N $.
\end{definition}
\begin{proposition}
  If $ (X,d) $ is a metric space with topology $ T_d $ then a sequence $ (x_n) $ converges in the metric sense if and only if it converges in the topological sense.
\end{proposition}
\pf Suppose it convergens in the metric sense to $ x $. Then for all $ \varepsilon>0 $ there exists a $ N $ such that for all $ n\ge N $ we have that $ x_n\in B_\varepsilon(x) $. If $ U $ is a neighourhood of $ x $ then there there is some $ \varepsilon $ such that the ball of radius $ \varepsilon $ centred at $ x $ is contained in $ U $. Conversely if $ (x_n) $ converges in the topological sense to $ x $, let $ \varepsilon>0 $ and consider the open ball centred at $ x $ with radius $ \varepsilon $. Now $ B_\varepsilon(x) $ is an open neighbourhood of $ x $ so there exists an integer $ N $ such that $ x_n\in B_\varepsilon(x) $ for all $ n> N $. Hence $ (x_n) $ converges to $ x $ in the metric sense.\qed
\par
Consider $ \R $ and $ (0,1) $ with the Euclidean metric and topology. Then the two spaces are related, by the function $ (0,1)\to \R $ by $ \tan^{-1}x $ which is invertible. Hence we say the two spaces are homeomorphic, and $ \R\cong (0,1) $. However the two spaces are not isometric since $ \R $ is not complete under the Euclidean metric and $ (0,1) $ is not. Hence the property of completeness is not a topological property: it is a property induced by the metric.
\par
\begin{definition}
	(Discrete topology) Let $ X $ be a set. The \textit{discrete} topology is the topology $ T_{\text{discrete}} = \mathcal P(X) $ (so every set is open).
\end{definition}
\begin{remark}
	Any function from $ (X,T_{\text{discrete}}) $ to any space si continuous. This toplogy can be induced by the discrete metric, where $ d(x,y) = \begin{cases}
		1 & x\ne y \\
		0 & x = y
	\end{cases} $. So $ B_{\frac 12}(x) = \{x\} $ so $ \{x\} $ is open, hence all the sets are open. 
\end{remark}
\begin{definition}
	(Indiscrete topology) Let $ X $ be a set. The \textit{indiscrete} topology $ T_{\text{indiscrete}} = \{\emptyset, X\} $ (as little as possible sets are open).
\end{definition}
\begin{remark}
	A function from any space to $ (X,T_{\text{indiscrete}}) $ is continuous. This topology does not come from a metric unless $ X $ is a singleton set. This is because if $ x\ne y $ then $ d(x,y)=\varepsilon > 0 $, so $ y\notin B_\varepsilon(x) $ and since $ y $ is arbitrary, then $ B_\varepsilon(x) = \{x\}=X $.
\end{remark}
Let $ X = \{o,c\} $. Then let $ T = \{ \emptyset, \{o,c\}, \{o\}\} $ be a topology of $ X $. This is called the Sierpinski space. It as the property that every sequence converges to $ c $. A continuous function $ f:T\to (X,T_{\text{Sierpinski}}) $ is exactly an open subset of $ Y $.\par
Let $ X = \R $ we'll define the right order topology on $ X $ as
\[
	T_{\text{ord}} = \{(a,\infty) \mid -\infty \le a \le \infty\}.
\]
Let $ \{(a,\infty)\}_{a\in I} $ be a collect of elements of $ T_{\text{ord}} $. Then
\begin{align*}
	\bigcup_{a\in I} (a,\infty) = (\inf_{a\in I}a,\infty)\in T_{\text{ord}}.
\end{align*}
Similarly for finite $ I $,
\begin{align*}
	\bigcap_{a\in I}(a,\infty) = (\max_{a\in I}a,\infty) \in T_{\text{ord}}
\end{align*}
\subsection{Bases and subbases}
\begin{definition}
(Basis) Let $ T $ be a topology of $ X $. A \textit{basis}, $ B\subseteq T $ for $ T $ is a subcollection such that every element of $ T $ is a union of elements in $ B $.
\end{definition}
\begin{definition}
	(Subbasis) Let $ T $ be a topology of $ X $. A \textit{subbasis}, $ S\subseteq T $ for $ T $ is a subcollection such that every element of $ T $ is a union of sets which are finite intersections of elements of $ S $.
\end{definition}
\begin{lemma}
  Let $ f:(X,T_X)\to (Y,T_Y) $ and $ S\subseteq T_Y $ is a subbasis. If $ \inv f(U) $ is open for all $ U\in S $ then $ f $ is continuous.
\end{lemma}
\pf If $ V\subseteq T_Y $, then $ V=\bigcup_{a\in I}V_a $ where $ V_a\in \bigcap_{b\in J_a}U_{a,b} $ with $U_{a,b}\in S $ and $ J_a $ finite. Then
\begin{align*}
	\inv f(V) = \bigcup_{a\in I}V_a &= \bigcup_{a\in I}\left(\bigcap_{b\in J_a}\inv f(U_{a,b})\right)\in T_X,
\end{align*}
by the axioms of the topology.\qed
\par
Consider the Euclidean topology on $ \R^n $. The collection $ B= \{B_r(x)\mid x\in \R^n, r>0\} $ is a basis. Likewise the collection of $ n $-cubes everywhere are also a basis. Interestingly the set $ QB \subseteq B $ with balls at rational points with rational radii is also a basis. This is interesting since $ QB $ is countable while $ B $ is uncountable and $ \mathcal P(\R^n) $ is $ \aleph_2 $.
\begin{definition}
	(Closed set) Let $ (X,T) $ be a topological space. A subset $ C\subseteq X $ is \textit{closed} if $ X\setminus C\in T $.
\end{definition}
\begin{proposition}
	Let $ (X,T) $ be a topological space and $ \mathcal F = \{ C\subseteq X \mid C \ \text{closed} \} $. Then
	\begin{enumerate}
		\item $ \emptyset, X\in \mathcal F $;
		\item $ \mathcal F $ is closed under (possibly uncountable) intersections;
		\item $ \mathcal F $ is closed under finite unions.
	\end{enumerate}
\end{proposition}
\begin{proposition}
  A function $ f:X\to Y $ between topological spaces is continuous if and only if the preimage of every closed set is closed.
\end{proposition}
\begin{definition}
  Let $ (X,T) $ be a topological space. Let $ A\subseteq X $ be a subset of $ X $. Then
  \begin{enumerate}
	  \item The closure $ \bar A $ is the smallest (by inclusion) closed set containing $ A $ so
		  \[
			  \bar A = \bigcap_{S\ \text{closed}, A\subseteq S} S.
		  \]
	  \item We say that $ A $ is dense in $ X $ if $ A=\bar A $.
	  \item The interior $ \mathring{A} $ is the largest open set contained in $ A $ so
		  \[
			  \mathring{A} = \bigcup_{S\ \text{open}, S\subseteq A} S.
		  \]
  \end{enumerate}
\end{definition}
\begin{definition}
	(Limit point) Let $ X $ be a topological space and $ A\subseteq X $. A \textit{limit point} of $ A $ is a point in $ X $ which is a limit of a sequence in $ A $.
\end{definition}
\begin{proposition}
  If $ C $ is a closed subset of $ (X,T) $, then the limit points of $ C $ lie in $ C $.
\end{proposition}
\pf Let $ \{x_n\} $ be a sequence in $ C $ with limit $ x_\infty $. If $ x_\infty\notin C $, then $ x_\infty \in X\setminus C $ which is open. Then if $ x_n\to x_\infty $ then we should have that $ x_n\in X\setminus C $ for $ n\ge N $ but $ x_n\in C $ so $ x_n\notin X\setminus C $ which is a contradiction.\qed
\begin{corollary}
  A limit point of a $ A $ lies in $ \bar A $.
\end{corollary}
For an example $ \overline\Q = \R $ since any real number is a limit of a sequence of rational numbers. We have that $ \overline{(0,1)} = [0,1] $ too. The cocountable topology on $ \R $ is the topology $ T_{\text{countable}} = \{\emptyset\} \cup \{\R \setminus C \mid C \ \text{countable}\} $ Let $ \{x_n\} $ be a sequence in $ \R $, for $ x \in \R $ consider $ \{x\} \cup\{\R - \{x_n\}\} $ is open and contains $ x $. If $ x_n\to x $, then $ x_n $ must be in a $ U $ for all $ n\ge \N $ so $ x_n=x $ for all $ n\ge N $. Hence the convergent sequences are exactly the eventually constant sequences with the limits being the value they are eventually constant to. So the limit points of a set $ A $ are $ A $ under this topology. However almost all $ A $ is not closed. For example $ (0,1) $ is not closed since $ \R\setminus (0,1) $ is not countable. But the closure of $ (0,1) $ must be closed, so it must be $ \R $ hence the sense of limit points and closure are actually two very different properties in topology instead of metric spaces.
\subsection{Hausdorff spaces}
\begin{definition}
	(Hausdorff) A space $ (X,T) $ is \textit{Hausdorff} if for $ x\ne y \in X $ there are open neighbourhoods $ x\in U $, $ y\in V $ with $ U\cap V=\emptyset $.
\end{definition}
\begin{remark}
  This is the notion that points are seperated by open sets.
\end{remark}
\begin{lemma}
  If the topology $ T $ is induced by a metric then it is Hausdorff.
\end{lemma}
\pf If $ x\ne y $ then $ d(x,y) = s>0 $. So consider $ U=B_{s/2}(x) $ and $ V= B_{s/2}(y) $. The triangle inequality shows that $ U\cap V = \emptyset $ and we know all balls are open.\qed
\begin{proposition}
  If a space is Hausdorff then a sequence in $ X $ has at most $ 1 $ limit.
\end{proposition}
\pf Let $ (x_n) $ be a sequence in $ X $. Suppose it has limits $ y\ne z\in X $. Let $ U $ and $ V $ be disjoint local neighbourhoods sfor $ y $ and $ z $ respectively. Then $ x_n\in U $ for all $ n\ge N_1 $ and $ x_n\in V $ for all $ n\ge N_2 $. So if we take that $ N=\max\{N_1,N_2\} $ then for all $ n\ge N $, we have that $ x_n\in U\cap V $ which is empty, hence we have a contradiction.\qed
\begin{proposition}
  If $ (X,T) $ is Hausorff then points are closed.
\end{proposition}
\pf Let $ x\in X $. We want to show that $ \{x\}=\overline{\{x\}} $. Let $ y\ne x $. Let $ U,V $ be disjoint neighbourhoods of $ x $ and $ y $ respectively. We know that $ x\in X\setminus V $ which is closed. Hence $ \overline{\{x\}}\subseteq X\setminus V $. But $ y\notin V $, so $ y $ is not in the closure of $ \{x\} $ hence the closure of $ \{x\} $ is just $ \{x\} $, so $ \{x\} $ is closed.\qed
\par
Let's see an example. Let $ X $ be an infinite set and consider the cofinite topology on $ X $. Take two non-empty open sets, so
\[
	(X\setminus F) \cap (X\setminus F') = X\setminus (F\cup F')
\]
which is non-empty since $ F\cup F' $ is finite and $ X $ is infinite so the set on the RHS is non-empty hence the space is not Hausdorff.
\subsection{Defining new topologies on existing ones}
We have three main ways to define new topologies when given a topology already.
\subsubsection{The subspace topology}
\begin{definition}
	(Subset topology) Let $ (X,T_X) $ be a topological space. Let $ Y\subseteq X $ a subset. The \textit{subset topology} on $ Y $ is 
	\[
		T\mid_Y = \{Y\cap U \mid U\in T\}.
	\]
\end{definition}
\begin{definition}
	(Subspace) A subspace of $ (X,T) $ is a subset equipped with the subspace topology.
\end{definition}
\begin{proposition}
  The subset topology is a topology.
\end{proposition}
\pf Simple exercise of the axioms.\qed
\begin{proposition}
  The inclusion map $ \iota: (Y,T\mid_Y)\to (X,T) $ is continuous. In fact $ T\mid_Y $ is the constant topology on $ Y $ such that the inclusion map is continuous.
\end{proposition}
\pf Let $ U\in T $ then $ \inv\iota (U) = U\cap Y\in T\mid_Y $ by definition. So it is continuous. Suppose $ \iota : (Y,T')\to (X,T)$ is continuous. For $ U\in T $, $ \inv\iota(U)\in T' $ so $ T\mid_Y\subseteq T' $.\qed
\par
A further point of view, a function $ f:(z,T_z)\to (Y,T\mid_Y) $ is continuous if and only if $ \iota \circ f$ is continuous. 
\begin{lemma}
	(Gluing Lemma) Let $ f:X\to Y $ be a function between topological spaces.
	\begin{enumerate}
		\item If $ \{U_\alpha\}_{\alpha\in I} $ are open subsets which cover $ X $ and each $ f\mid_{U_\alpha}:U_\alpha\to Y $ are continuous (where $ U_\alpha $ is given the subspace topology) then $ f $ is continuous.
	\item If $ \{C_\alpha\}_{\alpha\in I} $ is a finite collection of closed sets containing $ X $ and $ f\mid_{C_\alpha}:C_\alpha\to Y $ is continuous for each $ a\in I $ then $ f $ is continuous.
	\end{enumerate}
\end{lemma}
\pf Let $ V\subseteq Y $ be open. We want to show that $ \inv f(V) $ is open. We know that
\begin{align*}
	\inv f(V) = \inv(V) \cap X &= \inv f(V)\cap\left(\bigcup_{\alpha\in I}U_\alpha\right)\\
				   &= \bigcup_{\alpha\in I}\inv f(V)\cap U_\alpha
\end{align*}
Since $ f\mid_{U_\alpha} $ are continuous, we have that $ \inv f\mid_{U_\alpha} $ is open in $ U_\alpha $ in the subspace topology. So there exists a $ W $ open in $ X $ such that $ \inv f\mid_{U_\alpha}(V) = U_\alpha \cap W $ hence this is the intersection on open subsets of $ X $ so is open in $ X $, hence since the union of open subsets is open $ \inv f(V) $ is open, so $ f $ continuous.\par
The second part can be proved the same using the closed set definition of continuity.\qed
\par
If $ (X,d) $ is a metric space with topology $ T_d $ and $ Y\subseteq X $ then $ T_d\mid_Y $ is the topology induced by $ d\mid_Y $.\par
\subsubsection{The quotient topology}
\begin{definition}
	(Quotient topology) Let $ (X,T_X) $ be a topological space, $ \sim $ an equivalence relation on $ X $ and $ X/\sim $ is the set of equivalence classes, and $ \pi:X\to X/\sim $ the equivalence map. The \textit{quotient toplogy} on $ X/\sim $ is
	\[
		T_{X/\sim} = \{U\subset X/\sim \mid \inv\pi (U)\in T_X\}.
	\]
\end{definition}
\begin{proposition}
	$ T_{X/\sim} $ is indeed a topology.
\end{proposition}
\pf $ \emptyset = \inv\pi(\emptyset)\in T_X $ so $ \emptyset \in T_{X/\sim} $. $ X=\inv\pi(X/\sim)\in T_X $ so $ X/\sim \in T_{X/\sim} $. Let $ \{U_\alpha\} $ be a collection of sets of $ T_{X/\sim} $, then
\[
	\inv\pi\left(\bigcup_{\alpha\in I}U_\alpha\right) = \bigcup_{\alpha\in I}\inv \pi (U_\alpha),
\]
and $ \inv\pi(U_\alpha)\in T_X $, so the union is too. Hence $ \bigcup_{\alpha\in I}U_\alpha\in T_{X/\sim} $. We have a similar proof for finite intersections. \qed
\begin{proposition}
	The quotient map $ \pi:(X,T_X)\to (X/\sim, T_{X/\sim}) $ is continuous and $ T_{X/\sim} $ is the finest topology for which this is true.
\end{proposition}
\pf This is a tautology.\qed
\par
An alternative characterisation of the quotient topology is that $ f:X/\sim \to Y $ is continuous if and only if $ f\circ \pi:X\to Y $ is continuous.
\begin{definition}
	For a continuous function $ g:(X,T_X)\to (Y,T_Y) $ is a \textit{quotient map} if it surjective and $ U\in T_Y \iff \inv g(Y)\in T_X$.\par
	Given, this construct $ \sim $ on $ X $ by $ x\sim x' \iff g(x)=g(x')$. There is an induced function $ G:X/\sim \to Y $ sending $ G([x]) = g(x) $. 
\end{definition}
\begin{remark}
  This function $ G $ is a bijection and continuous with a continuous inverse. This means that $ G $ is a homeomorphism, so $ X/\sim\ \cong Y $.
\end{remark}
Let's see an example on $ \R $. Consider $ x\sim y\iff x-y\in \Z $. What is $ \R/\sim $? Consider $ f:\R\to \R^2 $ defined by $ x\to(\sin(2\pi x), \cos(2\pi x)) $. This is a continuous map so $ f:\R\to S^1\subseteq \R^2 $ is also continuous and surjective. By periodicity $ x\sim y\iff f(x)=f(y) $, so we get $ F:\R/\sim \to S^1 $ which we can check is a homeomorphism.
\par
Now take the example $ X=\R\times \{0,1\}\subseteq \R^2 $ with the standard subspace topology. Let $ (x,i)\sim (y,j)\iff (x,i)=(y,j) $ or $ x=y\ne 0 $. We can then think of $ X/\sim $ is a line with two origins. We cannot draw $ X/\sim $ since it is not Hausdorff. Any neighbourhood of $ [(0,0)]_\sim $ intersects any neighbourhood of $ [(1,0)]_\sim $ so not Hausdorff. Hence it is not subspace of any Euclidean space.
\subsubsection{The product topology}
For sets $ X,Y $ the projections functions are
\begin{align*}
	\pi_X : &X\times Y \to X\\
	      &(x,y) \to x
\end{align*}
and
\begin{align*}
	\pi_Y : &X\times Y \to Y\\
	      &(x,y) \to y
\end{align*}
\begin{definition}
	(Product topology) Let $ (X,T_X) $ and $ (Y,T_Y) $ be topological spaces. Then \textit{product topology} on $ X\times Y $ consists of open sets $ U\subseteq X\times Y $ such that for $ (x,y)\in U $ there is a $ V\in T_X $ and $ W\in T_Y $ such that $ (x,y)\in V\times W\in U $.
\end{definition}
\begin{proposition}
	This indeed is a topology and the sets $ V\times W $ are a basis for $ T_{X\times Y} $.
\end{proposition}
\pf Tautologically, we have that $ \emptyset \in T_{X\times Y} $. Taking $ V=X,W=Y $ we have that $ X\times Y\in T_{X\times Y} $. For a collection $ \{U_\alpha\}_{\alpha\in I} $ of elements of $ T_{X\times Y} $, let $ (x,y)\in \bigcup_{\alpha\in I}U_\alpha $. Then $ (x,y)\in U_\beta $ for \some $ \beta\in I $ so there exists neighbourhoods of $ x,y $ with their product a subset of $ U_\beta\subseteq \bigcup_{\alpha\in I} U_\alpha \in T_{X\times Y}$.  If $ I $ is finite and $ (x,y)\in \bigcap_{\alpha\in I}U_\alpha $. Then $ (x,y)\in V_\alpha\times W_\alpha \subseteq U_\alpha $ for each $ \alpha\in I $. So $ (x,y)\in \left(\bigcap_\alpha V_\alpha\right)\times\left(\bigcap_\alpha W_\alpha\right)\in \bigcap_\alpha U_\alpha $ and since these intersections are finite, these intersections are open.\qed
\begin{proposition}
  The projection maps 
  \[
	  \pi_X:(X\times Y, T_{X\times Y})\to (X,T_{X})\qquad \pi_Y:(X\times Y,T_{X\times Y})\to (Y,T_Y)
  \]
  are continuous and $ T_{X\times Y} $ is the coarsest topology for which this is true.
\end{proposition}
\pf Let $ V\in T_X $. Then $ \inv \pi_X(V) = V\times Y $, so this is open. Hence $ \pi_X,\pi_Y $ are continuous.\par
Suppose that $ T' $ is a topology on $ X\times Y $ such that $ \pi_X $ and $ \pi_Y $ are continuous, then $ \inv \pi_X(V)=V\times Y $ is open and $ \inv \pi_Y(W)=X\times W $ is open. So $ V\times W $ is open in $ T' $, so $ T_{X\times Y}\subseteq T' $.\qed 
\par
The universal property of the product topology is that the function
\[
	f:(Z,T_Z)\to (X\times Y,T_{X\times Y})
\]
is continuous if and only if $ \pi_X\circ f:(Z,T_Z)\to (X,T_X) $ and $ \pi_Y\circ f:(Z,T_Z)\to (Y,T_Y) $ are continuous. Equivalently $ f $ is componentwise continuous if and only if it is componentwise continuous.\par
We know from IA Analysis I, if $ f:[0,1]\to \R $ is continuous, $ f(0)<0<f(1) $ then $ f(t) = 0 $ for some $ t\in [0,1] $. This is a statement about continuous functions, but also about the interval $ [0,1] $. For example if we change the interval to $ [0,\frac 12) \cup(\frac 12, 1] $ then this does not satisfy the intermediate value theorem. The property of the interval we're using is connectedness.
\begin{definition}
	(Disconnected) A topological space $ X $ is \textit{disconnected} if $ X = U \cup V $ for $ U,V $ disjoint nonempty open sets.
\end{definition}
\begin{definition}
	(Connected) A topological space is \textit{connected} if it is not disconnected.
\end{definition}
If $ X = U\cup V $ is disconnected, then $ U $ and $ V $ are both open and also both closed.\par
Any set with the coarse topology is connected, due to the lack of non-trivial open sets. A set with the discrete topology is disconnected, if it has more than $ 1 $ point, since every set is open, so the result is trivial.\par
The set $ X = [0,\frac 12) \cup (\frac 12 ,1] \subseteq \R $ is disconnected since $ [0,\frac 12) $ is open in $ X $ and $ (\frac 12, 1] $ is open in $ X $ too. They are disjoint, hence $ X $ is disconnected.
\begin{proposition}
	A space $ X $ is disconnected if and only if, there is a continuous surjection $ f:X\to \{0,1\} $ where $ \{0,1\} $ is equipped with the discrete topology.
\end{proposition}
\pf Suppose that $ X $ is disconnected. So $ X = U\cup V $ disjoint. The define $ f$ such that
\[
  f(x) = \begin{cases}
	  0 & x\in U \\
	  1 & x\in V
  \end{cases}.
\]
This is well-defined since $ U $ and $ V $ are disjoint. Since $ U $ and $ V $ are non-empty, the function is surjective. The preimage of $ \{0\} $ and $ \{1\} $ are $ U $ and $ V $ respectively which we know is open. And the preimage of $ \{0,1\} $ and $ \emptyset $ are clearly open, so $ f $ is continuous.\par
Conversely suppose that $ f $ is continuous. Then define $ U = \inv f(\{0\}) $ and $ V = \int f(\{1\}) $. So since $ f $ is continuous, $ U $ and $ V $ are open. Clearly $ U $ and $ V $ is disjoint and non-empty since $ f $ is surjective. We have that $ X = U\cup V $ since $ X = \inv f(\{0,1\})= \inv f(0) \cup \inv f(1) = U \cup V $.\qed
\begin{theorem}
	The spaces $ [0,1] $, $ [0,1) $, $ (0,1) $ are all connected.
\end{theorem}
\pf Let's just consider $ [0,1] $, the rest of the proves are similar. If it was disconnected, then there is a continuous surjection
\[
	f:[0,1]\to \{0,1\}\subseteq \R.
\]
Then
\[
	f(\cdot)-\frac 12: [0,1]\to \R
\]
is continuous and takes the values $ \pm\frac 12 $ only. By the intermediate value theorem, we should have that $ f $ takes the value $ 0 $ which is a contradiction hence $ [0,1] $ is connected.\qed
\begin{theorem}
	(Generalised intermediate value theorem) Let $ X $ be a connected topological space and $ f:X\to \R $ continuous. If there exists $ x_0,x_1\in X $ such that $ f(x_0)< 0 < f(x_1) $ then there exists a $ x_2\in X $ such that $ f(x_2) = 0 $. 
\end{theorem}
\pf Consider the open sets $ U = \inv f((-\infty, 0)), V=\inv f((0,\infty)) $. $ f $ is continuous, so $ U,V $ are open. We know that $ x_0,x_1 $ exist hence $ U,V $ are non-empty. If $ f(x) $ is never zero, then $ X=U\cup V $ disjoint and open so $ X $ is disconnected. But $ X $ is connected hence $ \inv f(0) $ is non-empty, so pick $ x_2\in \inv f(0) $, so $ f(x_2) =0  $.\qed
\begin{proposition}
  Let $ f:X\to Y $ be a continuous surjection. Then $ X	$ connected implies that $ Y $ is connected.
\end{proposition}
\pf Let's show the contrapositive. Suppose that $ Y $ is disconnected. Then we have some $ h:Y\to \{0,1\} $ continuous and surjective. So
\[
	h\circ f:X\to \{0,1\}
\]
is also continuous and surjective, hence $ X $ is disconnceted.\qed
\begin{corollary}
  If $ X $ is connected and $ f: X\to Y $ is continuous then $ \ima(f) $ is connected.
\end{corollary}
\pf Apply the proposition to $ f:X\to\ima f $.
\par
For example if $ X $ is a connected space and $ \sim $ is an equivalence relation then $ \pi :X\to X/\sim $ is a continuous surjection so $ X/\sim $ is connected.
\begin{lemma}
  If $ f:X\to Y $ is a homeomorphism and $ Z\subseteq X $, then $ f\mid_Z:Z\to \ima(f\mid_Z) $ is a homeomorphism.
\end{lemma}
\pf Obvious.\qed
Let's use this to show that $ [0,1] $ is not homeomorphic to $ (0,1) $. Suppose they are. So we have a homeomorphism $ f:[0,1]\to (0,1) $. Let's now restrict $ f $ to $ (0,1] $. Then by the lemma we know that $ f\mid_{(0,1]} $ is a homeomorphism with
\[
	f\mid_{(0,1]}: (0,1]\to (0,1)\setminus \{f(0)\}
\]
for some $ 0<f(0)<1 $. But $ (0,1] $ is connected and $ (0,1)\setminus \{f(0)\}=(0,f(0))\cup (f(0),1) $ so $ (0,1)\setminus\{f(0)\} $ is disconnected which is a contradiction.\par
We can do a similar process to show that $ S^1 $ is not homeomorphic to $ \R $. We know that $ S^1 $ is connected since it is a quotient space of $ \R $ and $ \R $ is connected since $ \R\cong (0,1) $. Suppose that $ S^1 $ is homeomorphic to $ \R $. Then remove the point $ (1,0)\in S^1 $ and consider the restricted homeomorphism between the new spaces. $ \R $ is no longer connected since $ \R\setminus \{f(1,0)\} = (-\infty, f(1,0))\cup (f(1,0),\infty) $, but $ S^1\setminus \{f(1,0)\} $ is connected since it's homeomorphic to $ (0,1) $.






















\end{document}
