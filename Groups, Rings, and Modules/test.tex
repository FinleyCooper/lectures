\documentclass{article}
\usepackage{../header}
\title{Groups, Rings, and Modules}
\author{Notes made by Finley Cooper}

% Course specific commands
\newcommand{\nrm}{\triangleleft}
\DeclareMathOperator{\sym}{Sym}
\DeclareMathOperator{\orb}{orb}
\DeclareMathOperator{\stab}{stab}
\DeclareMathOperator{\aut}{Aut}
\DeclareMathOperator{\ccl}{ccl}
\DeclareMathOperator{\syl}{Syl}


\begin{document}
  \maketitle
  \newpage
  \tableofcontents
  \newpage
  \section{Review of IA Groups}
  \subsection{Definitions}
  We'll start with some simple definitions covered in IA Groups
  \begin{definition}
	  A group is a \textit{triple}, $ (G,\circ,e) $ consisting of a set $ G $, a binary operation $ \circ:G\times G\rightarrow G $ and an identity element $ e\in G $ where we have the following three properties,
          \begin{itemize}	  
		  \item $ \forall a,b,c\in G, (a\circ b) \circ c =a\circ(b\circ c) $
	          \item $ \forall a\in G, a\circ e = e\circ a = a $
		  \item $ \forall a\in G, \exists \inv a \in G, a \circ \inv a = \inv a \circ a = e $
          \end{itemize}
  We say that the \textit{order} of the group $ (G,\circ, e) $ is the size of the set $ G $	
  \end{definition}

\begin{proposition}
  Inverses are unique.
\end{proposition}
\textit{Proof.} Basic algebraic manipulation, covered in Part IA Groups.
\begin{definition}
	If $ G $ is a group, then a subset $ H\subseteq G $ is a \testit{subgroup} if the following hold,
\begin{itemize}
	\item $ e\in H $
	\item If $ a,b\in H $ then $ a\circ b\in H $
\item $ (H,\circ, e) $ forms a group.
\end{itemize}
\end{definition}
Now we'll give simple test for a subset being a subgroup
\begin{lemma}
  A non-empty subset, $ H $, of a group $ G $ is a subgroup if and only if $ \forall h_1,h_2\in H $ we have that $ h_1 \inv h_2\in H $
\end{lemma}
\textit{Proof.} Again covered in Part IA Groups
\begin{definition}
  A group $ G $ is abelian if $ \forall g_1,g_2\in G $ we have that $ g_1g_2=g_2g_1 $
\end{definition}
Let's look at some examples of groups.
\begin{itemize}
	\item The integers under addition, $ (\mathbb Z, +) $
	\item The integers modulo $ n $ under addition $ (\mathbb Z_n, +_n) $
	\item The rational numbers under addition $ (\mathbb Q, +) $
	\item The set of all bijections from $ \{1,\cdots,n\} $ to itself with the operation given by functional composition, $ S_n $
	\item The set of all bijections from a set $ X $ to itself under functional composition is a group $ \mathrm{Sym}(X) $
	\item The dihedral group, $ D_{2n} $ the set of symmetries of the regular $ n $-gon
\item The general linear group over $ \mathbb R $, $ \mathrm{GL}(n,\mathbb R) $, is the set of functions from $ \mathbb R\rightarrow \mathbb R $ which are linear and invertiable. Or we can think of the group as the set of $ n\times n $ invertiable matrices under matrix multiplication. We can view this group as a subgroup of $ \mathrm{Sym}(\mathbb R^n) $
\item The subgroup of $ S_n $ which are even permutations, so can be written as a product of evenly many transpositions, $ A_n $
\item The subgroup of $ D_{2n} $ which are only the rotation symmetries which is denoted by $ C_n $
\item The subgroup of $ \mathrm {GL}(n,\mathbb R) $ of matrices which have determinate $ 1 $ which is $ \mathrm {SL}(n,\mathbb R) $
\item The Klein four-group, which is $ K_4=C_2\times C_2 $, the symmetries of the non-square rectangle
\item The quaternions, $ Q_8 $ with the elements $ \{\pm 1, \pm i, \pm j, \pm k\} $ with multiplication defined with $ ij =k, ji = -k $, $ i^2=j^2=k^2=-1 $
	\subsection{Cosets}
	\begin{definition}
		Let $ G $ be a group and $ g\in G $. Let $ H $ be a subgroup of $ G $. The \textit{left coset}, written as $ gH $ is the set $ \{gh : h\in H\} $
	\end{definition}
Some observations we can make are,
\begin{itemize}
	\item Since $ e\in H $ we have that $ g\in gH $. So every element is in some coset
	\item The cosets partition, so if $ gH\cap g'H\ne \emptyset $ then $ gH=g'H $
	\item The function, $ f: H\rightarrow gH $ defined by $ f(h)= gh $ is a bijection, so all cosets are the same size
\end{itemize}
\begin{theorem}
	(Lagrange's Theorem) If $ G $ is a finite group, then for a subgroup $ H $ of $ G $, $ |G|=|H||G:H| $, where $ |G:H| $ is the number of left cosets of $ H $ in $ G $
\end{theorem}
\textit{Proof.} Obvious from the observations we've just made.

\begin{definition}
	Let $ G $ be a group, and take some element $ g\in G $. We define the \textit{order} of $ g $ as the smallest positive integer $ n $, such that $ g^n = e $. If no such $ n $ exists, we say the order of $ g $ is infinite. We denote the order by \mathrm {ord}(g).
\end{definition}
\begin{proposition}
	Let $ G $ be a group and $ g\in G $. Then $ \mathrm {ord}(g) $ divides $ |G| $
\end{proposition}
\textit{Proof.} Let $ g\in G $. Consider the subset, $ H=\{e, g, g^2,\cdots, g^{n-1}\} $ where $ n $ is the order of $ g $. We claim $ H $ is a subgroup. $ e\in H $ so $ H  $ is non-empty. Observe that $ g^rg^{-s}=g^{r-s}\in H $ so we have that $ H\le G $. Elements are distinct since if $ g_i=g_j, i\ne j, 0\le i<j<n $ then $ g{j-i}=e $ which contradicts the minimality of $ n $ since $ 0\le j-i\le n $. We have that $ |H|=n $, so by Lagrange, $ |H| $ divides $ |G| $.\qed
\subsection{Normal subgroups}
When does $ gH=g'H $? Then $ g\in g'H $, so we have that $ \inv {g'}g\in H $. The converse also holds.
\begin{lemma}
	For a group $ G $ with $ g, g'\in G $ and subgroup $ H $ we have that $ gH=g'H $ if and only if $ \inv {g'} g\in H $
\end{lemma}
\textit{Proof.} In Part IA Groups\\\\ 
Let $ G/H = \{gH:g\in G\} $ be the set of left cosets. This partitions $ G $. Does $ G/H $ have a natural group structure?\\
We propose the formula that $ g_1H\cdot g_2H=(g_1g_2)\cdot H $ for a group law on $ G/H $.\\
We need to check well definedness of this proposed formula.\\
\textit{Case 1:} Suppose that $ g_2H=g_2'H. $ Then $ g_2'=g_2h $ for some $ h\in H $. $ (g_1H)\cdot (g_2'H)=g_1g_2'H $ by the proposed formula. By the previous relation this is $ g_1g_2hH=g_1g_2H $.\\\\
\textit{Case 2:} Suppose that $ g_1H=g_1'H $ we have that $ g_1'=g_1h $ for some $ h\in H $. We need $ g_1g_2 H = \underbrace{g_1h}_{g'_1}g_2H $. Equivalently we need that $ \inv{(g_1g_2)}g_1hg_2\in H $. Or equivalently still, $ \inv{g_2}hg_2\in H $ for all $ g_2 $ and $ h $. This the definition of normality.
\begin{definition}
		(Normality) A subgroup $ H\le G $ is \textit{normal} if $ \forall g \in G $, $ h\in H $, we have that $ gh\inv g\in H $
\end{definition}
If $ H\le G $ is normal we write that $ H \triangleleft G $.
\begin{definition}
	(Quotient) Let $ H\triangleleft G $. The \textit{quotient group} is the set $ (G/H, \cdot, e=eH) $ where $ \cdot:G/H\times G/H \rightarrow G/H $ by $ (g_1H,g_2H)\rightarrow (g_1g_2)H $.
\end{definition}

\begin{definition}
	(Homomorphism) Let $ G $ and $ H $ be groups. A \textit{homomorphism} is a function $ f: G\rightarrow H $ such that for all $ g_1,g_2\in G $ we have that $ f(g_1g_2)=f(g_1)f(g_2) $
\end{definition}
This is a very constrained condition. For example $ f(e_G)=e_H $ always. To see this, observe $ e_G=e_Ge_G $, so we have that $ f(e_G)=f(e_G)f(e_G) $ so $ f(e_G)=e_H $ by multiplying by $ \inv{f(e_G)} $.
\begin{lemma}
	If $ f:G\rightarrow H $ is a homomorphism. Then $ f(\inv g)=\inv{f(g)} $
\end{lemma}
\textit{Proof.} Calculate $ f(g\inv g) $ in two ways.\\
	In the first way $ f(g\inv g)=f(e)=e, $ in the second way $ f(g\inv g)=f(g)f(\inv g). $\\
Equating gives that $ f(\inv g)=\inv{f(g)}. $\qed
\begin{definition}
	Let $ f:G\rightarrow H $ be a homomorphism. The \textit{kernal} of $ f $ is $ \ker f =\{g\in G: f(g)=e\} $. The \textit{image} of $ f $ is $ \ima f=\{ h\in H: h=f(g)\text{ for some } g\in G\} $.
	\end{definition}
\begin{proposition}
	Let $ f:G\rightarrow H $ be a homomorphism. Then $ \ker f \triangleleft G $ and $ \ima f \le H.$
\end{proposition}
\textit{Proof.} First let's proof that $ \ker f $ is a subgroup by the subgroup test. Observe by the lemma that $ e\in\ker f. $. If $ x,y \in\ker f $, then $ f(x\inv y)=f(x)\inv{f(y)}=e\implies x\inv y\in \ker f $. For normality, let $ x\in G $ and $ g\in\ker f $. Calculate $ f(xg\inv x)=f(x)f(g)\inv{f(x)} $. But $ f(g)=e $. So we just get the identity. Hence we have that $ xg\inv x\in\ker f. $ So $ ker f\nrm G $.\\
	To check that the $ \ima f \le H $, take $ a,b\in \ima f $, say that $ a=f(x), b=f(y) $. Then $ a\inv b=f(x)\inv{f(y)}=f(x\inv y) $. But $ x\inv y\in G $ so $ f(x\inv y)\in \ima f $. Also $ e\in \ima f $, so we have that $ \ima f\le H $.\qed
\begin{definition}
	(Isomorphism) A homomorphism $ f:G\rightarrow H $ is an \textit{isomorphism} if it is a bijection. Two groups are called \textit{isomorphic} if there exists an isomorphism between them.
\end{definition}
\begin{theorem}
	(First isomorphism theorem) Let $ f:G\to H $ be a homomorphism. Then $ \ker f $ is normal, and the function $ \varphi:G/\ker f\rightarrow \ima f $, by $ \varphi(g\ker f)=f(g) $, is a well-defined, isomorphism of groups.
\end{theorem}
\pf Already shown $ \ker f\nrm G $. Consider whenever $ \varphi $ is well-defined. Suppose that $ g\ker f=g'\ker f. $ Need to check $ \varphi(g\ker f)=\varphi(g'\ker f). $ We know that $ g\inv{g'}\in \ker f $, so $ f(g'\inv g)=e\iff f(g')=f(g) $. To see that $ \varphi $ is a homomorphism: $ \varphi(g\ker f g'\ker f)=\varphi(gg'\ker f)=f(gg')=f(g)f(g')=\varphi(g\ker f)\varphi(g'\ker f) $. So $ \varphi $ is a homomorphism.\\\\
Finally let's check $ \varphi $ is bijective. First for surjectivity, let $ h\in \ima f $, then $ h=f(g) $ for some $ g\in G $. So we have that $ h=\varphi(g\ker f) $.\\
Now for injectivity, $ \varphi(g\ker f)=\varphi(g'\ker f)\implies f(g)=f(g') \implies g'\inv g\in\ker f $. Hence the cosets are the same by the coset equality criterion, so we have that $ g\ker f=g'\ker f $, hence we have injectivity, so $ \varphi $ is an isomorphism.\\
\\
For an example of this theorem, consider the groups $ (\mathbb C, +) $ and $ (\mathbb C^*, \times) $ related by the homomorphism, $ \varphi(z)= e^z$. The kernal of $ \exp $ is exactly, $ 2\pi i \mathbb Z\le \mathbb C $, so the first isomorphism theorem gives that $ \frac{\mathbb C}{2\pi i \mathbb Z}\cong \mathbb C^* $. \textit{(Try to visualise this!)}\qed 
\begin{theorem}
	(Second isomorphism theorem) Let $ H\le G $ and $ K\nrm G $. Then $ HK=\{hk : h\in H, k\in K\} $ is a subgroup of $ G $, the set $ H\cap K$ is normal in H, and $ \frac{HK}K\cong \frac H{H\cap K} $.
\end{theorem}
\pf We take the statements in turn. First we can see that $ HK $ is a subgroup. Clearly it contains the identity, and take some $ x,y\in HK $, $ x=hk, y=h'k' $. We will show that $ y\inv x\in HK $. Observe that $ y\inv x = h'k'\inv k \inv h = h'(\inv h h)(k' \inv k)\inv h = (h'\inv h)h\underbrace{(k'\inv k)}_{k''}\inv h $. But we have that $ hk''\inv h \in K $ by the normality of $ K $, hence $ y\inv x \in HK $. So we have that $ HK\le G $.\\\\
Now we prove that $ H\cap K\nrm G $. Consider the homomorphism, $ \varphi: H\to G/K $, defined as $ \varphi(h)=hK $. This is a well defined homomorphism for the same reason that the group structure $ G/K $ is well-defined. The kernal of $ \varphi $, is $ \ker \varphi = \{h: hK=K\}=\{h:h\in K\}=H\cap K\nrm G $.\\\\
Now finally we're left to prove the isomorphism. Now apply the first isomorphism theorem to $ \varphi $. This tells us that $ \frac H{\ker \varphi}=\frac H {H\cap K}\cong \ima \varphi $. The image of the $ \varphi $ is exactly those coests of $ K $ in $ G $ that can be represented as $ hK $ which is exactly $ \frac{HK}K $.\qed\\
\begin{theorem}
	(Correspondence theorem). Consider a group $ G $ with $ K\nrm G $, with the homomorphism $ p:G\to G/K $, by $ p(g)=gK $. Then there is a bijection between the subgroups of $ G $ which contain $ K $ and the subgroups of $ G/K $.
\end{theorem}
\pf For some subgroup $ L $, we have $ K\nrm L \le G $, and we map $ L $ to $ L/K $, so we have that $ L/K\le G/K $. In the reverse direction, for a subgroup $ A\le G/K $, we map it to $ \{g\in G: gK\in A\} $.\\
We can think of this as taking $ L\to p(L) $ and $ \inv p(A)\leftarrow A $.
\\\\ Now we will state some facts without proof. (Although the proofs are fairly straightforward).
\begin{itemize}
	\item This is a bijection.
	\item This correspondence maps normal subgroups to normal subgroups.
\end{itemize}
\begin{theorem}
	(Third isomorphism theorem) Let $ K,L $ be normal subgroups of $ G $ with $ K\le L\le G $. Then we have that $ \frac{G/K}{L/K}\cong \frac GL $.
\end{theorem}
\pf Define a map $ \varphi: G/K \rightarrow G/L $, by $ \varphi(gK)=gL $. First we'll show that $ \varphi $ is a well-defined homomorphism, then we'll calculate the image and kernal, and finally apply the first isomophism theorem. To see well-definedness, if $ gK=g'K $, then $ g'\inv g\in K\subseteq L $, so $ g'L=gL $, so $ \varphi $ is well-defined. Obviously a homomorphism.\\
The kernal of $ \varphi $ is $ \ker \varphi = \{gK:gL=L\} =\{gK: g\in L\} = L/K $. $ \varphi $ is clearly surjective, so we conclude by the first isomorphism theorem that $ \frac{G/K}{L/K}\cong \frac GL $.\qed
\begin{definition}
	(Simple groups) A group $ G $ is called \textit{simple} if the only normal subgroups are $ G $ itself and $ \{ e \} $.
\end{definition}
\begin{proposition}
  Let $ G $ be an abelian group. Then $ G $ is simple if and only if  $ G\cong C_p $, for $ p $ prime.
\end{proposition}
\pf If $ G\cong C_p, $ then any $ g\in G, g\ne e $ is a generator of $ G $ by Lagrange. Conversely if $ G $ is simple and abelian, then take some non-identity, $ g\in G $, then $ \{g^n :n\in \mathbb Z\} $ is a subgroup, and because $ G $ is abelian, this subgroup is normal. Since $ g\ne e $, we must have $ G $ is cyclic, generated by $ g $. Now if $ G $ is infinitely cyclic, then $ G\cong\mathbb Z $, which is not simple since $ 2\mathbb Z\nrm \mathbb Z $, so we can't have this. Therefore $ G\cong C_m $ for some $ m\in \mathbb Z_{>0} $. Say $ q $ divides $ m $, then the subgroup of $ G $ generated by $ g^{\frac mq} $ is a normal subgroup, so we must have that $ q=m $ or $ q=1 $ by simplicity, hence we have that $ m $ is prime.\qed
\begin{theorem}
	(Composition series) Let $ G $ be a finite group. Then there exists subgroups such that, $ G=H_1\triangleright H_2\triangleright H_3\triangleright\cdots \triangleright H_n=\{e\} $, such that $ \frac {H_i}{H_{i+1}} $ is simple.
\end{theorem}
\pf If $ G $ is simple then take $ H_2=\{e\} $ and we're done. Otherwise, let $ H_2 $ be a proper normal subgroup of maximal order in $ G $. We claim that $ G/H_2 $ is simple. To see this, suppose not and consider $ \varphi: G\to G/H_2 $.
 By non-simplicity and correspondence between normal subgroups, we find a proper normal in $ G/H_2 $ and therefore a proper normal $ K\nrm G $. This leads to a contradiction as $ K $ contains $ H_2 $ non-trivally, so we contradict maximality, so $ G/H_2 $ is simple. Now we continue by replacing $ G $ with $ H_2 $ and iterate the process. Either we get that $ H_2 $ simple and we're done again, or we get find a proper normal subgroup $ H_3\nrm H_2 $ of maximal order. This process must terminate, since $ G $ is finite and the order is strictly decreasing in each step.\qed\\\\
 We know from Part IA groups that $ A_5 $ is simple. We see a series like this for $ S_5 $, namely, $ S_5\triangleright A_5\triangleright \{e\} $.
 \subsection{Groups actions and permutations}

\begin{definition}
	Let $ X $ be a set. Let $ \sym(x) $ denote the symmetric group of $ X $ and $ S_n=\sym([n]) $ where we have that $ [n]=\{1,2,\dots, n\} $.
\end{definition}
Reminders from IA Groups:
\begin{itemize}
	\item We can write any $ \sigma\in S_n $ as a product of disjoint cycles.
	\item If $ \sigma \in S_n $ we can write $ \sigma $ as a product of transpositions. The number of transpositions needed to write $ \sigma $ is well-defined modulo 2. This is called the sign of the transposition, denoted by $ \mathrm {sgn} $, where $ \mathrm{sgn}: S_n\to \{\pm 1\} $.
	\item \mathrm{sgn} is a homomorphism between the groups where $ \{\pm 1\} $ is given the unique group structure. When $ n\ge 3 $, the homomorphism is surjective.
\end{itemize}
\begin{definition}
	(Alternating group) The \textit{alternating group} $ A_n $ is the kernal of $ \mathrm{sgn} $.
\end{definition}
A homomorphism $ \varphi: G\to \sym(X) $ is called a permutation representation of $ G $.
\begin{definition}
	(Group action) An \textit{action} of $ G $ on a set $ X $ is a function $ \tau:G\times X\to X $ sending $ (g,x) \to \tau(g,x)\in X $ such that $ \tau(e,x)=x, \forall x\in X $, and $ \tau(g_1,\tau(g_2,x))=\tau(g_1g_2,x), \forall g_1g_2\in G, \forall x\in X $.
\end{definition}
How are actions and permmutation representations related?\\
For some homomorphism, $ \varphi: G\to\sym(X) $ we map the homomorphism to $ a(\varphi):G\times X\to X $, where $ (g,x)\to\varphi(g)(x) $.
\begin{proposition}
  The funtion $ a $ above is a bijection from the set of homomorphism from $ G\to\sym(X) $ to the set of actions from $ G $ on $ X $.
\end{proposition}
\pf We'll construct an inverse of $ a $. Given a group action $ *:G\times X \to X $. Define $ \varphi(*): G\to \sym(X) $ defined by sending $ g\to\varphi(*)(g) $, where $ \varphi(*)(g)(x)=g*x $. We aim to show that $ \varphi(*)(g):X\to X $ is a permutation. We have an inverse $ \varphi(*)(\inv g) $, and to see that it is a homomorphism $ \varphi(*)(g_1)\varphi(*)(g_2)(x)=g_1*(g_2*x)=(g_1g_2)*x=\varphi(*)(g_1g_2)(x) $. This is true for all $ x $, so the construction is a group homomorphism.\qed\\
\\  \textit{Notation}: Given a group action $ G $ acting on $ X $ given by $ \varphi:G\to\sym(X) $, denote $ G^X=\ima(\varphi) $, and $ G_X=\ker(\varphi) $. By the first isomorphism theorem we have that $ G_X\nrm G $ and $ G/G_X\cong G^X $.\\\\
For an example, consider the unit cube. Let $ G $ be the symmetric group it. Now let $ X $ be the set of (body) diagonals of the cube. Any element of $ G $ sends a diagonal to another diagonal, we get an action $ G\to\Sym(X)\cong S_4 $. The kernal $ G_X=\ker(\varphi)=\{\id, \text{ send each vertex to its opposite} \} $. Easy exercise to check that any diagonal can be sent to any other diagonal, so $ G^X=\ima (\varphi)=\sym(X) $. So by the first isomorphism theorem, we have that $ S_4\cong G^X\cong G/G_X\implies \frac{|G|}2 = 4!\implies |G|=48 $.\\\\
For the next example let's look at a group acting on itself. Let $ G $ act on itself by $ G\times G\to G $, sending $ (g,g_1)\to gg_1 $. This gives a homomorphism $ G\to \sym(G) $ (easy to check that $ \varphi $ is injective since the kernal is trival). By the first isomorphism theorem we get that every group is isomorphism to a subgroup of a symmetric group (Cayley's theorem).\\\\
Now let $ H\le G $ and let $ X=G/H $, let $ G $ act on $ X $ by $ g*g_1H=gg_1 H $. We get $ \varphi G\to\sym(X) $. Consider $ G_X=\ker \varphi $. If $ g\in G_X $, then $ gg_1H=g_1H, \forall g_1\in G $, so $ \inv {g_1}gg_1 H=H\implies G_X\subseteq \bigcap_{g_1\in G} g_1H\inv{g_1}.$ This argument is completely reversible, so if $ g\in\bigcap_{g_1}g_1H\inv g_1, $ then for each $ g_1\in G, $ we have \inv{g_1}gg_1\in H, so $ g\in G_X\implies G_X=\bigcap_{g_1\in G}g_1H\inv{g_1} $. Since $ G_X $ is a kernal and is a subset of $ H $, we've got a way of making $ H $ smaller and making it normal. This is the largest normal subgroup contained in $ H $.
\begin{theorem}
	  Let $ G $ be finite and $ H\le G $ of index $ n $. There exists a normal subgroup of $ G $, $ K\nrm G $, with $ K\le H $, such that $ G/K $ is isomorphic to a subgroup of $ S_n $. Thus, $ |G/K| $ divides $ n! $, and $ |G/K|\ge n$.
\end{theorem}
\pf Consider $ G $ acting on $ G/H $ in the previous example. So the kernal of $ \varphi:G\to\sym(G/H) $ is normal, denote it by $ K $. We've shown it is contained by $ H $. First isomorphism theorem gives that $ G/K\cong \ima(\varphi)\le Sym(X)\cong S_n $. Give that $ |G/K| $ divides $ n! $ by Lagrange. Since that $ K\le H, $ we have that $ |G/K|\ge |G/H|\implies |G/K|\ge n $.\qed
\begin{corollary}
	  Let $ G $ be non-abelian and simple. Let $ H\le G $ be a proper subgroup of index $ n>1 $. Then $ G $ is isomorphism to a subgroup $ A_n $. Moreover, $ n\ge 5, $ i.e. no subgroup of index less than $ 5 $.
\end{corollary}
\pf Action of $ G $ on the set $ X=G/H $ gives a homomorphism $ \varphi:G\to \sym(X)\cong S_n $. Since the kernal is normal, since $ G $ is simple it is either $ G $ or $ \{e\} $. Since $ H $ is a proper subgroup, for some $ g\in G $, $ gH\ne H $, so we must have that $ \ker \varphi=\{e\} $. So $ G\cong \ima\varphi\le S_n $. Now we want to show that $ \ima\varphi\le A_n $. To see this observe that $ A_n\nrm S_n $. Consider $ A_n\cap \ima\varphi\le \ima\varphi $. By the second isomorphism theorem, $ \ima\varphi\cap A_n\nrm \ima\varphi\implies\ima\varphi\cap A_n=\{e\} $ or $ \ima\varphi $ itself. By the rest of the second isomorphism theorem, if $ \ima\varphi\cap A_n=\{e\}\implies \ima\varphi\cong\frac{\ima\varphi}{\ima\varphi\cap A_n}\cong \frac{\ima\varphi A_n}{A_n}\le \frac{S_n}{A_n} \cong C_2 $, but $ G $ is non-abelian, so $ \ima \varphi $ is non-abelian, so we have a contradiction. So we have that $ \ima\varphi\cap A_n =\ima\varphi $, so $ \ima \varphi $ is a subgroup of $ A_n $.\\
For the next part of the corollary, $ S_1,S_2 $ are abelian and $ S_3,S_4 $ have no non-abelian simple subgroups, so we must have $ n\ge 5 $.\qed

\begin{definition}
	(Orbits and stabiliser) Let $ G $ act on some set $ X $. Then, the \textit{orbit} of $ x\in X $ is $ G\cdot x=\orb x=\{gx : g\in G\}\subseteq X $. And the \textit{stabiliser} of $ x\in X $ is $ G_x=\stab_G(x) = \{g\in G:gx=x\}\le G $.
\end{definition}
\begin{theorem}
	(Orbit-stabiliser) For a group $ G $ acting on a set $ X $. For all $ x\in X $, there is a bijection $ G\cdot x \to G/G_x $ given by $ g\cdot x \to gG_x $. In particular, if $ G $ is finite, then $ |G|=|G\cdot x||G_x|, \forall x\in X $.
\end{theorem}
\pf In the IA Groups course.
\subsection{Conjugacy, centralisers, and normalisers}
Let $ G $ be a group. The conjugation action of $ G $ acting on itself by $ G\times G\to G $, is $ (g,h)\to gh\inv g $. This is equivilent to a homomorphism $ G\to\sym(G) $.\\\\
Fix $ g\in G $. Then the permutation $ G\to G $ given by $ h\to gh\inv g $ is also a homomorphism.

\begin{definition}
	(Automorphism) Let $ G $ be a group. A permutation $ G\to G $ that is also a homomorphism is called an \textit{automorphism} of $ G $. The set of all automorphisms of $ G $, $ \aut(G) =\{f:G\to G: f \text{ is a automorphism}\}\subseteq \sym(G) $, is a subgroup, called the automorphism group of $ G $.
\end{definition}

\begin{definition}
	(Conjugacy classes and centralisers) Fix $ g\in G $. The \textit{conjugacy class} of $ g $ is the set $ \ccl_G(g) = \{hg\inv h: h\in G\} $, i.e it is the orbit under the conjugation action. The \textit{centraliser} of $ g\in G $ is $ C_G(g)=\{h\in G:hg\inv h=g\} $, i.e the stabiliser of $ g $ under the action.
\end{definition}
\begin{definition}
	(Centre) The \textit{centre} of $ G $ is $ Z(G)=\{z\in G:hz\inv h=z\forall h\in G\} $, i.e. it is the kernal of the conjugation action and the intersection of the centralisers.
\end{definition}
\begin{corollary}
	Let $ G $ be a finite group. Then $ |\ccl_G(x)|=|G:C_G(x)|=\frac{|G|}{|c_G(x)|} $.
\end{corollary}	
\pf Apply orbit-stabiliser to the conjugation action.
\begin{definition}
	(Normaliser) Let $ H\le G $. The \textit{normaliser} of $ H $ in $ G $ is $ N_G(H)=\{g\in G: gH\inv g = H\} $
\end{definition}
		We can see clearly that $ H\subseteq N_G(H) $ so $ N_G(H) $ is non-empty and we also have that $ N_G(H)\le G $.\\\\
In fact we have that $ N_G(H) $ is the largest subgroup containing $ H $ in which $ H $ is normal.
\subsection{Simplicity of $ A_n $ for $ n\ge 5 $}
Recall from Part IA groups that a conjugacy class in $ S_n $ consists of the set of all elements with a fixed cycle type.
\begin{theorem}
  Let $ n\ge 5 $. Then $ A_n $ is simple.
\end{theorem}
\pf We will prove the statement via these three claims:
\begin{itemize}
	\item $ A_n $ is generated by 3-cycles
	\item If $ H\nrm A_n $ that contains a 3-cycle then it contains all the 3-cycles
	\item Any non-trival $ H\nrm A_n $ contains a 3-cycle.
\end{itemize}
First we prove the first claim. Let $ g\in A_n $, when viewed in $ S_n $ it is the product of evenly many transposition. Consider a product of two transpositions:
\begin{itemize}
	\item $ (ab)(ab)=e\in A_n $
	\item $ (ab)(bc)=(abc)\in A_n $
	\item $ (ab)(cd)=(acb)(acd)\in A_n $.
\end{itemize}
In each case we can write all products of transpositions as a product of 3-cycles, hence we can write all elements in $ A_n $ as a product of 3-cycles.\\\\
Now for the second claim, any two 3-cycles in $ A_n $ are conjugate when viewed in $ S_n $. Let $ \delta, \delta' $ be 3-cycles and write $ \delta' =\sigma\delta\inv\sigma $, where $ \sigma\in S_n $. If $ \sigma $ is even, we're done since it's in $ A_n $. If $ \sigma $ is odd, observe since $ n\ge 5 $, there exists a transposition $ \tau $ disjoint from $ \delta $, now $ \delta'=\sigma(\tau\inv\tau)\delta\inv\sigma=(\sigma\tau)\delta\inv{(\sigma\tau)} $. Since $ \sigma\tau $ is even, we're done.\\\\
Finally for the last claim take some $ H\nrm A_n $ not trival. We break into cases
\begin{itemize}
	\item (a) If $ H $ contains an element on the form $ \sigma=(1\space 2\cdots r)\tau $ where $ \tau $ is disjoint from $ 1,\dots, r, $ and $ r\ge 4 $. Then let $ \delta = (1\space2\space3) $. Now consider $ \delta\sigma\inv\delta\in H $ (by normality). But then $ \inv\sigma\inv\delta\sigma\delta\in H $ as well. As $ \tau $ misses $ 1,2,3 $ and commutes with $ (1\space2\cdots r) $ we expand this: $ \inv\sigma\inv\delta\sigma\delta=(r\cdots 2\space 1)(1\space 3\space 2)(1\space 2\space 3\cdots r)(1\space 2\space 3)=(2\space 3\space r) $ so we find a 3-cycle.
	\item (b) Suppose $ H $ contains $ \sigma=(123)(456)\tau $ (or any relabeling of such). $ \tau $ is disjoint from $ 1,\cdots,6 $. Take $ \delta = (124) $ and calculate the conjugation $ \inv\sigma\inv\delta\sigma\delta=(124236) $ which is a 5-cycle so we're done by the first case.
	\item (c) Suppose that $ H $ contains $ \sigma $ of the form $ \sigma = (123)\tau $ where $ \tau $ is a product of disjoint transpositions. Note if $ \tau $ contains anything longer than a tranposition, we can just apply case (a) or (b). Then $ \sigma^2=(123)^2 $ which is a 3-cycle since the transpositions cancel.
	\item (d) Suppose that $ H $ contains $ \sigma=(12)(34)\tau $, where $ \tau $ is a product of transpositions. Let $ \delta = (123) $, consider $ \mu = \inv \sigma\inv\delta\sigma\delta=(14)(23) $. Let $ \nu = (152)\mu(125) = (13)(45) $. But observe that $ \mu\nu\in H $, but this is a 5-cycle, so we're done by case (a).
\end{itemize} 
Up to relabeling, we're covered all the cases. Hence any normal subgroup of $ A_5 $ must be trivial or $ A_5 $ itself, so $ A_5 $ is normal.\qed
\subsection{Finite $p$-groups}
\begin{definition}
	(Finite $ p $-groups) For $ p $ prime, a \textit{finite $ p $-group} is a group of order $ p^n $, $ n\in\N $.
\end{definition}

\begin{theorem}
  Let $ G $ be a finite $ p $-group. Then $ Z(G) $ is non-trival.
\end{theorem}
\pf Consider $ G $ acting on itself by conjugation. The centre of $ G $ is the union of orbits of size 1. The orbits partition $ G $, so
\[
	|G|=p^n=|Z(G)|+\sum\text{sizes of conjugacy classes of size} > 1  
\]
We know that the sizes of the non-trivial conjugacy classes always divide $ p^n $. So all the terms of size larger than one are divisible by $ p $. Hence we have that $ p $ divides $ |Z(G)| $. So since $ p\ge 2 $, the centre is non-trivial.\qed
\begin{theorem}
  A group of size $ p^2 $ must be abelian.
\end{theorem}
\pf Follows from an independently interesting technical result:
\begin{lemma}
	If $ G $ is any group and $ \frac G{Z(G)} $ is cyclic, then $ G $ is abelian.
\end{lemma}
\pf Let $ xZ(G) $ generate $ \frac G{Z(G)} $. Every coset of the form $ x^mZ(G), m\in Z $. Since any $ g\in G $ lies in some coset of $ Z(G) $, we can write $ g=x^mz $, for some $ z\in Z(G) $. Now for some $ g'\in G $, $ g'=x^nz' $, so $ gg'=x^mzx^nz'=x^{n+m}zz'=x^nz'x^mz=g'g $, so the group is abelian.\\\\
Our proof of the theorem follows since $ Z(G) $ is non-trivial, so it either has size $ p^2 $ or $ p $. If it has size $ p^2 $, the group is abelian so we're done. If it has size $ p $, the $ G/Z(G) $ also has size $ p $, so it's cyclic, hence it's abelian, so by the lemma we have that $ G $ is abelian.\qed
\begin{theorem}
  Let $ G $ be a group of size $ p^n $. Then for any $ 0\ge k \ge n$, $ G $ has a subgroup of size $ p^k $.
\end{theorem}
\pf (Inductive proof) The base case $ n=1 $ is clear because the group must be cyclic. Now suppose that $ n>1 $, if $ k=0 $, we take $ \{e\} $, so we're done, so assume that $ k\ge 1 $.  Note that $ Z(G) $ is non-trivial, let $ x\in Z(G) $ with $ x\ne e $. The order of $ x $ is a power of $ p $. By raising $ x $ to some power we can find an element with order $ p $ in $ Z(G) $. Replacing $ x $ with this element we can assume $ \mathrm{ord}(x)=p $. The subgroup generated by $ x $ is normal of size $ p $ because $ x $ is central of order $ p $. Now $ \frac{G}{\langle x\rangle} $ is a group of order $ p^{n-1} $ so inductive hypothesis allies. Let $ L\le \frac{G}{\langle x\rangle} $ of size $ p^{k-1} $. But by the subgroup correspondence result, we can find some $ K\le G $ containing $ \langle x\rangle $ such that $ \frac K{\langle x\rangle}=L $. So $ K $ has size $ p^k	 $, so we're done.\qed

\subsection {Finite abelian groups}
\begin{theorem}
	(Classification of finite abelian groups) Let $ G $ be a finite abelian group. There exists positive integers $ d_1,\cdots, d_r $ such that:
	\[
		G\cong C_{d_1}\times C_{d_2}\times \cdots\times C_{d_r}
	\]
	Moreover, we can choose $ d_i $ such that $ d_{i+1}\mid d_i $ in which case this is unique.
\end{theorem}
\pf To come later...\\\\
Abelian groups of order 8 are exactly $ C_8, C_4\times C_2, C_2\times C_2\times C_2 $.
\begin{lemma}
	(Chinese remainder theorem) If $ n $ and $ m $ are coprime, then $ C_n\times C_m\cong C_{nm} $
\end{lemma}
\pf Consider $ C_n\times C_m $. Suffices to produce an element of order $ nm $. Let $ g\in C_n $ and $ h\in C_m $ be generators of order $ n $ and $ m $ respectively. Consider $ (g,h) $. Say its order is $ k\implies (g,h)^k=(e,e) $. So $ n,m $ both divide $ k $, and since $ n,m $ are coprime we have that $ nm $ divides $ k $ and by Lagrange we have that $ k $ divides $ nm $, so we're done.\qed
\subsection{Sylow Theorems}
\begin{definition}
	(Sylow $ p $-subgroup) Let $ G $ be a finite group of order $ p^am$, where $ p\nmid m $, $ p $ is a prime. Then a \textit{Sylow p-subgroup} of $ G $ is a subgroup of size $ p^a $.
\end{definition}
\begin{theorem}
	(Sylow theorems) For a finite group $ G $ of order $ p^am $, where $ p\nmid m $, $ p $ is prime:
	\begin{itemize}
		\item The set $ \syl_p(G)=\{P\le G$ $ |$ $P \text{ is a Sylow p-subgroup of } G \}$ is non-empty.\\
		\item Any $ H,H'\in \syl_p(G) $ are conjugate, namely $ H=gH'\inv g $, for some $ g\in G $.\\
		\item If $ n_p=|\syl_p(G)| $ then $ n_p\equiv 1 \mod p $ \textit{and} $ n_p $ divides $ |G| $, so $ n_p\mid m $
	\end{itemize} 
\end{theorem}
Before we prove the statement, let's see why this theorem is useful.
\begin{lemma}
	If $ \syl_p(G)=\{P\} $, then $ P $ is normal in $ G $.
\end{lemma}
\pf For any $ g\in G $, the subgroup $ g P\inv g $ is isomorphic (as a group) to $ P $. So $ g P\inv g $ is in $ \syl_p(G) \implies gP\inv g=P$, which proves the claim.\qed
\begin{corollary}
	Let $ G $ be a non-abelian simple group, and $ p\mid |G| $, $ p $ prime. Then $ |G| $ divides $ \frac{n_p!}2 $ and $ n_p\ge 5 $.
\end{corollary}
Let $ G $ act by conjugation on $ \syl_p(G) $ which gives a homomorphism $ \varphi:G\to \sym(\syl_p(G))\cong S_{n_p} $. By simplicity, $ \ker \varphi=G $ or $ \{e\} $. If $ \ker \varphi=G $, then $ gP\inv g=P $ for all $ g\in G $ and all $ P\in\syl_p(G) $. So $ P $ is normal. Thus $ P $ is either $ \{e\} $ or $ G $. Well $ P$ is Sylow-$ p $ so it can't be $ \{e\} $, so $ P=G $. So $ G $ would be a $ p $-group. But from earlier, the centre of $ G $ is non-trivial proper since $ G $ is non-abelian, but the centre is always normal, so this contradicts simplicity, hence $ \ker \varphi=\{e\} $. So we have that $ \varphi $ is an injective homomorphism $ G\to S_{n_p} $, so by the first isomorphism theorem, $ G\cong \ima \varphi $. We'll show that $ \varphi $ lands in $ A_{n_p} $. Consider the composition $ G\to S_{n_p}\to \{\pm 1\} $. If this composition is surjective, then $ \ker (\mathrm{sgn} \circ \varphi) $ is index $ 5 $, but $ G $ simple so not possible. So $ \ima\varphi\subseteq \ker(\mathrm{sgn})=A_{n_p} $, so we're done by Lagrange. For the final statement we show all non-abelian subgroups of $ A_2, A_3, A_4 $ are not simple which finishes the statement which is just grunt work, and I pinky promise it's true, so we're done.\qed\\\\
Let's see a sample application. Let have $ G $ has size $ 11\times 12 $. If $ G $ is simple then there are exactly $ 12 $ Sylow 11-subgroups. Consider the number $ n_{11} $. We know from the Sylow theorems that $ n_{11}\equiv 1\mod 11 $ and $ n_{11}\mid 12 $. So $ n_{11} = 12 $ since $ G $ is simple. Similarly $ n_3\equiv 1 \mod 3 $ and $ n_3\mid 44 $. So either $ n_3 = 4 $ or $ 22 $. The corollary says that $ G $ divides $ \frac{n_3!}2 $, so $ n_3 $ can't be $ 4 $, so $ n_3=22 $. But this is a lot of elements. And 2 Sylow 11-subgroups interset only at the identity which leads to too many elements, so none of this even works, which seems confusing, but actually just means that $ G $ can't exist, hence all groups of order $ 132 $ are non-simple.\\\\
Finally we now prove the Sylow theorems.\\
\pf Let $ G $ be a group of order $ n=p^am $, with $ p\nmid m $, $ p $ prime. Define the set $ \Omega=\{X\subseteq G : |X|=p^a $. Let $ G $ act on $ \Omega $ by multiplying all elements of $ \Omega $ on the left by $ g\in G $ (we can see this obeys the axioms of the group action after some quick inspection. We have $ |\Omega|= \binom{n}{p^a}\equiv m\ne 0\mod p $. The proof of this can be seen by expanding out the binomial coeffient, but we'll assume it here. Suppose we have some $ U\in\Omega $, then let $ H\le G $ stabilise $ U $. Then $ |H|\mid |U| $. We can prove this by seeing that $ hU=U $ for all $ h\in H $. In other words for each $ u\in U $ the coset $ Hu $ is contained in $ U $. Every $ u\in U $ lies in some coset of $ H $, so the cosets partition $ U $, so $ |H|\mid |U| $. We know that $ |\Omega|\ne 0\mod p $. Since orbits partition, we know that
	\[
		|\Omega|=|O_1|+|O_2|+\cdots +|O_r|\text{, } O_i \text{ are the orbits}
	\]
So there exists an orbit $ \Theta $ whose size is prime to $ p $.
Let $ T\in\Theta $. By orbit-stabiliser, $ |G|=|\Theta||\stab(T)| $. So $ p^am=|\Theta||\stab(T)| $. By our previous lemma, $ |\stab T|\mid p^a $, so we're done because there are no factors of $ p $ in $ \Theta $, so we've prove the first part of the theorem.\\\\
Now for the second part, we actually show something stronger, that is, if $ Q\le G $ is a subgroup of size $ p^b $, where $ 0\le b\le a $, then there exists $ g\in G $ and $ P\in \syl_p(G) $, such that $ gQ\inv g\le P $. To prove this, let $ Q $ act on $ G/P $ by left coset multiplication. Note that the size of $ G/P $ does not divide by $ p $. Orbits have size dividing $ p^b $, so each orbit has size $ 1 $ or a power of $ p $. But $ p\nmid |G/P| $, so there exists a size 1 orbit. In other words, there exists some coset $ gP $ such that $ \forall q\in Q $, $ qgP=gP $, so rearranging gives that $ gQ\inv Q\le P $. So our second statement follows taking $ b=a $.\\\\
For the final theorem, we need to show that $ n_p\mid |G| $, and $ n_p\equiv 1\mod p $. For the first statement, consider $ G $ acting on $ \syl_p(G) $ by conjugation. By the second theorem, we know that there is one orbit of size $ n_p $, so the statement follows instantly from orbit-stabiliser. For the second statement, let $ P\in \syl_p(G) $. Consider $ P $ acting on $ \syl_p(G) $ by conjugation. By orbit-stabiliser, all the orbits have size 1 or $ p $. Since $ \{P\} $ is a size 1 orbit, to prove the statement is suffices to show that $ \{P\} $ is the only size 1 orbit. Say $ \{Q\} $ is another size 1 orbit. So $ \forall h\in P $, we have $ hQ\inv h=Q $. This means that $ N_G(Q) $ contains $ P $. Now observe if $ p^a $ is the largest power of $ p $ dividing $ |G| $, we know that it's the largest power of $ p $ dividing $ |N_G(Q)| $. But $ Q $ is normal in $ N_G(Q) $ by definition, and $ Q,P\in \syl_p(N_G(Q))\implies P=Q $, since normality $ \iff $ uniqueness for Sylow subgroups. So we've prove all the Sylow theorems and we're done. \qed
\newpage
\section{Rings}
\subsection{Definitions and examples}
\begin{definition}
	(Rings) A \textit{ring} is a quintuple $ (R,+,\circ,0_R,1_R) $, where $ R $ is a set with $ 0_R,1_R\in R $, and $ +:R\times R\to R $, and $ \circ:R\times R\to R $, called addition and multiplication are functions satisfying the following:
	\begin{itemize}
		\item $ (R,+,0_R) $ is an abelian group.
		\item $ \circ $ is associative, so $ a\circ(b\circ c)=(a\circ b)\circ c $.
		\item $ 1_R\circ a = a\circ 1_R=a $.
		\item We have distributivity, so $ r_1\circ (r_2+r_3)=(r_1\circ r_2)+(r_1\circ r_3) $ and $ (r_1+r_2)\circ r_3=(r_1\circ r_3)+(r_2\circ r_3) $.
	\end{itemize}
\end{definition}

Usually we just say "Let $ R $ by a ring..." with everything implicit. The symbol $ (-r) $ denotes the additive inverse of $ r $.\\\\
In IB Groups, Rings and Modules, rings will always be commutative, so $ r_1\circ r_2=r_2\circ r_1 $ for all $ r_1,r_2\in R $.

\begin{definition}
	(Subring) A \textit{subring} of a ring $ R $, is a subset $ S\subseteq R $, such that $ 0_R,1_R\in S $, $ S $ is closed under both multiplication and addition of the ring, and $ (S,+,\circ, 0_R,1_R) $ is a ring. 
\end{definition}
We notate this as $ S\le R $.\\\\
For examples we have $ \Z\le\Q\le\R\le\C $ which are all rings under usual multipliction and addition. Along a similar line, we also have the Gaussian integers, $ \Z[i]=\{a+ib:a,b\in \Z\} $ with multiplication and addition induced by $ \C $.\\
\\
Another example is $ \Z/n\Z $ which forms a ring under addition and multiplication modulo $ n $. In $ \Z/6 $ we have $ 2, 3\in \Z/6 $ such that $ 2\circ 3=0\mod 6 $ which is perfectly allowed.

\begin{definition}
	(Units) An element $ u\in R $, is called a \textit{unit} if there exists some $ v\in R $, such that $ uv=1_R\in R $.
\end{definition}
This notion does \textit{not} interact well with subrings, as we can take a unit in a subring without taking it's inverse, making it no longer a unit. For example 2 is a unit $ \Q $, but not in $ \Z $.
\\\\\textit{Discussion.} Does $ 0_R $ behave like it should? We would like $ 0\circ R=0_R $ for all $ r\in R $. In $ R $ we have that $ 0_R+0_R =0_R $, now multiplying by $ r\in R $, so $ r\circ 0_R+r\circ 0_R=r\circ 0_R $, hence cancelling a $ r\circ 0_R $ on both sides gives that $ r\circ 0_R=0_R $.\\
In particular this implies that if $ 1_R=0_R $ then for any $ r\in R $, $ r=r\circ 1_R=r\circ 0_R=0_R $ so for all $ r\in R $, $ r=0_R $, so $ R $ must be the zero ring, $ \{0_R\} $.\\\\

\begin{definition}
	(Polynomial) Let $ R $ be a ring. Then a \textit{polynomial}in $ x $ with coefficents in $ R $ in an expression:
	\[
		f(x)=a_0+a_1x+\cdots+a_nx^n
	\]
	and $ x^i $ are formal symbols. We will identify $ f(x) $ with $ f(x)+0\irc x^{n+1} $ as the same. The largest $ i $ such that $ a_i \ne 0$ is called the degree of the polynomial. A polynomial $ f(x) $ is monic of degree $ n $ if $ a_n=1 $ and it is of degree $ n $.
\end{definition} 
\begin{definition}
	(Polynomial ring) The \textit{polynomial ring} $ R[X] $ is given by:
	\[
		R[X]= \{f(X): \text{ f is a polynomial in } X \text { with coefficents in } R\}
	\]
	$+, \circ  $ are the usual operations, $ 0_{R[X]}=0_R $ and $ 1_{R[X]}=1_R $.
\end{definition}

\begin{definition}
	(Ring of formal power series) The \textit{ring of formal power series} is a ring in $ X $ with coefficents in $ R $ is:
	\[
	R[[X]]=\left\{\sum_{n=0}^\infty r_iX^i:a_i\in R,\forall i\ge 0, i\in\Z\right\}
	\]
	with the standard $ +,\circ $ of $ R $.
\end{definition}
For an example consider $ (1-x)\in R[X] $. Is it a unit? No! If $ g(x)(1-x)=1 $, then if $ g(x)=a_0+a_1x+\cdots a_nx^n $, $ a_n\ne 0 $, then $ (1-x)g(x) = a_0 + (a_1-a_0)x+\cdots (a_n-a_{n-1}x^n-a_nx^{n+1} $ which cannot be 1 since the highest power term has a non-zero coefficent.\\\\
However $ (1-x) $ is a unit in $ R[[X]] $! $ (1-x)(1+x+x^2+\cdots)=1\in R[[X]] $.\\
\begin{definition}
	(Laurent polynomials) If $ R $ is a ring then a \textit{Laurent polynomial} with coeffients in $ R $ is:
	\[
		R[X,\inv X] = \left\{\sum_{i\in\Z}a_iX^i: a_i\in R,\forall i\in\Z\right\}
	\]
	Where $ a_i $ is non-zero for at most finitely many $ i $ and with standard multiplication and addition.
\end{definition}

If $ R $ is a ring, and $ X $ is a set the set of $ R $-valued functions, namely, $ \{f:X\to R\} $ is a ring with "pointwise" addition and multiplication as given by the ring $ R $. (So $ (f+g)(x)=f(x)+g(x) $)
\subsection{Homomorphisms, ideals, and quotients}
\begin{definition}
	(Ring homomorphism) Let $ R $ and $ S $ be rings. A function $ f:R\to S $ is a \textit{ring homomorphism} if for all $ r_1,r_2\in R $:
	\begin{itemize}
		\item $ f(r_1+r_2)=f(r_1)+f(f_2) $
		\item $ f(0_R)=0_S $
		\item $ f(r_1r_2)=f(r_1)f(r_2) $
		\item $ f(1_R)=1_S $.
	\end{itemize}
\end{definition}
These first two conditions are the conditions for $ f $ to be a group homomorphism with the addition operation. Note that the second condition is not required and it follows from the first condition. But non-symmetrically the fourth condition is not implied by the third condition.
\begin{definition}
	(Isomorphism) An \textit{isomorphism} $ f:R\to S $ is a bijective ring homomorphism. The inverse function is also a ring homomorphism.
\end{definition}
\begin{definition}
	(Kernal) The \textit{kernal} of a ring homomorphism $ f: R\to S $ is the set $ \ker f = \{r\in R: f(r)=0_S\} $.
\end{definition}
\begin{definition}
	(Image) The \textit{image} of a ring homomorphism $ f: R\to S $ is $ \ima f = \{s\in S:s=f(r) \text{ for some } r\in R \}$.
\end{definition}

\begin{lemma}
	A homomorphism $ f:R\to S $ is injective if and only if $ \ker f = \{0\} $.
\end{lemma}
\pf Follows from the corresponding fact about groups. \qed
\begin{definition}
	(Ideal) A subset $ I\subseteq R $ is an \textit{ideal}, written as $ I\nrm R $, if $ I $ is a subgroup and if $ a\in I $ and $ b\in R $, then $ ab\in I $.
\end{definition}
Keep in mind that an ideal is usually not a subring, since if $ 1_R\in I $ then $ I=R $.
\begin{lemma}
  If $ f:R\to S $ is a ring homomorphism then $ \ker f \nrm R$.
\end{lemma} 
\pf Since $ f $ is also a group homomorphism, then $ \ker f $ is a subgroup. If $ a\in \ker f $ and $ b\in R $ then $ f(ab)=f(a)(b)=0f(b)=0 $, so $ ab\in \ker f $. \qed\\\\
Now we'll look at some examples.\\\\
If $ \Z $ is the ring of integers then $ n\Z $ are ideals for all $ n\in\N \cup \{0\} $. In fact, every ideal of $ \Z $ has this form. To see this $ I\ne \{0\} $ is an ideal. Let $ n\in \Z $ be the smallest postive element of $ I $. We claim that $ I=n\Z $. Let $ m\in I $. We claim that it's divisible by $ n $. Apply the Euclidean algorithm so $ m=qn+r $ where $ 0\le r<n $. But $ qn\in I $ by the absorbing property so $ r\in I $ since $ I $ is a subgroup which contradicts minimality unless $ r=0 $.\qed
\begin{definition}
  Let $ A\subseteq R $. The ideal generated by $ A $ is
  \[
	  (A)=\left\{\sum_{a\in A}r_aa,\quad r_a \in R, \quad \text{all but finitely many } r_a \text{ are } 0\right\}
  \]
\end{definition}
\begin{definition}
	(Principle) An ideal $ I\nrm R $ is \textit{principle} if there exists $ r\in R $ such that $ (r)=I $.
\end{definition}
For another example let $ \R[X] $ be the polynomial ring in one variable over $ \R $. The subset $ \{f\in \R[X]: \text{constant term is } 0\} $, is an ideal. It is actually principle, generated by $ (X) $.

\begin{definition}
	(Quotient) Let $ I\nrm R $ be an ideal. Then the \textit{quotient ring} $ R/I $ is the set of cosets $ r+I $ with $ 0_R/I=0_R+I $ and $ 1_R/I=1_R+I $, and operations $ (r_1+I)+(r_2+I)=(r_1+r_2)+I $ and $ (r_1+I)(r_2+I)=r_1r_2+I $.
\end{definition}

\begin{proposition}
  The quotient ring is a ring. The function $ f:R\to R/I $ sending $ r $ to $ r+ I $ is a ring homomorphism.
\end{proposition}
\pf Obviously an abelian group. Multiplication is well-defined. To see this suppose $ r_1+I=r_1'+I $ and $ r_2+I=r_2'+I $. Then $ r_1-r_1'=a_1\in I $, and $ r_2-r_2'=a_2\in I $, so $ r_1'r_2'=(r_1+a_1)(r_2+a_2)=r_1r_2+r_1a_2+r_2a_1+a_1a_2 $. By the absorbing property the last three terms are contained in $ I $, so $ r_1r_2+ I=r_1'r_2'+I $. The rest is straightforward.\qed\\\\
For another example, we have $ n\Z\nrm\Z $. The quotient $ \Z/n\Z $ is the usual ring of integers modulo $ n $.\\\\
Take $ (X)\nrm \C[X] $. The elements of $ \C[X]/(X) $ are represeneted by:
\[
	a_0+a_1X+\cdots a_n X^n+(X),\  \text{but} \ \sum_{i=1}^na_iX^i\in (X)
\]
so each coset is represented equivalently by $ a_0+(X) $, so we have that $ \C[X]/(X)\cong \C $.\\\\
Similarly $ (X^2)\nrm \C[X] $, the ring $ \C[X]/(X^2) $ consists of elements represented by linear polynomials $ a_0+a_1X+(X) $ with the following multiplication given by $ (a_0+a_1X)(b_0+b_1X)=a_0b_0+(a_1b_0+a_0b_1)X $.\\
This ring is quite weird. For example if we take $ X\in \C[X]/(X^2) $. Then $ 0\ne X $ but $ X^2=0 $. We say that $ X $ is nilpotent.

\begin{proposition}
	(Euclidean algorithm for polynomials in $ X $) Let $ K $ be a field and $ f,g\in K[X] $. Then there exists polynomials $ r,q\in K[X] $ such that $ f=gq+r $ with $ \mathrm{deg}(r)<\mathrm{deg}(g) $.
\end{proposition}
\pf Let $ n $ be the degree of $ f $. So $ f=\sum_{i=0}^na_iX^i $ with $ a_i\in K,a_n\ne 0 $. Similarly $ g=\sum_{i=0}^mb_iX^i $ with $ b_i\in K $ and $ b_m\ne 0 $.\\
If $ n< m $ set $ q=0 $ and $ r=f $ so we're finished.\\
If instead $ n\ge m $, proceed by induction on the degree. Let $ f_1=f-a_n\inv b_mX^{n-m}g $. Observe that $ \mathrm{deg}(f_1)<n $. If $ n=m $ then $ \mathrm{deg}(f_1)<n=m $. So write $ f=(a_\inv b_mX^{n-m})g+f_1 $, so we're done. Otherwise if $ n>m $, then because $ \mathrm{deg}(f_1)<n $, by induction we cab wrute write $ f_1=gq_1+r_1 $ where $ \mathrm{deg}(r_1)<\mathrm{deg}(g)=m $. Then $ f=(a_n\inv b_m)X^{n-m}g+q_1g+r_1=(a_n\inv b_mX^{n-m}+q_1)g+r_1 $\qed

\begin{corollary}
	If $ K $ is a field then $ K[X] $ every ideal is principle.
\end{corollary}
\pf Identical to the case of $ \Z $ using the proposition.\\\\
This proof fails for $ \Z [X] $ (since $ \Z $ is not a field) and for $ K[X,Y] $.
\begin{theorem}
	(First isomorphism theorem) Let $ \varphi:R\to S $ be a ring homomorphism. Then the function $ f:R/\ker\varphi\to\ima\varphi\le S $ sending $ r+\ker\varphi\to\varphi(r) $ is well-defined and an isomorphism of rings.
\end{theorem}
\pf Well-definedness, bijective, additive homomorphism property all follow from the group statement. We check multiplicativity. $ f((f+\ker\varphi)(t+\ker\varphi))=f(rt+\ker\varphi)=\varphi(rt)=\varphi(r)\varphi(t)=f(r+\ker\varphi)(f+t+\ker\varphi) $ since $ \varphi $ is a ring homomorphism.\qed
\\\\
For an example consider the homomorphism $ \varphi:\R[X]\to \C $. sending $ f(X) $ to $ f(i) $. Clearly this is a surjective ring homomorphism since $ a+bX\to a+bi $ under $ \varphi $. The kernal is exactly real polynomials $ f(X) $ such that $ f(i)=0 $ i.e $ i $ is a root. But since $ f $ has real coefficents that means that $ (X+i)(X-i)\mid f(X) $ i.e. $ (X^2+1)\mid f(X) $. So in fact $ \ker\varphi=(X^2+1) $, the ideal generated by $ X^2+1 $. Now applying the first isomorphism theorem $ \frac{\R[X]}{(X^2+1)}\cong \C $.

\begin{theorem}
	(Second isomorphism theorem) Let $ R\le S $ and $ J\nrm S $. Then $ J\cap R\nrm R $ and $ \frac{R+J}{J}=\{r+J:r\in R\}\le \frac SJ $. Furthermore,
	\[
		\frac{R}{R\cap J}\cong \frac{R+J}J.
	\]
\end{theorem}
\pf Define a function $ \varphi:R\to S/J $ by $ r\to r+J $. The kernal is $ \{r:r+J=0\}=\{r\in J\}=R\cap J $. The image $ \ima\varphi=\{r+J:r\in R\}=\frac{R+J}J $, so apply the first isomorphism theorem to conclude.\qed
\\\\
Again similar to groups we have a correspondence result.
\begin{theorem}
	(Correspondence theorem) If $ I\nrm R $ is an ideal there is a bijection between subrings of $ R/I $ and subrings of $ R $ which contain $ I $. This is given by sending $ L\le R/I\to \{r\in R: r+I\in L\} $ and conversely $ I\nrm S\le R\to S/I\le R/I $
\end{theorem}
\pf Same as from groups.
\\
Similar for ideals there is a bijection betwen ideals in $ R/I $ and ideals in $ R $ that contain $ I $.
\begin{theorem}
	(Third isomorphism theorem) Let $ I\nrm R $ and $ J\nrm R $ with $ I\subseteq J $. Then $ \frac JI\nrm \frac RI $ and we have that,
	\[
		\frac{R/I}{J/I}\cong R/J.
	\]
\end{theorem}
\pf Define a function $\varphi: R/I\to R/J $ sending $ r+I $ to $ r+J $. Well-definedness follows from the same argument as from groups. Easy verification to see it is a ring homomorphism.
The kernal is $ \ker\varphi=\{r+I:r+J=J\} $, i.e. that $ \ker\varphi=J/I $. So apply the first isomorphism theorem to get the result.\qed
\begin{claim}
  Let $ R $ be any ring. There is a unique ring homomorphism
  \[
	  i:\Z\to R
  \]
\end{claim}
The kernal of $ i $, $ \ker i $ is an ideal $ n\Z\nrm Z $. The number $ |n| $ is called the characteristic of $ R $. The rings $ \Z,\R,\C,\C[X] $ all have characteristic 0. $ \Z/k\Z $ has characteristic $ k $.
\subsection{Integral domains}
In the ring $ \Z/6 $ we have that $ 2\cdot 3 = 0 $. In an integral domain this will not happen.

\begin{definition}
	(Integral domain) A nonzero ring $ R $ is an integral domain if $ \forall a,b\in R $, if $ ab=0 $ then $ a=0 $ or $ b=0 $.
\end{definition}
An element that violates this is called a zero divisor, i.e. a zero divisor is a non-zero element $ a\in R $ such that $ \exists b\in R, b\ne 0 $ where $ ab=0 $.\\
All fields are integral domains, since if $ ab=0, b\ne 0 $ then $ a(b\inv b)=0\inv b=0 $ so $ a=0 $.\\
Any subring of an integral domain is an integral domain. To list a set of examples we have $ \Z,\Z[i],\Q,\C,\R[X],\Z[X], $ etc.
For a set of non-examples we have $ \Z/6, \Z/pq, \C[X]/(X^2) $ etc.
\begin{lemma}
  Let $ R $ be a finite integral domain. Then $ R $ is a field.
\end{lemma}
\pf Let $ a\in R $ be non-zero. Consider the function $ \mu_a: R\to R $ sending $ r\to ar $. It's easy to verify that $ \mu_a $ is an (additive) group homomorphism for all $ a $ non-zero. Since $ R $ is an integral domain, $ \ker\mu_a $ is trivial so the map is injective. So since $ R $ is finite, $ \mu_a $ is also surjective. In particular $ 1=ab $ for some $ b\in R $ hence this is an inverse of $ a $, so $ R $ is a field.\qed
\begin{definition}
	Let $ R $ be an integral domain. A \textit{text of fractions} for $ R $ is a field $ F $ such that:
	\begin{itemize}
		\item $ R\le F $ is a subring,
		\item every $ x\in F $ can be written as $ a\inv b $, where $ a,b\in R, $ where $ \inv b $ is the multiplictive inverse to $ b $ in $ F $.
	\end{itemize}
\end{definition}
$ \Q $ is a field of fractions for $ \Z $.
\begin{theorem}
  Every integral domain has a field of fractions.
\end{theorem}
\pf Define a set $ S=\{(a,b)\in R\times R:b\ne 0\} $. Place an equivalence relation $ \sim $, defined as $ (a,b)\sim (c,d)\iff ad=bc $ on $ S $. We can check this is an equivalence relation, the only non-trivial axiom to check is transitivity. Suppose that $ (a,b)\sim (c,d) $ and $ (c,d)\sim (e,f) $. So we have that $ ad=bc $ and $ cf=de $. We wish to deduce that $ af=be $. Multiple the first equality by $ f $ and the second by $ b $. So we get that $ adf=bcf $ and $ bcf=bed $. Rearranging we get $ d(af-be)=0 $ since $ d $ is non-zero and $ R $ is an integral domain we know that $ af=be $. So $ \sim $ is an equivalence relation. Now define $ F=\frac S\sim $ with notation $ \frac ab = [(a,b)]_\sim $. Now we turn $ F $ into a ring. Take the operations to be $ \frac ab + \frac cd = \frac {ad+bc}{bd} $ and $ \frac ab\frac cd = \frac{ac}{bd} $. Some elementary operations show that these operations are well-defined and makes $ F $ into a ring. To see that $ F $ is a field, if $ \frac ab\ne 0_F $ i.e. $ \frac ab\ne \frac 01\implies a\cdot 1\ne b\cdot 0 =0  $, so $ a\ne 0 $. now $ \frac ba \in F $ and $ \frac ba\frac ab=1_F $, so $ F $ is a field.\\\\
We now construct an injective homomorphism $ R\to F $ by $ r\to \frac r1 $. Straightforward to check that this is a ring homomorphism. The kernal is $ \{r\in R: \frac r1 = 0 \text { in } F\} =(0)$. By the first isomorphism theorem $ R $ is isomorphic to the image of $ R\to F $, in other words $ R\le F $.
Finally since $ \frac ab\in F $ is $ \frac ab= \frac a1\cdot \frac 1b\implies \frac a1\inv{(\frac b1)}=a\inv b $\qed

Sometimes we write $ \mathrm{FF}(R) $ for a field of fractions of $ R $.
\begin{proposition}
  Let $ R $ be a ring. Then $ R $ is a field if and only if the only ideals in $ R $ are $ (0) $ and $ R $.
\end{proposition}
\pf If $ R $ is a field and $ I\nrm R $ is non-zero then $ I $ contains a unit $ u $. Since $ 1=uv $ we have that $ 1\in I $. But for any $ r\in R $, we have $ 1\cdot r =r\in I $, so $ I=R $.\\
Conversely suppose that $ (0) $ and $ R $ are the only ideals of $ R $. Take $ r\in R $ non-zero. We know that $ (r)=R $ since $ r $ is non-zero. Since $ 1\in (r) $ we know that $ r\cdot b = 1 $ for some $ b\in R $ so $ r $ is a unit hence $ R $ is a field.

\begin{definition}
	(Maximal ideal) An ideal $ I\nrm R $ is called \textit{maximal} if it is not $ R $ itself and if for any $ J\nrm R $ with $ I\subseteq J\subseteq R $, either $ J=I $ or $ J=R $.
\end{definition}

\begin{proposition}
  An ideal $ I\nrm R $ is maximal if and only if $ R/I $ is a field.
\end{proposition}
\pf $ R/I $ is a field if and only if the ideals are $ R/I $ and $ (0) $. Now apply the ideal correspondence theorem.\qed
\begin{definition}
	(Prime ideal) An ideal $ I\nrm R $ is \textit{prime} if whenever $ ab\in I $ either $ a $ or $ b $ lies in $ I $.
\end{definition}
An ideal $ n\Z\nrm \Z $ is a prime ideal if and only if $ n $ is a prime number (or zero). We can see this since if $ n=p $ is prime, and $ ab\in p\Z $ then $ ab $ is a multiple of $ p $ so either $ a $ or $ b $ must be a multiple of $ p $ hence in $ p\Z $. Conversely if $ n $ is not prime and wlog positive (zero case is trivial) we know that $ n=m_1m_2 $, $ 1<m_1,m_2<n $. Then $ m_1,m_2\notin n\Z $ but $ m_1m_2\in n\Z $ so the ideal is not a prime ideal.
\\ Interestingly $ p\Z\nrm \Z $ for $ p $ non-zero prime, then $ \Z/p\Z $ is a field so $ p\Z $ is maximal.

\begin{proposition}
  An ideal $ I\nrm R$ is prime if and only if $ R/I $ is an integral domain.
\end{proposition}
\pf If $ I \nrm R $ is prime, then let $ (a+I) $ and $ (b+I)\in R/I $. Suppose $ (a+I)\cdot(b+I)=(ab+I)=0+I $ (recall $ 0+I $ is the zero element in $ R/I $). This means that $ ab\in I $ but $ I $ is prime so $ a $ or $ b\in I $, so $ a+I $ or $ b+I $ is $ 0 $.\\
Conversely if $ R/I $ is an integral domain, consider $ ab\in I $. Then $ ab + I = 0 $. So either $ a+I $ or $ b+I $ is zero so $ a $ or $ b $ lies in $ I $. So $ I $ is a prime ideal.\qed

\begin{corollary}
  If $ R $ is a prime and $ I\nrm R $ is maximal, then $ I $ is prime.
\end{corollary}
\pf Since $ I\nrm R $ is maximal then $ R/I $ is a field. Hence $ R/I $ is an integral domain so $ I $ is prime by the proposition.\qed\\\\
Every nonzero ring $ R $ has a maximal ideal and therefore a prime ideal (proof is very set theoretic, equivalent to the axiom of choice through Zorn's lemma)

\subsection{Factorisation in integral domains}
From now on we let $ R $ be a general integral domain
\begin{definition}
	(Division) Let $ a,b\in R $ we say that a \textit{divides} b, written as $ a\mid b $ if there exists some $ c \in R $ such that $ b=ac $. Equivalently we have that $ (b)\subseteq (a) $.
\end{definition}
\begin{definition}
	(Associates) We say that $ a $ and $ b $ in $ R $ are \textit{associates} if $ a=bc $ for $ c\in R $ a unit. Equivalent to $ (a)=(b) $ and also equivalent to that $ a\mid b $ and $ b\mid a $.
\end{definition}
In $ \Z $ for example, we want to factorise up to units, i.e $ 6=2\times 3 = (-2)\times (-3) $. But as $ 2 $ and $ -2 $ are associates we declare some amount of uniqueness.
\begin{definition}
	(Irreducible) An element $ a\in R $ is called \textit{irreducible} if $ a\ne 0 $, $ a $ is not a unit, and if $ a=xy $ then either $ x $ or $ y $ is a unit.
\end{definition}
In the special case of $ \Z $ irreducible and prime are the same thing. But this is NOT always the case.
\begin{definition}
	(Prime element) We say that an element $ p\in R $ is \textit{prime} if $ p\ne 0 $, not a unit and if $ p\mid xy $, then either $ p\mid x $ or $ p\mid y $.
\end{definition}
\begin{proposition}
  Let $ r\in R $. Then $ r\ne 0 $ is prime if and only if $ (r) $ is a prime ideal.
\end{proposition}
\pf Suppose that $ (r) $ is a prime ideal. Then it is proper by definition, so $ r $ is not a unit. Suppose that $ r\mid xy $, so $ xy\in (r) $ so by primality either $ x $ or $ y $ lies in $ (r) $ so $ r\mid x $ or $ r\mid y $.
Conversely let $ r\in R $ be a prime. Suppose $ xy\in (r) $ then $ r\mid xy $ so $ r|x $ or $ r|y $ so $ x\in (r) $ or $ y\in (r) $ \qed\\\\
Again irreducible and prime are not the same thing. However...
\begin{proposition}
  Let $ r\in R $ be prime. Then $ r $ is irreducible.
\end{proposition}
\pf Let $ r\in R $ be a prime and suppose can write $ r $ as $ r=xy $. Since $ r=1_Rr $ we have that $ r\mid xy $ so either $ r\mid x $ or $ r\mid y $. Assume by symmetry that $ r\mid x $. This means that $ x=rz $ for $ \in R $. So $ r=xy=rzy $. So since we're in an integral domain and $ r\ne 0 $ we have that $ zy=1 $ hence $ y $ is a unit\qed\\\\
Now let's look at an example.\\\\
Let $ R=\Z[\sqrt{-5}]\le \C $, i.e. elements of the form $ a+b\sqrt{-5} $ for $ a,b\in \Z $. Observe that $ R $ is an integral domain since it is a subring of a field. Let's discuss the units. We define a "norm", $ N:R\to \Z_{\ge 0} $ sending $ a+b\sqrt{-5}\to a^2+5b^2 $. This is a function and importantly it is multiplicative, so $ N(ab)=N(a)N(b) $. Notice that all units have norm 1, since if $ 1=uv $, then $ N(1)=N(u)N(v)=1 $, so we must have that $ N(u)=N(v)=1 $. This implies the units are $ \pm 1 $.
\begin{claim}
  $ 2\in R $ is an irreducible element
\end{claim}
\pf If $ 2=ab $ then $ N(2)=4=N(a)N(b) $. But no element in $ R $ has norm of $ 2 $. Therefore either either $ a $ or $ b $ has norm 1, which means either $ a $ or $ b $ is a unit.\qed\\\\
A similar calculation shows that $ 3, 1\pm \sqrt{-5} $ are all also irreducible.
But are they prime?\\
Observe that $ 6=(1+\sqrt{-5})(1-\sqrt{-5})=2\times 3 $
\begin{claim}
	2 does not divide $ 1\pm \sqrt{-5} $
\end{claim}
\pf If it did then $ N(2)\mid N(1\pm\sqrt{-5}) $ but $ N(2)=4 $ and $ N(1\pm \sqrt{-5})=6 $ but $ 4\nmid 6 $ so 2 is no longer a prime in $ \Z[\sqrt{-5}] $.\\
In this same example, we see unique factorisation of 6 no longer holds.

\begin{definition}
	An integral domain $ R $ is called a \textit{Euclidean domain} if there exists a Euclidean function $ \varphi:R\setminus \{0\}\to\Z_{\ge 0} $ such that:
	\begin{itemize}
	\item $ \varphi(ab)\ge \varphi(b) $ for all $ a,b\ne 0 $.
	\item If $ a,b\in R $ with $ b\ne 0 $, then there exists $ q,r\in R $ such that $ a=bq+r $ and either $ r=0 $ or $ \varphi(r)<\varphi(b) $.
	\end{itemize}
\end{definition}
This definition is just saying we can run the Euclidean algorithm (or some equivalent form of it) on the ring.\\\\
We've already seen $ \Z $ is an integral domain where $ \varphi(x)=|x| $. Also seen, that if we take $ K $ a field, then $ K[X] $ is a Euclidean domain with a Euclidean function given by the degree of the polynomial.\\\\

Now take $ R=\Z[i]\le \C $ (Gaussian integers). This is a Euclidean domain with Euclidean function $ \varphi(z)=|z|^2 $
\begin{claim}
  $\varphi$ is a Euclidean function of $ R $
\end{claim} 
\pf The first requirement is obvious. For the second requirement, consider $ a,b\in\Z[i] $, with $ b\ne 0 $. Consider the ratio $ \frac ab\in \C $. There is a point $ q\in \Z[i] $ that has distance at most 1 from $ \frac ab $. So we have that $ \left|\frac ab - q\right|<1 $. Then write $ \frac ab = q+c $ where $ |c|<1 $. Then we have that $ a=bq+bc $, now set $ r=bc $. We know that $ r=a-bq\in R $. And finally $ \varphi(r)=\varphi(b)\varphi(c)<\varphi(b) $ \qed
\end{document}






