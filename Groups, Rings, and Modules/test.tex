\documentclass{article}
\usepackage{../header}
\title{Groups, Rings, and Modules}
\author{Notes made by Finley Cooper}

% Course specific commands
\newcommand{\nrm}{\triangleleft}
\newcommand{\F}{\mathbb{F}}
\DeclareMathOperator{\sym}{Sym}
\DeclareMathOperator{\orb}{orb}
\DeclareMathOperator{\stab}{stab}
\DeclareMathOperator{\aut}{Aut}
\DeclareMathOperator{\ccl}{ccl}
\DeclareMathOperator{\syl}{Syl}
\DeclareMathOperator{\Ann}{Ann}
\DeclareMathOperator{\Fit}{Fit}
\begin{document}
  \maketitle
  \newpage
  \tableofcontents
  \newpage
  \section{Review of IA Groups}
  \subsection{Definitions}
  We'll start with some simple definitions covered in IA Groups
  \begin{definition}
	  A group is a \textit{triple}, $ (G,\circ,e) $ consisting of a set $ G $, a binary operation $ \circ:G\times G\rightarrow G $ and an identity element $ e\in G $ where we have the following three properties,
          \begin{enumerate}	  
		  \item $ \forall a,b,c\in G, (a\circ b) \circ c =a\circ(b\circ c) $
	          \item $ \forall a\in G, a\circ e = e\circ a = a $
		  \item $ \forall a\in G, \exists \inv a \in G, a \circ \inv a = \inv a \circ a = e $
          \end{enumerate}
  We say that the \textit{order} of the group $ (G,\circ, e) $ is the size of the set $ G $	
  \end{definition}

\begin{proposition}
  Inverses are unique.
\end{proposition}
\textit{Proof.} Basic algebraic manipulation, covered in Part IA Groups.
\begin{definition}
	If $ G $ is a group, then a subset $ H\subseteq G $ is a \textit{subgroup} if the following hold,
\begin{enumerate}
	\item $ e\in H $
	\item If $ a,b\in H $ then $ a\circ b\in H $
\item $ (H,\circ, e) $ forms a group.
\end{enumerate}
\end{definition}
Now we'll give simple test for a subset being a subgroup
\begin{lemma}
  A non-empty subset, $ H $, of a group $ G $ is a subgroup if and only if $ \forall h_1,h_2\in H $ we have that $ h_1 \inv h_2\in H $
\end{lemma}
\textit{Proof.} Again covered in Part IA Groups
\begin{definition}
  A group $ G $ is abelian if $ \forall g_1,g_2\in G $ we have that $ g_1g_2=g_2g_1 $
\end{definition}
Let's look at some examples of groups.
\begin{enumerate}
	\item The integers under addition, $ (\mathbb Z, +) $
	\item The integers modulo $ n $ under addition $ (\mathbb Z_n, +_n) $
	\item The rational numbers under addition $ (\mathbb Q, +) $
	\item The set of all bijections from $ \{1,\cdots,n\} $ to itself with the operation given by functional composition, $ S_n $
	\item The set of all bijections from a set $ X $ to itself under functional composition is a group $ \mathrm{Sym}(X) $
	\item The dihedral group, $ D_{2n} $ the set of symmetries of the regular $ n $-gon
\item The general linear group over $ \mathbb R $, $ \mathrm{GL}(n,\mathbb R) $, is the set of functions from $ \mathbb R\rightarrow \mathbb R $ which are linear and invertible. Or we can think of the group as the set of $ n\times n $ invertible matrices under matrix multiplication. We can view this group as a subgroup of $ \mathrm{Sym}(\mathbb R^n) $
\item The subgroup of $ S_n $ which are even permutations, so can be written as a product of evenly many transpositions, $ A_n $
\item The subgroup of $ D_{2n} $ which are only the rotation symmetries which is denoted by $ C_n $
\item The subgroup of $ \mathrm {GL}(n,\mathbb R) $ of matrices which have determinant $ 1 $ which is $ \mathrm {SL}(n,\mathbb R) $
\item The Klein four-group, which is $ K_4=C_2\times C_2 $, the symmetries of the non-square rectangle
\item The quaternions, $ Q_8 $ with the elements $ \{\pm 1, \pm i, \pm j, \pm k\} $ with multiplication defined with $ ij =k, ji = -k $, $ i^2=j^2=k^2=-1 $
\end{enumerate}
	\subsection{Cosets}
	\begin{definition}
		Let $ G $ be a group and $ g\in G $. Let $ H $ be a subgroup of $ G $. The \textit{left coset}, written as $ gH $ is the set $ \{gh : h\in H\} $
	\end{definition}
Some observations we can make are,
\begin{enumerate}
	\item Since $ e\in H $ we have that $ g\in gH $. So every element is in some coset
	\item The cosets partition, so if $ gH\cap g'H\ne \emptyset $ then $ gH=g'H $
	\item The function, $ f: H\rightarrow gH $ defined by $ f(h)= gh $ is a bijection, so all cosets are the same size
\end{enumerate}
\begin{theorem}
	(Lagrange's Theorem) If $ G $ is a finite group, then for a subgroup $ H $ of $ G $, $ |G|=|H||G:H| $, where $ |G:H| $ is the number of left cosets of $ H $ in $ G $
\end{theorem}
\textit{Proof.} Obvious from the observations we've just made.

\begin{definition}
	Let $ G $ be a group, and take some element $ g\in G $. We define the \textit{order} of $ g $ as the smallest positive integer $ n $, such that $ g^n = e $. If no such $ n $ exists, we say the order of $ g $ is infinite. We denote the order by $ \mathrm {ord}(g) $.
\end{definition}
\begin{proposition}
	Let $ G $ be a group and $ g\in G $. Then $ \mathrm {ord}(g) $ divides $ |G| $
\end{proposition}
\textit{Proof.} Let $ g\in G $. Consider the subset, $ H=\{e, g, g^2,\cdots, g^{n-1}\} $ where $ n $ is the order of $ g $. We claim $ H $ is a subgroup. $ e\in H $ so $ H  $ is non-empty. Observe that $ g^rg^{-s}=g^{r-s}\in H $ so we have that $ H\le G $. Elements are distinct since if $ g_i=g_j, i\ne j, 0\le i<j<n $ then $ g{j-i}=e $ which contradicts the minimality of $ n $ since $ 0\le j-i\le n $. We have that $ |H|=n $, so by Lagrange, $ |H| $ divides $ |G| $.\qed
\subsection{Normal subgroups}
When does $ gH=g'H $? Then $ g\in g'H $, so we have that $ \inv {g'}g\in H $. The converse also holds.
\begin{lemma}
	For a group $ G $ with $ g, g'\in G $ and subgroup $ H $ we have that $ gH=g'H $ if and only if $ \inv {g'} g\in H $
\end{lemma}
\textit{Proof.} In Part IA Groups
\par
Let $ G/H = \{gH:g\in G\} $ be the set of left cosets. This partitions $ G $. Does $ G/H $ have a natural group structure?

We propose the formula that $ g_1H\cdot g_2H=(g_1g_2)\cdot H $ for a group law on $ G/H $.

We need to check well definedness of this proposed formula.
\textit{Case 1:} Suppose that $ g_2H=g_2'H. $ Then $ g_2'=g_2h $ for some $ h\in H $. $ (g_1H)\cdot (g_2'H)=g_1g_2'H $ by the proposed formula. By the previous relation this is $ g_1g_2hH=g_1g_2H $.
\par
\textit{Case 2:} Suppose that $ g_1H=g_1'H $ we have that $ g_1'=g_1h $ for some $ h\in H $. We need $ g_1g_2 H = \underbrace{g_1h}_{g'_1}g_2H $. Equivalently we need that $ \inv{(g_1g_2)}g_1hg_2\in H $. Or equivalently still, $ \inv{g_2}hg_2\in H $ for all $ g_2 $ and $ h $. This is the definition of normality.
\begin{definition}
		(Normality) A subgroup $ H\le G $ is \textit{normal} if $ \forall g \in G $, $ h\in H $, we have that $ gh\inv g\in H $
\end{definition}
If $ H\le G $ is normal we write that $ H \triangleleft G $.
\begin{definition}
	(Quotient) Let $ H\triangleleft G $. The \textit{quotient group} is the set $ (G/H, \cdot, e=eH) $ where $ \cdot:G/H\times G/H \rightarrow G/H $ by $ (g_1H,g_2H)\rightarrow (g_1g_2)H $.
\end{definition}

\begin{definition}
	(Homomorphism) Let $ G $ and $ H $ be groups. A \textit{homomorphism} is a function $ f: G\rightarrow H $ such that for all $ g_1,g_2\in G $ we have that $ f(g_1g_2)=f(g_1)f(g_2) $
\end{definition}
This is a very constrained condition. For example $ f(e_G)=e_H $ always. To see this, observe $ e_G=e_Ge_G $, so we have that $ f(e_G)=f(e_G)f(e_G) $ so $ f(e_G)=e_H $ by multiplying by $ \inv{f(e_G)} $.
\begin{lemma}
	If $ f:G\rightarrow H $ is a homomorphism. Then $ f(\inv g)=\inv{f(g)} $
\end{lemma}
\textit{Proof.} Calculate $ f(g\inv g) $ in two ways.

In the first way $ f(g\inv g)=f(e)=e, $ in the second way $ f(g\inv g)=f(g)f(\inv g). $

Equating gives that $ f(\inv g)=\inv{f(g)}. $\qed
\begin{definition}
	Let $ f:G\rightarrow H $ be a homomorphism. The \textit{kernal} of $ f $ is $ \ker f =\{g\in G: f(g)=e\} $. The \textit{image} of $ f $ is $ \ima f=\{ h\in H: h=f(g)\text{ for some } g\in G\} $.
	\end{definition}
\begin{proposition}
	Let $ f:G\rightarrow H $ be a homomorphism. Then $ \ker f \triangleleft G $ and $ \ima f \le H.$
\end{proposition}
\textit{Proof.} First let's proof that $ \ker f $ is a subgroup by the subgroup test. Observe by the lemma that $ e\in\ker f. $. If $ x,y \in\ker f $, then $ f(x\inv y)=f(x)\inv{f(y)}=e\implies x\inv y\in \ker f $. For normality, let $ x\in G $ and $ g\in\ker f $. Calculate $ f(xg\inv x)=f(x)f(g)\inv{f(x)} $. But $ f(g)=e $. So we just get the identity. Hence we have that $ xg\inv x\in\ker f. $ So $ ker f\nrm G $.

To check that the $ \ima f \le H $, take $ a,b\in \ima f $, say that $ a=f(x), b=f(y) $. Then $ a\inv b=f(x)\inv{f(y)}=f(x\inv y) $. But $ x\inv y\in G $ so $ f(x\inv y)\in \ima f $. Also $ e\in \ima f $, so we have that $ \ima f\le H $.\qed
\begin{definition}
	(Isomorphism) A homomorphism $ f:G\rightarrow H $ is an \textit{isomorphism} if it is a bijection. Two groups are called \textit{isomorphic} if there exists an isomorphism between them.
\end{definition}
\begin{theorem}
	(First isomorphism theorem) Let $ f:G\to H $ be a homomorphism. Then $ \ker f $ is normal, and the function $ \varphi:G/\ker f\rightarrow \ima f $, by $ \varphi(g\ker f)=f(g) $, is a well-defined, isomorphism of groups.
\end{theorem}
\pf Already shown $ \ker f\nrm G $. Consider whenever $ \varphi $ is well-defined. Suppose that $ g\ker f=g'\ker f. $ Need to check $ \varphi(g\ker f)=\varphi(g'\ker f). $ We know that $ g\inv{g'}\in \ker f $, so $ f(g'\inv g)=e\iff f(g')=f(g) $. To see that $ \varphi $ is a homomorphism: $ \varphi(g\ker f g'\ker f)=\varphi(gg'\ker f)=f(gg')=f(g)f(g')=\varphi(g\ker f)\varphi(g'\ker f) $. So $ \varphi $ is a homomorphism.

\par
Finally let's check $ \varphi $ is bijective. First for surjectivity, let $ h\in \ima f $, then $ h=f(g) $ for some $ g\in G $. So we have that $ h=\varphi(g\ker f) $.

Now for injectivity, $ \varphi(g\ker f)=\varphi(g'\ker f)\implies f(g)=f(g') \implies g'\inv g\in\ker f $. Hence the cosets are the same by the coset equality criterion, so we have that $ g\ker f=g'\ker f $, hence we have injectivity, so $ \varphi $ is an isomorphism.

\par
For an example of this theorem, consider the groups $ (\mathbb C, +) $ and $ (\mathbb C^*, \times) $ related by the homomorphism, $ \varphi(z)= e^z$. The kernal of $ \exp $ is exactly, $ 2\pi i \mathbb Z\le \mathbb C $, so the first isomorphism theorem gives that $ \frac{\mathbb C}{2\pi i \mathbb Z}\cong \mathbb C^* $. \textit{(Try to visualise this!)}\qed 
\begin{theorem}
	(Second isomorphism theorem) Let $ H\le G $ and $ K\nrm G $. Then $ HK=\{hk : h\in H, k\in K\} $ is a subgroup of $ G $, the set $ H\cap K$ is normal in H, and $ \frac{HK}K\cong \frac H{H\cap K} $.
\end{theorem}
\pf We take the statements in turn. First we can see that $ HK $ is a subgroup. Clearly it contains the identity, and take some $ x,y\in HK $, $ x=hk, y=h'k' $. We will show that $ y\inv x\in HK $. Observe that $ y\inv x = h'k'\inv k \inv h = h'(\inv h h)(k' \inv k)\inv h = (h'\inv h)h\underbrace{(k'\inv k)}_{k''}\inv h $. But we have that $ hk''\inv h \in K $ by the normality of $ K $, hence $ y\inv x \in HK $. So we have that $ HK\le G $.

\par
Now we prove that $ H\cap K\nrm G $. Consider the homomorphism, $ \varphi: H\to G/K $, defined as $ \varphi(h)=hK $. This is a well defined homomorphism for the same reason that the group structure $ G/K $ is well-defined. The kernal of $ \varphi $, is $ \ker \varphi = \{h: hK=K\}=\{h:h\in K\}=H\cap K\nrm G $.

\par
Now finally we're left to prove the isomorphism. Now apply the first isomorphism theorem to $ \varphi $. This tells us that $ \frac H{\ker \varphi}=\frac H {H\cap K}\cong \ima \varphi $. The image of the $ \varphi $ is exactly those coests of $ K $ in $ G $ that can be represented as $ hK $ which is exactly $ \frac{HK}K $.\qed
\begin{theorem}
	(Correspondence theorem). Consider a group $ G $ with $ K\nrm G $, with the homomorphism $ p:G\to G/K $, by $ p(g)=gK $. Then there is a bijection between the subgroups of $ G $ which contain $ K $ and the subgroups of $ G/K $.
\end{theorem}
\pf For some subgroup $ L $, we have $ K\nrm L \le G $, and we map $ L $ to $ L/K $, so we have that $ L/K\le G/K $. In the reverse direction, for a subgroup $ A\le G/K $, we map it to $ \{g\in G: gK\in A\} $.
We can think of this as taking $ L\to p(L) $ and $ \inv p(A)\leftarrow A $.
\par Now we will state some facts without proof. (Although the proofs are fairly straightforward).
\begin{enumerate}
	\item This is a bijection.
	\item This correspondence maps normal subgroups to normal subgroups.
\end{enumerate}
\begin{theorem}
	(Third isomorphism theorem) Let $ K,L $ be normal subgroups of $ G $ with $ K\le L\le G $. Then we have that $ \frac{G/K}{L/K}\cong \frac GL $.
\end{theorem}
\pf Define a map $ \varphi: G/K \rightarrow G/L $, by $ \varphi(gK)=gL $. First we'll show that $ \varphi $ is a well-defined homomorphism, then we'll calculate the image and kernal, and finally apply the first isomophism theorem. To see well-definedness, if $ gK=g'K $, then $ g'\inv g\in K\subseteq L $, so $ g'L=gL $, so $ \varphi $ is well-defined. Obviously a homomorphism.

The kernal of $ \varphi $ is $ \ker \varphi = \{gK:gL=L\} =\{gK: g\in L\} = L/K $. $ \varphi $ is clearly surjective, so we conclude by the first isomorphism theorem that $ \frac{G/K}{L/K}\cong \frac GL $.\qed
\begin{definition}
	(Simple groups) A group $ G $ is called \textit{simple} if the only normal subgroups are $ G $ itself and $ \{ e \} $.
\end{definition}
\begin{proposition}
  Let $ G $ be an abelian group. Then $ G $ is simple if and only if  $ G\cong C_p $, for $ p $ prime.
\end{proposition}
\pf If $ G\cong C_p, $ then any $ g\in G, g\ne e $ is a generator of $ G $ by Lagrange. Conversely if $ G $ is simple and abelian, then take some non-identity, $ g\in G $, then $ \{g^n :n\in \mathbb Z\} $ is a subgroup, and because $ G $ is abelian, this subgroup is normal. Since $ g\ne e $, we must have $ G $ is cyclic, generated by $ g $. Now if $ G $ is infinitely cyclic, then $ G\cong\mathbb Z $, which is not simple since $ 2\mathbb Z\nrm \mathbb Z $, so we can't have this. Therefore $ G\cong C_m $ for some $ m\in \mathbb Z_{>0} $. Say $ q $ divides $ m $, then the subgroup of $ G $ generated by $ g^{\frac mq} $ is a normal subgroup, so we must have that $ q=m $ or $ q=1 $ by simplicity, hence we have that $ m $ is prime.\qed
\begin{theorem}
	(Composition series) Let $ G $ be a finite group. Then there exists subgroups such that, $ G=H_1\triangleright H_2\triangleright H_3\triangleright\cdots \triangleright H_n=\{e\} $, such that $ \frac {H_i}{H_{i+1}} $ is simple.
\end{theorem}
\pf If $ G $ is simple then take $ H_2=\{e\} $ and we're done. Otherwise, let $ H_2 $ be a proper normal subgroup of maximal order in $ G $. We claim that $ G/H_2 $ is simple. To see this, suppose not and consider $ \varphi: G\to G/H_2 $.
 By non-simplicity and correspondence between normal subgroups, we find a proper normal in $ G/H_2 $ and therefore a proper normal $ K\nrm G $. This leads to a contradiction as $ K $ contains $ H_2 $ non-trivally, so we contradict maximality, so $ G/H_2 $ is simple. Now we continue by replacing $ G $ with $ H_2 $ and iterate the process. Either we get that $ H_2 $ simple and we're done again, or we get find a proper normal subgroup $ H_3\nrm H_2 $ of maximal order. This process must terminate, since $ G $ is finite and the order is strictly decreasing in each step.\qed

\par
 We know from Part IA groups that $ A_5 $ is simple. We see a series like this for $ S_5 $, namely, $ S_5\triangleright A_5\triangleright \{e\} $.
 \subsection{Groups actions and permutations}

\begin{definition}
	Let $ X $ be a set. Let $ \sym(x) $ denote the symmetric group of $ X $ and $ S_n=\sym([n]) $ where we have that $ [n]=\{1,2,\dots, n\} $.
\end{definition}
Reminders from IA Groups:
\begin{enumerate}
	\item We can write any $ \sigma\in S_n $ as a product of disjoint cycles.
	\item If $ \sigma \in S_n $ we can write $ \sigma $ as a product of transpositions. The number of transpositions needed to write $ \sigma $ is well-defined modulo 2. This is called the sign of the transposition, denoted by $ \mathrm {sgn} $, where $ \mathrm{sgn}: S_n\to \{\pm 1\} $.
	\item $ \mathrm{sgn} $ is a homomorphism between the groups where $ \{\pm 1\} $ is given the unique group structure. When $ n\ge 3 $, the homomorphism is surjective.
\end{enumerate}
\begin{definition}
	(Alternating group) The \textit{alternating group} $ A_n $ is the kernal of $ \mathrm{sgn} $.
\end{definition}
A homomorphism $ \varphi: G\to \sym(X) $ is called a permutation representation of $ G $.
\begin{definition}
	(Group action) An \textit{action} of $ G $ on a set $ X $ is a function $ \tau:G\times X\to X $ sending $ (g,x) \to \tau(g,x)\in X $ such that $ \tau(e,x)=x, \forall x\in X $, and $ \tau(g_1,\tau(g_2,x))=\tau(g_1g_2,x), \forall g_1g_2\in G, \forall x\in X $.
\end{definition}
How are actions and permmutation representations related?

For some homomorphism, $ \varphi: G\to\sym(X) $ we map the homomorphism to $ a(\varphi):G\times X\to X $, where $ (g,x)\to\varphi(g)(x) $.
\begin{proposition}
  The funtion $ a $ above is a bijection from the set of homomorphism from $ G\to\sym(X) $ to the set of actions from $ G $ on $ X $.
\end{proposition}
\pf We'll construct an inverse of $ a $. Given a group action $ *:G\times X \to X $. Define $ \varphi(*): G\to \sym(X) $ defined by sending $ g\to\varphi(*)(g) $, where $ \varphi(*)(g)(x)=g*x $. We aim to show that $ \varphi(*)(g):X\to X $ is a permutation. We have an inverse $ \varphi(*)(\inv g) $, and to see that it is a homomorphism $ \varphi(*)(g_1)\varphi(*)(g_2)(x)=g_1*(g_2*x)=(g_1g_2)*x=\varphi(*)(g_1g_2)(x) $. This is true for all $ x $, so the construction is a group homomorphism.\qed

\par
\par \textit{Notation}: Given a group action $ G $ acting on $ X $ given by $ \varphi:G\to\sym(X) $, denote $ G^X=\ima(\varphi) $, and $ G_X=\ker(\varphi) $. By the first isomorphism theorem we have that $ G_X\nrm G $ and $ G/G_X\cong G^X $.

\par
For an example, consider the unit cube. Let $ G $ be the its symmetric group. Now let $ X $ be the set of (body) diagonals of the cube. Any element of $ G $ sends a diagonal to another diagonal, we get an action $ G\to\sym(X)\cong S_4 $. The kernal $ G_X=\ker(\varphi)=\{e, \text{ send each vertex to its opposite} \} $. Easy exercise to check that any diagonal can be sent to any other diagonal, so $ G^X=\ima (\varphi)=\sym(X) $. So by the first isomorphism theorem, we have that $ S_4\cong G^X\cong G/G_X\implies \frac{|G|}2 = 4!\implies |G|=48 $.

\par
For the next example let's look at a group acting on itself. Let $ G $ act on itself by $ G\times G\to G $, sending $ (g,g_1)\to gg_1 $. This gives a homomorphism $ G\to \sym(G) $ (easy to check that $ \varphi $ is injective since the kernal is trival). By the first isomorphism theorem we get that every group is isomorphism to a subgroup of a symmetric group (Cayley's theorem).

\par
Now let $ H\le G $ and let $ X=G/H $, let $ G $ act on $ X $ by $ g*g_1H=gg_1 H $. We get $ \varphi G\to\sym(X) $. Consider $ G_X=\ker \varphi $. If $ g\in G_X $, then $ gg_1H=g_1H, \forall g_1\in G $, so $ \inv {g_1}gg_1 H=H\implies G_X\subseteq \bigcap_{g_1\in G} g_1H\inv{g_1}.$ This argument is completely reversible, so if $ g\in\bigcap_{g_1}g_1H\inv g_1, $ then for each $ g_1\in G, $ we have $ \inv{g_1}gg_1\in H $, so $ g\in G_X\implies G_X=\bigcap_{g_1\in G}g_1H\inv{g_1} $. Since $ G_X $ is a kernal and is a subset of $ H $, we've got a way of making $ H $ smaller and making it normal. This is the largest normal subgroup contained in $ H $.
\begin{theorem}
	  Let $ G $ be finite and $ H\le G $ of index $ n $. There exists a normal subgroup of $ G $, $ K\nrm G $, with $ K\le H $, such that $ G/K $ is isomorphic to a subgroup of $ S_n $. Thus, $ |G/K| $ divides $ n! $, and $ |G/K|\ge n$.
\end{theorem}
\pf Consider $ G $ acting on $ G/H $ in the previous example. So the kernal of $ \varphi:G\to\sym(G/H) $ is normal, denote it by $ K $. We've shown it is contained by $ H $. First isomorphism theorem gives that $ G/K\cong \ima(\varphi)\le Sym(X)\cong S_n $. Give that $ |G/K| $ divides $ n! $ by Lagrange. Since that $ K\le H, $ we have that $ |G/K|\ge |G/H|\implies |G/K|\ge n $.\qed
\begin{corollary}
	  Let $ G $ be non-abelian and simple. Let $ H\le G $ be a proper subgroup of index $ n>1 $. Then $ G $ is isomorphism to a subgroup $ A_n $. Moreover, $ n\ge 5, $ i.e. no subgroup of index less than $ 5 $.
\end{corollary}
\pf Action of $ G $ on the set $ X=G/H $ gives a homomorphism $ \varphi:G\to \sym(X)\cong S_n $. Since the kernal is normal, since $ G $ is simple it is either $ G $ or $ \{e\} $. Since $ H $ is a proper subgroup, for some $ g\in G $, $ gH\ne H $, so we must have that $ \ker \varphi=\{e\} $. So $ G\cong \ima\varphi\le S_n $. Now we want to show that $ \ima\varphi\le A_n $. To see this observe that $ A_n\nrm S_n $. Consider $ A_n\cap \ima\varphi\le \ima\varphi $. By the second isomorphism theorem, $ \ima\varphi\cap A_n\nrm \ima\varphi\implies\ima\varphi\cap A_n=\{e\} $ or $ \ima\varphi $ itself. By the rest of the second isomorphism theorem, if $ \ima\varphi\cap A_n=\{e\}\implies \ima\varphi\cong\frac{\ima\varphi}{\ima\varphi\cap A_n}\cong \frac{\ima\varphi A_n}{A_n}\le \frac{S_n}{A_n} \cong C_2 $, but $ G $ is non-abelian, so $ \ima \varphi $ is non-abelian, so we have a contradiction. So we have that $ \ima\varphi\cap A_n =\ima\varphi $, so $ \ima \varphi $ is a subgroup of $ A_n $.\par
For the next part of the corollary, $ S_1,S_2 $ are abelian and $ S_3,S_4 $ have no non-abelian simple subgroups, so we must have $ n\ge 5 $.\qed

\begin{definition}
	(Orbits and stabiliser) Let $ G $ act on some set $ X $. Then, the \textit{orbit} of $ x\in X $ is $ G\cdot x=\orb x=\{gx : g\in G\}\subseteq X $. And the \textit{stabiliser} of $ x\in X $ is $ G_x=\stab_G(x) = \{g\in G:gx=x\}\le G $.
\end{definition}
\begin{theorem}
	(Orbit-stabiliser) For a group $ G $ acting on a set $ X $. For all $ x\in X $, there is a bijection $ G\cdot x \to G/G_x $ given by $ g\cdot x \to gG_x $. In particular, if $ G $ is finite, then $ |G|=|G\cdot x||G_x|, \forall x\in X $.
\end{theorem}
\pf In the IA Groups course.
\subsection{Conjugacy, centralisers, and normalisers}
Let $ G $ be a group. The conjugation action of $ G $ acting on itself by $ G\times G\to G $, is $ (g,h)\to gh\inv g $. This is equivilent to a homomorphism $ G\to\sym(G) $.

\par
Fix $ g\in G $. Then the permutation $ G\to G $ given by $ h\to gh\inv g $ is also a homomorphism.

\begin{definition}
	(Automorphism) Let $ G $ be a group. A permutation $ G\to G $ that is also a homomorphism is called an \textit{automorphism} of $ G $. The set of all automorphisms of $ G $, $ \aut(G) =\{f:G\to G: f \text{ is a automorphism}\}\subseteq \sym(G) $, is a subgroup, called the automorphism group of $ G $.
\end{definition}

\begin{definition}
	(Conjugacy classes and centralisers) Fix $ g\in G $. The \textit{conjugacy class} of $ g $ is the set $ \ccl_G(g) = \{hg\inv h: h\in G\} $, i.e it is the orbit under the conjugation action. The \textit{centraliser} of $ g\in G $ is $ C_G(g)=\{h\in G:hg\inv h=g\} $, i.e the stabiliser of $ g $ under the action.
\end{definition}
\begin{definition}
	(Centre) The \textit{centre} of $ G $ is $ Z(G)=\{z\in G:hz\inv h=z\forall h\in G\} $, i.e. it is the kernal of the conjugation action and the intersection of the centralisers.
\end{definition}
\begin{corollary}
	Let $ G $ be a finite group. Then $ |\ccl_G(x)|=|G:C_G(x)|=\frac{|G|}{|c_G(x)|} $.
\end{corollary}	
\pf Apply orbit-stabiliser to the conjugation action.
\begin{definition}
	(Normaliser) Let $ H\le G $. The \textit{normaliser} of $ H $ in $ G $ is $ N_G(H)=\{g\in G: gH\inv g = H\} $
\end{definition}
We can see clearly that $ H\subseteq N_G(H) $ so $ N_G(H) $ is non-empty and we also have that $ N_G(H)\le G $.

\par
In fact we have that $ N_G(H) $ is the largest subgroup containing $ H $ in which $ H $ is normal.
\subsection{Simplicity of $ A_n $ for $ n\ge 5 $}
Recall from Part IA groups that a conjugacy class in $ S_n $ consists of the set of all elements with a fixed cycle type.
\begin{theorem}
  Let $ n\ge 5 $. Then $ A_n $ is simple.
\end{theorem}
\pf We will prove the statement via these three claims:
\begin{enumerate}
	\item $ A_n $ is generated by 3-cycles
	\item If $ H\nrm A_n $ that contains a 3-cycle then it contains all the 3-cycles
	\item Any non-trival $ H\nrm A_n $ contains a 3-cycle.
\end{enumerate}
First we prove the first claim. Let $ g\in A_n $, when viewed in $ S_n $ it is the product of evenly many transposition. Consider a product of two transpositions:
\begin{enumerate}
	\item $ (ab)(ab)=e\in A_n $
	\item $ (ab)(bc)=(abc)\in A_n $
	\item $ (ab)(cd)=(acb)(acd)\in A_n $.
\end{enumerate}
In each case we can write all products of transpositions as a product of 3-cycles, hence we can write all elements in $ A_n $ as a product of 3-cycles.

\par
Now for the second claim, any two 3-cycles in $ A_n $ are conjugate when viewed in $ S_n $. Let $ \delta, \delta' $ be 3-cycles and write $ \delta' =\sigma\delta\inv\sigma $, where $ \sigma\in S_n $. If $ \sigma $ is even, we're done since it's in $ A_n $. If $ \sigma $ is odd, observe since $ n\ge 5 $, there exists a transposition $ \tau $ disjoint from $ \delta $, now $ \delta'=\sigma(\tau\inv\tau)\delta\inv\sigma=(\sigma\tau)\delta\inv{(\sigma\tau)} $. Since $ \sigma\tau $ is even, we're done.

\par
Finally for the last claim take some $ H\nrm A_n $ not trival. We break into cases
\begin{enumerate}
	\item (a) If $ H $ contains an element on the form $ \sigma=(1\space 2\cdots r)\tau $ where $ \tau $ is disjoint from $ 1,\dots, r, $ and $ r\ge 4 $. Then let $ \delta = (1\space2\space3) $. Now consider $ \delta\sigma\inv\delta\in H $ (by normality). But then $ \inv\sigma\inv\delta\sigma\delta\in H $ as well. As $ \tau $ misses $ 1,2,3 $ and commutes with $ (1\space2\cdots r) $ we expand this: $ \inv\sigma\inv\delta\sigma\delta=(r\cdots 2\space 1)(1\space 3\space 2)(1\space 2\space 3\cdots r)(1\space 2\space 3)=(2\space 3\space r) $ so we find a 3-cycle.
	\item (b) Suppose $ H $ contains $ \sigma=(123)(456)\tau $ (or any relabeling of such). $ \tau $ is disjoint from $ 1,\cdots,6 $. Take $ \delta = (124) $ and calculate the conjugation $ \inv\sigma\inv\delta\sigma\delta=(124236) $ which is a 5-cycle so we're done by the first case.
	\item (c) Suppose that $ H $ contains $ \sigma $ of the form $ \sigma = (123)\tau $ where $ \tau $ is a product of disjoint transpositions. Note if $ \tau $ contains anything longer than a tranposition, we can just apply case (a) or (b). Then $ \sigma^2=(123)^2 $ which is a 3-cycle since the transpositions cancel.
	\item (d) Suppose that $ H $ contains $ \sigma=(12)(34)\tau $, where $ \tau $ is a product of transpositions. Let $ \delta = (123) $, consider $ \mu = \inv \sigma\inv\delta\sigma\delta=(14)(23) $. Let $ \nu = (152)\mu(125) = (13)(45) $. But observe that $ \mu\nu\in H $, but this is a 5-cycle, so we're done by case (a).
\end{enumerate} 
Up to relabeling, we're covered all the cases. Hence any normal subgroup of $ A_5 $ must be trivial or $ A_5 $ itself, so $ A_5 $ is normal.\qed
\subsection{Finite $p$-groups}
\begin{definition}
	(Finite $ p $-groups) For $ p $ prime, a \textit{finite $ p $-group} is a group of order $ p^n $, $ n\in\N $.
\end{definition}

\begin{theorem}
  Let $ G $ be a finite $ p $-group. Then $ Z(G) $ is non-trival.
\end{theorem}
\pf Consider $ G $ acting on itself by conjugation. The centre of $ G $ is the union of orbits of size 1. The orbits partition $ G $, so
\[
	|G|=p^n=|Z(G)|+\sum\text{sizes of conjugacy classes of size} > 1  
\]
We know that the sizes of the non-trivial conjugacy classes always divide $ p^n $. So all the terms of size larger than one are divisible by $ p $. Hence we have that $ p $ divides $ |Z(G)| $. So since $ p\ge 2 $, the centre is non-trivial.\qed
\begin{theorem}
  A group of size $ p^2 $ must be abelian.
\end{theorem}
\pf Follows from an independently interesting technical result:
\begin{lemma}
	If $ G $ is any group and $ \frac G{Z(G)} $ is cyclic, then $ G $ is abelian.
\end{lemma}
\pf Let $ xZ(G) $ generate $ \frac G{Z(G)} $. Every coset of the form $ x^mZ(G), m\in Z $. Since any $ g\in G $ lies in some coset of $ Z(G) $, we can write $ g=x^mz $, for some $ z\in Z(G) $. Now for some $ g'\in G $, $ g'=x^nz' $, so $ gg'=x^mzx^nz'=x^{n+m}zz'=x^nz'x^mz=g'g $, so the group is abelian.

\par
Our proof of the theorem follows since $ Z(G) $ is non-trivial, so it either has size $ p^2 $ or $ p $. If it has size $ p^2 $, the group is abelian so we're done. If it has size $ p $, the $ G/Z(G) $ also has size $ p $, so it's cyclic, hence it's abelian, so by the lemma we have that $ G $ is abelian.\qed
\begin{theorem}
  Let $ G $ be a group of size $ p^n $. Then for any $ 0\ge k \ge n$, $ G $ has a subgroup of size $ p^k $.
\end{theorem}
\pf (Inductive proof) The base case $ n=1 $ is clear because the group must be cyclic. Now suppose that $ n>1 $, if $ k=0 $, we take $ \{e\} $, so we're done, so assume that $ k\ge 1 $.  Note that $ Z(G) $ is non-trivial, let $ x\in Z(G) $ with $ x\ne e $. The order of $ x $ is a power of $ p $. By raising $ x $ to some power we can find an element with order $ p $ in $ Z(G) $. Replacing $ x $ with this element we can assume $ \mathrm{ord}(x)=p $. The subgroup generated by $ x $ is normal of size $ p $ because $ x $ is central of order $ p $. Now $ \frac{G}{\langle x\rangle} $ is a group of order $ p^{n-1} $ so inductive hypothesis allies. Let $ L\le \frac{G}{\langle x\rangle} $ of size $ p^{k-1} $. But by the subgroup correspondence result, we can find some $ K\le G $ containing $ \langle x\rangle $ such that $ \frac K{\langle x\rangle}=L $. So $ K $ has size $ p^k	 $, so we're done.\qed

\subsection {Finite abelian groups}
\begin{theorem}
	(Classification of finite abelian groups) Let $ G $ be a finite abelian group. There exists positive integers $ d_1,\cdots, d_r $ such that:
	\[
		G\cong C_{d_1}\times C_{d_2}\times \cdots\times C_{d_r}
	\]
	Moreover, we can choose $ d_i $ such that $ d_{i+1}\mid d_i $ in which case this is unique.
\end{theorem}
\pf To come later...

\par
Abelian groups of order 8 are exactly $ C_8, C_4\times C_2, C_2\times C_2\times C_2 $.
\begin{lemma}
	(Chinese remainder theorem) If $ n $ and $ m $ are coprime, then $ C_n\times C_m\cong C_{nm} $
\end{lemma}
\pf Consider $ C_n\times C_m $. Suffices to produce an element of order $ nm $. Let $ g\in C_n $ and $ h\in C_m $ be generators of order $ n $ and $ m $ respectively. Consider $ (g,h) $. Say its order is $ k\implies (g,h)^k=(e,e) $. So $ n,m $ both divide $ k $, and since $ n,m $ are coprime we have that $ nm $ divides $ k $ and by Lagrange we have that $ k $ divides $ nm $, so we're done.\qed
\subsection{Sylow Theorems}
\begin{definition}
	(Sylow $ p $-subgroup) Let $ G $ be a finite group of order $ p^am$, where $ p\nmid m $, $ p $ is a prime. Then a \textit{Sylow p-subgroup} of $ G $ is a subgroup of size $ p^a $.
\end{definition}
\begin{theorem}
	(Sylow theorems) For a finite group $ G $ of order $ p^am $, where $ p\nmid m $, $ p $ is prime:
	\begin{enumerate}
		\item The set $ \syl_p(G)=\{P\le G$ $ |$ $P \text{ is a Sylow p-subgroup of } G \}$ is non-empty.
		\item Any $ H,H'\in \syl_p(G) $ are conjugate, namely $ H=gH'\inv g $, for some $ g\in G $.
		\item If $ n_p=|\syl_p(G)| $ then $ n_p\equiv 1 \mod p $ \textit{and} $ n_p $ divides $ |G| $, so $ n_p\mid m $
	\end{enumerate} 
\end{theorem}
Before we prove the statement, let's see why this theorem is useful.
\begin{lemma}
	If $ \syl_p(G)=\{P\} $, then $ P $ is normal in $ G $.
\end{lemma}
\pf For any $ g\in G $, the subgroup $ g P\inv g $ is isomorphic (as a group) to $ P $. So $ g P\inv g $ is in $ \syl_p(G) \implies gP\inv g=P$, which proves the claim.\qed
\begin{corollary}
	Let $ G $ be a non-abelian simple group, and $ p\mid |G| $, $ p $ prime. Then $ |G| $ divides $ \frac{n_p!}2 $ and $ n_p\ge 5 $.
\end{corollary}
Let $ G $ act by conjugation on $ \syl_p(G) $ which gives a homomorphism $ \varphi:G\to \sym(\syl_p(G))\cong S_{n_p} $. By simplicity, $ \ker \varphi=G $ or $ \{e\} $. If $ \ker \varphi=G $, then $ gP\inv g=P $ for all $ g\in G $ and all $ P\in\syl_p(G) $. So $ P $ is normal. Thus $ P $ is either $ \{e\} $ or $ G $. Well $ P$ is Sylow-$ p $ so it can't be $ \{e\} $, so $ P=G $. So $ G $ would be a $ p $-group. But from earlier, the centre of $ G $ is non-trivial proper since $ G $ is non-abelian, but the centre is always normal, so this contradicts simplicity, hence $ \ker \varphi=\{e\} $. So we have that $ \varphi $ is an injective homomorphism $ G\to S_{n_p} $, so by the first isomorphism theorem, $ G\cong \ima \varphi $. We'll show that $ \varphi $ lands in $ A_{n_p} $. Consider the composition $ G\to S_{n_p}\to \{\pm 1\} $. If this composition is surjective, then $ \ker (\mathrm{sgn} \circ \varphi) $ is index $ 5 $, but $ G $ simple so not possible. So $ \ima\varphi\subseteq \ker(\mathrm{sgn})=A_{n_p} $, so we're done by Lagrange. For the final statement we show all non-abelian subgroups of $ A_2, A_3, A_4 $ are not simple which finishes the statement which is just grunt work, and I pinky promise it's true, so we're done.\qed

\par
Let's see a sample application. Let have $ G $ has size $ 11\times 12 $. If $ G $ is simple then there are exactly $ 12 $ Sylow 11-subgroups. Consider the number $ n_{11} $. We know from the Sylow theorems that $ n_{11}\equiv 1\mod 11 $ and $ n_{11}\mid 12 $. So $ n_{11} = 12 $ since $ G $ is simple. Similarly $ n_3\equiv 1 \mod 3 $ and $ n_3\mid 44 $. So either $ n_3 = 4 $ or $ 22 $. The corollary says that $ G $ divides $ \frac{n_3!}2 $, so $ n_3 $ can't be $ 4 $, so $ n_3=22 $. But this is a lot of elements. And 2 Sylow 11-subgroups interset only at the identity which leads to too many elements, so none of this even works, which seems confusing, but actually just means that $ G $ can't exist, hence all groups of order $ 132 $ are non-simple.

\par
Finally we now prove the Sylow theorems.

\pf Let $ G $ be a group of order $ n=p^am $, with $ p\nmid m $, $ p $ prime. Define the set $ \Omega=\{X\subseteq G : |X|=p^a $. Let $ G $ act on $ \Omega $ by multiplying all elements of $ \Omega $ on the left by $ g\in G $ (we can see this obeys the axioms of the group action after some quick inspection. We have $ |\Omega|= \binom{n}{p^a}\equiv m\ne 0\mod p $. The proof of this can be seen by expanding out the binomial coeffient, but we'll assume it here. Suppose we have some $ U\in\Omega $, then let $ H\le G $ stabilise $ U $. Then $ |H|\mid |U| $. We can prove this by seeing that $ hU=U $ for all $ h\in H $. In other words for each $ u\in U $ the coset $ Hu $ is contained in $ U $. Every $ u\in U $ lies in some coset of $ H $, so the cosets partition $ U $, so $ |H|\mid |U| $. We know that $ |\Omega|\ne 0\mod p $. Since orbits partition, we know that
	\[
		|\Omega|=|O_1|+|O_2|+\cdots +|O_r|\text{, } O_i \text{ are the orbits}
	\]
So there exists an orbit $ \Theta $ whose size is prime to $ p $.
Let $ T\in\Theta $. By orbit-stabiliser, $ |G|=|\Theta||\stab(T)| $. So $ p^am=|\Theta||\stab(T)| $. By our previous lemma, $ |\stab T|\mid p^a $, so we're done because there are no factors of $ p $ in $ \Theta $, so we've prove the first part of the theorem.

\par
Now for the second part, we actually show something stronger, that is, if $ Q\le G $ is a subgroup of size $ p^b $, where $ 0\le b\le a $, then there exists $ g\in G $ and $ P\in \syl_p(G) $, such that $ gQ\inv g\le P $. To prove this, let $ Q $ act on $ G/P $ by left coset multiplication. Note that the size of $ G/P $ does not divide by $ p $. Orbits have size dividing $ p^b $, so each orbit has size $ 1 $ or a power of $ p $. But $ p\nmid |G/P| $, so there exists a size 1 orbit. In other words, there exists some coset $ gP $ such that $ \forall q\in Q $, $ qgP=gP $, so rearranging gives that $ gQ\inv Q\le P $. So our second statement follows taking $ b=a $.

\par
For the final theorem, we need to show that $ n_p\mid |G| $, and $ n_p\equiv 1\mod p $. For the first statement, consider $ G $ acting on $ \syl_p(G) $ by conjugation. By the second theorem, we know that there is one orbit of size $ n_p $, so the statement follows instantly from orbit-stabiliser. For the second statement, let $ P\in \syl_p(G) $. Consider $ P $ acting on $ \syl_p(G) $ by conjugation. By orbit-stabiliser, all the orbits have size 1 or $ p $. Since $ \{P\} $ is a size 1 orbit, to prove the statement is suffices to show that $ \{P\} $ is the only size 1 orbit. Say $ \{Q\} $ is another size 1 orbit. So $ \forall h\in P $, we have $ hQ\inv h=Q $. This means that $ N_G(Q) $ contains $ P $. Now observe if $ p^a $ is the largest power of $ p $ dividing $ |G| $, we know that it's the largest power of $ p $ dividing $ |N_G(Q)| $. But $ Q $ is normal in $ N_G(Q) $ by definition, and $ Q,P\in \syl_p(N_G(Q))\implies P=Q $, since normality $ \iff $ uniqueness for Sylow subgroups. So we've prove all the Sylow theorems and we're done. \qed
\newpage
\section{Rings}
\subsection{Definitions and examples}
\begin{definition}
	(Rings) A \textit{ring} is a quintuple $ (R,+,\circ,0_R,1_R) $, where $ R $ is a set with $ 0_R,1_R\in R $, and $ +:R\times R\to R $, and $ \circ:R\times R\to R $, called addition and multiplication are functions satisfying the following:
	\begin{enumerate}
		\item $ (R,+,0_R) $ is an abelian group.
		\item $ \circ $ is associative, so $ a\circ(b\circ c)=(a\circ b)\circ c $.
		\item $ 1_R\circ a = a\circ 1_R=a $.
		\item We have distributivity, so $ r_1\circ (r_2+r_3)=(r_1\circ r_2)+(r_1\circ r_3) $ and $ (r_1+r_2)\circ r_3=(r_1\circ r_3)+(r_2\circ r_3) $.
	\end{enumerate}
\end{definition}

Usually we just say "Let $ R $ by a ring..." with everything implicit. The symbol $ (-r) $ denotes the additive inverse of $ r $.\par
In IB Groups, Rings and Modules, rings will always be commutative, so $ r_1\circ r_2=r_2\circ r_1 $ for all $ r_1,r_2\in R $.

\begin{definition}
	(Subring) A \textit{subring} of a ring $ R $, is a subset $ S\subseteq R $, such that $ 0_R,1_R\in S $, $ S $ is closed under both multiplication and addition of the ring, and $ (S,+,\circ, 0_R,1_R) $ is a ring. 
\end{definition}
We notate this as $ S\le R $.\par
For examples we have $ \Z\le\Q\le\R\le\C $ which are all rings under usual multipliction and addition. Along a similar line, we also have the Gaussian integers, $ \Z[i]=\{a+ib:a,b\in \Z\} $ with multiplication and addition induced by $ \C $.\par
Another example is $ \Z/n\Z $ which forms a ring under addition and multiplication modulo $ n $. In $ \Z/6 $ we have $ 2, 3\in \Z/6 $ such that $ 2\circ 3=0\mod 6 $ which is perfectly allowed.

\begin{definition}
	(Units) An element $ u\in R $, is called a \textit{unit} if there exists some $ v\in R $, such that $ uv=1_R\in R $.
\end{definition}
This notion does \textit{not} interact well with subrings, as we can take a unit in a subring without taking it's inverse, making it no longer a unit. For example 2 is a unit $ \Q $, but not in $ \Z $.
\par\textit{Discussion.} Does $ 0_R $ behave like it should? We would like $ 0\circ R=0_R $ for all $ r\in R $. In $ R $ we have that $ 0_R+0_R =0_R $, now multiplying by $ r\in R $, so $ r\circ 0_R+r\circ 0_R=r\circ 0_R $, hence cancelling a $ r\circ 0_R $ on both sides gives that $ r\circ 0_R=0_R $.\par
In particular this implies that if $ 1_R=0_R $ then for any $ r\in R $, $ r=r\circ 1_R=r\circ 0_R=0_R $ so for all $ r\in R $, $ r=0_R $, so $ R $ must be the zero ring, $ \{0_R\} $.\par

\begin{definition}
	(Polynomial) Let $ R $ be a ring. Then a \textit{polynomial} in $ x $ with coefficents in $ R $ in an expression:
	\[
		f(x)=a_0+a_1x+\cdots+a_nx^n
	\]
	and $ x^i $ are formal symbols. We will identify $ f(x) $ with $ f(x)+0\circ x^{n+1} $ as the same. The largest $ i $ such that $ a_i \ne 0$ is called the degree of the polynomial. A polynomial $ f(x) $ is monic of degree $ n $ if $ a_n=1 $ and it is of degree $ n $.
\end{definition} 
\begin{definition}
	(Polynomial ring) The \textit{polynomial ring} $ R[X] $ is given by:
	\[
		R[X]= \{f(X): \text{ f is a polynomial in } X \text { with coefficents in } R\}
	\]
	$+, \circ  $ are the usual operations, $ 0_{R[X]}=0_R $ and $ 1_{R[X]}=1_R $.
\end{definition}

\begin{definition}
	(Ring of formal power series) The \textit{ring of formal power series} is a ring in $ X $ with coefficents in $ R $ is:
	\[
	R[[X]]=\left\{\sum_{n=0}^\infty r_iX^i:a_i\in R,\forall i\ge 0, i\in\Z\right\}
	\]
	with the standard $ +,\circ $ of $ R $.
\end{definition}
For an example consider $ (1-x)\in R[X] $. Is it a unit? No! If $ g(x)(1-x)=1 $, then if $ g(x)=a_0+a_1x+\cdots a_nx^n $, $ a_n\ne 0 $, then $ (1-x)g(x) = a_0 + (a_1-a_0)x+\cdots (a_n-a_{n-1}x^n-a_nx^{n+1} $ which cannot be 1 since the highest power term has a non-zero coefficent.\par
However $ (1-x) $ is a unit in $ R[[X]] $! $ (1-x)(1+x+x^2+\cdots)=1\in R[[X]] $.\par
\begin{definition}
	(Laurent polynomials) If $ R $ is a ring then a \textit{Laurent polynomial} with coeffients in $ R $ is:
	\[
		R[X,\inv X] = \left\{\sum_{i\in\Z}a_iX^i: a_i\in R,\forall i\in\Z\right\}
	\]
	Where $ a_i $ is non-zero for at most finitely many $ i $ and with standard multiplication and addition.
\end{definition}

If $ R $ is a ring, and $ X $ is a set the set of $ R $-valued functions, namely, $ \{f:X\to R\} $ is a ring with "pointwise" addition and multiplication as given by the ring $ R $. (So $ (f+g)(x)=f(x)+g(x) $)
\subsection{Homomorphisms, ideals, and quotients}
\begin{definition}
	(Ring homomorphism) Let $ R $ and $ S $ be rings. A function $ f:R\to S $ is a \textit{ring homomorphism} if for all $ r_1,r_2\in R $:
	\begin{enumerate}
		\item $ f(r_1+r_2)=f(r_1)+f(f_2) $
		\item $ f(0_R)=0_S $
		\item $ f(r_1r_2)=f(r_1)f(r_2) $
		\item $ f(1_R)=1_S $.
	\end{enumerate}
\end{definition}
These first two conditions are the conditions for $ f $ to be a group homomorphism with the addition operation. Note that the second condition is not required and it follows from the first condition. But non-symmetrically the fourth condition is not implied by the third condition.
\begin{definition}
	(Isomorphism) An \textit{isomorphism} $ f:R\to S $ is a bijective ring homomorphism. The inverse function is also a ring homomorphism.
\end{definition}
\begin{definition}
	(Kernal) The \textit{kernal} of a ring homomorphism $ f: R\to S $ is the set $ \ker f = \{r\in R: f(r)=0_S\} $.
\end{definition}
\begin{definition}
	(Image) The \textit{image} of a ring homomorphism $ f: R\to S $ is $ \ima f = \{s\in S:s=f(r) \text{ for some } r\in R \}$.
\end{definition}

\begin{lemma}
	A homomorphism $ f:R\to S $ is injective if and only if $ \ker f = \{0\} $.
\end{lemma}
\pf Follows from the corresponding fact about groups. \qed
\begin{definition}
	(Ideal) A subset $ I\subseteq R $ is an \textit{ideal}, written as $ I\nrm R $, if $ I $ is a subgroup and if $ a\in I $ and $ b\in R $, then $ ab\in I $.
\end{definition}
Keep in mind that an ideal is usually not a subring, since if $ 1_R\in I $ then $ I=R $.
\begin{lemma}
  If $ f:R\to S $ is a ring homomorphism then $ \ker f \nrm R$.
\end{lemma} 
\pf Since $ f $ is also a group homomorphism, then $ \ker f $ is a subgroup. If $ a\in \ker f $ and $ b\in R $ then $ f(ab)=f(a)(b)=0f(b)=0 $, so $ ab\in \ker f $. \qed\par
Now we'll look at some examples.\par
If $ \Z $ is the ring of integers then $ n\Z $ are ideals for all $ n\in\N \cup \{0\} $. In fact, every ideal of $ \Z $ has this form. To see this $ I\ne \{0\} $ is an ideal. Let $ n\in \Z $ be the smallest postive element of $ I $. We claim that $ I=n\Z $. Let $ m\in I $. We claim that it's divisible by $ n $. Apply the Euclidean algorithm so $ m=qn+r $ where $ 0\le r<n $. But $ qn\in I $ by the absorbing property so $ r\in I $ since $ I $ is a subgroup which contradicts minimality unless $ r=0 $.\qed
\begin{definition}
  Let $ A\subseteq R $. The ideal generated by $ A $ is
  \[
	  (A)=\left\{\sum_{a\in A}r_aa,\quad r_a \in R, \quad \text{all but finitely many } r_a \text{ are } 0\right\}
  \]
\end{definition}
\begin{definition}
	(Principle) An ideal $ I\nrm R $ is \textit{principle} if there exists $ r\in R $ such that $ (r)=I $.
\end{definition}
For another example let $ \R[X] $ be the polynomial ring in one variable over $ \R $. The subset $ \{f\in \R[X]: \text{constant term is } 0\} $, is an ideal. It is actually principle, generated by $ (X) $.

\begin{definition}
	(Quotient) Let $ I\nrm R $ be an ideal. Then the \textit{quotient ring} $ R/I $ is the set of cosets $ r+I $ with $ 0_R/I=0_R+I $ and $ 1_R/I=1_R+I $, and operations $ (r_1+I)+(r_2+I)=(r_1+r_2)+I $ and $ (r_1+I)(r_2+I)=r_1r_2+I $.
\end{definition}

\begin{proposition}
  The quotient ring is a ring. The function $ f:R\to R/I $ sending $ r $ to $ r+ I $ is a ring homomorphism.
\end{proposition}
\pf Obviously an abelian group. Multiplication is well-defined. To see this suppose $ r_1+I=r_1'+I $ and $ r_2+I=r_2'+I $. Then $ r_1-r_1'=a_1\in I $, and $ r_2-r_2'=a_2\in I $, so $ r_1'r_2'=(r_1+a_1)(r_2+a_2)=r_1r_2+r_1a_2+r_2a_1+a_1a_2 $. By the absorbing property the last three terms are contained in $ I $, so $ r_1r_2+ I=r_1'r_2'+I $. The rest is straightforward.\qed\par
For another example, we have $ n\Z\nrm\Z $. The quotient $ \Z/n\Z $ is the usual ring of integers modulo $ n $.\par
Take $ (X)\nrm \C[X] $. The elements of $ \C[X]/(X) $ are represeneted by:
\[
	a_0+a_1X+\cdots a_n X^n+(X),\  \text{but} \ \sum_{i=1}^na_iX^i\in (X)
\]
so each coset is represented equivalently by $ a_0+(X) $, so we have that $ \C[X]/(X)\cong \C $.\par
Similarly $ (X^2)\nrm \C[X] $, the ring $ \C[X]/(X^2) $ consists of elements represented by linear polynomials $ a_0+a_1X+(X) $ with the following multiplication given by $ (a_0+a_1X)(b_0+b_1X)=a_0b_0+(a_1b_0+a_0b_1)X $.\par
This ring is quite weird. For example if we take $ X\in \C[X]/(X^2) $. Then $ 0\ne X $ but $ X^2=0 $. We say that $ X $ is nilpotent.

\begin{proposition}
	(Euclidean algorithm for polynomials in $ X $) Let $ K $ be a field and $ f,g\in K[X] $. Then there exists polynomials $ r,q\in K[X] $ such that $ f=gq+r $ with $ \mathrm{deg}(r)<\mathrm{deg}(g) $.
\end{proposition}
\pf Let $ n $ be the degree of $ f $. So $ f=\sum_{i=0}^na_iX^i $ with $ a_i\in K,a_n\ne 0 $. Similarly $ g=\sum_{i=0}^mb_iX^i $ with $ b_i\in K $ and $ b_m\ne 0 $.\par
If $ n< m $ set $ q=0 $ and $ r=f $ so we're finished.\par
If instead $ n\ge m $, proceed by induction on the degree. Let $ f_1=f-a_n\inv b_mX^{n-m}g $. Observe that $ \mathrm{deg}(f_1)<n $. If $ n=m $ then $ \mathrm{deg}(f_1)<n=m $. So write $ f=(a_{\inv b_m}X^{n-m})g+f_1 $, so we're done. Otherwise if $ n>m $, then because $ \mathrm{deg}(f_1)<n $, by induction we cab wrute write $ f_1=gq_1+r_1 $ where $ \mathrm{deg}(r_1)<\mathrm{deg}(g)=m $. Then $ f=(a_n\inv b_m)X^{n-m}g+q_1g+r_1=(a_n\inv b_mX^{n-m}+q_1)g+r_1 $\qed

\begin{corollary}
	If $ K $ is a field then $ K[X] $ every ideal is principle.
\end{corollary}
\pf Identical to the case of $ \Z $ using the proposition.\par
This proof fails for $ \Z [X] $ (since $ \Z $ is not a field) and for $ K[X,Y] $.
\begin{theorem}
	(First isomorphism theorem) Let $ \varphi:R\to S $ be a ring homomorphism. Then the function $ f:R/\ker\varphi\to\ima\varphi\le S $ sending $ r+\ker\varphi\to\varphi(r) $ is well-defined and an isomorphism of rings.
\end{theorem}
\pf Well-definedness, bijective, additive homomorphism property all follow from the group statement. We check multiplicativity. $ f((f+\ker\varphi)(t+\ker\varphi))=f(rt+\ker\varphi)=\varphi(rt)=\varphi(r)\varphi(t)=f(r+\ker\varphi)(f+t+\ker\varphi) $ since $ \varphi $ is a ring homomorphism.\qed
\par
For an example consider the homomorphism $ \varphi:\R[X]\to \C $. sending $ f(X) $ to $ f(i) $. Clearly this is a surjective ring homomorphism since $ a+bX\to a+bi $ under $ \varphi $. The kernal is exactly real polynomials $ f(X) $ such that $ f(i)=0 $ i.e $ i $ is a root. But since $ f $ has real coefficents that means that $ (X+i)(X-i)\mid f(X) $ i.e. $ (X^2+1)\mid f(X) $. So in fact $ \ker\varphi=(X^2+1) $, the ideal generated by $ X^2+1 $. Now applying the first isomorphism theorem $ \frac{\R[X]}{(X^2+1)}\cong \C $.

\begin{theorem}
	(Second isomorphism theorem) Let $ R\le S $ and $ J\nrm S $. Then $ J\cap R\nrm R $ and $ \frac{R+J}{J}=\{r+J:r\in R\}\le \frac SJ $. Furthermore,
	\[
		\frac{R}{R\cap J}\cong \frac{R+J}J.
	\]
\end{theorem}
\pf Define a function $ \varphi:R\to S/J $ by $ r\to r+J $. The kernal is $ \{r:r+J=0\}=\{r\in J\}=R\cap J $. The image $ \ima\varphi=\{r+J:r\in R\}=\frac{R+J}J $, so apply the first isomorphism theorem to conclude.\qed
\par
Again similar to groups we have a correspondence result.
\begin{theorem}
	(Correspondence theorem) If $ I\nrm R $ is an ideal there is a bijection between subrings of $ R/I $ and subrings of $ R $ which contain $ I $. This is given by sending $ L\le R/I\to \{r\in R: r+I\in L\} $ and conversely $ I\nrm S\le R\to S/I\le R/I $
\end{theorem}
\pf Same as from groups.
\par
Similar for ideals there is a bijection betwen ideals in $ R/I $ and ideals in $ R $ that contain $ I $.
\begin{theorem}
	(Third isomorphism theorem) Let $ I\nrm R $ and $ J\nrm R $ with $ I\subseteq J $. Then $ \frac JI\nrm \frac RI $ and we have that,
	\[
		\frac{R/I}{J/I}\cong R/J.
	\]
\end{theorem}
\pf Define a function $\varphi: R/I\to R/J $ sending $ r+I $ to $ r+J $. Well-definedness follows from the same argument as from groups. Easy verification to see it is a ring homomorphism.
The kernal is $ \ker\varphi=\{r+I:r+J=J\} $, i.e. that $ \ker\varphi=J/I $. So apply the first isomorphism theorem to get the result.\qed
\begin{claim}
  Let $ R $ be any ring. There is a unique ring homomorphism
  \[
	  i:\Z\to R
  \]
\end{claim}
The kernal of $ i $, $ \ker i $ is an ideal $ n\Z\nrm Z $. The number $ |n| $ is called the characteristic of $ R $. The rings $ \Z,\R,\C,\C[X] $ all have characteristic 0. $ \Z/k\Z $ has characteristic $ k $.
\subsection{Integral domains}
In the ring $ \Z/6 $ we have that $ 2\cdot 3 = 0 $. In an integral domain this will not happen.

\begin{definition}
	(Integral domain) A nonzero ring $ R $ is an integral domain if $ \forall a,b\in R $, if $ ab=0 $ then $ a=0 $ or $ b=0 $.
\end{definition}
An element that violates this is called a zero divisor, i.e. a zero divisor is a non-zero element $ a\in R $ such that $ \exists b\in R, b\ne 0 $ where $ ab=0 $.\par
All fields are integral domains, since if $ ab=0, b\ne 0 $ then $ a(b\inv b)=0\inv b=0 $ so $ a=0 $.\par
Any subring of an integral domain is an integral domain. To list a set of examples we have $ \Z,\Z[i],\Q,\C,\R[X],\Z[X], $ etc.
For a set of non-examples we have $ \Z/6, \Z/pq, \C[X]/(X^2) $ etc.
\begin{lemma}
  Let $ R $ be a finite integral domain. Then $ R $ is a field.
\end{lemma}
\pf Let $ a\in R $ be non-zero. Consider the function $ \mu_a: R\to R $ sending $ r\to ar $. It's easy to verify that $ \mu_a $ is an (additive) group homomorphism for all $ a $ non-zero. Since $ R $ is an integral domain, $ \ker\mu_a $ is trivial so the map is injective. So since $ R $ is finite, $ \mu_a $ is also surjective. In particular $ 1=ab $ for some $ b\in R $ hence this is an inverse of $ a $, so $ R $ is a field.\qed
\begin{definition}
	Let $ R $ be an integral domain. A \textit{text of fractions} for $ R $ is a field $ F $ such that:
	\begin{enumerate}
		\item $ R\le F $ is a subring,
		\item every $ x\in F $ can be written as $ a\inv b $, where $ a,b\in R, $ where $ \inv b $ is the multiplictive inverse to $ b $ in $ F $.
	\end{enumerate}
\end{definition}
$ \Q $ is a field of fractions for $ \Z $.
\begin{theorem}
  Every integral domain has a field of fractions.
\end{theorem}
\pf Define a set $ S=\{(a,b)\in R\times R:b\ne 0\} $. Place an equivalence relation $ \sim $, defined as $ (a,b)\sim (c,d)\iff ad=bc $ on $ S $. We can check this is an equivalence relation, the only non-trivial axiom to check is transitivity. Suppose that $ (a,b)\sim (c,d) $ and $ (c,d)\sim (e,f) $. So we have that $ ad=bc $ and $ cf=de $. We wish to deduce that $ af=be $. Multiple the first equality by $ f $ and the second by $ b $. So we get that $ adf=bcf $ and $ bcf=bed $. Rearranging we get $ d(af-be)=0 $ since $ d $ is non-zero and $ R $ is an integral domain we know that $ af=be $. So $ \sim $ is an equivalence relation. Now define $ F=\frac S\sim $ with notation $ \frac ab = [(a,b)]_\sim $. Now we turn $ F $ into a ring. Take the operations to be $ \frac ab + \frac cd = \frac {ad+bc}{bd} $ and $ \frac ab\frac cd = \frac{ac}{bd} $. Some elementary operations show that these operations are well-defined and makes $ F $ into a ring. To see that $ F $ is a field, if $ \frac ab\ne 0_F $ i.e. $ \frac ab\ne \frac 01\implies a\cdot 1\ne b\cdot 0 =0  $, so $ a\ne 0 $. now $ \frac ba \in F $ and $ \frac ba\frac ab=1_F $, so $ F $ is a field.\par
We now construct an injective homomorphism $ R\to F $ by $ r\to \frac r1 $. Straightforward to check that this is a ring homomorphism. The kernal is $ \{r\in R: \frac r1 = 0 \text { in } F\} =(0)$. By the first isomorphism theorem $ R $ is isomorphic to the image of $ R\to F $, in other words $ R\le F $.
Finally since $ \frac ab\in F $ is $ \frac ab= \frac a1\cdot \frac 1b\implies \frac a1\inv{(\frac b1)}=a\inv b $\qed

Sometimes we write $ \mathrm{FF}(R) $ for a field of fractions of $ R $.
\begin{proposition}
  Let $ R $ be a ring. Then $ R $ is a field if and only if the only ideals in $ R $ are $ (0) $ and $ R $.
\end{proposition}
\pf If $ R $ is a field and $ I\nrm R $ is non-zero then $ I $ contains a unit $ u $. Since $ 1=uv $ we have that $ 1\in I $. But for any $ r\in R $, we have $ 1\cdot r =r\in I $, so $ I=R $.\par
Conversely suppose that $ (0) $ and $ R $ are the only ideals of $ R $. Take $ r\in R $ non-zero. We know that $ (r)=R $ since $ r $ is non-zero. Since $ 1\in (r) $ we know that $ r\cdot b = 1 $ for some $ b\in R $ so $ r $ is a unit hence $ R $ is a field.

\begin{definition}
	(Maximal ideal) An ideal $ I\nrm R $ is called \textit{maximal} if it is not $ R $ itself and if for any $ J\nrm R $ with $ I\subseteq J\subseteq R $, either $ J=I $ or $ J=R $.
\end{definition}

\begin{proposition}
  An ideal $ I\nrm R $ is maximal if and only if $ R/I $ is a field.
\end{proposition}
\pf $ R/I $ is a field if and only if the ideals are $ R/I $ and $ (0) $. Now apply the ideal correspondence theorem.\qed
\begin{definition}
	(Prime ideal) An ideal $ I\nrm R $ is \textit{prime} if whenever $ ab\in I $ either $ a $ or $ b $ lies in $ I $.
\end{definition}
An ideal $ n\Z\nrm \Z $ is a prime ideal if and only if $ n $ is a prime number (or zero). We can see this since if $ n=p $ is prime, and $ ab\in p\Z $ then $ ab $ is a multiple of $ p $ so either $ a $ or $ b $ must be a multiple of $ p $ hence in $ p\Z $. Conversely if $ n $ is not prime and wlog positive (zero case is trivial) we know that $ n=m_1m_2 $, $ 1<m_1,m_2<n $. Then $ m_1,m_2\notin n\Z $ but $ m_1m_2\in n\Z $ so the ideal is not a prime ideal.
\par Interestingly $ p\Z\nrm \Z $ for $ p $ non-zero prime, then $ \Z/p\Z $ is a field so $ p\Z $ is maximal.

\begin{proposition}
  An ideal $ I\nrm R$ is prime if and only if $ R/I $ is an integral domain.
\end{proposition}
\pf If $ I \nrm R $ is prime, then let $ (a+I) $ and $ (b+I)\in R/I $. Suppose $ (a+I)\cdot(b+I)=(ab+I)=0+I $ (recall $ 0+I $ is the zero element in $ R/I $). This means that $ ab\in I $ but $ I $ is prime so $ a $ or $ b\in I $, so $ a+I $ or $ b+I $ is $ 0 $.\par
Conversely if $ R/I $ is an integral domain, consider $ ab\in I $. Then $ ab + I = 0 $. So either $ a+I $ or $ b+I $ is zero so $ a $ or $ b $ lies in $ I $. So $ I $ is a prime ideal.\qed

\begin{corollary}
  If $ R $ is a prime and $ I\nrm R $ is maximal, then $ I $ is prime.
\end{corollary}
\pf Since $ I\nrm R $ is maximal then $ R/I $ is a field. Hence $ R/I $ is an integral domain so $ I $ is prime by the proposition.\qed\par
Every nonzero ring $ R $ has a maximal ideal and therefore a prime ideal (proof is very set theoretic, equivalent to the axiom of choice through Zorn's lemma)

\subsection{Factorisation in integral domains}
From now on we let $ R $ be a general integral domain
\begin{definition}
	(Division) Let $ a,b\in R $ we say that a \textit{divides} b, written as $ a\mid b $ if there exists some $ c \in R $ such that $ b=ac $. Equivalently we have that $ (b)\subseteq (a) $.
\end{definition}
\begin{definition}
	(Associates) We say that $ a $ and $ b $ in $ R $ are \textit{associates} if $ a=bc $ for $ c\in R $ a unit. Equivalent to $ (a)=(b) $ and also equivalent to that $ a\mid b $ and $ b\mid a $.
\end{definition}
In $ \Z $ for example, we want to factorise up to units, i.e $ 6=2\times 3 = (-2)\times (-3) $. But as $ 2 $ and $ -2 $ are associates we declare some amount of uniqueness.
\begin{definition}
	(Irreducible) An element $ a\in R $ is called \textit{irreducible} if $ a\ne 0 $, $ a $ is not a unit, and if $ a=xy $ then either $ x $ or $ y $ is a unit.
\end{definition}
In the special case of $ \Z $ irreducible and prime are the same thing. But this is NOT always the case.
\begin{definition}
	(Prime element) We say that an element $ p\in R $ is \textit{prime} if $ p\ne 0 $, not a unit and if $ p\mid xy $, then either $ p\mid x $ or $ p\mid y $.
\end{definition}
\begin{proposition}
  Let $ r\in R $. Then $ r\ne 0 $ is prime if and only if $ (r) $ is a prime ideal.
\end{proposition}
\pf Suppose that $ (r) $ is a prime ideal. Then it is proper by definition, so $ r $ is not a unit. Suppose that $ r\mid xy $, so $ xy\in (r) $ so by primality either $ x $ or $ y $ lies in $ (r) $ so $ r\mid x $ or $ r\mid y $.
Conversely let $ r\in R $ be a prime. Suppose $ xy\in (r) $ then $ r\mid xy $ so $ r|x $ or $ r|y $ so $ x\in (r) $ or $ y\in (r) $ \qed\par
Again irreducible and prime are not the same thing. However...
\begin{proposition}
  Let $ r\in R $ be prime. Then $ r $ is irreducible.
\end{proposition}
\pf Let $ r\in R $ be a prime and suppose can write $ r $ as $ r=xy $. Since $ r=1_Rr $ we have that $ r\mid xy $ so either $ r\mid x $ or $ r\mid y $. Assume by symmetry that $ r\mid x $. This means that $ x=rz $ for $ \in R $. So $ r=xy=rzy $. So since we're in an integral domain and $ r\ne 0 $ we have that $ zy=1 $ hence $ y $ is a unit\qed\par
Now let's look at an example.\par
Let $ R=\Z[\sqrt{-5}]\le \C $, i.e. elements of the form $ a+b\sqrt{-5} $ for $ a,b\in \Z $. Observe that $ R $ is an integral domain since it is a subring of a field. Let's discuss the units. We define a "norm", $ N:R\to \Z_{\ge 0} $ sending $ a+b\sqrt{-5}\to a^2+5b^2 $. This is a function and importantly it is multiplicative, so $ N(ab)=N(a)N(b) $. Notice that all units have norm 1, since if $ 1=uv $, then $ N(1)=N(u)N(v)=1 $, so we must have that $ N(u)=N(v)=1 $. This implies the units are $ \pm 1 $.
\begin{claim}
  $ 2\in R $ is an irreducible element
\end{claim}
\pf If $ 2=ab $ then $ N(2)=4=N(a)N(b) $. But no element in $ R $ has norm of $ 2 $. Therefore either either $ a $ or $ b $ has norm 1, which means either $ a $ or $ b $ is a unit.\qed\par
A similar calculation shows that $ 3, 1\pm \sqrt{-5} $ are all also irreducible.
But are they prime?\par
Observe that $ 6=(1+\sqrt{-5})(1-\sqrt{-5})=2\times 3 $
\begin{claim}
	2 does not divide $ 1\pm \sqrt{-5} $
\end{claim}
\pf If it did then $ N(2)\mid N(1\pm\sqrt{-5}) $ but $ N(2)=4 $ and $ N(1\pm \sqrt{-5})=6 $ but $ 4\nmid 6 $ so 2 is no longer a prime in $ \Z[\sqrt{-5}] $.\par
In this same example, we see unique factorisation of 6 no longer holds.

\begin{definition}
	An integral domain $ R $ is called a \textit{Euclidean domain} if there exists a Euclidean function $ \varphi:R\setminus \{0\}\to\Z_{\ge 0} $ such that:
	\begin{enumerate}
	\item $ \varphi(ab)\ge \varphi(b) $ for all $ a,b\ne 0 $.
	\item If $ a,b\in R $ with $ b\ne 0 $, then there exists $ q,r\in R $ such that $ a=bq+r $ and either $ r=0 $ or $ \varphi(r)<\varphi(b) $.
	\end{enumerate}
\end{definition}
This definition is just saying we can run the Euclidean algorithm (or some equivalent form of it) on the ring.\par
We've already seen $ \Z $ is an integral domain where $ \varphi(x)=|x| $. Also seen, that if we take $ K $ a field, then $ K[X] $ is a Euclidean domain with a Euclidean function given by the degree of the polynomial.\par

Now take $ R=\Z[i]\le \C $ (Gaussian integers). This is a Euclidean domain with Euclidean function $ \varphi(z)=|z|^2 $
\begin{claim}
  $\varphi$ is a Euclidean function of $ R $
\end{claim} 
\pf The first requirement is obvious. For the second requirement, consider $ a,b\in\Z[i] $, with $ b\ne 0 $. Consider the ratio $ \frac ab\in \C $. There is a point $ q\in \Z[i] $ that has distance at most 1 from $ \frac ab $. So we have that $ \left|\frac ab - q\right|<1 $. Then write $ \frac ab = q+c $ where $ |c|<1 $. Then we have that $ a=bq+bc $, now set $ r=bc $. We know that $ r=a-bq\in R $. And finally $ \varphi(r)=\varphi(b)\varphi(c)<\varphi(b) $ \qed

\begin{definition}
	(Principal ideal domain) A ring $ R $ is a \textit{principle ideal domain} (PID) if it is an integral domain, and every ideal is a principal ideal, i.e for all $ I\nrm R $, there is some $ a $ such that $ I=(a) $.
\end{definition}
\begin{proposition}
  Every Euclidean domain is a principal ideal domain.
\end{proposition}
\pf Identical to the case of $ \Z $ just with a general Euclidean function $ \varphi $ instead of $ |x| $.
\par
$\Z, \R[X], \C[X], \Z[i] $ are all examples of PIDs.
\par
For a non-example we have $ R=\Z[X] $ and let $ I= (2, X) $. Suppose that $ I $ is principal, so $ I=(f) $, hence $ 2\in (f) $. So we have that $ 2=fg $ for some $ g\in R $. Hence we have that $ f $ is of degree zero, so $ f\in \{\pm 1, \pm 2\} $. But we can't have $ f=\pm 1 $ since $ \pm 1 $ are units in $ \Z[X] $ (since $ (2,X)\ne R $). But now since $ X\in (f) $ we must have that $ \pm 2\mid X $ which is false, hence $ (2,X) $ is not a principal ideal.\par
Simiarly $ \C[X,Y] $, $ K[X_1,\cdots, X_n $, $ n\ge 2 $ are not PIDs
\begin{definition}
	(Unique factorisation domain) An integral domain $ R $ is a \textit{unique factorisation domain} (UFD) if:
	\begin{enumerate}
		\item Every non-unit in $ R $ can be written as a product of irreducibles.
		\item If $ p_1\dots p_n=q_1\cdots q_m $, where $ p_i, q_i $ are irreducible, then $ n=m $ and up to reordering $ p_i $ are $ q_i $ are associates.
	\end{enumerate} 
\end{definition}
Now we aim to show that PID $ \implies $ UFD.
\begin{lemma}
  In a PID, any irreducible element is also prime.
\end{lemma}
\pf Let $ R $ be a PID and $ p\in R $ irreducible. Suppose $ p\mid ab $ and $ p\nmid a $ then we need to show that $ p\mid b $. Let's consider the ideal $ (p, a) $. This is principal so $ (p,a)=(d) $ for some $ d\in R $. So we must have that $ d\mid p $ and $ d\mid a $. So we have that $ p=q_1d $ so since $ p $ is irreducible we know that either $ d $ or $ q_1 $ is a unit. If $ q_1 $ is a unit, then $ d=\inv q_1 p $ and this divides $ a $ so we have that $ a=\inv q_1px $ which is a contradiction since $ p\nmid a $. So we have that $ d $ is a unit, so $ 1_R\in (p,a) $ so we can write $ 1_R=rp+sa $ for some $ r,s\in R $, so $ b=rpb+sab $. But now we can see that $ ab $ is divisible by $ p $ so $ b $ is divisible by $ p $.\qed
\begin{lemma}
	(PIDs are Noetherian) Let $ R $ be a PID. If $ I_1\subseteq I_2\subseteq\cdots $ ideals in $ R $ then for some $ N\in \Z_{> 0} $, we have for all $ n\ge N $, $ I_n=I_{n+1} $.
\end{lemma}
\pf Consider $ I=\bigcup_iI_i $. This is an ideal and $ I\nrm R $ so $ I=(a) $ for some $ a\in R $. But $ a\in I_N $ for some $ N $, so the result follows. \qed
\begin{theorem}
  Let $ R $ be a principal ideal domain. Then $ R $ is a unique factorisation domain.
\end{theorem}
\pf First we show that any $ r\in R $ is a product of irreducibles. If $ r $ is irreducible, we're done. If not, we can write $ r=r_1s_1 $ where either $ r_1 $ or $ s_1 $ are units. If both are products of irreducibles, then we're done. We can therefore assume by relabeling that $ r_1 $ is not a product of irreducibles. So we can write $ r_1=r_2s_2 $ with $ r_2,s_2 $ not units. Again without loss of generality we suppose that $ r_2 $ cannot be factored as a product of irreducibles. We continue in this way. So we can write that $ (r)\subseteq (r_1)\subseteq (r_2)\subseteq\cdots $. But by our lemma this chain stabilises so there is some $ n $ such that $ (r_n)=(r_{n+1})=\cdots $, so we have that $ s_{n+1} $ is a unit which is a contradiction so $ r $ must be a product of irreducibles.\par
For uniqueness let $ p_1p_2\dots p_n=q_1q_2\dots q_m $ with $ p_i,q_i $ irreducibles. So in particular we have that $ p_1\mid q_1\dots q_m $. Since $ p_1 $ is irreducible it is prime by our lemma so $ p_1 $ divides some $ q_i $. We reorder and suppose that $ p_1\mid q_1 $. So $ q_1=p_1a $ for some $ a $. But since $ q_1 $ is irreducible, $ a $ must be a unit so $ p_1 $ and $ q_1 $ are asssociates. Since $ R $ is a principal ideal domain, it is an integral domain so we can cancel $ p_1 $ to get that $ p_2p_3\dots p_n=(aq_2)q_3\dots q_m $. Now we can rename $ aq_2 $ as $ q_2 $ and continue as above show that $ p_i, q_i $ are associates for all $ i $. This also shows that $ n=m $ as if we had a leftover product, suppose $ p_{k_1}\dots p_n = 1$ which is a contradiction since they are irreducible so a product of them cannot be a unit as that would imply that each $ p_i $ was a unit for $ k+1\le i \le n $.\qed

\begin{claim}
  Let $ R $ be a unique factorisation domain. Then every irreducible in $ R $ is prime.
\end{claim}
If $ p\in R $ is irreducible and $ ab\in (p) $. Write $ ab=pc $. Now compare unique factorisation so $ (q_1\dots q_r)(s_1\dots s_k)=p(t_1\dots t_\ell) $ So $ p $ is an associate of soem $ q_i $ or $ s_j $. So done because either $ a $ or $ b $ is $ p $ times some element so lies in $ (p) $.
\begin{definition}
	(Greatest common divisor) Let $ R $ be an integral domain and $ a_1,\dots, a_n\in R $. We say that $ d\in  R $ is a \textit{greatest common divisor} (GCD) of $ a_1, \dots, a_n $ if:
        \begin{enumerate}
 	         \item $ d\mid a_i $ for all $ i $.
	         \item If $ d'\mid a_i $ for all $ i $ then $ d'\mid d $.
        \end{enumerate}
\end{definition}
\begin{definition}
	(Least common multiple) Let $ R $ be an integral domain and $ a_1,\dots, a_n\in R $. We say that $ m\in R $ is a \textit{least common multiple} (LCM) of $ a_1,\dots, a_n $ if:
	\begin{enumerate}
		\item $ a_i\mid m $ for all $ i $.
		\item If $ a_i\mid m' $ for all $ i $ then $ m\mid m' $.
	\end{enumerate}	
\end{definition}
\begin{theorem}
  Let $ R $ be a unique factorisation domain. Then gcd's and lcm's exist and are unique up to associates.
\end{theorem}
\pf Let $ a_1,\dots,a_n\in R $. Let $ p_1,\dots,p_nn $ be list of all irreducible factors of the $ a_i $ and no two are associates of each other. So we can write
\[
	a_i=u_i\prod_{j=1}^mp_j^{n_{ij}},
\]
where $ n_{ij}\in \N $ and $ u_i $ are units. Now we can let
\[
	m_j=\min_i\{n_{ij}\},
\]
and set
\[
	d=\prod_{i=1}^mp_u^{m_j}.
\]
So clearly, for all $ i $ we have that $ d\mid a_i $. Suppose we have some $ d'|a_i $, $ \forall i $, can set write $ d' $ as
\[
	d'=v\prod_{j=1}^np_j^{t_j},
\]
for some unit $ v $. So we must have that $ t_j\le n_{ij} $ for all $ i,j $. So we must have that $ t_j\le m_j $ for all $ j $. So $ d'\mid d $.\par
Now uniqueness up to associates follows from the fact that and two greatest common divisors must divide each other by definition, hence they must be associates. The argument for least common multiples is similar.\qed
\par We've created a lot of 'special' types of rings, so to make this section clearer use the following class inclusion.\par
$
	\text{rings} \supset \text{integral domains}\supset\text{unique factorisation domains} \supset \text{principal ideal domains}\supset \text{Euclidean domains} \supset \text{fields}
$\par
It should be very obvious why fields are Euclidean domains.
\subsection{Factorisation in polynomial rings}
Recall that, for $ F $ a field, $ F[X] $ is a Euclidean domain. Hence it is also a principal ideal domain and therefore a unique factorisation domain. This gives a few consequences.
\begin{enumerate}
	\item If $ I\nrm F[X] $, then $ I=(f) $ for some $ f\in F[X] $.
	\item If $ f\in F[X] $, then $ f $ is irreducible if and only if $ f $ is prime.
	\item If $ f $ is irreducible suppose $ (f)\subseteq J\subseteq F[X] $. Then $ J=(g) $ for some $ g\in F[X] $. Since $ (f)\subseteq (g) $ we must have that $ f=gh $. But since $ f $ is irreducible, either $ g $ or $ h $ is a unit. If $ h $ is a unit, then $ (f)=(g) $ and if $ g $ is a unit, then $ J=F[X] $ so $ (f) $ is a maximal ideal.
	\item $ J\nrm F[X] $ is a non-zero prime ideal if and only if it is a maximal ideal. We've seen that maximal ideals implies prime ideals, so conversely suppose $ (f) $ is a prime ideal, non-zero. Hence $ f $ is prime and therefore irreducible so be the previous point $ (f) $ is maximal.
	\item Finally we have that $ f\in F[X] $ is irreducible if and only if $ F[X]/(f) $ is a field.
\end{enumerate} 
We have that $ X^2+1 $ is irreducible in $ \R[X] $. Hence we have that $ \frac{\R[X]}{X^2+1}\cong\C $.
\begin{definition}
	(Content) Let $ R $ be a unique factorisation domain and let $ f=a_0+a_1X+\cdots+a_nX^n\in R[X] $. We define the \textit{content} as
	\[
	  c(f)=\gcd(a_0,\dots,a_n)\in R.
	\]
\end{definition}
Since the gcd is only defined up to a unit, so is the content.
\begin{definition}
	(Primitive polynomials) A polynomial is called \textit{primitive} if $ c(g) $ is a unit.
\end{definition}
Now we want to introduce Gauss' Lemma which provides an equivalency for reducibility using the field of fractions of a unique factorisation domain. But before that we require some preparation.
\begin{lemma}
	Let $ R $ be a unique factorisation domain. If $ f,g\in R[X] $ are primitive, then so is $ fg $.
\end{lemma}
\pf Let
\begin{gather*}
  f=a_0+a_1X+\cdots a_nX^n \\
  g=b_0+b_1X+\cdots b_mX^m
\end{gather*}
be primitive, with $ a_n,b_m\ne 0 $. Suppose for contradiction, $ c(fg) $ is not a unit. Since $ R $ is a UFD, we can find irreducible $ p\in R $ which divides $ c(fg) $. By assumption, $ c(g) $ and $ c(f) $ are units, so $ p\nmid c(g) $, $ p\nmid c(f) $. Let $ k $ be minimal such that $ p\nmid a_k $ and let $ \ell $ be minimal such that $ p\nmid b_\ell $. Consider the coefficient of $ X^{k+\ell} $ in fg, given by
\[
	\sum_{i+j=k+\ell}a_ib_j.
\]
Since $ p\mid c(fg) $ we have that
\[
	p\mid \sum_{i+j=k+\ell}a_ib_j.
\]
However $ p|a_{k+\ell}b_0+\dots a_{k+1}b_{\ell -1} $ and $ p |a_{k-1}b_{\ell +1} + \dots a_0b_{\ell +k} $, therefore $ p\mid a_kb_\ell $ so either $ p\mid a_k $ or $ p\mid b_\ell $ which in either case is a contradiction, so $ c(fg) $ is a unit.\qed
\begin{corollary}
	Let $ R $ be a unique factorisation domain. Then for $ f,g\in R[X] $, $ c(fg) $ is an associate of $ c(f)c(g) $.
\end{corollary}
\pf Write $ f=c(f)f_1 $ and $ g=c(g)g_1 $ so we have that $ f_1,g_1 $ are primitive. Then 
\[
  fg=c(f)c(g)f_1g_1
\]
so therefore $ c(fg)=c(f)c(g)c(f_1g_1) $ and $ c(f_1g_2) $ is a unit so they are associates.
\par Finally we can now prove Gauss' lemma.
\begin{lemma}
	(Gauss' lemma) Let $ R $ be a unique factorisation domain with $ F $ its field of fractions. Let $ f\in R[X] $ be primitive. Then $ f $ is reducible in $ R[X] $ if and only if $ f $ is reducible in $ F[X] $.
\end{lemma}
\pf First for the forwards direction, let $ f=gh $ be a product in $ R[X] $ with $ g,h $ not units. Since $ f $ is primitive so are $ g $ and $ h $. So both have non-zero degree hence they are not units. So $ f $ is reducible in $ F[X] $.\par
For the other direct let $ f=gh $ in $ F[X] $ with $ g $ and $ h $ not units. So we can clear denominators so we pick $ a,b\in R $ such that $ ag,bh\in R[X] $. then we have that $ abf=(ag)(bh) $.\par
Let
\begin{gather*}
  ag=c(ag)g_1,\\
  bh=c(bh)h_1,
\end{gather*}
where $ g_1 $ and $ h_1 $ are primitive. So $ ab=uc(abf)=uc((ag)(bh))=u'c(ag)c(bh) $. But $ abf=c(ag)(bh)g_1h_1=\inv u abg_1h_1 $, since we're in a integral domain we can cancel $ ab $ to get $ f=\inv u g_1h_1\in R[X] $. So $ f $ is reducible in $ R[X] $.\qed\par

Now for an example. Consider $ f=X^3+X+1\in \Z[X] $. We can see that $ c(f) =1 $ so $ f $ is primitive. Suppose for contradiction that $ f $ is reducible in $ \Q[X] $. So by Gauss' lemma, $ f $ is reducible in $ \Z[X] $, so $ X^3+X+1=gh $ where $ g,h\in \Z[X] $ not units. Hence $ \deg(g),\deg(h)\ge 1 $. Since $ \deg(f)=3=\deg(g)+\deg(h) $, suppose that $ \deg(g)=1 $ and $ \deg(h)=2 $. Hence let $ g=b_0+b_1X $ and let $ h=c_0+c_1X+c_2X^2 $. Multiplying out and equating coefficents we get that $ b_0c_0=1 $ and $ c_2b_1=1 $. Hence $ b_0,b_1 $ must be $ \pm 1 $, so we must have that $ g $ is either $ 1\pm X $ or $ -1\pm X $. Hence $ \pm 1 $ is a root of $ g $ which is a contradiction since $ f $ does not have a root of $ \pm 1 $. So $ f $ is not reducible in $ \Q[X] $, hence it has no root in $ \Q $ and we have that $ \Q[X]/(X^3+X+1) $ is a field.
\begin{proposition}
	Let $ R $ be a unique factorisation domain and $ F $ its field of fractions. Let $ g\in R[X] $ be primitive. Then a polynomial $ f\in R[X] $ is divisble by $ g $ in $ R[X] $ if and only if it is divisble by $ g $ in $ F[X] $. Or in other words if $ J=(g)\nrm R[X] $ and $ I=(g)\nrm F[X] $ then $ J=I\cap R[X] $.
\end{proposition}
\pf We'll prove the second formualation. Certainly we have that $ J\subseteq I\cap R[X] $. So let $ f\in I\cap R[X] $. So we can write
\[
	f=gh \quad\text{with }h\in F[X] 
\]
Now we clear denominators by choosing $ b \in R $ such that $ bh\in R[X] $. We know by multiplying by $ b $ so that $ bf=g(bh) $. We let $ (bh)=c(bh)h_1 $ where $ h_1 $ is primitive and $ h_1\in R[X] $. So $ bf = c(bh)gh_1 $, but $ g $ and $ h_1 $ are both primitive, so $ gh_1 $ is also primitive. So $ c(bh) = c(bf)u $ where $ u $ is a unit. Since $ bf $ is a product in $ R[X] $
\[
  c(bf)=c(b)c(f)=b\cdot c(f)
\]
This gives that $ bf = ub\cdot c(f)gh_1 $ so cancelling $ b $ gives that $ f $ is divisble by $ g $.\qed
\begin{theorem}
	If $ R $ is a unique factorisation domain, then $ R[X] $ is also a unique factorisation domain.
\end{theorem}
\pf First we prove that factorisations exist. Let $ f\in R[X] $. We can write $ f = c(f)f_1 $ where $ f_1 $ is primitive. Since $ F $ is a UFD, factorise $ c(f) $ as $ p_1\dots p_n $ for $ p_i\in R $ irreducible. Now we deal with $ f_1 $. If $ f_1 $ is not irreducible then write $ f_1=f_2f_3 $ where $ f_2,f_3 $ are not units. Since $ f_1 $ is primitive neither $ f_2 $ nor $ f_3 $ can be constant. So $ \deg(f_2),\deg(f_3)>0 $ and also $ \deg(f_1)=\deg(f_2)+\deg(f_3) $ using the fact that we're working in an integral domain \textit{(think!)}. Induct on the degree, if $ f_2 $ and $ f_3 $ are irreducible, we're done otherwise repeat the same steps until the process terminates, which will happen in finitely many steps since the degree is strictly decreasing. Putting this together we can write $ f=p_1\dots p_nq_1\cdots q_m $ all irreducibles.\par
Now for uniqueness. First we deal with the $ p $'s. The content has a unique factorisation $ c(f)=p_1\dots p_n $. So cancelling the content suffices to show uniqueness of factorisation for $ f_1=q_1\dots q_m $. Suppose $ f_1=q_1\dots q_m=r_1\dots r_\ell $ are two factorisations in $ R[X] $. Viewing this in $ F[X] $, where $ F $ is the fraction field of $ R $, since $ F[X] $ is a Euclidean domain, we know that $ \ell=m $ and up to reordering $ q_i $ and $ r_i $ are associates in $ F[X] $. So $ q_i\mid r_i $ and $ r_i\mid q_i $ in $ F[X] $. But from the previous proposition we get the same statement in $ R[X] $\qed\par
This gives us that rings such as $ \Z[X] $ and $ \C[X,Y]  $ are not principal ideal domains, but they are unique factorisation domains.
\begin{theorem}
	(Eisenstein's criterion) Let $ R $ be a unique factorisation domain and 
	\[
		f=a_0+a_1X+a_2X^2+\cdots+a_nX^n\in R[X]
	\]
	is primitive, with $ a_n\ne 0 $. Let $ p\in R $ be irreducible such that
	\begin{enumerate}
		\item $ p\nmid a_n $.
		\item $ p\mid a_i $ for $ 0\le i\le n-1 $.
		\item $ p^2 \nmid a_0 $.
	\end{enumerate}
	Then $ f $ is irreducible in $ R[X] $.
\end{theorem}
\pf Suppose we have $ f=gh $, with $ g=r_0+r_1X+\cdots r_kX^k $ and $ h = s_0+s_1X + \cdots s_\ell X^\ell $ and $ r_k, s_\ell \ne 0 $. We know that and since $ p\nmid a_n $ it does not divide $ r_k $ nor $ s_\ell $. Similarly $ r_0s_0=a_0 $ and $ p^2\nmid a_0 $ so $ p $ divides exactly one of $ r_0 $ and $ s_0 $. Let assume $ p\mid r_0 $ so $ p\nmid s_0 $. Let $ j $ be the index such that
\[
	p\mid r_0, \quad p\mid r_1,\cdots, \quad p\mid r_{j-1},\quad p\nmid r_j.
\]
Consider $ a_j $. We know $ a_j=r_0s_j+r_1s_{j-1}+\cdots +r_{j-1}s_1+r_js_0 $. We know that $ p \mid r_0s_j+\cdots +r_{j-1}s_1 $, also $ p\nmid r_j $ and $ p\nmid s_0 $, because $ p $ is prime, $ p\nmid a_j $. So we must have that $ j=n $. We also have that $ j\le k \le n $, so $ j=k=n $. Hence $ \deg g = n $ and $ \deg h =0 $. Since $ f $ is primitive we must have that $ h $ is a unit, so this is not a proper factorisation. Hence $ f $ is irreducible in $ R[X] $.\qed

For an example consider the polynomial $ X^n-p\in \Z[X] $ with $ p $ prime. Apply Eisenstein's criterion with $ p\in \Z $ and observe all the conditions hold. This is certainly primitive, since this is monic. So $ X^n-p $ is irreducible in $ \Z[X] $, hence it is also irreducible in the fraction field polynomial, namely $ \Q[X] $. In particular $ X^n-p $ has no rational roots, so $ \sqrt[n]{p} $ is irrational.\par
Next, consider $ f=X^{p-1}+X^{p-2}+\cdots + X + 1\in \Z[X] $ for $ p $ prime. See that we can write $ f $ as
\[
	f=\frac{X^p-1}{X-1}.
\]
So perchance we should write $ Y=X-1 $. Then we get a new polynomial
\[
	\hat f(Y)=\frac{(Y+1)^p-1}Y = Y^{p-1}+\binom p1Y^{p-2}+\cdots+\binom p{p-1}.
\]
So now we can apply Eisenstein's criterion so $ \hat f $. So $ \hat f $ is irreducible in $ \Z[X] $. If we had a factorisation $ f(X)=g(X)h(X)\implies \hat f(Y)=g(Y+1)h(Y+1) $, but we know that $ \hat f $ cannot be factorised, so $ f $ is irreducible.
\subsection{Gaussian integers}
This is the ring $ \Z[i]=\{a+ib:a,b\in \Z\}\le \C $ and we have the usual norm $ N(a+ib)=a^2+b^2 $ which helpfully is a Euclidean function. This implies that $ \Z[i] $ is a Euclidean domain, hence it is a principal ideal domain and therefore a unique factorisation domain. So irreducible and prime are the same in the Gaussian integers. The units in $ \Z[i] $ are $ \pm 1 $ and $ \pm i $ which are exactly the norm $ 1 $ elements.\par
We make the following observations.
\begin{enumerate}
	\item We have $ 2 = (1+i)(1-i)$ so $ 2 $ is not a prime.
	\item Similiarly $ 5=(1+2i)(1-2i) $ so also not a prime.
	\item We claim that $ 3 $ is a prime in $ \Z[i] $. The norm of $ 3 $ is $9 $ so if $ 3=uv $ then either $ N(u) $ or $ N(v) $ is $ 1 $ (so one is a unit) or $ N(u)=N(v)=3 $. But if $ u=a+bi $ then we'd need $ a^2+b^2=3 $ which has no solutions for $ a,b\in \Z $. So one of $ u $ or $ v $ is a unit, hence $ 3 $ is prime. 
\end{enumerate} 
So what are the primes in $ \Z[i] $?
\begin{proposition}
	A prime number $ p\in \Z $ remains prime when viewed in $ \Z[i] $ if and only if $ p\ne a^2+b^2 $ for $ a,b\in \Z\setminus \{0\} $.
\end{proposition}
\pf If $ p=a^2+b^2 $ then in $ \Z[i] $ we can write $ p=(a+bi)(a-bi) $ so $ p $ is not irreducible and hence not prime.\par
Conversely suppose $ p = uv $ is a product of non-units in $ \Z[i] $ then $ p^2=N(u)N(v) $. Since $ u $ and $ v $ are non-units then $ N(u)=N(v)=p $. Write $ u=a+bi $. Then we have that $ a^2+b^2=p $.\qed
\par
Now we want to classify all of the primes in $ \Z[i] $. But before that we need a lemma.
\begin{lemma}
	Let $ p $ be a prime number. Then the group $ \mathbb F_p^\times $ of non-zero elements $ \mod p $ under multiplication is a cyclic group. So $ \mathbb F_p^\times \cong C_{p-1}. $
\end{lemma}
\pf
Certainly $ \mathbb F_p^\times $ is finite abelian. By the classification of finite abelian groups, if $ \mathbb F_p^\times $ is not cyclic then it contains $ C_m\times C_m  $ as a subgroup for some $ m>1 $. This means there are at least $ m^2 $ elements $ x\in \mathbb F_p^\times $ satisfying $ x^m-1=0 $. But a polynomial with coefficients in $ \mathbb F_p $ has number of solutions bounded above by the degree, which is a contradiction. \qed\par
Now let's prove the theorem.
\begin{theorem}
	(Classification of primes in $ \Z[i] $) The primes in $ \Z[i] $ up to associates are exactly
	\begin{enumerate}
		\item The primes $ p\in\Z $ such that $ p\equiv 3 \mod 4 $.
		\item Gaussian integers $ z\in \Z[i] $ with $ N(z)=p $ where $ p $ is a prime that is either $ 2 $ or $ 1 \mod 4$.
	\end{enumerate}
\end{theorem}
\pf Firstly we prove the two types of Gaussian integers are prime. If $ p\in \Z $ is $ 3\mod 4 $ then $ p\ne a^2+b^2 $ for $ a,b\in \Z\setminus \{0\} $ (by arithmetic) so by our proposition these are primes in $ \Z[i] $. Now for $ z\in\Z[i] $ if $ N(z)=p $, with $ p $ prime congruent to $ 2 $ or $ 1 \mod 4 $, we'll show that $ z $ is irreducible. Write $ z=uv $ in $ \Z[i] $, then $ p=N(z)=N(u)N(v) $ in $ \Z_{\ge 0 } $. So either $ N(u) $ or $ N(v) $ is $ 1 $ so $ u $ or $ v $ is a unit, so $ z $ is irreducible hence prime.\par
Now we show all primes in $ \Z[i] $ are of this form. Let $ z\in \Z[i] $ be prime. Observe that $ N(z)=z\overline z $ is a factorisation of $ N(z) $ viewed inside $ \Z[i] $ into irreducibles. Let $ p $ be a prime number that divides $ N(z) $ in $ \Z_{\ge 0} $. Suppose $ p\equiv 3\mod 4 $ so $ p $ is prime in $ \Z[i] $. So $ p=z\overline z $ hence $ p\mid z $ or $ p\mid \overline z $ therefore because $ z $ is conjugate to $ \overline z $ so $ p\mid z$ and $ p\mid \overline z $. But then $ p=z $ up to associates. Now assume that $ p\equiv 2 $ or $ 1\mod 4 $. The $ p\equiv 2 \mod 4 $ case can be worked out by hand, so assume $ p=4k+1 $ for $ k \in \Z $. By the lemma we know that $ \mathbb F_p^\times \cong C_{4k} $. There is a unique order $ 2 $ element here (which is $ -1 $). If $ a\in \mathbb F_p^\times $ is order $ 4 $, then we have $ a^2\equiv -1\mod p $. So $ p $ divides $ (a+i)(a-i) $ but not $ a+i $ or $ a-i $ individually. So $ p $ is not irreducible in $ \Z[i] $. So now we can write $ p=z_1z_2 $ non-units. Taking the norm we get that $ N(p)=p^2=N(z_1)N(z_2) $. Because these are non-units, we must have that $ N(z_1)=N(z_2)=p $. So $ \overline z_1=z_2 $. Finally we have that $ p= , z_1\overline z_1\mid N(z)=z\overline z $ all of $ z,z_i $ are irreducible so $ z,z_1 $ or $ z,\overline z_1 $ are associates. Hence $ N(z)=p. $\qed
\begin{corollary}
  An non-negative integer $ n $ can be written as a sum of two squares if and only if when we write $ n $ as a product of distinct primes,
  \[
	  n=p_1^{n_1}p_2^{n_2}\cdots p_k^{n_k}
  \]
  $ p_i\equiv 3\mod 4 $ implies that $ n_i $ is even.
\end{corollary}
\pf If $ n=x^2+y^2 $ we have that $ n=N(z) $ where $ z=x+iy\in \Z[i] $. Factorise $ z $ into primes, $ z=\alpha_1\cdots\alpha_q $. Either $ \alpha_i $ is a prime number in $ \Z $ that is $ 3\mod 4 $ or it has norm $ p $ where $ p $ is $ 2 $ or $ 1\mod 4 $. Taking norms we get that $ n=x^2+y^2=N(z)=N(\prod \alpha_i)=\prod_i N(\alpha_i) $. Each factor is this product is either $ p^2 $ or $ p\equiv 3\mod 4 $ or $ p $ with $ p=2 $ or $ p\equiv 1\mod 4 $. For the converse let $ n=p_1^{n_1}\cdots p_k^{n_k} $ satisfying hypothesis given. Now we know that $ p_i=N(\alpha_i) $ if $ p_i\equiv 3\mod 4 $, $ p_i^{n_i}=N(p_i)^{\frac{n_i}2} $ otherwise $ p_i=2 $ or $ p_i=1\mod 4 $ so $ p_i=N(\alpha_i) $ for some $ \alpha_i $ so $ p_i^{n_i}=N(\alpha_i^{n_i}) $ so write $ n $ as $ N(z) $ in this way.\qed

\subsection{Algebraic integers}
\begin{definition}
	(Algebraic integer) A complex number $ \alpha\in \C $ is an \textit{algebraic integer} if it is a root of a monic polynomial in $ \Z[X] $.
\end{definition}
For a piece of notation we write $ \Z[\alpha]\le \C $ for an algebraic integer $ \alpha $, as the smallest subring containing $ \alpha $ so
\[
	\Z[\alpha]=\bigcap_{\substack{R\le \C \\ \alpha\in R}}R.
\]
Or equivalently $ \Z[\alpha] $ is the image of $ e_\alpha:\Z[X]\to \C $ given by $ g\to g(\alpha) $
\begin{proposition}
	Let $ \alpha\in \C $ be an algebraic integer. The ideal $ I=\ker (e_\alpha)\nrm \Z[X] $ is principal, and $ I=(f_\alpha) $ with $ f_\alpha $ monic and irreducible.
\end{proposition}
Recall that $ \Z[X] $ is not a principal ideal domain, so this proposition is not trival.\par From our proposition we can use the following definition.
\begin{definition}
	(Minimal polynomial) The \textit{minimal polynomial} of an algebraic integer $ \alpha $ is the irreducible monic $ f_\alpha $ such that $ (f_\alpha)=\ker (e_\alpha) $.
\end{definition}
Now for the proof of the proposition\par
\pf By definition $ I=\ker e_\alpha $ is non-zero. Take $ f_\alpha\in I $ of minimal degree. We can assume $ f_\alpha $ is primiative (since monic). Now we aim to show that $ I=(f_\alpha) $. Let $ h\in I $. In $ \Q[X] $ we have the Euclidean algorithm so we can write $ h=f_\alpha q+r $ with $ r=0 $ or $ \deg r<\deg f_\alpha $. Now clearing denominators so for some $ a\in \Z $ we get that $ ah=f_\alpha(aq)+(ar) $. Now plugging in $ \alpha $ we get that
\[
  ah(\alpha)=f_\alpha(\alpha)aq(\alpha)+ar(\alpha)
\]
Since that $ f_\alpha,h\in I $ we get that $ f_\alpha(\alpha)=h(\alpha)=0 $, so we have that $ ar(\alpha)=0 $. So $ (ar)\in I $. But since we have that $ f_\alpha\in I $ has minimal degree, and $ \deg(r)=\deg(ar) $ we must have that $ r=0 $. So we know that $ ah=f_\alpha\cdot (aq) $. This almost what we want, we just need to factorise $ h $ instead of $ ah $. Taking the contents we know that $ c(f_\alpha) $ is a unit so $ c(ah)=c(aq) $, so $ a\mid c(aq) $. Therefore $ aq=a\overline q $ for $ \overline q\in \Z[X] $. So this means that $ q=\overline q\in \Z[X] $. So we know that $ h=f_\alpha q\in (f_\alpha) $. Hence $ I=(f_\alpha) $.\par
Now we're left to show that $ f_\alpha $ is irreducible. We know that
\[
	\frac{\Z[X]}{(f_\alpha)}\cong\ima(e_\alpha)\le \C.
\]
But since $ \C $ is an integral domain all of its subrings are integral domains, so $ \Z[X]/(f_\alpha) $ is an integral domain, so $ (f_\alpha) $ is prime hence $ f_\alpha $ is prime, so it's irreducible.\qed\par
We have plenty of examples of algebraic integers.
\begin{enumerate}
	\item $ \alpha=i $ is an algebraic integer with $ f_\alpha=X^2+1 $.
	\item $ \alpha=\frac 12(1+\sqrt{-3}) $ is an algebraic integer with $ f_\alpha=X^2-X-1 $.
	\item The monic polynomial $ X^5-X+d $ with $ d\in \Z_{\ge 1} $ has exactly one real root which is an algebraic integer.
\end{enumerate}
\begin{theorem}
	The real root of $ X^5-X+d $ cannot be written using integers and the operations $ \times, +,-,\sqrt[n]{},\div $.
\end{theorem}
\pf II Galois Theory
\begin{proposition}
  Let $ \alpha\in \Q $ be an algebraic integer. Then $ \alpha\in \Z $.
\end{proposition}
\pf Let $ f_\alpha $ be the minimal polynomial of $ \alpha $ irreducible. So by Gauss' lemma we have that the polynomial is also irreducible in $ \Q[X] $. But $ (X-\alpha) $ divides it in $ \Q[X] $ so $ f_\alpha=(X-\alpha) $ hence $ \alpha\in\Z $.\par
It should be noted that the set of algebraic integers form a ring, but this is not obvious and will be proved later.
\subsection{Hilbert's basis theorem}
\begin{definition}
	(Noetherian ring). A ring is \textit{Noetherian} if for any chain of ideals
	\[
	  I_1\subseteq I_2\subseteq I_3\subseteq \cdots
	\]
	then there is some $ N $ such that $ I_N=I_{N+1}=\cdots $. This is known as the \textit{ascending chain condition} (ACC).
\end{definition}
\begin{definition}
	(Finitely generated ideal) An ideal $ I\nrm R $ is finitely generated if it can be written as $ I=(r_1,\dots, r_n) $ for some $ r_1,\dots, r_n\in R $.
\end{definition}
\begin{proposition}
  A ring $ R $ is Noetherian if and only if every ideal is finitely generated.
\end{proposition}
\pf Suppose all ideals are finitely generated. Given the chain $ I_1\subseteq I_2\subseteq \cdots $ Consider
\[
	I=\bigcap_{j=1}^\infty I_j.
\]
$ I $ is an ideal (Example Sheet 2) so we know that $ I $ is finitely generated, so $ I=(r_1,\dots,r_n) $ with $ r_i\in I_{k_i} $. So let
\[
	K=\max_{i=1,\dots n}\{k_i\}.
\]
Then we have that $ r_1,\dots,r_n\in I_K $, so $ I_K=I=I_{K+1}=\cdots $ so $ R $ is Noetherian.\par
For the converse if $ I $ is not finitely generated, pick $ r_1\in I $ such that $ I\ne (r_1) $. Pick $ r_2\in I\setminus (r_1) $ such that $ (r_1,r_2)\ne I $, and so on to get $ (r_1)\subseteq (r_1,r_2)\subseteq \cdots $ strictly increasing so $ R $ is not Noetherian. \qed
\begin{theorem}
	(Hilbert's basis theorem) Let $ R $ be a Noetherian ring. Then $ R[X] $ is also Noetherian.
\end{theorem}
\pf Let $ J\nrm R[X] $ be an ideal. Let $ f_1\in J $ have minimal degree. If $ J \ne (f_1) $ pick $ f_2\in J\setminus (f_1) $ of minimal degree, and so on. If $ J=(f_1,\dots, f_k) $ then we're done so suppose for contradiction we don't get this. Let $ a_i $ be the leading coefficient of $ f_i $. The chain $ (a_1)\subseteq (a_1,a_2)\subseteq \cdots $ stabilises by ACC for $ R $. So for some $ m $ we have that $ (a_1,a_2,\dots, a_m)=(a_1,a_2,\dots) $. Now we let
\[
	a_{m+1}=\sum_{i=1}^ma_ib_i
\]
and consider the polynomial
\[
	g=\sum_{i=1}^mb_if_iX^{\deg f_{m+1}-\deg f_i}.
\]
This has the same degree and leading coefficient as $ f_{m+1} $. Then $ f_{m+!}-g $ has smaller degree and lies in $ J $, but not in $ (f_1,\dots f_m) $. This contradicts minimality of $ \deg f_{m+1} $. So the process must terminate, hence $ R[X] $ is Noetherian.\qed
\par
We also have that $ R[[X]] $ is Noetherian
\begin{corollary}
	$ \Z[X_1,X_2,\dots, X_n] $ is Noetherian, and for $ F $ a field $ F[X_1,X_2,\dots, X_n]  $ is Noetherian.
\end{corollary}
\pf Let $ J\nrm \frac RI  $ and let $ J'\nrm R $ be the corresponding ideal in $ R $. If $ J' $ is generated by $ r_1,\dots, r_m $ then $ J $ is generated by $ (r_1+I),\dots, (r_m+I) $.\qed
\par
In fact all finitely-generated rings is a quotient of some $ \Z[X_1,\dots, X_n] $ giving the final corollary of this section.
\begin{corollary}
  All finitely-generated rings are Noetherian.
\end{corollary}
\pf Obvious \qed
\section{Modules}
\subsection{Definitions and examples}
An example of a module is a vector space over a field $ K $. (Recall a vector space is an abelian group $ V $ with a 'scaling', $ K\times V\to V $.)\par
Another example is that if $ A $ is any abelian group, we can think of $\Z\times A\to A $ sending $ (n,a)\to \underbrace{a+a+\cdots +a}_{n \text{ times} } $
\begin{definition}
	(R-module) Let $ R $ be a communiative ring. Then a quadruple $ (M,+,0_M,\circ) $ is an \textit{R-module} if
	\begin{enumerate}
		\item $ (M,+,0_M) $ is an abelian group.
		\item $ \circ:R\times M\to M $ satisfies
			\begin{enumerate}
				\item $ (r_1+r_2)\circ m = r_1m+r_2m $;
				\item $ r\circ (m_1+m_2) = (r\circ m_1)+(r\circ m_2)$;
				\item $ r_1\circ (r_2\circ m)=(r_1\circ r_2)\circ m $; and
				\item $ 1_R\circ M = m $.
			\end{enumerate}
	\end{enumerate}
	for all $ m_1,m_2,m\in M $ and $ r_1,r_2,r\in R $.
\end{definition}
If $ R=K $ is a field, then an $ R $-module is a $ K $-vector space. Any abelian group is naturally a $ \Z $-module.\par
For any $ R $ the self product $ R^m=\underbrace{R\times\cdots\times R}_{m \text{ times}} $ is an $ R $-module via $ R\times R^m\to R^m $ given by $ (r,r_1,\dots,r_m)\to (rr_1,\dots, rr_m) $.\par
If $ I\nrm R $ then $ I $ is an $ R $-module via multiplication.\par
Similarly the group $ R/I $ is also an $ R $-module via $ R\times R/I\to R/I $ given by $ (r,s+I)\to (rs+I) $.\par
Let $ V $ be a $ K $-vector space and let $ \alpha\in \mathrm{End} $(V)	so $ \alpha $ is a linear map. We can give $ V $ the structure of a module over $ K[X] $ by
\begin{align*}
	K[X]\times V\to V \\
	(f,v)\to f(\alpha)(v)
\end{align*}
giving a $ K[X] $-module.\par
Another important example. Let $ \varphi:R\to S $ be a ring homomorphism and let $ M $ be an $ S $-module. We can turn $ M $ into an $ R $-module via:
\begin{align*}
  R\times M\to M\\
  (r,m)\to \varphi(r)m.
\end{align*}
In particular this turns $ S $ into an $ R $-module.
\begin{definition}
	(Submodule) Let $ M $ be an $ R $-module. A subgroup $ N\subseteq M $ is an $ R $\textit{-submodule} if for every $ n\in N $ and $ r\in R $, we have $ rn\in N $. We write $ N\le M $.
\end{definition}
For a non-example $ \R^n $ is an $ \R $-vector space hence it is an $ \R $-module. But $ \Q^n $ is not a $ \R $-submodule because it's not closed under the scaling operation.\par
If $ R=K $ then a $ K $-submodule of $ V $ is just a $ K $-vector subspace.

\begin{definition}
	(Quotient module) Let $ N\le M $ be an $ R $-module. The \textit{quotient module} is the group $ M/N $ equipped with
	\begin{align*}
	  R\times M/N\to M/N \\
	  (r,m+N)\to (rm+N).
	\end{align*}
\end{definition}
It can be see that this is well defined and itself an $ R $-module.

\begin{definition}
  Let $ M,N $ be. $ R $-modules. An $ R $-module homomorphism is group homomorphism $ \varphi: M\to N $ such that $ \varphi(rm)=r\varphi(m) $ for all $ r\in R $ and $ m \in M$
\end{definition}
\begin{definition}
	(Isomorphism) An \textit{isomorphism} of $ R $-modules is a bijective homomorphism. We say that the modules are \textit{isomorphic}.
\end{definition}
Again we have the isomorphism theorems and a correspondence theorem for modules.
\begin{theorem}
	(First isomorphism theorem) Let $ \varphi: M\to N$ be a homomorphism of $ R $-modules. Then
	\[
		\ker\varphi=\{m\in M:\varphi(m)=0\}
	\]
	is a submodule of $ M $. The image
	\[
		\ima\varphi= \{\varphi(m):m\in M\}
	\]
	is a submodule of $ N $. Furthermore there is an isomorphism
	\[
	  M/\ker\varphi\cong\ima\varphi
	\]
\end{theorem}
\pf Immediate from the first isomorphism theorem of groups.\qed

\begin{theorem}
	(Second isomorphism theorem) Let $ L,K\le M $ be $ R $-submodules. Then
	\[
		K+L=\{k+\ell :k\in K, \ell\in L\} 
	\]
	is an $ R $-submodule of $ M $. Moreover
	\[
		\frac{K+L}K\cong \frac L{L\cap K}
	\]
	is an isomorphism of $ R $-modules.
\end{theorem}
\pf Immediate from the second isomorphism theorem of groups.\qed
\begin{theorem}
	(Third isomorphism theorem) Let $ N\le L\le M $. Then
	\[
		M/L\cong\frac{M/N}{L/N}
	\]
	is an isomorphism of $ R $-modules.
\end{theorem}
\pf Immediate from the third isomorphism theorem of groups.\qed
\begin{theorem}
	(Correspondence theorem) We have the correspondence
	\[
		\{\text{Submodules of } M/N\}\leftrightarrow{} \{\text{Submodules of } M \text{ which contain } N \}.
	\]
\end{theorem}
\pf Same as the correspondence theorem for groups.\qed
\begin{definition}
	(Annihilator) Let $ M $ be an $ R $-module. For $ m\in M $ its \textit{annihilator} is 
	\[
		\Ann(m)=\{r\in R:rm=0\}.
	\]
	For any set $ S\subseteq M $ we define
	\[
		\Ann(S)=\{r\in R:rm=0\text{ for all } m \in S\}=\bigcap_{s\in S}\Ann(s).
	\]
\end{definition}
We observe that the annihilator of $ S\subseteq M $ is an ideal in $ R $.
\begin{definition}
(Submodule generated by an element) Let $ M $ be a $ R $-module and $ m\in M $. The submodule generated by $ m $ is
\[
	Rm=\{rm\in M:r\in R\}.
\]
\end{definition}
\begin{proposition}
  For $ m\in M $ there is an isomorphism $ R/\Ann(m)\cong Rm $.
\end{proposition}
\pf Consider the function $ R\to M $ sending $ r\to rm $. This is clearly a $ R $-module homomorphism, so the result follows from the first isomorphism theorem.\qed
\begin{definition}
	(Finite generation) An $ R $-module $ M $ is finitely generated if there exists elements $ m_1,\dots,m_k $ such that
	\[
		M=Rm_1+Rm_2+\cdots Rm_k=\{r_1m_1+\cdots r_km_k:r_i\in R\}.
	\]
\end{definition}
\begin{lemma}
  An $ R $-module $ M $ is finitely generated if and only if there is a surjective $ R $-module homomorphism from $ R^k\to M $.
\end{lemma}
\pf If $ M =Rm_1+\cdots+Rm_k $ then consider
\begin{align*}
  R^k\to M \\
  (r_1,\dots,r_k)\to \sum r_im_i,
\end{align*}
clearly surjective homomorphism.
Conversely if $ \varphi:R^K\to M $ is given, let $ e_i=(0,\dots, 0,1,0,\dots, 0) $ with a $ 1 $ in the $ i $th position and $ m_i =\varphi(e_i) $ and now we can check that $ m_i $ generates $ M $ by surjectivity. \qed
\begin{corollary}
  If $ M $ is finitely generated $ R $-module and $ N\le M $ then $ M/N $ is finitely generated.
\end{corollary}
\pf Using the previous lemma we can compose surjection $ R^k\to M $ and surjection $ M\to M/N $ hence we have a surjection $ R^k\to M/N $.\qed\par
However a submodule of a finitely-generated module need not be finitely-generated. For example let $ R=\C[X_1,X_2,\dots] $. And consider the module $ M=R $ so the submodule $ I $ generated by $ (X_1,X_2,\dots) $ is not finitely-generated but $ M $ is finitely-generated (by $ 1 $).
\subsection{Direct sums and free modules}
Now we'll slug through some definitions.
\begin{definition}
	(Direct sum) Let $ M_1,\dots, M_k $ be $ R $-modules. The \textit{direct sum} is the abelian group $ M_1\times\cdots\times M_k $ with scaling
	\begin{align*}
	  R\times M_1\times\cdots\times M_k\to M_1\times \cdots \times M_k \\
	  (r,m_1,\dots, m_k)\to (rm_1\dots, rm_k)
	\end{align*}
	We notate this as $ M_1\oplus \cdots \oplus M_k $.
\end{definition}
\begin{definition}
	(Linear independence) Let $ m_1,\dots, m_k\in M $. Then $ \{m_1,\dots, m_k\} $ is $ R $\textit{-linearly independent} if
	\[
		\sum r_im_i=0\implies r_i=0\quad\text{for all } i 
	\]
\end{definition}

\begin{definition}
	(Free generation) A subset $ S\subseteq M $ \textit{freely generates} $ M $ if 
	\begin{enumerate}
		\item $ S $ generates $ M $
		\item Any set function $ f:S\to N $ with $ N $ an $ R $-module extends to an $ R $-module homomorphism $ \varphi_f:M\to N $ with the condition $ \varphi_f(s)=f(s)\quad\forall s\in S $.
	\end{enumerate}
\end{definition}
\begin{definition}
	(Free module and basis) A module $ M $ is \textit{free} if it is freely generated by some subset $ S\subseteq M $, and $ S $ is called a \textit{basis}.
\end{definition}
Now let's look at an example. Consider the $ \Z $-module $ \Z/2 $. Suppose $ \Z/2 $ was generated by some $ S\subseteq \Z/2 $. This can only happen with $ S=\{1\} $. Then this implies that there is a homomorphism $ \varphi:\Z/2\to \Z $ sending $ 1 $ to $ 1 $. But it it does not send $ 0 $ to $ 1+1 $ since $ \varphi(0)=0 $. So $ \Z/2 $ is not free.
\begin{proposition}
	Let $ S=\{m_1,\dots,m_k\}\subseteq M $. Then the following three statement are equivalent
	\begin{enumerate}
		\item $ S $ generates $ M $ freely.
		\item $ S $ generates $ M $ and $ S $ is linearly independent.
		\item Every $ m\in M $ is uniquely expressible as $ m=r_1m_1+\cdots r_km_k $.
	\end{enumerate}
\end{proposition}
\pf The fact that (ii) $ \iff $ (ii) follows from IB Linear Algebra.\par
For the first statement, suppose that $ S $ genreates $ M $ freely. If $ S $ is not linearly independent we can write
\[
  0=r_1m_1+r_2m_2+\cdots +r_km_k
\]
and we can assume with reordering that $ r_1\ne 0 $. Consider the set function $ f:S\to R $ sending $ m_1\to 1 $ and $ m_i\to 0 $ for all $ i>1 $. Now compute $ 0=\varphi(0)=\varphi\left(\sum r_im_i\right)=r_1 $ which is contradiction.
For the converse assume (iii). Since every $ m $ is uniquely $ m=\sum^k_{i=1}r_im_i $. Given any set function $ f:S\to N $ we define $ \varphi_f:M\to N $ by
\[
  \varphi_f(r_1m_1+\cdots r_km_k)=r_1\varphi_f(m_1)+\cdots+r_k\varphi_f(m_k).
\]
We have well-definedness by uniqueness and is clearly an $ R $-module homomorphism. So $ S $ generates $ M $.\qed
\par
Now take $ \{2,3\}\subseteq \Z $. This set generates $ \Z $ by Bezout's, but
\[
  0=3\cdot 2 + (-2)\cdot 3
\]
so the set does not generate $ \Z $ freely. But what differs from linear algebra vector spaces, is that neither $ 2 $ or $ 3 $ generate $ \Z $ so no it's not like a linearly independent list of elements in a vector space where we can chuck elements out to get a basis.
\begin{definition}
	(Relations) Let $ M $ be finitely-generated with generators given by $\theta: R^k\to M $. Then $ \ker \theta $ is called the \textit{module of relations} of $ M $ with respect to $ \theta $.
\end{definition}
\begin{definition}
	(Finitely presented module) We say that a module is \textit{finitely-presented} if $ \ker\theta $ is finitely generated.
\end{definition}
And for someone completely different.
\begin{proposition}
	(Invariance of dimension/rank) Let $ R $ be a non-zero ring. Then if $ R^n\cong R^m $ then $ n=m $.
\end{proposition}
\pf We will prove the statement by reducing to $ R $ to a field where we know the result follows.\par
If $ I\nrm R $ is an ideal and $ M $ is an $ R $-module, consider $ IM = \{ r\cdot m:r\in I, m\in M\}\le M $. So we can consider $ M/IM $ as an $ R $-module. If $ r\in I $ then its action on $ M/IM $ is
\[
  r(m+IM)=rm+IM=0+IM=IM.
\]
So we can make $ M/IM $ into an $R/I$-module by
\[
	(r+I)\cdot (m+IM)=r\cdot m+IM.
\]
Now let $ I\nrm R $ be a maximal ideal So we have an isomorphism $ (R/I)^n\cong (R/I)^m $ of $ R/I $-modules. Since $ R/I $ is a field we have $ n=m $ by the invariance of dimension for vector spaces.\qed
\subsection{Matrices over Euclidean domains}
For the rest of the course we'll fix a Euclidean domain $ R $ along with a Euclidean function $ \varphi: R\setminus \{0\}\to \Z_{\ge 0} $. We know that for $ a,b\in R $ we can find $ x,y\in R $ such that $ ax+by=\gcd(a,b) $.
\begin{definition}
	(Elementary row operations) Let $ A $ be an $ m\times n $ matrix with entries in $ R $. The \textit{elementary row opertaions} are the following
	\begin{enumerate}
		\item [(ER1)] Add $ c\in R $ times row $ i $ to row $ j $
		\item [(ER2)] Swap row $ i $ and row $ j $
		\item [(ER3)] Multiply the $ i $th row by a \textit{unit} $ c\in R $. 
	\end{enumerate}
	A similarly we can do column operations in the same way.
\end{definition}
How 'nice' can we make $ A $?
\begin{definition}
	(Equivalent matrices) Two matrices $ A,B $ over $ R $ are equivalent if $ A $ can be obtained from $ B $ by a sequence of row and column operations. In particular
	\[
	  B=QA\inv T
	\]
	for $ Q,T $ invertible.
\end{definition}
Notice that the matrix $
\begin{pmatrix}
	2 & 0 \\
	0 & 0 
\end{pmatrix} $
over $ \Z $ cannot be reduced any further.

\begin{theorem}
	(Smith normal form) Any $ m\times n $ matrix over $ R $ is equivalent to one of the form
	\[
	  \begin{pmatrix}
		  d_1 & & & & \\
		  & d_2 & & & \\
		  & & \ddots & & \\
		  & & & \ddots & \\
		  & & & & 0 
	  \end{pmatrix}
	\]
	with
	\[
	  d_1\mid d_2\mid \cdots \mid d_r
	\]
	These $ d_i $ are the \textit{invariant factors} of $ A $
\end{theorem}
\pf If $ A=0 $ we're done so assume that $ A\ne 0  $. Let $ A_{ij} $ be some non-zero entry. By row and column operations we can move this to the $ (1,1) $ position, so we can assume that $ A_{11}\ne 0 $. Now we have two basic moves.
\begin{enumerate}
\item If $ A_{1j} $ is not divisible by $ A_{11} $ so we can write $ A_{1j}=A_{11}q+r $ with $ \varphi(r)<\varphi(A_{11}) $. By column operations we can make the $ (i,j) $-entry equal to $ r $. Now swap the $ (1,j) $-entry with the $ (1,1) $-entry. The result of this is that $ \varphi $ of the $ (1,1) $ entry has got smaller.
\item Similar move for $ A_{j1} $. Using row operations instead of column operations.
\end{enumerate}
The consequence of (i) and (ii) is that after finitely many operations the $ (1,1) $ entry divides everything in row 1 and column 1. By more operations we get the matrix of the form
\[
  \begin{pmatrix}
	  d & 0 & \cdots & 0 \\
	  0 & & & \\
	  \vdots & & C & \\
	  0 & & &
  \end{pmatrix}
\]
for a $ (m-1)\times(n-1) $ matrix $ C $. Replace $ A $ with this matrix. Now suppose that $ A_{ij} $ in $ C $ is \textit{not} divisible by $ d $. Then
\[
	A_{ij} = dq + r,\quad \varphi(r)<\varphi(d).
\]
Now we add column 1 to column $ j $ and subtract $ q $ times row 1 from row $ i $. This gives $ r $ in the new $ (i,j) $-entry. Swap again to make $ r $ the $ (1,1) $-entry. Now we can repeat (i) and (ii) to get a matrix of the form
\[
  \begin{pmatrix}
	  d' & 0 & \cdots & 0 \\
	  0 & & & \\
	  \vdots & & C' & \\
	  0 & & & 
  \end{pmatrix}
\]
Where $ d'=r $. By applying this finitely many times we get a matrix
\[
  \begin{pmatrix}
	  d & 0 & \cdots & 0 \\
	  0 & & & \\
	  \vdots & & C' & \\
	  0 & & &
  \end{pmatrix}
\]
where $ d $ divides every entry in $ C $.\par
Now recurse with $ C $ replacing $ A $ from the start of the proof. Then the output matrix is as claimed.\qed
\begin{remark}
  However we haven't shown the invariant factors are well-defined.
\end{remark}
Recall for a matrix $ A $ a $ k\times k $ minor is the determinant of a $ k\times k $ submatrix.
\begin{definition}
	(Fitting ideals) For $ A $ a matrix over $ R $, the $ k $th \textit{fitting ideal} is the ideal 
	\[
	  \Fit_k(A)\nrm R
	\]
	generated by the $ k\times k $ minors of $ A $.
\end{definition}
\begin{proposition}
  If $ A $ and $ B $ are equivalent matrices then 
  \[
    \Fit_k(A)=\Fit_k(B),\quad \forall k.
  \]
\end{proposition}
\pf We'll check that $ \Fit_k(A) $ is unchanged by row and column operations. Since $ \Fit_k(A)=\Fit_k(A^T) $ it suffices to check row operations. Consider $ C $ a $ k\times k $-submatrix of $ A $. Consider adding $ c $ times row $ i $ to row $ j $.
\begin{enumerate}
	\item If both rows lie inside $ C $ then the minor is unchanged by properties of the determinant.
	\item If the $ j $th row is outside $ C $ then $ C $ is unchanged obviously.
	\item Suppose the $ j $th row in $ C $ and the $ i $th row is not. If row $ i $ is $ [f_1,\dots, f_k] $ then the new row $ i $ is $ [C_{j1}+cf_1,\cdots, C_{jk}+cf_k] $. Now computing the determinant of this new submatrix along this row. So
		\[
		  \det C'=\det C+c\det D
		\]
		where $ D $ is obtained from $ C $ by replacing row $ j $ with $ [f_1,\cdots, f_k] $ But $ D $ is a $ k\times k $ submatrix of $ A $ so we're done since ideals are closed under addition and multiplication.
\end{enumerate}
The remaining operations are straightforward to check. We conclude that $ \Fit_k(A)\subseteq \Fit_k(B) $ and since all row operations are invertible we have that $ \Fit_k(A)=\Fit_k(B) $\qed
\begin{corollary}
  If $ A $ has Smith normal form then
  \[
    \Fit_k(A)=(d_1d_2\cdots d_k)
  \]
\end{corollary}
\pf Trivial \qed
\begin{proposition}
  Let $ R $ be a principal ideal domain. Any submodule of $ R^m $ can be generated by $ m $ or fewer elements.
\end{proposition}
\pf We will induct on $ m $ since the $ m=1 $ case is the definition of a PID. Let $ N\le R^m $. Consider $ I=\{r\in R:(r,r_2,\dots, r_m)\in N \text{ for some } r_2,\dots, r_m\in R\}$. Certainly this is an ideal so $ I\nrm R $. Since $ R $ is a PID we have that $ I $ is generated by a single element $ (a) $. Choose $ n=(a,a_2,\dots, a_m)\in N $. for $ (r_1,\dots, r_m)\in N $ we know that $ a\mid r_1 $.So $ (r_1,\dots, r_m)-rn=(0,r_2-ra_2,\dots, r_m-ra_m) $. So everything in $ N $ is a multiple of $ n $ plus something in $ N\cap (\{0\}\times R^{m-1})\le R^{m-1}  $. Now induct.\qed
\begin{theorem}
	Let $ R $ be a Euclidean domain. $ N\le R^m $. There exists a basis $ v_1,\dots, v_m $ for $ R^m $ such that $ N $ is generated by $ d_1v_1,\dots, d_rv_r $ for some $ 0\le r\le m $, $ d_i\in R $ such that $ d_i\mid d_{i+1} $.
\end{theorem}
\pf By the previous proposition we have that $ N $ is generated by $ x_1,\dots, x_n $ with $ n\le m $. Viewing this as a matrix with $ x_i $ as the $ i $th column. Putting this into Smith normal form we note that row operations change the basis for $ R^m $ and column operations for $ N $. The matrix has the form
\[
  \begin{pmatrix}
	  d_1 & & & & & & & \\
	      & d_2 & & & & & \\
	      & & \ddots & & & & \\
	      & & & d_r & & & \\
	      & & & & 0 & & \\
	      & & & & & \ddots & \\
	      & & & & & & 0 \\
	      & & & & & & 0 \\
	      & & & & & & \vdots \\
	      & & & & & & 0
  \end{pmatrix}
\]
So in this new basis $ N $ is generated by $ \{d_iv_i\}^r_{i=1} $.\qed
\begin{theorem}
	If $ R $ is a Euclidean domain, then any submodule of $ R^m $ is free.
\end{theorem}
\pf In the proof of the previous theorem we have a generating set $ d_1v_1,\dots, d_rv_r $. Any linear dependence between $ \{d_iv_i\} $ gives a linear dependence between the $ v_i $'s themselves, but $ \{v_i\} $ is a basis so it's linear indepedent.\qed
\begin{theorem}
  Let $ R $ be a Euclidean domain. Let $ M $ be a finitely generated $ R $-module. Then
  \[
	  M\cong R\oplus \cdots \oplus R\oplus \frac R{(d_1)}\oplus \cdots \oplus\frac R{(d_r)}
  \]
  where $ d_i\ne 0 $ and $ d_i\mid d_{i+1} $.
\end{theorem}
\begin{remark}
  If $ R=\Z $ and $ M $ is finite we recover the classification of finite abelian groups. We get a generalisation of this statement to finitely generated abelian groups which may be possibly infinite.
\end{remark}
\pf Write $ \gamma:R^m\to M $ a generating surjection. We know $ \ker \gamma\le R^m $ and
\[
	M\cong \frac{R^m}{\ker \gamma}.
\]
Pick a basis $ v_1,\dots,v_m $ for $ R^m $ such that $ \ker\gamma $ is generated by $ d_1v_1,\dots,d_rv_r $ for $ 0\le r\le m) $ with $ d_i\mid d_{i+1} $. This means that \[
	M\cong\frac{R^m}{\left\langle(d_1,0,\dots,0),(0,d_2,0,\dots,0),\cdots,(0,\dots,d_r,0,\dots,0)\right\rangle}\cong \frac R{(d_1)}\oplus\cdots\oplus \frac R{(d_r)}\oplus R\oplus\cdots\oplus R
\]
hence we're done.\qed
\par
Suppose that $ G $ is abelian generated by $ a,b,c\in G $ with the following relations
\begin{align*}
  2a+3b+c=0 \\
  a+2b=0 \\
  5a+6b+7c=0
\end{align*}
We have a surjection of abelian groups
\begin{align*}
  \Z^3\to G\\
  e_1\to a\\
  e_2\to b\\
  e_3\to c.
\end{align*}
The associated matrix is
\[
  \begin{pmatrix}
	  2 & 1& 5 \\
	  3 & 2 & 6\\
	  1& 0 & 7
  \end{pmatrix},
\]
and the homomorphism $ \Z^3\to \Z^3 $ has $ G\cong \Z^3/\ima(a) $. 
We need to find the "$ d_i $'s" from the theorem. To do this calculate $ \Fit_k(A) $. This comes out to
\[
  \Fit_1(A)=\Fit_2(A)=(1),\quad \Fit_3(A)=(3).
\]
So $ G\cong \Z/3 $
\begin{proposition}
	(Chinese remainder theorem) For $ R $ a Euclidean domain with $ a,b\in R $ such that $ \gcd(a,b)=1 $. Then
  \[
	  \frac R{(a)}\oplus \frac R{(b)}\cong \frac R{(ab)}.
  \]
\end{proposition}
\pf Consider $ \varphi: R/(a)\oplus R/(b)\to R/(ab) $, given by $ (r_1+(a),r_2+(b))\to (br_1+ar_2+(ab)) $. To show well-definedness, say $ (r_1+(a),r_2+(b))=(r_1'+(a),r_2'+(b)) $. So $ r_1=r_1'+za $ and $ r_2=r_2'+wr_2=r_2'+wb $. So by computation we get that $ br_1+ar_2+(ab)=br_1'+bza+ar_2'+awb=br_1'+ar_2'+(ab) $. Hence $ \varphi $ is well-defined.\par
By the Euclidean algorithm we can write $ 1=ax+by $. Observe that
\[
  \varphi(y+(a),x+(v))=1+(ab)
\]
and since $ R/(ab) $ is generated by $ 1_{R/(ab)} $ and $ \varphi $ is an $ R $-module homomorphism, the map is surjective.\par
Finally for injectivity we calculate $ \ker\varphi $. If $ br_1+ar_2\in (ab) $ then write
\[
  br_1+ar_2=abx
\]
for some $ x\in R $. We know that $ a \mid ar_2 $ and $ a \mid ab\implies a\mid br_1 $. $ a\mid b $ are coprime so $ a\mid r_1 $. Symmetrically we have $ b\mid r_2 $ So $ r_1\equiv 0 \mod (a) $ and $ r_2\equiv 0\mod (b) $ so $ \ker\varphi = \{(0,0)\} $.\qed
\begin{theorem}
	(Prime decomposition theorem) Let $ R $ be a Euclidean domain and $ M $ be a finitely generated $ R $-module. Then
\[
  M\cong N_1\oplus \cdots \oplus N_t
\]
where each $ N_i $ is either $ R $ or $ R/(p^n) $ for some prime $ p $.
\end{theorem} 
\pf We may write $ M\cong \frac R{(d_1)}\oplus \cdots \oplus \frac R{(d_r)}\oplus R\oplus \cdots \oplus R $. For each $ d_i $ write as a product of prime powers and apply the previous proposition.\qed
\subsection{Modules over $ \F $ and forms of matrices}
Let $ V $ be a vector space over $ \F $ and $ \alpha\in\mathrm{End}(V) $ so $ \alpha:V\to V $ linear. We can make $ V $ into an $ \F[X] $-module where
\begin{align*}
	\F[X]\times V\to V \\
	(f(X), v)\to f(\alpha)v.
\end{align*}
For notation we'll write $ V_\alpha $ for this module.
\begin{lemma}
	If $ \dim V $ is finite then $ V_\alpha $ is a finitely generated $ \F[X] $-module.
\end{lemma}
If we have a generating set for $ V $ as an $ \F $-module then they also generate $ V_\alpha $ since $ \F\le \F[X] $.\qed
\par
For an example suppose that $ V_\alpha\cong \F[X]/(X^r) $. As a vector space ove r $ \F $, $ V_\alpha $ is spanned by $ 1,X,X^2,\dots, X^{r-1} $ which is a basis. As a matrix $ \alpha $ i.e. multiplication by $ X $ is given by
\[
  \begin{pmatrix}
	  0 & 0 & \cdots & 0 & 0 \\
	  1 & 0 & \cdots & 0 & 0 \\
	  0 & 1 & \cdots & 0 & 0 \\
	  \vdots & \vdots & \ddots & \vdots & \vdots \\
	  0 & 0 & \cdots & 1 & 0
  \end{pmatrix}.
\]
Similarly if we have $ V_\alpha\cong \F[X]/(X-\lambda)^r $, for $ \lambda\in \F $, then write $ \beta:V\to V $ as $ \alpha-\lambda\cdot \mathrm{id} $. Now by the previous example $ \beta $ can be written in some basis as the matrix in the previous example. So $ \alpha $ can be written in some basis as
\[
  \begin{pmatrix}
	  \lambda & 0 & \cdots & 0 & 0 \\
	  1 & \lambda & \cdots & 0 & 0 \\
	  0 & 1 & \cdots & 0 & 0 \\
	  \vdots & \vdots & \ddots & \vdots & \vdots \\
	  0 & 0 & \cdots & 1 & \lambda
  \end{pmatrix}.
\]
Now for another example, let $ p(X) = a_0 + a_1X+\cdots +a_{r-1}X^{r-1} + X^r$. Suppose that $ V_\alpha $ is an $ \F[X] $ module and $ V_\alpha\cong \F[X]/(p(X)) $. As a vector space over $ \F $ we have $ \dim \F[X]/(p(X)) = r $ (in particular both $ V_\alpha $ and this as vector spaces over $ \F $ are isomorphic to $ \F^r $). There is a basis for $ \F[X]/(p(X)) $ given by $ \{1,X,X^2,\cdots,X^{r-1}\} $. The matrix $ \alpha $ in this basis is
\[
  \begin{pmatrix}
	  0 & 0 & \cdots & 0 & -a_0 \\
	  1 & 0 & \cdots & 0 & -a_1 \\
	  0 & 1 & \cdots & 0 & -a_2 \\
	  \vdots & \vdots & \ddots & \vdots & \vdots \\
	  0 & 0 & \cdots & 1 & -a_{r-1}
  \end{pmatrix}.
\]
\textit{This is sometimes called the companion matrix for} $ p(X) $.
\begin{theorem}
	(Rational canonical form) Let $ V $ be a finite dimensional vector space over $ \F $ and let $ \alpha\in\mathrm{End}(V) $ giving an $ \F[X] $-module $ V_\alpha $. Then
	\[
		V_\alpha\cong\frac{\F[X]}{(f_1)}\oplus \cdots \oplus \frac{\F[X]}{(f_s)} 
	\]
with $ f_1\mid f_2\mid \cdots\mid f_s $ and there exists a basis for $ V $ where $ \alpha $ has block diagonal matrix
	\[
	  \begin{pmatrix}
		  c(f_1) & & \\
			 & \ddots & \\
			 & & c(f_s)
	  \end{pmatrix}
	\]
	where $ c(f_i) $ is the companion matrix for $ f_i $.
\end{theorem}
\pf Apply classification of modules over $ \F[X] $ and the previous example.\qed
\begin{remark}
	In this rational canonical form, the minimal polynomial of $ \alpha $ is $ f_s $ and the characteristic polynomial is $ \prod_i f_i $. This is beacuse $ f(\alpha) $ acts by 0 on all direct summands. No smaller degree polynomial than $ \alpha $ acts by 0 on $ \F[X]/(f_s) $.
\end{remark}
\begin{lemma}
	The primes in $ \C[X] $ are exactly $ (X-\alpha) $ for $ \alpha\in \C $.
\end{lemma}
\pf Any constant is either a unit or 0 so not prime. Any quadratic or above has a root so not irreducible. Clearly $ (X-\alpha) $ are always prime since $ \C[X]/(X-\alpha)\cong \C $ which is a field.\qed\par
Now for the final theorem! (woah)
\begin{theorem}
	(Jordan canonical form)	Let $ V $ be a vector space over $ \C $ and let $ \alpha\in \mathrm{End}(V) $. Then
	\[
		V_\alpha\cong \frac{\C[X]}{(X-\lambda_1)^{a_1}}\oplus\cdots\oplus\frac{\C[X]}{(X-\lambda_t)^{a_t}}
	\]
	and there is a basis where $ \alpha $ is given by the matrix 
	\[
	  \begin{pmatrix}
		  J_{a_1}(\lambda_1) & & \\
				     & \ddots & \\
				     & & J_{a_t}(\lambda_t) 
	  \end{pmatrix}
	\]
	where $ J_a(\lambda) $ is the $ a\times a $ matrix given by
	\[
	  \begin{pmatrix}
		  \lambda & 0 & \cdots & 0 \\
		  1 & \lambda & \cdots & 0 \\
		  \vdots & \vdots & \ddots & \vdots \\
		  0 & \cdots & 1 & \lambda
	  \end{pmatrix}
	\]
	which is the Jordan canonical form.
\end{theorem}
\pf Apply the prime decomposition theorem using the lemma above to $ V_\alpha $\qed
\end{document}

























