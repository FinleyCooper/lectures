\documentclass{article}
\usepackage{../header}
\title{Fluid Dynamics}
\author{Notes by Finley Cooper}
\begin{document}
  \maketitle
  \newpage
  \tableofcontents
  \newpage
  \section{Kinematics}
  \subsection{Streamlines and pathlines}
  There are two natural ways to think of flow.
  \begin{enumerate}
	  \item A stationary observer watching flow go past. This is the Eulerian perspective. This is the approach used through this course. We define a velocity field (continuum field) $ \mathbf u(\x, t) $.
	  \item A moving observing, travelling along with the flow. This is the Lagrangian perspective.   
  \end{enumerate}
  \begin{definition}
	  (Streamlines) These are curves that are everywhere parallel to the flow at a given instant.
  \end{definition}
  \begin{remark}
    The streamline that goes through $ \mathbf x_0 $ at time $ t_0 $ is given parametrically as $ \x = \x(s, \x_0,t_0) $ and
    \[
	    \frac{d\x}{ds}=\mathbf u(\x, t_0)
    \]
    (with $ \x=\x_0 $ at $ s=0 $).
  \end{remark}
  The set of streamlines shows the direction of flow a \textit{given} instant a time (all fluid particle at one given time).
  Take the example $ \mathbf u = (1,t) $. So at $ t=0 $ we have $ \mathbf u=(1,) $ so the streamlines are horizontal lines. At $ t=1 $ we have $ \mathbf u = (1,1) $, so the streamlines are diagonal.
  \begin{definition}
	  (Pathlines) A \textit{pathline} is the trajectory of a fluid particle (a very small bit of fluid). The pathline $ \x=\x(t,\x_0) $ of a fluid which is at $ \x_0 $ at $ t=0 $ is such that
	  \[
		  \frac{d\x}{dt}=\mathbf u(\x,t)
	  \]
	  with $ \x(0,\x_0)=\x_0 $.
  \end{definition}
  Again if we take $ \mathbf u = (1,t) $ we get
  \[
    \begin{cases}
	    \frac{dx}{dt}=1 \\
	    \frac{dy}{dt}=t
    \end{cases}\rightarrow \begin{cases}
      x=x_0+t\\
      y=y_0 +\frac{t^2}2
    \end{cases}
  \]
  which describes the path $ y-y_0=\frac12(x-x_0)^2 $.
\begin{remark}
  Pathlines are often called "Lagrangian trajectories". The applications are very useful to characterise transport (infecious diseases and pollution simulations).
\end{remark}
	If the flow is \textit{steady} (so $ \mathbf u $ does not depend on time). Then pathlines and streamlines are the same.
\subsection{The material derivative}
We will characterise the rate of change of "stuff" moving with a fluid. Consider a quantity $ F(\x,t) $ in a fluid flow (intution is $ F $ is temperature). We want to measure how the temperature chnages as we move through the field $ F $ along the flow. Let compute the rate of change of $ \F $ (in time) seen by a moving observer. We will call this $ \frac{DF}{Dt} $. Take a small time interval $ \delta t $. Then
\begin{align*}
	\delta F &= F(\x+\delta\x,t+\delta t)-F(\x,t)\\
		 &= \delta t\frac{\partial F}{\partial t}+(\delta\x\cdot \nabla)F + \text{(higher order terms)}.
\end{align*}
We have that $ \delta\x=\mathbf u\delta t $, so
\[
	\frac{\delta F}{\delta t}=\frac{DF}{Dt}=\frac{\partial F}{\partial t}+(\mathbf u \cdot \nabla)F.
\]
We have the derivative and the convected derivative. This should be thought of as moving along gradients of a field.
\subsection{Conservation of mass}
Consider the flow through a straight rigid pipe with constant cross section. Suppose we have a $ \mathbf u_{\text{in}} $ and a $ \mathbf u_{\text{out}} $. Can we have $ \mathbf u_{\text{in}}\ne \mathbf u_{\text{out}} $? For a gas, yes we can since they can be compressed. For a fluid, we cannot, since they are incompressible.\par
Define $ \rho(\x,t) $ as the mass density with $ [\rho]=\frac{\text{M}}{ \text{L}^3 }$. We want a relation between $ \rho $ and $ \mathbf u $. Consider a fixed volume $ V $ and compute the rate of change of its mass, $ M $.
\begin{align*}
	M &=\int_V\rho\mathrm dV
\end{align*}
Assume that mass can only change due to the flow of mass across the boundary surface $ \partial V $. Take a small surface element $ \delta A $ with normal $ \mathbf n $. The volume out of $ V $ during $ \delta t $ is $ (\mathbf u \cdot \mathbf n)\delta A\delta t $. Hence the mass out is $ \rho(\mathbf u\cdot\mathbf n)\delta A\delta t $, so we get that
\[
	\frac{dM}{dt}=-\int_{\partial V}\rho(\mathbf u\cdot \mathbf n)\mathrm dA.
\]
The divergence theorem will allow us to rewrite this as
\begin{align*}
	\frac{\partial \rho}{\partial t}+\nabla\cdot(\rho \mathbf u) = 0.
\end{align*}
We know from IA Vector Calculus that $ \nabla\cdot(\rho\mathbf u) = \rho\nabla\cdot \mathbf u +\mathbf u \cdot\nabla\rho $, so we can write that
\[
	\frac{D\rho}{Dt}=-\rho\nabla\cdot \mathbf u.
\]
\begin{definition}
	(Incompressible) A fluid flow is \textit{incompressible} if $ \frac{D\rho}{Dt}=0 $.
\end{definition}
This is then equivalent to $ \nabla\cdot\mathbf u =0 $ which is the equivalent condition we'll use for the course.\par
For this course we will assume that $ \rho $ is constant. This means as a consequence that $ \nabla\cdot\mathbf u =0 $.
\subsection{Kinematic boundary condition}
Consider the material boundary, with unit norm $ \mathbf n $, of a body of fluid has a given velocity $ \mathbf U(\x,t) $. At a point $ \x $ on the boundary, the fluid velocity relative to the surface is $ \mathbf u -\mathbf U $. Applying mass conversation on the interface over a small surface element $ \delta A $ in time $ \delta t $. So
\begin{align*}
  \rho(\mathbf u-\mathbf U)\cdot\mathbf n \delta A\delta t = 0.
\end{align*}
Hence we require $ \mathbf u \cdot n = \mathbf U \cdot \mathbf n $ at the interface. This is the kinematic boundary condition.
\begin{remark}
  $ \mathbf n $ occurs on both sides, hence we don't need $ \mathbf n $ to be a unit vector.
\end{remark}
We have some consequences of this condition.
\begin{enumerate}
	\item If the boundary is fixed, $ \mathbf U=0 $ implies that $ \mathbf u \cdot \mathbf n = 0 $. This is called the no penetration condition.
	\item Consider an air/water interface (free surface). Suppose the surface is defined by $ z=\xi(x,y,t) $. Then can think of the free space as $ F(x,y,z,t)=0 $ where $ F(x,y,z,t) = z-\deta(x,y,t) $. So $ \mathbf n $ is perp to $ \nabla F = (-\xi_x,-\xi_y,1) $. Then if $ \mathbf u = (u,v,w) $ so $ \mathbf U = (0,0,\xi_t) $. Then the kinematic boundary condition becomes $ -u\xi_x-c\xi_y+w=\xi_t $, so $ w=\xi_t+u\xi_x+v\xi_y=\frac{D\xi}{Dt} $. This is equivalent to $ \frac{DF}{Dt}=0 $.
\end{enumerate}
\subsection{Streamfunction for 2D incompressible flow}
We know that $ \nabla\cdot \mathbf u = 0 $ which is equivalent to there existing a vector potential $ \mathbf A $ such that $ \mathbf u = \nabla\times \mathbf A $. In 2D if $ \mathbf u = (u,v,0) $ then $ \mathbf A = (0,0,\psi(x,y)) $. So
\[
	u=\frac{\partial \psi}{\partial y},\quad v = -\frac{\partial \psi}{\partial x}.
\]
We call $ \psi $ a \textit{streamfunction}. Looking at dimensions we have that $ [\psi]=\mathrm{L}^2 \ \mathrm{T}^{-1} $. Now we'll see an example.\par
Let $ \mathbf u = (y,x) $ (which we can see is incompressible) so 
\[
	\frac{\partial\psi}{\partial y} = u = y
\],
hence $ \psi = \frac 12 y^2 + f(x) $. We also have that $ -\frac{\partial \psi}{\partial x}=-f'(x)=x $, so $ \psi = \frac 12(y^2-x^2)+C $.\par
We have some properties about the streamfunction,
\begin{enumerate}
	\item Streamlines are given by $ \psi=\text{constant} $.
	\item $ |\mathbf u| = |\nabla\psi| $, so the flow is faster if the streamlines are closer together.
	\item If we take two points $ \mathbf x_0,\mathbf x_1 $, then then the volume flux crossing the line between $ \mathbf x_0 $ and $ \mathbf x_1 $ is
		\[
			\int_{\mathbf x_0}^{\mathbf x_1} \mathbf u \cdot \mathbf n \mathrm d\ell = \psi(\mathbf x_1)-\psi(\mathbf x_0).
		\]
	\item $ \psi $ is constant at rigid boundaries.
\end{enumerate}
We can do the same in polar coordinates. So $ \mathbf u = (u_r(r,\theta), u_\theta(r,\theta),0) $. We have that
\[
	\mathbf u = \nabla\times \mathbf A =\left(\frac 1r\frac{\partial \psi}{\partial\theta}, -\frac{\partial \psi}{\partial r},0\right),
\]
so we can check that $ \nabla\cdot\mathbf u = \frac 1r\frac{\partial}{\partial r}(ru_r)+\frac1r \frac{\partial}{\partial \theta}=0 $
\section{Dynamics of inviscid flow}
\subsection{Surface and volume forces}
There a two types of forces exerted on a fluid.
\begin{enumerate}
	\item Forces proportional to the volume (gravity);
	\item Forces proportional to the surface area (pressure, viscous stresses).
\end{enumerate}
We'll first look at the first type, called volume forces. We'll denote $ F(\x,t)\delta V $ as the force acting on a small volume element $ \partial V $. Let's take gravity as an example, so $ \mathbf F=\rho \mathbf g $. Often we have that $ \mathbf F $ is conservative, so $ \mathbf F=-\nabla\chi $ for some function $ \chi $ (we know gravity is $ \chi=\rho gz $).\par
Now for surface forces. Consider a small element of area of $ \mathbf n \delta A $. Denote the surface force exerted by the positive side on the negative side by $ \boldsymbol \tau(\x,t,\mathbf n) \delta A $. We say that $ \boldsymbol\tau $ is "stress" acting on a surface element. Note that $ \boldsymbol \tau $ depends on orientation. By Newton's third law, we have that $ \boldsymbol \tau(\x,t,-\mathbf n) = -\boldsymbol \tau(\x,t,\mathbf n) $.\par
There are many phenomona where friction inside a fluid (viscous stress) is negligible. For example a 10cm box of water, it takes hours for viscosity to bring the fluid to rest once disturbed.
\begin{definition}
	(Inviscid) A fluid is said to be \textit{inviscid} if we can neglect viscosity.
\end{definition}
For inviscid flow, $ \boldsymbol\tau $ has no tangential component and its magnitude is independent of orientation. So $ \boldsymbol \tau(\x,t,\mathbf n) =-p(\x,t)\mathbf n $, where $ p $ is the pressure. Note we have the minus sign because the positive side pushes with pressure $ p $ 	towards the negative side when $ p>0 $. 
\subsection{The Euler Momentum equation}
The idea here is that we do a similar calculation for mass conservation, but now for momentum instead. Consider an arbitrary fixed volume, $ V $ with boundary $ \partial V $. Hence the momentum inside $ V $ is
\[
  \int_V \rho \mathbf u\ \mathrm dV.
\]
The momentum inside $ V $ can change due to
\begin{enumerate}
	\item Flux of momentum across $ \partial V $;
	\item Force acting on $ V $ or $ \partial V $.
\end{enumerate}
The volume out of $ \delta A $ in $ \delta t $ is $ (\mathbf u\cdot\mathbf n )\delta A\ \delta t $. So the momentum out of $ \delta A $ in time $ \delta t $ is $ \rho \mathbf u (\mathbf u \cdot\mathbf n)\delta A\ \delta t $. Hence we get the following.
\begin{theorem}
	(Euler momentum integral equation)
\begin{align*}
	\frac d{dt} \int_V\rho \mathbf u\  \mathrm dV &= -\int_{\partial V}\rho \mathbf u (\mathbf u \cdot \mathbf n)\ \mathrm d A + \underbrace{\int_V \mathbf f\mathrm dV}_{\text{volume force}} + \underbrace{\int_{\partial V} - p\mathbf n \ \mathrm dA}_{\text{surface force}}.
\end{align*}
\end{theorem}
In components,
\begin{align*}
	\int_V\frac\partial{\partial t}(\rho u_i)\ \mathrm dV = -\int_{\partial V} \rho u_iu_jn_j\ \mathrm dA + \int_{\partial V} - pn_i\ \mathrm dA + \int_V f_i \ \mathrm dV.
\end{align*}
Sometimes books call $ \rho u_i u_j $ the momentum flux tensor. We can apply the divergence theorem to the first two integrals which gives that those two integrals become
\[
	\int_V\left[-\frac{\partial}{\partial x_j}(\rho u_iu_j)-\frac{\partial p}{\partial x_i}\right]\mathrm dV.
\]
Given that this true for any fixed volume $ V $, the integrand must be zero, hence we have that
\[
	\frac\partial{\partial t}(\rho u_i)+\frac{\partial}{\partial x_j}(\rho u_iu_j)=-\frac{\partial p}{\partial x_i}+f_i.
\]
This becomes
\[
	u_i\left[\frac{\partial \rho}{\partial t}+\frac{\partial}{\partial x_j}(\rho u_j)\right]+\rho\left[\frac{\partial u_i}{\partial t} + u_j\frac{\partial}{\partial x_j} u_i\right]=-\frac{\partial p}{\partial x_i} + f_i.
\]
So using mass conservation on the first part we get some simplification, so going back to vector form we have the Euler momentum equation,
\[
	\rho\frac{D\mathbf u}{D t} = -\nabla p+ \mathbf f.
\]
This is the equation of motion for inviscid fluid flow.\par
Fluid particles accelerate under differences in pressure and volume (body) forces (inviscid flows only).
\begin{remark}
  Note that at the surface, the stress exerted by the fluid at the surface is $ p\mathbf n $.
\end{remark}
Let's see an application of the momentum equation. Consider a $ 90^{\text o} $ bent pipe with a flow $ U $. What is the force exerted by flow on the pipe for a steady flow without gravity? We will use the Euler momentum integral equation. Since the flow is steady, the LHS term is  zero and since there is no gravity, the volume force term is zero. Hence we have that
\[
	\int_{\text{walls}} + \int_{\text{end}}[\rho \mathbf u (\mathbf u\cdot n ) p\mathbf n]\mathrm dA = 0
\]
and because of our kinetmatic boundary condition, $ \mathbf u\cdot\mathbf n = 0	$, so our integral across the walls becomes
\begin{align*}
	\int_{\text{walls}} p\mathbf n\ \mathrm d A =\mathbf F = \text{Force exerted by fluid flow on the pipe}.
\end{align*}
Across the ends,
\[
	\int_{\text{in} +\text{ out}} [\rho\mathbf u(\mathbf u\cdot\mathbf n)+p\mathbf ][\mathrm dA  = \rho(-U)(-U\mathbf n_1)A_1+p_1\mathbf n_1+\rho U(U\mathbf n_2) A_2 + p_2\mathbf n_2 A_2.
\]
W ehave that $ p=p_1=p_2 $ and $ A_1=A_2=A $ so the integral becomes
\[
	=A[(p+\rho U^2)(\mathbf n_1+\mathbf n_2)],
\]
hence
\[
  \mathbf F = -A(p+\rho U^2)(\mathbf n_1+\mathbf n_2).
\]
\subsection{Bernoulli equation for steady flow with potential forces}
There are \textit{two} Bernoullli equations in the course. We'll look at the first one for steady flow here.\par
Recall from the Euler equation that
\[
	\rho\left(\frac{\partial\mathbf u}{\partial t} + (\mathbf u \cdot\nabla)\mathbf u\right) = -\nabla p+ \mathbf F.
\]
We will assume that
\begin{enumerate}
	\item We have steady flow so $ \frac{\partial}{\partial t}= 0 $.
	\item $ \rho $ is constant (as always in IB Fluid Dynamics).
	\item $ \mathbf F = - \nabla \chi $ (conservative force).
\end{enumerate}
So the Euler equation gives that
\[
  \rho(\mathbf u \cdot \nabla \mathbf u) = -\nabla(p+\chi).
\]
Now we have the identity
\[
	(\mathbf u\cdot \nabla)\mathbf u = \nabla\left(\frac 12 u^2\right) -\mathbf u \times(\nabla\times \mathbf u),
\]
which gives that 
\begin{align*}
	\rho\left[\nabla\left(\frac 12 u^2\right)-\mathbf u\times(\nabla\times \mathbf u)\right] = -\nabla(p+\chi).
\end{align*}
Now since $ \rho $ is constant we can move it inside the $ \nabla $. We'll now define $ \boldsymbol\omega = \nabla\times \mathbf u $, the \textit{vorticity} of the fluid, giving,
\[
	\nabla\left[\frac 12 \rho u^2+p+\chi\right]= \rho\mathbf u\times \boldsymbol\omega.
\]
We like to dot this with $ \mathbf u $ to get the Steady Bernoulli equation,
\[
\mathbf u \cdot\nabla\left[\frac 12 \rho u^2 + p+\chi\right]=0.
\]
So this value $ H = \frac 12 \rho u^2 + p +\chi $ is constant on streamlines for steady flow. The physical consequence of this is that when $ u $ increases, $ p $ decreases (ignoring gravity). Let's now see a simple application.
\subsubsection{Tank of fluid with small drain}
Take a tank full of fluid with a small hole at the bottom. The water height has height $ h $ and the fluid exists the tank with speed $ u $. We assume that the size of the hole is small enough so the flow is steady. At the top of the tank, the pressure is atmospherical pressure and at the exit it is the same. Taking a streamline from the top on the tank to the exist we'll use the steady flow Bernoulli equation. We have gravity so $ \chi = - \rho gh $. At the top we have $ H_1 $ and at the bottom we have $ H_2 $, from Bernoulli, $ H_1=H_2 $,
\[
  \frac 12 \rho u^2 = \rho gh,
\]
so 
\[
	u=\sqrt{2gh}.
\]
\begin{remark}
  We take the hole very small, so $ u=0 $ (approximately) at the top of the tank.
\end{remark}
\subsubsection{Venturi meter}
This is a device to measure flow rates with no moving parts. We ignore gravity and assume the flow is steady and uniform across any cross section (gentle variations in the cross section $ A $). The device is a pipe which is pinched in the middle to a much smaller cross section area. Before the pinch we have $ A_1,u_1,p_1 $ and in the pinched area we have $ A_2,u_2,p_2 $. By conservation of mass, $ A_1u_1=A_2u_2 $. Now we attach two linked tubes of fluid which have a difference in height of fluid $ h $ one at the first point, and the other at the second point. From the Steady Bernoulli equation, $ \frac 12 \rho u_1^2 + p_1 = \frac \rho u_2^2 + p_2 $. So using our mass conservation,
\[
	p_1-p_2 = \frac 12 \rho u_1^2\left(\frac{A_1^2}{A_2^2} -1\right).
\]
If we measure $ h $ using hydrostatic balance,
\[
	\rho gh = \frac 12 \rho u_1^2 \left(\frac{A_1^2}{A_2^2} - 1\right)
\]
hence
\[
	A_1u_1 = \sqrt{2gh}\frac{A_1A_2}{\sqrt{A_1^2 -A_2^2}}.
\]
Now for the final example, let's see a water jet on a 2D oblique plane. Let the angle under the jet and the angled plane be $ \beta $. We will ignore gravity. We let the incoming fluid flow rate be $ U $ and the flow going up be $ a_2 $ and the flow going down be $ a_1 $. We also want to know what the force exerted on the plane.\par
First we apply Bernoulli on the surface streamline. So since $ p $ is atmospherical everywhere  hence $ u = U $ everywhere. By mass conservation, $ aU = a_1 U + a_2 U $, hence
\[
  a = a_1 + a_2.
\]
Now for more information, we apply the integral form of the Euler momentum equation. We'll take the volume $ V $ to be the entire volume of fluid.
\[
	\frac d{dt} \int_V\rho \mathbf u \mathrm dt = \int_{\partial V} -\rho \mathbf u (\mathbf u \cdot\mathbf n )\mathrm dA + \int_{\partial V} -p \mathbf n\mathrm dA.
\]
We've assumed the flow is steady, so the LHS is zero. $ p $ is atmospherical pressure everywhere except from the surface. On the surface streamlines due to the kinematic boundary condition, $ \mathbf u \cdot\mathbf u = 0 $. Components the Euler momentum equation on the surface gives
\[
	\rho a U^2\cos\beta = \rho a_2 U^2 - \rho a_1 U^2
\]
hence
\[a_2 = a_1 + a\cos\beta \]
so solving gives that
\begin{align*}
  a_1 = \frac a2(1-\cos\beta) \qquad a_2 = \frac a2(1+\cos\beta).
\end{align*}
Now instead if we calculate the component perpendicular to the surface,
\[
  F = \int p\mathbf n \mathrm dA = \rho a U^2\sin\beta.
\]
\subsection{Hydrostatic pressure and Archimedes}
What if we set $ \mathbf u = 0 $ in the Euler equation? Then $ 0 = -\nabla p + \mathbf f $. However there is a non-trivial distribution of pressure in the fluid. For example if we take gravity, so $\mathbf f = \rho \mathbf g $, hence 
\[
  0 = -\nabla p + \rho \mathbf g = -\nabla(p+\chi).
\]
So $ p+\chi $ is constant, so $ p = p_0 - \rho g z $. If we take $ z=0 $ to the top of the water, then we have $ p_0 $ as atmospherical pressure. This is called hydrostatic pressure. We can use this to mathematically derive Archimedes principle.\par
We want to calculate the total pressure force on a body exerted due to a fluid with no flow. Let $ \mathbf n $ be a normal to the body surface $ \partial V $. $ \mathbf F $ is the force exerted by the hydrostatic pressure field on the body. Hence
\begin{align*}
	\mathbf F &= \int_{\partial V} -p\mathbf n \mathrm dS\\
		  &= \int_{\partial V} -(p_0 - \rho g z)\mathbf n \mathrm dS\\
		  &= -\int_V \nabla (p_0 - \rho g z) \mathrm dV\\
		  &= \int_V \rho g \hat z \mathrm dV\\
		  &= \rho g V\hat z.
\end{align*}
Where we've extended the fluid pressure field into the body. $ \rho g V $ is the weight of the displaced fluid which is the buoyancy force.
\begin{remark}
  Suppose we have the Euler equation
  \[
	  \rho \frac{D\mathbf u }{Dt} = - \nabla p + \rho \mathbf g.
  \]
  We will write $ p = p_{\text{static}}+p' $, where $ p' $ is called the dynamic pressure, so our Euler equation becomes
  \[
	  \rho \frac{D\mathbf u}{Dt} = -\nabla p'
  \]
  hence $ p' $ denotes the pressure in fluid to motion only. 
\end{remark}
\subsection{Vorticity}
Recall we defined $ \boldsymbol \omega = \nabla\times \mathbf u $ as the \textit{vorticity} of the flow. Let's see some examples of vorticity.
\begin{itemize}
	\item Let $ \mathbf u = \boldsymbol \Omega \times \z $ (solid body rotation). Then using IA Vector Calculus identities, we have that $ \boldsymbol \omega = 2\boldsymbol\Omega $.
	\item Let $ \mathbf u$ be the \textit{shear flow} $ \mathbf u = (0,\gamma x,0) $, so $ \boldsymbol \omega = (0,0\gamma) $.
	\item Let $ \mathbf u = \frac{k}{2\pi r} \mathbf e_\phi $ in cylindrical coordinates, representing a line singular vortex. Then $ \boldsymbol \omega = 0 $ everywhere except at $ r = 0 $, where
		\[
		  \oint_r \mathbf u \cdot d\boldsymbol \rho = k =\iint \boldsymbol\omega \cdot\mathbf n \mathrm d\mathbf S
		\]
		by Stokes' theorem, giving that $ \boldsymbol\omega = (0,0,k\delta(r)) $
\end{itemize}
\subsubsection{Interpretation of $ \boldsymbol\omega $}
Here we will show $ \boldsymbol\omega $ is equivalent to twice the local rotation rate of fluid particles.\par
Consider a material line $ \delta\boldsymbol \ell $. How does $ \delt\boldsymbol\ell $ change from $ t $ to $ t+\delta t $.
\[
  \x + \delta\boldsymbol \ell \to \x+\delta\boldsymbol\ell + \mathbf u(\x+\delta\boldsymbol\ell,t) \delta t
\]
and
\[
  \x\to \x + \mathbf u(\x,t)\delta t.
\]
Hence our difference $ \delta\boldsymbol\ell $ goes to
\begin{align*}
	\delta\boldsymbol\ell &\to \x+\delta\boldsymbol\ell + \mathbf u(\x,\delta\boldsymbol\ell,t)\delta t-[\x+\mathbf u(\x,t)\delta t]\\
			      &= \delta\boldsymbol\ell + (\delta\boldsymbol \ell\cdot\nabla)\mathbf u\delta t.
\end{align*}
Hence
\[
	\frac{D}{Dt}\delta\boldsymbol\ell = (\delta \boldsymbol\ell \cdot\nabla)\mathbf u.
\]
Hence the tensor $ \frac{\partial u_i}{\partial x_j} $ determines changes in material lines (local rate of deformation). Write
\begin{align*}
	\frac{\partial u_i}{\partial x_j} &= \frac 12\left(\frac{\partial u_i}{\partial x_j}+\frac{\partial u_j}{\partial x_i}\right) +\frac 12 \left(\frac{\partial u_i}{\partial x_j}-\frac{\partial u_j}{\partial x_i}\right)\\
					  &= e_{ij} +\frac 12\varepsilon_{ijk}w_k,
\end{align*}
where $ \boldsymbol\omega = \nabla\times \mathbf u $ by the standard decomposition of the antisymmetric tensor as a vector. The contribution of the second term to $ \frac D{Dt}\partial \ell_i $ is
\begin{align*}
	\frac 12 \varepsilon_{jik}w_k \delta \ell_j &= \frac 12 \varepsilon_{ikj}w_k\delta\ell_j\\
						    &=\frac 12(\boldsymbol\omega\times \delta\boldsymbol\ell)_i.
\end{align*}
This is the rigid body rotation with angular velocity $ \boldsymbol\Omega = \frac 12 \boldsymbol\omega $. So $ \frac 12\boldsymbol\omega(\x,t) $ represents the average rotation rate of fluid particles. This is \textit{not} the same as having flow with circular streamline.
\subsubsection{Vorticity equation}
Let's start with the Euler momentum equation,
\[
	\rho\left(\frac{\partial \mathbf u}{dt} +(\mathbf u \cdot \nabla)\mathbf u \right) = -\nabla p+\mathbf f.
\]
Assume that $ \rho $ is constant and $ \mathbf f = -\nabla\chi $. We know that $ \mathbf u \cdot\nabla)\mathbf u = \nabla(\frac 12 u^2) - \mathbf u\times\boldsymbol\omega $, so taking the curl of the Euler equation we get that
\[
	\frac{\partial\boldsymbol\omega}{\partial t} -\nabla\times[\mathbf u \times \boldsymbol\omega]=0.
\]
Evaluating the $ i $th component of this equation,
\begin{align*}
	[\nabla\times(\mathbf u\times\boldsymbol\omega)]_i &= \varepsilon_{ijk} \frac{\partial}{\partial x_j} [\varepsilon_{kmn}u_m\omega_n]\\
	&= (\delta_{im}\delta_{jn}-\delta_{in}\delta_{jm})\left(\frac{\partial u_m}{\partial x_j}\omega_n + u_m\frac{\partial \omega_n}{\partial x_j}\right)\\
	&= \omega_j \frac{\partial u_i}{\partial x_j} - \omega_i \frac{\partial u_j}{\partial x_j}+u_i\frac{\partial\omega_j}{\partial x_j}-u_j\frac{\partial \omega_i}{\partial x_j}.
\end{align*}
Mass is conserved and the divergence of the curl is zero, so the second and third terms are zero, hence this is equal to
\[
	(\boldsymbol\omega\cdot \nabla)u_i-(\mathbf u \cdot\nabla)\omega_i.
\]
So the vorticity equation is
\[
	\frac{\partial \boldsymbol\omega}{\partial t} + (\mathbf u \cdot \nabla)\boldsymbol\omega = (\boldsymbol\cdot\nabla)\mathbf u.
\]
This is equivalent to
\[
	\frac{D}{Dt}\boldsymbol\omega = (\boldsymbol\omega \cdot\nabla)\mathbf u.
\]
The material derivative recall is the rate of change of the field while moving with the flow. 
\subsubsection{Vortex stretching}
Let's compare the two equations
\[
	\frac{D}{Dt}\delta\boldsymbol\ell = (\delta\boldsymbol\ell\cdot\nabla)\mathbf u\qquad \frac D{Dt}\boldsymbol\omega = (\boldsymbol\omega\cdot\nabla)\mathbf u.
\]
We can see these are the same equation. So moving with the fluid, $ \boldsymbol\omega $ changes just like a material line element initially parallel to $ \boldsymbol\omega $. In particular, if $ \delta\boldsymbol\ell $ gets longer (stretching) then $ \boldsymbol\omega $ gets larger. This is conservation of angular momentum. When the fluid is stretched in the direction of $ \boldsymbol\omega $, we get an increase in vorticity.
\begin{definition}
	(Circulation) Given a closed curve $ \Gamma $, the \textit{circulation} around $ \Gamma $ is defined as
	\[
	  C(t)=\int_\Gamma \mathbf u \cdot\mathrm d\boldsymbol\ell.
	\]
\end{definition}
By Stokes' theorem we have that
\[
  C(t) = \iint_S \boldsymbol \omega \cdot\mathrm \mathbf S
\]
where $ S $ is an open surface with $ \partial S = \Gamma $.
\begin{theorem}
	\textit{Non-examinable}	(Kelvine circulation theorem) For an inviscid fluid with constant density and conservative body forces then if we take $ \Gamma $ to be a material line then
	\[
		\frac{d}{dt}C = 0
	\]
\end{theorem}
\begin{remark}
	If the flow is irrotatoinal at $ t=0 $ then it is irrotional at all times.
\end{remark}
\section{Viscosity}
The previous chapter only covered flows which were inviscid and we neglected any friction in fluid flow. We had that
\begin{enumerate}
	\item $ \boldsymbol\tau = -p\mathbf n $
	\item In our momentum balance equation (Euler) we had $ \rho\frac{D\mathbf u}{Dt}\sim (\nabla p;\ \mathbf F) $.
\end{enumerate}
Now we will include viscosity; we will find
\begin{enumerate}
	\item New \textit{tangential} stress between fluid layers or between fluid and the boundary.
	\item New term in the momentum balance equation.
\end{enumerate}
Here in IB Fluid Dynamics we will only focus on 2D parallel viscious flows. So we will have that
\[
  \mathbf u = (u(y,t),0,0).
\]
We instantly get that $ \nabla\cdot\mathbf u = 0 $.
\subsection{Plane Couette flow}
Consider a a steady flow between two parallel plates, driven only by the motion of the top plate parallel to the bottom one with speed $ U $. Experiments report that for the wide variety of fluids (Newtonian fluids) the find that
\begin{enumerate}
	\item The fluid velocity at the top is $ U $ and at the bottom it is zero.
	\item The fluid flow velocity varies linearly between $ 0 $ and $ V $, so $ u(y)=U\frac yh $.
	\item The tangential force $ \tau $ required to move the top plane is linear in $ U $ and inversely proportional to $ h $.
\end{enumerate}
So we have that
\[
  \tau \propto \frac Uh
\]
\begin{definition}
	(Dynamic viscosity) The \textit{dynamic viscosity}, $ \mu $, of a fluid is the proportionality constant from above, so
	\[
	  \tau = \mu \frac Uh.
	\]
	We usually write this as
	\[
		\tau = \mu \frac{\partial u}{\partial n},
	\]
	the viscous tangential stress exerted by the positive side to the negative side, where the normal is pointing from the interface to the positive side.
\end{definition}
\begin{remark}
  The viscosity is a material property of the fluid.
\end{remark}
We call $ \frac Uh $ the shear rate of the fluid.
\subsection{2D parallel viscous flow}
\subsubsection{Steady case with $ \mathbf F = 0 $}
Consider an infinitesimal volume of fluid $ \delta x \delta y $. Consider a flow $ u(y,t) $ moving left to right in the volume. On the left and right side we only have the pressure force and no tangential forces. On the top and bottom we have the tangetial frictional forces equal to
\[
	\pm\mu \frac{\partial u}{\partial y}.
\]
Using the balance of forces in the $ x $-direction we have that
\[
	p(x)\delta y - p(x+\delta x) \delta y + \left( \mu \frac{\partial y}{\partial y}(y+\delta y) - \mu \frac{\partial u}{\partial y}(y)\right)\delta x = 0.
\]
Expanding the $ \delta x\delta y $ terms we get that
\[
	\delta x \delta y\left(- \frac{\partial p }{\partial x} + \mu \frac{\partial ^2 u}{\partial y^2}\right) = 0
\]
which gives that
\[
	-\frac{\partial p}{\partial x} + \mu \frac{\partial^2 u}{\partial y^2} = 0.
\]
In the $ y $-direction we get that
\[
	\frac{\partial p}{\partial y}  =0
\]
\subsubsection{General case with $ u(y,t) $ and $ \mathbf F $}
On Example Sheet 2 we will see that the general case solution is
\begin{align*}
	\rho \frac{\partial u}{\partial t} &= -\frac{\partial p }{\partial x} + \mu\frac{\partial ^2 u}{\partial y^2} + F_x\\
	0 &= -\frac{\partial p}{\partial y} + F_y.
\end{align*}
\begin{remark}
  For flows of the form $ \mathbf u = (u(y,t), 0,0) $ we have that $ (\mathbf u \cdot \nabla) \mathbf u = 0 $ so this explains why this result is much simplier than it seems it would be.
\end{remark}
\subsubsection{The no-slip boundary condition}
Notice that we've now increased the order of our differential equations. This means that we now need another boundary condition to solve our equations. \par
Experiements tell us that for viscious flow at a rigid boundary $ \mathbf u = \mathbf U $, so the tagnetial components of the flow match the velocity of the rigid boundary. This is called the no-slip boundary condition and is \textit{only} applicable to viscious flows.\par
Recall we still have our other boundary condition: the no penetration boundary condition
\[
  \mathbf u \cdot\mathbf n = \mathbf U \cdot\mathbf n.
\]
\subsubsection{Examples} 
Poiseuille flow in a channel: This is steady flow in a (2D) cannel driven by a pressure gradient. Say the channel has a length $ L $ and height $ h $, with pressure $ p_0 $ at the input of the channel and $ p_1 $ outwards. We have steady flow, so $ \frac \partial{\partial t} = 0 $ and our no-slip boundary conditions gives that $ u(0) = 0 $ and $ u(h) = 0 $. We have that $ \frac{\partial p}{\partial y} = 0 $ so $ p=p(x) $. Further we have that
\[
	\frac{\partial p}{\partial x} = \mu \frac{\partial^2 u}{\partial y^2}
\]
hence both are constant functions. Solving gives that 
\[
	u(y) = \frac G{2\mu} y(h-y)
\]
where we define
\[
	G = -\left(\frac{p_1-p_0}2\right).
\]
So our flow is a parabola, with zero flow at the boundaries of the channel walls. We can calculate the flow rate, $ q $, as
\[
  q = \int_0^h u\ \mathrm dy = \frac{(p_1-p_0)h^3}{12\mu L}.
\]
\begin{remark}
  We can see if we balance forces that we have that
  \[
	  p_1 h - p_0 h + \tau_{\text{top}} L - \tau_{\text{bottom}} L = 0
  \]
  which we can calculate and expand out to see using the fact that
  \[
	  \tau_{\text{top}} = \mu \frac{\partial u}{\partial \mathbf n} = \mu \frac{\partial u}{\partial y}\mid_{y=h}.
  \]
\end{remark}
Now let's look at a viscous flow driven down an inclined plane by gravity (body force). Suppose the plane is inclined by an angle $ \alpha $ and let the $ x $ direction be parallel to the plane and the $ y $ direction be perpendicular to the plane. Let the fluid thickness be $ h $.
\[
  \mathbf F = \rho \mathbf g = (\rho g \sin\alpha, -\rho g \cos\alpha).
\]
In the $ y $ direction we have that
\[
	0 = -\frac{\partial p}{\partial y} - \rho g \cos\alpha
\]
with boundary conditions
\[
	p(h) = p_a
\]
so we get that
\[
  p = p_a +\rho g (h-y)\cos\alpha
\]
which resembles the hydrostatic balance equation (but now tilted). In the $ x $ direction we have that
\[
   0 = -\frac{\partial p}{\partial x} + \mu \frac{\partial^2 u}{\partial y^2} + \rho g \sin \alpha.
\]
Since $ h $ is constant the first term is zero. We assume there is no shear stress exerted by the fluid on the air hence we can see that
\[
	\mu \frac{\partial u}{\partial y}\mid_{y=h} = 0.
\]
So solving
\[
	\frac{\partial^2 u}{\partial y^2} = -\frac{\rho g}\mu \sin\alpha
\]
gives that
\[
	u(y) = \frac{\rho g \sin\alpha}{2\mu } y(2h-y).
\]
\subsubsection{Boundary condition at an interface}
Suppose we have two fluids. Fluid 1 has viscosity $ \mu_1 $ and fluid 2 has viscosity has viscosity $ \mu_2 $. What are the boundary conditions at the interface of the fluid?
\begin{enumerate}
	\item No slip: $ u_1=u_2 $ at the interface\\
	\item Continuity of stress: In the normal direction $ p_1=p_2 $ and in the tangential direction \[ \mu_1\frac{\partial u_1}{\partial y} = \mu_2 \frac{\partial u_2}{\partial y} \].
\end{enumerate}
\subsection{Unsteady parallel viscous flows and viscous diffusion}
\begin{example}
  Fundamental example: Rayleigh's problem (Stokes' 1st problem). This is the impulsively started plate. Consider a semi-infinite viscous fluid in $ y>0 $ that is initially at rest with no applied pressure gradient, $ \mathbf f = 0 $. At $ t=0^+ $ the plate starts to move at constant velocity $ (U,0) $. What are the equations of motion? We have that
  \[
	  \frac{\partial p }{\partial y} = 0
  \]
  so $ p = p_0 $. In the $ x $-direction,
  \[
	  \rho\frac{\partial u }{\partial t} = -\frac{\partial p}{\partial x} + \mu\frac{\partial ^2 u }{\partial y^2} + f_x
  \]
  which reduces to the diffusion equation
  \[
	  \frac{\partial u}{\partial t} = \frac\mu\rho \frac{\partial^2 u}{\partial y^2}.
  \]
  The initial condition is $ u(y,t) = 0 $ at $ t=0 $. And the boundary conditions are 
  \begin{align*}
	  u(0,t) &= U\\
	  u(\infty, t ) = 0
  \end{align*}
  with the first equation coming from the no-slip condition. We define $ \frac \mu \rho=\nu $ as the kinematic viscosity. If $ \mu = 0 $ nothing happens. We see that $ \nu $ is the diffusivity for momentum (and vorticity: see later). So all we have to do is solve the differential equation using a Fourier series or seperation of variables (IB Methods). Here we use dimensional analysis so $ u(y,t) = Uf(t,y,\nu) $ so $ f(t,y,\nu) $ is dimensionless. Hence
  \[
	  \frac uU = f\left(\frac y{\sqrt{\nu t}}\right)
  \]
  so $ y\sim \sqrt{\nu t} $. Define our \textit{similarity variable} as
  \[
	  \eta = \frac{y}{\sqrt{vt}}
  \]
  so putting this into our diffusion equation we get that
  \[
	  -\frac 12 \frac \eta t f' U = \frac{\nu}{\nu t}Uf''
  \]
  and solving using $ f(0) = 1 $ and $ f(\infty) = 0 $ gives that
  \[
	  u(y,t) = U \mathrm{erfc}\left(\frac y{2\sqrt {\nu t}}\right)
  \]
  where
  \[
	  \mathrm{erfc}(x) = \frac 2{\sqrt \pi} \int^\infty_x e^{-t^2} \mathrm dt.
  \]
  So the force required to move the plate at $ y=0 $ is
  \[
	  \tau = \mu\frac{\partial u}{\partial y}\mid_{y=0} = -\frac{\mu U}{\sqrt{\nu t}} f'(0) = -\frac{\mu U}{\sqrt{\pi \nu t}}.
  \]
  We have that $ \mathbf u = (u,0,0) $, so $ \boldsymbol\omega = (0,0,-\frac{\partial u}{\partial y} $ hence the vorticity also diffuses.
\end{example}
Let's see some more examples
\begin{example}
  Now add a boudary at $ y=h $. Now our similarity dimensionless solution is no longer possible.
  \begin{enumerate}
	  \item At small times $ (\nu t)^{\frac 12} \ll h $ this is approximately like the previous example.
	  \item At long times eventually we have Couette flow.
  \end{enumerate}
  The characteristic time-scale is $ t\sim \frac {h^2}\nu $ for diffusion of momentum.
\end{example}





\end{document}
